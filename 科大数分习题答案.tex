% !TEX TS-program = pdflatex
% !TEX encoding = UTF-8 Unicode

% This is a simple template for a LaTeX document using the "article" class.
% See "book", "report", "letter" for other types of document.

\documentclass[oneside]{ctexbook} % use larger type; default would be 10pt

\usepackage[utf8]{inputenc} % set input encoding (not needed with XeLaTeX)

\usepackage{amsmath}
\usepackage{amssymb}

\usepackage{color}
\definecolor{lightblue}{RGB}{70,130,196}

\usepackage[colorlinks=true, allcolors=lightblue]{hyperref}

%%% Examples of Article customizations
% These packages are optional, depending whether you want the features they provide.
% See the LaTeX Companion or other references for full information.

%%% PAGE DIMENSIONS
\usepackage{geometry} % to change the page dimensions
\geometry{a4paper} % or letterpaper (US) or a5paper or....
% \geometry{margin=2in} % for example, change the margins to 2 inches all round
% \geometry{landscape} % set up the page for landscape
%   read geometry.pdf for detailed page layout information

\usepackage{graphicx} % support the \includegraphics command and options

% \usepackage[parfill]{parskip} % Activate to begin paragraphs with an empty line rather than an indent

%%% PACKAGES
\usepackage{booktabs} % for much better looking tables
\usepackage{array} % for better arrays (eg matrices) in maths
\usepackage{paralist} % very flexible & customisable lists (eg. enumerate/itemize, etc.)
\usepackage{verbatim} % adds environment for commenting out blocks of text & for better verbatim
\usepackage{subfig} % make it possible to include more than one captioned figure/table in a single float
% These packages are all incorporated in the memoir class to one degree or another...

%%% HEADERS & FOOTERS
\usepackage{fancyhdr} % This should be set AFTER setting up the page geometry
\pagestyle{fancy} % options: empty , plain , fancy
\renewcommand{\headrulewidth}{0pt} % customise the layout...
\lhead{}\chead{}\rhead{}
\lfoot{}\cfoot{\thepage}\rfoot{}

%%% ToC (table of contents) APPEARANCE
\usepackage[nottoc,notlof,notlot]{tocbibind} % Put the bibliography in the ToC
\usepackage[titles,subfigure]{tocloft} % Alter the style of the Table of Contents
\renewcommand{\cftsecfont}{\rmfamily\mdseries\upshape}
\renewcommand{\cftsecpagefont}{\rmfamily\mdseries\upshape} % No bold!

%%% END Article customizations

%%% The "real" document content comes below...

\title{科大数分习题答案}
\author{johnsmith0x3f}
\date{\today} % Activate to display a given date or no date (if empty),
              % otherwise the current date is printed

\begin{document}

\maketitle

\newpage

\frontmatter

\tableofcontents

\newpage

\mainmatter

\part*{数学分析讲义 \ 第一册}
\addcontentsline{toc}{part}{数学分析讲义 \ 第一册}

\thispagestyle{empty}

\newpage

\chapter{极限}

\section{实数}

\begin{enumerate}
    \item[1.]
    假设 $a \oplus b = c$ 为有理数,则 $b = a \otimes c$,其中 $\oplus$ 和 $\otimes$ 为任意一对互逆的四则运算. 这与有理数集对四则运算的封闭性矛盾,故假设不成立,$c$ 为无理数.
    \item[3.]
    若 $\sqrt 2$ 为有理数,则存在最小的正整数 $q$ 使得 $\sqrt 2 q$ 是正整数,然而令 $r = \sqrt 2 q - q < q$,则 $\sqrt 2 r$ 亦为正整数,故 $\sqrt 2$ 为无理数. 同理可得 $\sqrt 3,\ \sqrt 6$ 均为无理数.
    
    假设 $\sqrt 2 + \sqrt 3$ 为有理数,则 $(\sqrt 2 + \sqrt 3)^2 = 5 + 2 \sqrt 6$ 为有理数,与 $\sqrt 6$ 为无理数矛盾,故 $\sqrt 2 + \sqrt 3$ 为无理数.
    \item[6.]
    设数列 $\{ x_n \} = (1+a_1)(1+a_2)\cdots(1+a_n) - (1+a_1+a_2+\cdots+a_n)$,则 $x_{n+1} - x_n = a_{n+1} \left[ (1+a_1)(1+a_2)\cdots(1+a_n)-1 \right]$.
    
    若 $a_i \geqslant 0$,显然有 $x_{n+1} \geqslant x_n$;若 $-1 \leqslant a_i < 0$,此时 $(1+a_1)(1+a_2)\cdots(1+a_n) < 1$,亦有 $x_{n+1} > x_n$,故 $\{ x_n \}$ 为递增数列. 可得 $x_n \geqslant x_1 = 0$.
    \item[7.]
    注意到 $(a-1)(b-1) > 0$ 和 $(a+1)(b+1) > 0$,即 $1+ab > \pm(a+b)$.
    
    故 $|1+ab| = 1+ab > |a+b|$,即 $\left| \frac{a+b}{1+ab} \right| < 1$.
\end{enumerate}

\section{数列极限}

\begin{enumerate}
    \item[9.]
    不能,易举出反例. 如 $a_n = \frac 1 {2^n}$.
    
    事实上,只有当 $a \neq 0$ 时,才能通过极限的四则运算得到 $\displaystyle\lim_{n \to \infty} \frac{a_{n+1}}{a_n} = 1$.
    \item[12.]
    \begin{enumerate}
        \item[(1)]
        未证 $\{ a_n \}$ 收敛.
        \item[(2)(3)]
        极限的四则运算仅在有限次内成立.
    \end{enumerate}
    \item[15.]
    \begin{enumerate}
        \item[(2)]
        注意到 $0 < (n+1)^k - n^k = n^k \left[(1 + \frac 1 n)^k - 1 \right] < n^k (1 + \frac 1 n - 1) = n^{k-1}$.
        
        由 $\displaystyle \lim_{n \to \infty} n^{k-1} = 0$ 及夹逼原理知,$\displaystyle\lim_{n \to \infty} \left[(n+1)^k - n^k \right] = 0$.
    \end{enumerate}
    \item[16.]
    不妨设 $A = \max\{ a_1,\ a_2,\ \ldots,\ a_m \}$,则 $1 < \frac 1 A \sqrt[n]{a_1^n+a_2^n+\cdots+a_m^n} \leqslant \sqrt[n]{m}$.
    
    由 $\displaystyle \lim_{n \to \infty} \sqrt[n]{m} = 1$ 及夹逼原理知,$\displaystyle\lim_{n \to \infty} \frac 1 A \sqrt[n]{a_1^n+a_2^n+\cdots+a_m^n} = 1$.
    
    故 $\displaystyle \lim_{n \to \infty} \sqrt[n]{a_1^n+a_2^n+\cdots+a_m^n} = A$.
    \item[17.]
    \begin{enumerate}
        \item[(3)(4)]
        利用 Cauchy 收敛准则易证.
    \end{enumerate}
    \item[18.]
    \begin{enumerate}
        \item[(2)(4)]
        实质上是求数列不动点,由数学归纳法结合定义法易求得极限.
    \end{enumerate}
    \item[21.]
    由题设知 $\frac{a_{n+1}}{b_{n+1}} \leqslant \frac{a_n}{b_n}$,又 $\frac{a_n}{b_n} > 0$,由单调有界原理知 $\{ \frac{a_n}{b_n} \}$ 收敛.
    
    则 $\displaystyle\lim_{n \to \infty} a_n = \left( \lim_{n \to \infty} \frac{a_n}{b_n} \right) \left(\lim_{n \to \infty} b_n \right)$,故 $\{ a_n \}$ 收敛.
    \item[25.]
    由题设知 $a_n>0$,故有 $a_{n+1}>a_n$.
    
    假设存在 $M>0$,使 $a_n \leqslant M$ 恒成立,则 $a_{M^2+1} = 1 + \frac 1 {a_1} + \frac 1 {a_2} + \cdots + \frac 1 {a_{M^2}} \ge 1 + M^2 \cdot \frac 1 M > M$,与假设矛盾.
    
    故数列 $\{ a_n \}$ 单调递增且无界,即 $ a_n \to +\infty \ (n \to \infty)$.
    \item[26.]
    由题设知 $\forall \varepsilon>0,\ \exists N \in \mathbf{N^*}$ 使得 $n>N$ 时有 $A-\varepsilon < \frac{a_{n+1}-a_n}{b_{n+1}-b_n} < A+\varepsilon$.
    
    由于 $\{ b_n \}$ 严格单调递减,则有 $(A-\varepsilon)(b_n-b_{n+1}) < a_n - a_{n+1} < (A+\varepsilon)(b_n-b_{n+1})$.
    
    取正整数 $m>n$,累加得 $(A-\varepsilon)(b_n-b_m) < a_n - a_m < (A+\varepsilon)(b_n-b_m)$.
    
    即 $\left| \frac{a_n-a_m}{b_n-b_m} - A \right| < \varepsilon$,令 $m \to \infty$ 即得 $\left| \frac{a_n}{b_n} - A \right| < \varepsilon$.
\end{enumerate}

\section{函数极限}

\begin{enumerate}
    \item[7.]
    $\displaystyle\lim_{n \to \infty} \left( \sin\frac \alpha {n^2} + \sin\frac{2\alpha}{n^2} + \cdots + \sin\frac{n\alpha}{n^2} \right) = \lim_{n \to \infty} \left( \frac \alpha {n^2} + \frac{2\alpha}{n^2} + \cdots + \frac{n\alpha}{n^2} \right) = \frac \alpha 2$.
    \item[9.]
    \begin{enumerate}
        \item[(2)]
        $\displaystyle\lim_{x \to 0} \frac{\cos x-\cos 3x}{x^2} = \lim_{x \to 0} \frac{2\sin 2x\sin x}{x^2} = \lim_{x \to 0} \frac {2 \cdot 2x \cdot x}{x^2} = 4$.
        \item[(4)]
        $\displaystyle\lim_{x \to \infty} \left(\frac{x^2+1}{x^2-1}\right)^{x^2} = \lim_{x \to \infty} \left(1+\frac 1 {x^2}\right)^{x^2} \cdot \left(1+\frac 1 {x^2-1}\right)^{x^2-1} \cdot \left(1+\frac 1 {x^2-1}\right) = e^2$.
    \end{enumerate}
    \item[10.]
    \begin{enumerate}
        \item[(2)]
        $\forall \ \varepsilon > 0$,当 $0 < |x| < \varepsilon$ 时有 $\left|x^2 \sin \frac 1 x - 0\right| < |x^2| \cdot \left| \frac 1 x \right| = |x| < \varepsilon$.
        
        故 $\displaystyle \lim_{x \to 0} x^2 \sin \frac 1 x = 0$.
    \end{enumerate}
    \item[18.]
    设常数 $k \in \mathbf{N^*}$,则 $\displaystyle\lim_{x \to 0} \frac{\sqrt[k]{1+x}}{1+\frac x k} = \sqrt[k]{\frac{1+x}{1+x+\frac {k-1} 2 x^2+\cdots+x^k}} = 1$.
    
    即 $x \to 0$ 时 $\sqrt[k]{1+x} \sim 1 + \frac x k$. 由此易解 (3)(5)(6).
\end{enumerate}

\section*{第 1 章综合习题}
\addcontentsline{toc}{section}{第 1 章综合习题}

\begin{enumerate}
    \item[5.]
    \begin{enumerate}
        \item[(1)]
        $A_n < A_{n+1} \leqslant M$.
        \item[(2)]
        $|a_m - a_n| \leqslant |a_{n+1}-a_n|+\cdots+|a_m-a_{m-1}| = |A_{m-1}-A_n| < \varepsilon \ (m>n)$.
    \end{enumerate}
    \item[6.]
    \begin{enumerate}
        \item
        若 $\{ a_n \}$ 收敛,则 $\displaystyle\lim_{n \to \infty} a_n^{\alpha} = \lim_{n \to \infty} a_{n+1}^{\alpha} = \left(\lim_{n \to \infty} a_n\right)^{\alpha}$,即 $\displaystyle\lim_{n \to \infty} \left(a_{n+1}^{\alpha}-a_n^{\alpha}\right) = 0$.
        \item
        否则不妨设 $a_{n+1}-a_n \leqslant k a_1 \ (k>0)$,则 $\frac{a_{n+1}}{a_n} \leqslant 1 + \frac{ka_1}{a_n} \leqslant 1 + k$.
        则 $0 < a_{n+1}^{\alpha}-a_n^{\alpha} \leqslant a_n^{\alpha}\left((1+k)^{\alpha}-1 \right) < k a_n^{\alpha}$.
        
        由 $\displaystyle\lim_{n \to \infty} ka_n^{\alpha} = 0$ 及夹逼原理知,$\displaystyle\lim_{n \to \infty} \left(a_{n+1}^{\alpha}-a_n^{\alpha}\right) = 0$.
    \end{enumerate}
    \item[8.]
    $a \neq 0$ 时,有 $\displaystyle\lim_{n \to \infty} \frac 1 {a_n} = \frac 1 a$. 又由均值不等式有
    
    $$
    \frac{n}{\frac 1 {a_1} + \frac 1 {a_2} + \cdots + \frac 1 {a_n}} \leqslant \sqrt[n]{a_1a_2\cdots a_n} \leqslant \frac {a_1+a_2+\cdots+a_n} n
    $$
    
    由 Cauchy 命题易知
    
    $$
    \lim_{n \to \infty} \frac{n}{\frac 1 {a_1} + \frac 1 {a_2} + \cdots + \frac 1 {a_n}} = \frac 1 {\frac 1 a} = a = \lim_{n \to \infty} \frac {a_1+a_2+\cdots+a_n} n
    $$
    
    故由夹逼原理知,$\displaystyle\lim_{n \to \infty} \sqrt[n]{a_1a_2\cdots a_n} = a$.
    
    $a = 0$ 时运用不等式 $0 < \sqrt[n]{a_1a_2\cdots a_n} \leqslant \frac {a_1+a_2+\cdots+a_n} n$ 即可.
    \item[12.]
    若 $\{ b_n \}$ 收敛,则
    
    $$
    \lim_{n \to \infty} c_n = \lim_{n \to \infty} \frac{\frac {a_1b_1+a_2b_2+\cdots+a_nb_n} n}{\frac {b_1+b_2+\cdots+b_n} n} = a
    $$
     
    否则由 Stolz 定理知,$\displaystyle\lim_{n \to \infty} c_n = \lim_{n\to \infty} \frac{a_{n+1}b_{n+1}}{b_{n+1}} = a$.
    \item[16.]
    由 Dirichlet 逼近定理知,对任意 $k \in \mathbf{N^*}$,存在 $i,\ j \in \mathbf{N^*}$ 满足 $0<|\{ i\xi \}-\{ j\xi \}|<\frac 1 k$.
    
    不妨令 $k>\frac 1 {b-a}$,取最小正整数 $n$ 满足 $n|\{ i\xi \}-\{ j\xi \}|>a$.
    
    若 $n|\{ i\xi \}-\{ j\xi \}| \geqslant b > a + \frac 1 k > a + |\{ i\xi \}-\{ j\xi \}|$,则 $(n-1)|\{ i\xi \}-\{ j\xi \}|>a$,矛盾.
    
    故存在 $n|\{ i\xi \}-\{ j\xi \}| \in (a,\ b)$.
\end{enumerate}

\newpage

\chapter{单变量函数的连续性}

\section{连续函数的基本概念}

\begin{enumerate}
    \item[15.]
    易知 $f(nx) = nf(x) \ (n \in \mathbf{N})$,进而有 $f\left(\frac p q x\right) = \frac p q f(x) \ (p,\ q \in \mathbf{Z},\ q \neq 0)$.
    
    对任意无理数 $r$,以有理数逼近之,得到收敛于 $r$ 的无穷数列.
    
    又知 $f(x)$ 连续,故 $f(\lambda x) = \lambda f(x)$ 对任意实数 $\lambda$ 成立,故有 $f(x) = x f(1)$.
    \item[17.]
    \begin{enumerate}
        \item[(6)]
        $x \to \infty$ 时,$\sqrt{x^2+k} = |x| \sqrt{1+\frac k {x^2}} \sim |x|(1+\frac k {2x^2})$.
        
        故 $\displaystyle\lim_{x \to -\infty} x(\sqrt{x^2+100}+x) = -50$.
        \item[(7)]
        $\displaystyle\lim_{x \to \infty} \left( \sin\sqrt{x+1} - \sin\sqrt x \right) = \lim_{x \to \infty} 2 \cos{\frac {\sqrt{x+1}+\sqrt x} 2}\sin{\frac {\sqrt{x+1}-\sqrt x} 2} = 0$.
    \end{enumerate}
\end{enumerate}

\section{闭区间上连续函数的性质}

\begin{enumerate}
    \item[6.]
    设 $g(x) = f(x)-f(x+a) \ (0 \leqslant x \leqslant a)$,则 $g(0) = -g(a)$,由介值定理可得欲证.
    \item[7.]
    不妨设 $f(x_1) \leqslant f(X_2) \leqslant \cdots \leqslant f(x_n)$,则 $f(x_1) < f(\xi) < f(x_n)$,显然存在这样的 $\xi$.
    \item[16.]
    $f(x) = \sin x^2$.
\end{enumerate}

\section*{第 2 章综合习题}
\addcontentsline{toc}{section}{第 2 章综合习题}

\begin{enumerate}
    \item[5.]
    设 $g(x) = f(x) - f(x+\frac 1 n) \ (0 \leqslant x \leqslant 1 - \frac 1 n)$,则 $g(0) + g(\frac 1 n) + \cdots + g(1-\frac 1 n) = 0$.
    
    则或 $g(0) = 0$;或存在 $g\left( \frac i n \right) \cdot g\left( \frac j n \right) < 0 \ (0 \leqslant i < j < n)$,由零点定理可得欲证.
    \item[8.]
    \begin{enumerate}
        \item[(2)]
        由题设知 $|x_{n+1}-x_0| \le k |x_n-x_0|$,故 $0 < |x_n-x_0| \le k^{n-1}|x_1-x_0|$.
        由夹逼原理知,$\displaystyle\lim_{n \to \infty} |x_n-x_0| = 0$,即 $\displaystyle\lim_{n \to \infty} x_n = x_0$.
    \end{enumerate}
    \item[10.]
    事实上可以证明 $f(b)$ 即为 $f(x)$ 在 $[a,\ b]$ 上能取到的最大值.
\end{enumerate}

\newpage

\chapter{单变量函数的微分学}

\section{导数}

\begin{enumerate}
    \item[17.]
    \begin{enumerate}
        \item[(1)]
        设 $\displaystyle f(x) = x + x^2 + \cdots + x^n = \frac{x(1-x^n)}{1-x} \ (x \neq 1)$,
        
        则 $\displaystyle P_n = f'(x) = \frac{nx^{n+1}+(x+1)x^n+1}{(x-1)^2}$.
        \item[(2)]
        $\displaystyle Q_n = f'(x)+xf''(x) = \frac{n^2x^{n+2}-(2n^2+2n-1)x^{n+1}+(n+1)^2x^n-x-1}{(x-1)^3}$.
        \item[(3)]
        设 $\displaystyle g(x) = \sin x + \sin 2x + \cdots + \sin nx = \frac{\cos\frac x 2 - \cos(\frac {2n+1} 2 x)}{2\sin\frac x 2}$,
        
        则 $\displaystyle R_n = g'(1) = \frac{((n+1)\cos(n)-n\cos(n+1)-1)}{4\sin^2\frac 1 2}$.
    \end{enumerate}
\end{enumerate}

\section{微分}

\begin{enumerate}
    \item[3.]
    \begin{enumerate}
        \item[(1)]
        $\displaystyle \begin{cases} \mathrm dx = \frac{2t}{1+t^2} \mathrm dt \\ \mathrm dy = \frac 1 {1+t^2} \mathrm dt \end{cases} \ \Rightarrow \ \begin{cases} \frac{\mathrm dy}{\mathrm dx} = \frac 1 {2t} \\ \frac{\mathrm d^2y}{\mathrm dx^2} = \frac{\mathrm d \left( \frac{\mathrm dy}{\mathrm dx} \right)}{\mathrm dx} = -\frac{1+t^2}{4t^3} \end{cases}$.
    \end{enumerate}
\end{enumerate}

\section{微分中值定理}

\begin{enumerate}
    \item[4.]
    \begin{enumerate}
        \item[(3)]
        对 $f(x) = x \ln x$ 分别在 $\left( a,\ \frac {a+b} 2 \right)$ 和 $\left( \frac {a+b} 2,\ b \right)$ 上运用 Lagrange 中值定理易得.
    \end{enumerate}
    \item[18.]
    由题设知对任意 $x>0$ 存在 $\xi \in (0,\ x)$ 使得 $\frac{f(x)-f(0)}{x-0} = f'(\xi) < f'(x)$.
    
    则 $\frac{\mathrm d \frac{f(x)}{x}}{\mathrm dx} = \frac{xf'(x)-f(x)}{x^2} > 0$,即 $\frac{f(x)}{x}$ 严格递增.
    \item[20.]
    设 $g(x)=e^x\left[ f(x)-f'(x) \right]$,则 $g(0) = g(1) = 0$.
    
    故由 Rolle 定理知,存在 $\xi \in (0,\ 1)$,满足 $g'(\xi) = e^{\xi}\left[ f(\xi) - f''(\xi) \right] = 0$,即 $f(\xi) = f''(\xi)$.
    \item[23.]
    \begin{enumerate}
        \item[(1)]
        显然有 $x^p + (1-x)^p \leqslant x + 1-x = 1$.
        
        不妨设 $x < \frac 1 2$,则 $\displaystyle \frac 1 {2^{p-1}} \leqslant x^p + (1-x)^p \ \Leftrightarrow \ \frac 1 {2^p} - x^p \leqslant (1-x)^p - \frac 1 {2^p}$.
        
        由 Lagrange 中值定理知,存在 $x < \xi_1 < \frac 1 2 < \xi_2 < 1-x$,满足
        
        $$
        \frac{\frac 1 {2^p} - x^p}{\frac 1 2 - x} = p\xi_1^{p-1} < p\xi_2^{p-1} = \frac{(1-x)^p - \frac 1 {2^p}}{1-x-\frac 1 2}
        $$
        
        则原式得证.
        \item[(3)]
        由 Lagrange 中值定理知,存在 $0 < \xi_1 < x_1 < \xi_2 < x_2 < \frac \pi 2$,满足
        
        $$
        \frac{\tan x_1-0}{x_1-0} = \frac 1 {\cos^2 \xi_1} < \frac 1 {\cos^2 \xi_2} = \frac{\tan x_2 - \tan x_1}{x_2 - x_1}
        $$
        
        整理后即得原式.
        \item[(5)]
        注意到不等式两侧均为关于 $x$ 的偶函数,而易知 $x = 0$ 时等号成立,故不妨设 $x > 0$.
        
        由 Lagrange 中值定理知,存在 $\xi \in (0,\ x)$,满足
        
        $$
        \frac{\ln\left( x+\sqrt{1+x^2} \right)-0}{x-0} = \frac 1 {\sqrt{1+\xi^2}} > \frac 1 {\sqrt{1+x^2}}
        $$
        
        带入原式易证.
    \end{enumerate}
\end{enumerate}

\section{未定式的极限}

\begin{enumerate}
    \item[4.]
    由 Cauchy 中值定理知,存在 $\xi \in (a,\ b)$ 满足
    
    $$
    \frac{\frac {f(b)} b - \frac {f(a)} a}{\frac 1 b - \frac 1 a} = \frac{\frac{\xi f'(\xi) - f(\xi)}{\xi^2}}{-\frac 1 {\xi^2}} = f(\xi) - \xi f'(\xi)
    $$
\end{enumerate}

\section{函数的单调性和凸性}

\begin{enumerate}
    \item[1.]
    显然 $n = 1$ 时成立.
    
    假设 $n=k \ (\geqslant 1)$ 时结论成立,设 $\alpha_1 + \alpha_2 + \cdots + \alpha_{k+1} = 1$,则
    
    $$
    f(\frac{\alpha_1x_1 + \alpha_2x_2 + \cdots + \alpha_kx_k}{1-\alpha_{k+1}}) \leqslant \frac{\alpha_1f(x_1) + \alpha_2f(x_2) + \cdots + \alpha_kf(x_k)}{1-\alpha_{k+1}}
    $$
    
    则由凸函数定义知
    
    $$
    f\left( (1-\alpha_{k+1}) \cdot \frac{\sum_{i=1}^k \alpha_ix_i}{1-\alpha_{k+1}} + \alpha_{k+1}x_{k+1} \right) \leqslant (1-\alpha_{k+1}) \cdot \frac{\sum_{i=1}^k \alpha_if(x_i)}{1-\alpha_{k+1}} + \alpha_{k+1}f(x_{k+1})
    $$
    
    整理后即得 $n = k+1$ 时成立.
    
    故由数学归纳法知,命题成立.
\end{enumerate}

\section{Taylor 展开}

\begin{enumerate}
    \item[8.]
    对任意 $x \in (0,\ 2)$,由 Taylor 公式得
    
    $$
    \begin{cases}
        f(0) = f(x) - f'(x)x + \frac {f''(\xi_1)} 2 (0-x)^2 ,\ \quad \xi_1 \in (0,\ x) \\
        f(2) = f(x) + f'(x)(2-x) + \frac {f''(\xi_2)} 2 (2-x)^2 ,\ \quad \xi_2 \in (x,\ 2) \\
    \end{cases}
    $$
    
    两式相减得 $f'(x) = \frac {f(2)-f(0)} 2 + \frac {x^2 f''(\xi_1) - (x-2)^2 f''(\xi_2)} 4 \leqslant 1 + \frac{x^2 + (x-2)^2} 4 \leqslant 2$.
    
    同理可得 $f'(x) \geqslant -2$,即 $|f'(x)| \leqslant 2$.
\end{enumerate}

\section*{第 3 章综合习题}

\begin{enumerate}
    \item[5.]
    \textbf{引理}:对任意 $i,\ n \in \mathbf{N_+},\ 0 \leqslant i \leqslant 2^n$,有
    
    $$
    f\left( \frac{ix_1+(2^n-i)x_2}{2^n} \right) \leqslant \frac i {2^n} f(x_1) + \left( 1 - \frac i {2^n} \right) f(x_2)
    $$
    
    \textbf{证明} \ 用数学归纳法证明.
    
    \begin{enumerate}
        \item
        当 $n = 1$ 时,显然成立.
        \item
        当 $n = k (\geqslant 2)$ 时,假设成立,则对于 $0 < j < 2^{k+1}$:
        
        若 $j$ 为偶数,显然成立;若 $j$ 为奇数,不妨设 $j < 2^k$,则
        
        $$
        \begin{cases}
            f\left( \frac{(j-1)x_1+(2^{k+1}-j+1)x_2}{2^{k+1}} \right) \leqslant \frac {j-1} {2^{k+1}} f(x_1) + \left( 1 - \frac {j-1} {2^{k+1}} \right) f(x_2) \\
            f\left( \frac{(j+1)x_1+(2^{k+1}-j-1)x_2}{2^{k+1}} \right) \leqslant \frac {j+1} {2^{k+1}} f(x_1) + \left( 1 - \frac {j+1} {2^{k+1}} \right) f(x_2)
        \end{cases}
        $$
        
        在 $f\left( \frac {x_1+x_2} 2 \right) \leqslant \frac {f(x_1)+f(x_2)} 2$ 中,令 $\begin{cases} x_1 = \frac{(j-1)x_1+(2^{k+1}-j+1)x_2}{2^{k+1}} \\ x_2 = \frac{(j+1)x_1+(2^{k+1}-j-1)x_2}{2^{k+1}} \end{cases}$,即可得欲证.
    \end{enumerate}
    
    对任意实数 $\lambda \in (0,\ 1)$,令 $a_1 = 0,\ b_1 = 1$.
    
    当 $\lambda < \frac {a_n+b_n} 2$ 时,令 $a_{n+1} = a_n,\ b_{n+1} = \frac {a_n+b_n} 2$,否则令 $a_{n+1} = \frac {a_n+b_n} 2,\ b_{n+1} = b_n$.
    
    则由区间套定理知,$\displaystyle\lim_{n \to \infty} a_n = \lim_{n \to \infty} b_n = \lambda$.
    
    又由引理知,$f[ a_n x_1 + (1-a_n) x_2 ] \leqslant a_nf(x_1) + (1-a_n)f(x_2)$.
    
    则由 $f(x)$ 连续知 $f[ \lambda x_1 + (1-\lambda) x_2 ] \leqslant \lambda f(x_1) + (1-\lambda)f(x_2)$,即 $f(x)$ 为凸函数.
    \item[6.]
    设 $y = f'(x)$,则 $f''(x) = \frac{2f'(x)}{1-x} \ \Leftrightarrow \ \frac {\mathrm dy} y = \frac {2\mathrm dx} {1-x} \ (0 < x < 1)$.
    
    两边同时积分,得 $\left| \ln y \right| = -2 \ln(1-x) + C$,解得 $f'(x)(x-1)^{\pm 2} = e^C$.
    
    设 $g(x) = f'(x)(x-1)^2$,由 Rolle 定理知存在 $\eta \in (0,\ 1)$,满足 $f'(\eta) = 0$.
    
    则 $g(\eta) = g(1) = 0$,故由 Rolle 定理知存在 $\xi \in (\eta,\ 1)$,满足 $g'(\xi) = 0$,即 $f''(\xi) = \frac{2f'(\xi)}{1-\xi}$.
    \item[9.]
    设 $g(x) = \frac 1 {f(x)} - x$,则 $g(0) = g(1) = 1$.
    
    由 Rolle 定理知,存在 $\xi \in (0,\ 1)$,满足 $g'(\xi) = - \frac{f'(\xi)}{f^2(\xi)} - 1 = 0$,即 $f^2(\xi) + f'(\xi) = 0$.
    \item[19.]
    设 $g(x) = f(x) - \frac 1 2 x^3 + \left[ f(0) - \frac 1 2 \right] x^2$,则 $g(-1) = g(0) = g(1)$.
    
    则由 Rolle 定理知存在 $\xi_1 \in (-1,\ 0)$ 及 $\xi_2 \in (0,\ 1)$,满足 $g'(\xi_1) = g'(\xi_2) = g(0) = 0$.
    
    故知存在 $\eta_1 \in (\xi_1,\ 0)$ 及 $\eta_2 \in (0,\ \xi_2)$,满足 $g''(\eta_1) = g''(\eta_2) = 0$.
    
    故又知存在 $\zeta \in (\eta_1,\ \eta_2)$,满足 $g'''(\zeta) = 0$,即 $f'''(\zeta) = 3$.
    \item[20.]
    假设命题不成立,即存在 $x_0 > 0$,使得 $x > x_0$ 时均有 $f'(x) \geqslant f(ax) > 0$.
    
    则 $f(x)$ 在 $(x_0,\ +\infty)$ 上单调递增.
    
    由 Lagrange 中值定理知,存在 $\xi \in (x,\ ax)$,满足 $\frac{f(ax)-f(x)}{(a-1)x} = f'(\xi) \geqslant f(a\xi) > f(ax)$.
    
    故有 $(1-ax+x)f(ax) \geqslant f(x)$,当 $x > \frac 1 {a-1}$ 时,得 $f(x) < 0$,矛盾,故命题成立.
    \item[21.]
    \textbf{引理}:设 $x,\ y > 0$,$p,\ q$ 是大于 $1$ 的正数,且 $\frac 1 p + \frac 1 q = 1$,则
    
    $$
    xy \leqslant \frac 1 p x^p + \frac 1 q y^q
    $$
    
    \textbf{证明} \ 由 $\ln x$ 凹性知 $\ln xy = \frac 1 p \ln x^p + \frac 1 q \ln y^q \leqslant \ln\left( \frac 1 p x^p + \frac 1 q y^q \right)$,即 $xy \leqslant \frac 1 p x^p + \frac 1 q y^q$.
    
    令 $x = \frac{a_i}{\left( \sum_{i=1}^n a_i^p \right)^{\frac 1 p}},\ y = \frac{b_i}{\left( \sum_{i=1}^n b_i^q \right)^{\frac 1 q}}$,则有 $\frac{a_ib_i}{\left( \sum_{i=1}^n a_i^p \right)^{\frac 1 p} \left( \sum_{i=1}^n b_i^q \right)^{\frac 1 q}} \leqslant \frac{a_i^p}{p \left( \sum_{i=1}^n a_i^p \right)} + \frac{b_i^q}{q \left( \sum_{i=1}^n b_i^q \right)}$.
    
    则 $\displaystyle \frac{\sum_{i=1}^n a_ib_i}{\left( \sum_{i=1}^n a_i^p \right)^{\frac 1 p} \left( \sum_{i=1}^n b_i^q \right)^{\frac 1 q}} \leqslant \frac 1 p + \frac 1 q = 1$,即 $\displaystyle \sum_{i=1}^n a_ib_i \leqslant \left( \sum_{i=1}^n a_i^p \right)^{\frac 1 p} \left( \sum_{i=1}^n b_i^q \right)^{\frac 1 q}$.
\end{enumerate}

\newpage
\chapter{不定积分}

\section{不定积分及其基本计算方法}

\begin{enumerate}
    \item[3.]
    \begin{enumerate}
        \item[(12)]
        $$
        \begin{aligned}
            \int \frac 1 {x^8(1+x^2)} \mathrm dx &= \int \left( \frac 1 {x^8} - \frac 1 {x^6} + \frac 1 {x^4} - \frac 1 {x^2} + \frac 1 {1+x^2} \right) \mathrm{d}x \\ &= - \frac 1 {7x^7} + \frac 1 {5x^5} - \frac 1 {3x^3} + \frac 1 x + \arctan x + C
        \end{aligned}
        $$
    \end{enumerate}
    \item[7.]
    \begin{enumerate}
        \item[(26)]
        $$
        \begin{aligned}
            \int \frac{xe^x}{(1+x)^2} \mathrm dx &= - \frac{xe^x}{1+x} + \int e^x \mathrm dx \\
            &= \frac{e^x}{1+x} + C
        \end{aligned}
        $$
    \end{enumerate}
\end{enumerate}

\section{有理函数的不定积分}

\begin{enumerate}
    \item[1.]
    \begin{enumerate}
        \item[(8)]
        $$
        \begin{aligned}
            \int \frac{x^{15}}{(x^8 + 1)^2} \mathrm dx &= - \frac{x^8}{8(x^8 + 1)} + \int \frac{x^7}{x^8 + 1} \mathrm dx \\
            &= \frac 1 8 \ln(x^8 + 1) - \frac{x^8}{8(x^8 + 1)} + C
        \end{aligned}
        $$
    \end{enumerate}
    \item[2.]
    \begin{enumerate}
        \item[(10)]
        $$
        \begin{aligned}
            \int \frac{\cos x}{a\sin x + b\cos x} \mathrm dx &= \int \frac{\cos x}{\sqrt{a^2 + b^2} \cos(x + \varphi)} \mathrm dx \quad (\tan \varphi = - \frac a b) \\
            &= \frac 1 {a^2 + b^2} \int \frac{\sqrt{a^2 + b^2} \cos (t - \varphi)}{\cos t} \mathrm dt \quad (t = x + \varphi) \\
            &= \frac 1 {a^2 + b^2} \int \frac{b\cos t - a\sin t}{\cos t} \mathrm dt \\
            &= \frac{b(x + \varphi) + a \ln|\cos(x + \varphi)|}{a^2 + b^2} + C
        \end{aligned}
        $$
    \end{enumerate}
\end{enumerate}

\newpage

\chapter{单变量函数的积分学}

\section{积分}

\begin{enumerate}
    \item[2.]
    对任意 $x_{i-1} < x_i$,总能找到 $\xi_i$ 使得 $f(\xi_i) \equiv 0$ 或 $f(\xi_i) \equiv 1$,则极限 $S_n(T) - I$ 不存在.
    \item[3.]
    $f(x) = 2D(x) - 1$,其中 $D(x)$ 为 Dirchlet 函数.
    \item[9.]
    \begin{enumerate}
        \item[(1)]
        设 $m \leqslant f(x) \leqslant M$,由 $g(x) \geqslant 0$ 知 $\int_a^b m g(x) \mathrm dx \leqslant \int_a^b f(x)g(x) \mathrm dx \leqslant \int_a^b M g(x) \mathrm dx$.
        
        即存在 $\lambda \in [m, M]$ 使得 $\int_a^b f(x)g(x) \mathrm dx = \lambda \int_a^b g(x) \mathrm dx$.
        
        又 $f(x)$ 连续,则存在 $\xi \in [a, b]$ 使得 $f(\xi) = \lambda$.
        \item[(2)]
        令 $f(x) = (x+1)^2,\ g(x) = x$,则 $\int_a^b f(x)g(x) \mathrm dx = \frac 4 3$,而 $\int_{-1}^1 g(x) \mathrm dx = 0$.
        
        故不存在满足条件的 $\xi$.
    \end{enumerate}
    \item[13.]
    $$
    \begin{aligned}
        &F(x) = \int_0^x f(t) \mathrm dt = x \int_0^x f(t) \mathrm dt \\ 
        \Rightarrow \ &F'(x) = x f(x) + \int_0^x f(t) \mathrm dt = xf(x) + \varphi(x)
    \end{aligned}
    $$
    \item[18.]
    \begin{enumerate}
        \item[(3)]
        $$
        \lim_{n \to \infty} \left[ \frac 1 {\sqrt{n^2}} + \frac 1 {\sqrt{n^2 - 1^2}} + \frac 1 {\sqrt{n^2 - 2^2}} + \cdots + \frac 1 {\sqrt{n^2 - (n-1)^2}} \right] = \int_0^1 \frac 1 {\sqrt{1 - x^2}} = \frac \pi 2
        $$
        \item[(4)]
        $$
        \lim_{n \to \infty} \frac{1^p + 2^p + \cdots + n^p}{n^{p+1}} = \int_0^1 x^p = \frac 1 {1+p}
        $$
    \end{enumerate}
    \item[19.]
    \begin{enumerate}
        \item[(2)]
        易知 $0 < \frac{x^n}{1+x} < x^n$.
        
        又 $\displaystyle\lim_{n \to \infty} \int_0^1 x^n = \lim_{n \to \infty} \left. \frac{x^{n+1}}{n+1} \right|_0^1 = \lim_{n \to \infty} \frac 1 {n+1} = 0$.
        
        由夹逼原理知 $\displaystyle\lim_{n \to \infty} \int_0^1 \frac{x^n}{1+x} = 0$.
    \end{enumerate}
    \item[22.]
    \begin{enumerate}
        \item[(14)]
        $$
        \begin{aligned}
            \int_0^{\pi} \frac{\sec^2 x}{2 + \tan^2 x} \mathrm dx &= \int_0^{\frac \pi 2} \frac{\sec^2 x}{2 + \tan^2 x} \mathrm dx + \int_{\frac \pi 2}^{\pi} \frac{\sec^2 x}{2 + \tan^2 x} \mathrm dx \\
            &= \int_{0}^{+\infty} \frac 1 {t^2 + 2} \mathrm dt + \int_{-\infty}^0 \frac 1 {t^2 + 2} \mathrm dt \quad (t = \tan x) \\
            &= \lim_{a \to +\infty} \left. \frac 1 {\sqrt 2} \arctan \left( \frac t {\sqrt 2} \right) \right|_0^a + \lim_{a \to -\infty} \left. \frac 1 {\sqrt 2} \arctan \left( \frac t {\sqrt 2} \right) \right|_a^0 \\
            &= \frac \pi {\sqrt 2}
        \end{aligned}
        $$
    \end{enumerate}
    \item[23.]
    \textbf{证}
    
    $$
    \begin{aligned}
        \int_0^{\pi} x f(\sin x) \mathrm dx &= \int_0^{\frac \pi 2} x f(\sin x) \mathrm dx + \int_{\frac \pi 2}^{\pi} x f(\sin x) \mathrm dx \\
        &= \int_0^{\frac \pi 2} x f(\sin x) \mathrm dx + \int_0^{\frac \pi 2} (\pi - x) f( \sin(\pi - x) ) \mathrm dx \\
        &= \pi \int_0^{\frac \pi 2} f(\sin x) \mathrm dx
    \end{aligned}
    $$
    
    则
    
    $$
    \begin{aligned}
        \int_0^{\pi} \frac{x \sin x}{1 + \cos^2 x} \mathrm dx &= \pi \int_0^{\frac \pi 2} \frac{\sin x}{1 + 1 + \cos^2 x} \\
        &= -\arctan(\cos x) \bigg|_0^{\frac \pi 2} \\
        &= \frac \pi 4
    \end{aligned}
    $$
    \item[24.]
    易知在 $(0, 1)$ 上有 $\frac {x^2} 2 < \sin x^2 < x^2$,则原式易证.
    \item[27.]
    设 $g(\alpha) = \int_0^{\alpha} f(x) \mathrm dx - \alpha \int_0^1 f(x) \mathrm dx \ (0 < \alpha < 1)$,则
    $$
    \begin{cases}
        g(0) = g(1) = 0 \\
        g'(\alpha) = f(\alpha) \ \Rightarrow \ g''(\alpha) = f'(\alpha) < 0
    \end{cases}
    \ \Rightarrow \ g(\alpha) \geqslant 0
    $$
\end{enumerate}

\section{函数的可积性}

\begin{enumerate}
    \item[3.]
    由题设知存在 $m \leqslant f(x) \leqslant M \ (a \leqslant x \leqslant b)$,则 $0 \leqslant |f(x)| \leqslant \max\{ M, -m \}$,则 $|f(x)|$ 在 $[a, b]$ 上可积.
    
    不妨把 $f(x)$ 按取值正负分为两部分,分别记作 $f_1(x)$ 和 $f_2(x)$;具体地,令
    
    $$
    f_1(x) = \max\{ f(x), 0 \},\ f_2(x) = \min\{ f(x), 0 \}
    $$
    
    则
    
    $$
    \begin{aligned}
        \left| \int_a^b f(x) \mathrm dx \right| &= \left| \int_a^b f_1(x) \mathrm dx + \int_a^b f_2(x) \mathrm dx \right| \\ &\leqslant \left| \int_a^b f_1(x) \mathrm dx \right| + \left| \int_a^b f_2(x) \mathrm dx \right| \\
        &= \int_a^b |f(x)| \mathrm dx
    \end{aligned}
    $$
\end{enumerate}

\section{积分的应用}

\begin{enumerate}
    \item[4.]
    $$
    \begin{aligned}
        V &\sim \int_0^h \pi \left( R^2 - (R-h+x)^2 \right)  \mathrm dx \\
        &= -\pi \int_0^h \left( x^2 + 2 (R-h) x + h (h-2R) \right) \mathrm dx \\
        &= \left. -\pi \left( \frac {x^3} 3 + (R-h) x^2 +  h(h-2R)x \right) \right|_0^h \\
        &= \pi h^2 (R - \frac h 3)
    \end{aligned}
    $$
\end{enumerate}

\section{广义积分}

\begin{enumerate}
    \item[1.]
    \begin{enumerate}
        \item[(12)]
        $$
        \begin{aligned}
            \int_0^1 (\ln x)^n \mathrm dx &= (-1)^{n+1} \int_0^{+\infty} t^n e^{-t} \mathrm dt \ (x = e^{-t}) \\
            &= (-1)^{n+1}\Gamma(n+1)
        \end{aligned}
        $$
    \end{enumerate}
    \item[2.]
    \begin{enumerate}
        \item[(1)]
        由 $\frac x {1+x^2}$ 奇性知 $\displaystyle \int_{-b}^b \frac x {1+x^2} \mathrm dx \equiv 0$.
        
        然而
        
        $$
        \begin{aligned}
            \int_{-\infty}^{+\infty} \frac x {1+x^2} \mathrm dx &= \int_{-\infty}^0 \frac x {1+x^2} \mathrm dx + \int_0^{+\infty} \frac x {1+x^2} \mathrm dx \\
            &= \left. \frac 1 2 \ln (1+x^2) \right|_{-\infty}^0 + \left. \frac 1 2 \ln (1+x^2) \right|_0^{+\infty}
        \end{aligned}
        $$
        
        则原积分发散.
    \end{enumerate}
    \item[3.]
    \begin{enumerate}
        \item[(2)]
        \begin{enumerate}
            \item[(a)]
            当 $\alpha = 1$ 时,$\displaystyle \int_0^{+\infty} \frac {\mathrm dx} x = \ln x \big|_0^{+\infty}$ 发散.
            \item[(b)]
            当 $\alpha \neq 1$ 时,易知 $\displaystyle \int_0^{+\infty} \frac 1 {x^\alpha} = \left. \frac{x^{1-\alpha}}{1-\alpha} \right|_0^{+\infty}$,则
            
            $$
            \begin{aligned}
                \lim_{x \to +\infty} \frac{x^{1-\alpha}}{1-\alpha} = +\infty \ (\alpha < 1) \\
                \lim_{x \to 0_+} \frac{x^{1-\alpha}}{1-\alpha} = -\infty \ (\alpha > 1)
            \end{aligned}
            $$
        \end{enumerate}
        综上,无论 $\alpha$ 取何实数值,$\displaystyle \int_0^{+\infty} \frac{\mathrm dx}{x^{\alpha}} $ 必发散.
    \end{enumerate}
\end{enumerate}

\section*{第 5 章综合习题}

\begin{enumerate}
    \item[1.]
    \begin{enumerate}
        \item[(1)]
        设 $t = 2 \pi - x$,则
        
        $$
        \begin{aligned}
            \int_0^{2 \pi} \sin mx \cdot \cos nx \mathrm dx &= \int_0^\pi \sin mx \cdot \cos nx \mathrm dx + \int_0^\pi \sin mt \cdot \cos nt \mathrm dx \\
            &= \int_0^\pi \sin mx \cdot \cos nx - \int_0^\pi \sin mx \cdot \cos nx \\
            &= 0
        \end{aligned}
        $$
        \item[(2)]
        $$
        \begin{aligned}
            \int_0^{2\pi} \sin mx \cdot \sin nx &= \frac 1 2 \int_0^{2\pi} \bigg( \cos( mx - nx ) - \cos( mx + nx ) \bigg) \\
            &= \frac 1 2 \int_0^{2\pi} \cos\big( (m - n)x \big) - \frac 1 2 \int_0^{2\pi} \cos\big( (m + n)x \big)
        \end{aligned}
        $$
        
        则结论显然.
    \end{enumerate}
    \item[2.]
    \begin{enumerate}
        \item[(2)]
        由分部积分法知
        
        $$
        \begin{aligned}
            B(m, n) &= \int_0^1 x^m (1-x)^n \mathrm dx \\
            &= \left. \frac{x^{m+1}(1-x)^n}{m+1} \right|_0^1 + \frac n {m+1} \int_0^1 x^{m+1} (1-x)^{n-1} \mathrm dx \\
            &= \frac n {m+1} B(m+1, n-1) \\
            &= \frac n {m+1} B(m, n-1) - \frac n {m+1} B(m, n)
        \end{aligned}
        $$
        
        即 $B(m, n) = \frac n {m+n+1} B(m, n-1)$.
        
        易知 $m,\ n \geqslant 0$ 时 $B(m, n)$ 均有意义,则
        
        $$
        \begin{aligned}
            B(m, n) &= \frac{n!(m+1)!}{(m+n+1)!} B(m, 0) \\
            &= \frac{n!(m+1)!}{(m+n+1)!} B(0, m) \\
            &= \frac{n!(m+1)!}{(m+n+1)!} \cdot \frac{m!}{(m+1)!} \\
            &= \frac{n!m!}{(m+n+1)!}
        \end{aligned}
        $$
    \end{enumerate}
    \item[3.]
    \begin{enumerate}
        \item[(c)]
        $$
        \begin{aligned}
            f(x) &= \int_x^{x+2\pi} (1 + e^{\sin t} - e^{-\sin t}) \mathrm dt + \frac 1 {1+x} \int_0^1 f(t) \mathrm dt \\
            &= \int_x^{x+\pi} \left( 1 + e^{\sin t} - e^{-\sin t} + 1 + e^{\sin(t+\pi)} - e^{-\sin(t+\pi)} \right) \mathrm dt + \frac 1 {1+x} \int_0^1 f(t) \mathrm dt \\
            &= 2\pi + \frac 1 {1+x} \int_0^1 f(t) \mathrm dt \\
        \end{aligned}
        $$
        
        则 $\displaystyle \int_0^1 f(x) \mathrm dx = (1 + x)(f(x) - 2\pi)$.
    \end{enumerate}
    \item[4.]
    $$
    \int_0^{\frac \pi 4} \tan^n x \mathrm dx = \int_0^1 \frac{t^n}{1+t^2} \mathrm dt \ (x = \arctan t)
    $$
    
    而
    
    $$
    \frac 1 {2n+2} = \int_0^1 \frac {t^n} 2 \mathrm dt < \int_0^1 \frac{t^n}{1+t^2} \mathrm dt < \int_0^1 \frac{t^n}{2t} \mathrm dt = \frac 1 {2n}
    $$
    \item[10.]
    设 $\displaystyle \varphi(t) = \int f(t) \mathrm dt$,则 $\displaystyle F(x) = \left. \frac {\varphi(xt)} x \right|_{t=0}^1 = \frac {\varphi(x) - \varphi(0)} x$.
    
    故 $F(x)$ 可导,$F'(x) = \frac{xf(x) - \varphi(x) + \varphi(0)}{x^2}$.
    \item[11.]
    \begin{enumerate}
        \item[(2)]
        $$
        \begin{aligned}
            f'_+(0) &= \lim_{x \to 0_+} \frac 1 x \int_{\frac 1 x}^{+\infty} \frac{\cos u}{u^2} \mathrm du \ (u = \frac 1 t) \\
            &= \lim_{x \to 0_+} - \frac 2 x \int_{\frac 1 x}^{+\infty} \frac{\sin u}{u^3} \mathrm du \\
            &= 2 \lim_{x \to 0_+} \frac{\int_0^x t \cos \frac 1 t}{x} \\
            &= 0
        \end{aligned}
        $$
    \end{enumerate}
    \item[12.]
    设 $\displaystyle \varphi(x) = \int f(x) \mathrm dx$,则
    
    $$
    \begin{aligned}
        LHS &= \lim_{h \to 0} \frac 1 h \left( \varphi(b+h) - \varphi(b) - \varphi(a+h) + \varphi(a) \right) \\
        &= \lim_{h \to 0} \frac {\varphi(b+h) - \varphi(b)} h - \lim_{h \to 0} \frac {\varphi(a+h) - \varphi(a)} h \\
        &= f(b) - f(a)
    \end{aligned}
    $$
    \item[13.]
    $$
    \lim_{\lambda \to \infty} \int_a^b f(x) \sin(\lambda x) \mathrm dx = \lim_{\lambda \to \infty} \left( \frac {f(x) \cos(\lambda x)} \lambda - \frac 1 \lambda \int_a^b f'(x) \cos(\lambda x) \mathrm dx \right) = 0
    $$
    \item[14.]
    $$
    \begin{aligned}
        \lim_{x \to +\infty} \frac 1 x \int_0^{x} |\sin t| \mathrm dt &= \lim_{x \to +\infty} \frac 1 x \int_0^{k\pi} \sin t \mathrm dt + \lim_{x \to +\infty} \frac 1 x \int_{k\pi}^x |\sin t| \mathrm dt \ \left( k = \left\lfloor \frac x \pi \right\rfloor \right) \\
        &= \lim_{x \to \infty} \frac {2k} x + 0 \\
        &= \frac 2 \pi
    \end{aligned}
    $$
    \item[16.]
    设 $f(x_0) = M$,对任意 $\varepsilon > 0$,存在 $\delta$ 使得 $f(x) > M - \varepsilon \ \left(x \in U(x_0, \delta) \right)$.
    
    设 $g_\delta(x) = \begin{cases} M - \varepsilon & x \in U(x_0, \delta) \\ 0 & x \not\in U(x, \delta) \end{cases}$,则 $0 < g(x) < f(x) \leqslant M$.
    
    则对任意 $\varepsilon > 0$ 均有
    
    $$
    \lim_{n \to \infty} \left( \int_a^b g_\delta^n(x) \mathrm dx \right)^{\frac 1 n} = \lim_{n \to \infty} (M - \varepsilon) \delta^{\frac 1 n} = M - \varepsilon \leqslant \lim_{n \to \infty} \int_a^b f(x) \mathrm dx \leqslant M
    $$
    
    由夹逼原理知 $\displaystyle \lim_{n \to \infty} \int_a^b f(x) \mathrm dx = M$.
    \item[18.]
    考虑 $h(x) = f(x) + \lambda g(x)$,显然有
    
    $$
    \int_a^b h^2(x) \mathrm dx = \int_a^b f^2(x) \mathrm dx + 2\lambda \int_a^b f(x)g(x) \mathrm dx + \lambda^2 \int_a^b g^2(x) \mathrm dx \geqslant 0
    $$
    
    上式可看作关于 $\lambda$ 的二次不等式,则有
    
    $$
    \Delta = 4 \left( \int_a^b f(x)g(x) \mathrm dx \right)^2 - 4 \int_a^b f^2(x) \mathrm dx \int_a^b g^2(x) \mathrm dx \leqslant 0
    $$
    
    整理后即得欲证.
    \item[19.]
    
    设 $\min\{ |f(x)| \} = |f(x_0)|$,则
    
    $$
    |f(a)| = \left| f(x_0) + \int_{x_0}^a f'(x) \mathrm dx \right| \leqslant |f(x_0)| + \left| \int_{x_0}^a f'(x) \mathrm dx \right| \leqslant \int_0^1 |f(x)| \mathrm dx + \int_0^1 |f'(x)| \mathrm dx
    $$
    \item[21.]
    $$
    \begin{aligned}
        \left| \int_0^1 f(x) \mathrm dx - \frac 1 n \sum_{k=1}^n f\left( \frac k n \right) \right| &= \left| \sum_{k=1}^n \left( \int_{\frac {k-1} n}^{\frac k n} f(x) \mathrm dx - \frac 1 n f\left( \frac k n \right) \right) \right| \\
        &\leqslant \sum_{k=1}^n \left| \int_{\frac {k-1} n}^{\frac k n} f(x) \mathrm dx - \frac 1 n f\left( \frac k n \right) \right| \\
        &\leqslant \sum_{k=1}^n \frac M 2 (\frac k n - \frac {k-1} n)^2 \\
        &= \frac M {2n}
    \end{aligned}
    $$
    \item[22.]
    见\href{https://www.zhihu.com/question/611041671/answer/3109739799}{此解}.
\end{enumerate}

\newpage
\chapter{常微分方程初步}

\section{一阶微分方程}

\begin{enumerate}
    \item[4.]
    \begin{enumerate}
        \item[(3)]
        整理原式得
        
        $$
        \frac{\mathrm dx}{\mathrm dy} = \frac x y + y^2
        $$
        
        若 $y \neq 0$,作代换 $u = \frac x y$,等式化为
        
        $$
        u + y \frac{\mathrm du}{\mathrm dy} = u + y^2 \ \left( u = \frac x y,\ y \neq 0 \right)
        $$
        
        则可进一步解得 $x = \frac 1 2 y^3 + Cy$.
        
        验证知 $y = 0$ 亦为方程的解.
    \end{enumerate}
    \item[6.]
    \begin{enumerate}
        \item[(1)]
        不妨设 $u = \frac y {x^2}$,则原式化为
        
        $$
        2xu + x^2 \frac{\mathrm du}{\mathrm dx} = x \left( \sqrt{1+u} - 1 \right)
        $$
        
        进一步整理得
        
        $$
        - \frac{\mathrm du}{2u - \sqrt{1+u} + 1} = \frac {\mathrm dx} x
        $$
        
        再设 $v = \sqrt{1+u}$,化为
        
        $$
        - \frac{2v \mathrm dv}{(v-1)(2v+1)} = \frac {\mathrm dx} x
        $$
        
        两边积分,得
        
        $$
        -\frac 1 3 \left( 2 \ln(v-1) + \ln(2v+1) \right) = \ln|x| + C
        $$
        
        即
        
        $$
        (v-1)^2(2v+1) = \frac{C'}{x^3}.
        $$
        
        则 $y$ 可解. 或设 $u = \sqrt{x^2 + y}$,详见\href{https://math.stackexchange.com/questions/3374263/how-to-solve-the-following-ordinary-differential-equation}{此解}.
    \end{enumerate}
    \item[7.]
    $$
    \begin{aligned}
        \frac{\mathrm dy}{\mathrm dx} + P(x) y &= Q(x) y^n \ (n \not\in \{ 0, 1 \} ) \\
        \frac{\mathrm du}{\mathrm dx} + (1-n) P(x) u &= (1-n) Q(x) \ (u = y^{1-n}) \\
    \end{aligned}
    $$
    
    设 $u = C(x) e^{(n-1) \int P(x) \mathrm dx}$,则
    
    $$
    \begin{aligned}
        \frac{\mathrm d C(x)}{\mathrm dx} &= (1-n)Q(x)e^{(1-n) \int P(x) \mathrm dx} \\
        C(x) &= \int (1-n)Q(x)e^{(1-n) \int P(x) \mathrm dx} \mathrm dx + C
    \end{aligned}
    $$
    
    则 $u = e^{(n-1) \int P(x) \mathrm dx} \left( \int (1-n)Q(x)e^{(1-n) \int P(x) \mathrm dx} \mathrm dx + C \right)$,进一步可得 $y$.
    \item[13.]
    \begin{enumerate}
        \item[(2)]
        设 $p = y'$,则原式化为
        
        $$
        y^3 p \frac{\mathrm dp}{\mathrm dy} = -1
        $$
        
        也即 $\displaystyle p \mathrm dp = - \frac{\mathrm dy}{y^3}$,易解得 $y = \sqrt{Cx^2 - \frac 1 C}$.
    \end{enumerate}
\end{enumerate}

\section{二阶线性微分方程}

\begin{enumerate}
    \item[1.]
    \begin{enumerate}
        \item[(3)]
        由题设知 $y_1(x) = x,\ p(x) = \frac{2x}{x^2 - 1}$,则
        
        $$
        y_2(x) = x \int \frac 1 {x^2} e^{\int_{x_0}^x \frac{2t}{t^2 - 1} \mathrm dt} \mathrm dx = \frac{x^2 + 1}{x_0^2 - 1}
        $$
        
        故通解 $y_0(x) = c_1x + c_2 (x^2 + 1)$.
    \end{enumerate}
    \item[2.]
    \begin{enumerate}
        \item[(2)]
        $y = x + 1$ 为一特解.
    \end{enumerate}
    \item[3.]
    其对应的齐次方程为
    
    $$
    y'' + \frac{2x}{1 + x^2} y' = 0
    $$
    
    观察知 $y_1 = \arctan x$ 为一特解,则可进一步求出 $y_2$.
    \item[5.]
    \begin{enumerate}
        \item[(2)]
        易得对应齐次方程的通解为 $y = c_1 e^{3x} + c_2 x e^{3x}$.
        
        则
        
        $$
        y_0(x) = \int_{x_0}^x \frac{e^{3t} \cdot xe^{3x} - te^{3t} \cdot e^{3x}}{W(t)} f(t) \mathrm dt
        $$
        
        进而可得原方程通解.
    \end{enumerate}
    \item[9.]
    \begin{enumerate}
        \item[(4)]
        设 $x = e^{\lambda t}$,则得到特征方程 $\lambda^4 + 2\lambda^2 + 1 = 0$.
        
        解得 $\lambda = \pm i$,故通解为 $x = c_1 \cos x + c_2 \sin x$.
    \end{enumerate}
    
\end{enumerate}

\newpage

\chapter{无穷级数}

\section{数项级数}

\begin{enumerate}
    \item[2.]
    \begin{enumerate}
        \item[(13)]
        对任意 $0 < k < \frac 1 4$,存在 $N \in \mathbf N_+$ 使得当 $n \geqslant N$ 时有 $\ln n < n^k$.
        
        故
        
        $$
        \sum_{n=N}^{\infty} \frac{\ln n}{\sqrt[4]{n^5}} < \sum_{n=N}^{\infty} \frac 1 {n^{\frac 5 4 - k}}
        $$
        
        由比较审敛法知 $\displaystyle \sum_{n=1}^{\infty} \frac{\ln n}{\sqrt[4]{n^5}}$ 收敛.
        \item[(14)]
        易知
        
        $$
        \int_3^{+\infty} \frac 1 {x \ln x (\ln \ln x)^k} = 
        \begin{cases}
            \ln \ln \ln x \bigg|_3^{+\infty} = +\infty & k = 1 \\
            \left. \frac{(\ln \ln x)^{1-k}}{1-k} \right|_3^{+\infty} = +\infty & k \neq 1
        \end{cases}
        $$
        
        故由 Cauchy 积分判别法知级数 $\displaystyle \sum_{n=2}^{\infty} \frac 1 {n \ln n (\ln \ln n)^k}$ 发散.
        \item[(15)]
        $n \to \infty$ 时
        
        $$
        \left( \cos \frac 1 n \right)^{n^3} \sim \left( 1 - \frac 1 {2n^2} \right)^{n^3} = \frac 1 {e^{\frac n 2}}
        $$
        
        故原级数收敛.
        \item[(16)]
        当 $n \to \infty$ 时
        
        $$
        \left( \frac{an}{n+1} \right)^n = \frac{a^n}{\left( 1 + \frac 1 n \right)^n} \sim \frac {a^n} e
        $$
        
        故 $a \geqslant 1$ 时原级数发散,$a < 1$ 时原级数收敛.
    \end{enumerate}
    \item[4.]
    \begin{enumerate}
        \item[(3)]
        设 $\displaystyle A_n = \sum_{k=1}^n k (a_k - a_{k+1})$,则
        
        $$
        \lim_{n \to \infty} A_n = \lim_{n \to \infty} \left( \sum_{k=1}^n a_k - ka_{k+1} \right)
        $$
        
        又由 $\displaystyle \lim_{n \to \infty} n a_n = a$ 知 $\displaystyle \lim_{n \to \infty} a_n = 0$.
        
        则 $\lim_{n \to \infty} \sum_{k=1}^n a_k = \lim_{n \to \infty} A_n + a - 0$,原级数收敛.
    \end{enumerate}
    \item[6.]
    设 $\displaystyle c_n = a_n - a_1 - \sum_{k=1}^{n-1} b_k$,则 $\{ c_n \}$ 单调有界,故 $\{ c_n \}$ 收敛,进而 $\{ a_n \}$ 收敛.
    \item[7.]
    前两项易证.

    对最后一项,取 $b_n = \frac 1 n$ 即证.
    \item[12.]
    \begin{enumerate}
        \item[(8)]
        由 $\ln\left( 1 + \frac 1 n \right) = \frac 1 n + o \left( \frac 1 {n^2} \right)$ 知 $\displaystyle \lim_{n \to \infty} \left( \frac 1 n - \ln \left( 1 + \frac 1 n \right) \right) = 0$.

        故原级数收敛.
    \end{enumerate}
    \item[15.]
    \begin{enumerate}
        \item[(1)]
        易知 $\displaystyle \sum_{k=1}^n \sin kx = \frac{\cos \frac x 2 - \cos \left( nx + \frac x 2 \right)}{2 \sin \frac x 2}$ 有界,$\{ \frac 1 n \}$ 单调递减趋于 $0$,则由 Dirichlet 判别法知级数 $\displaystyle \sum_{n=1}^{\infty} \frac {\sin nx} n$ 收敛.
        \item[(2)]
        同理,取 $a_n = \cos \frac {n \pi} 4,\ b_n = \frac 1 {\ln n}$,由 Dirichlet 判别法易证.
        \item[(3)]
        由 Dirichlet 判别法易知级数 $\displaystyle \sum_{n=1}^{\infty} \frac{|\sin n|}{\sqrt n}$ 收敛.
        进而由 Abel 判别法知级数 $\displaystyle \sum_{n=1}^{\infty} \frac{|\sin n|}{\sqrt n} \left( 1 + \frac 1 n \right)^n$ 收敛.
        \item[(4)]
        由 $\displaystyle \lim_{n \to \infty} \frac{n-1}{n+1} \cdot \frac 1 {\sqrt[100]{n}} = 0$ 知原级数收敛.
    \end{enumerate}
\end{enumerate}

\section{函数项级数}

\begin{enumerate}
    \item[2.]
    \begin{enumerate}
        \item[(6)]
        由 Stirling 公式知 $n \to \infty$ 时有

        $$
        n! \left( \frac x n \right)^n \sim \sqrt{2 \pi n} \left( \frac n e \right)^n \left( \frac x n \right)^n = \sqrt{2 \pi n} \left( \frac x e \right)^n
        $$

        故级数收敛域为 $(-e, e)$.
    \end{enumerate}
    \item[8.]
    $$
    \lim_{x \to +\infty} \left( \sum_{n=1}^{\infty} \frac{x^n \cos \frac {n\pi} x}{(1+2x)^n} \right) = \sum_{n=1}^{\infty} \left( \lim_{x \to +\infty} \frac{x^n \cos \frac {n\pi} x}{(1+2x)^n} \right) \lim_{x \to +\infty} = \sum_{n=1}^{\infty} \frac 1 {2^n} = 1
    $$
    \item[10.]
    若有 $\displaystyle \lim_{n \to \infty} f_n(x) = f(x)$,则

    $$
    f'(x) = f^2(x)
    $$

    结合 $f(0) = 1$ 解得 $f(x) = \frac 1 {1 - x}$,则考虑证 $\displaystyle \lim_{n \to \infty} f_n(x) = \frac 1 {1 - x}$.

    由 $f_{n+1}'(x) = f_n(x) f_{n+1}(x)$ 及 $f_n(0) = 1$ 解得 $\displaystyle f_{n+1}(x) = \exp\left( \int_0^x f_n(t) \mathrm dt \right)$.

    \begin{enumerate}
        \item[a)]
        当 $n = 1$ 时,$\displaystyle f_2(x) = \exp\left( \int_0^x f_1(t) \mathrm dt \right) = e^x$,则 $f_1(x) \leqslant f_2(x) \leqslant \frac 1 {1 - x}$.
        \item[b)]
        若当 $n = k \ (\geqslant 2)$ 时,$f_{k-1}(x) \leqslant f_k(x) \leqslant \frac 1 {1 - x}$ 成立,则

        $$
        \begin{cases}
            f_{k+1}(x) = \exp\left( \int_0^x f_k(t) \mathrm dt \right) \leqslant \exp\left( \int_0^x \frac{\mathrm dt}{1 - t} \right) = \frac 1 {1 - x} \\
            \frac{f_{k+1}(x)}{f_k(x)} = \exp\left( \int_0^x \left( f_k(t) - f_{k-1}(t) \right) \mathrm dt \right) \geqslant e^0 = 1
        \end{cases}
        $$

        即 $f_k(x) \leqslant f_{k+1}(x) \leqslant \frac 1 {1 - x}$,故 $f_n(x)$ 单调有界,收敛至 $\frac 1 {x - 1}$.
    \end{enumerate}

    \item[11.]
    \begin{enumerate}
        \item[(1)]
        由题设知 $u_n(x)$ 在 $[a, b]$ 上有极值,且极值单调递减趋于 $0$,故一致连续.
        \item[(2)]
        充分性见定理 7.34.

        对于必要性,令 (1) 中的 $u_n(x) = S(x) - S_n(x)$ 即得.
    \end{enumerate}
\end{enumerate}

\section{幂级数与 Taylor 展式}

\begin{enumerate}
    \item[1.]
    \begin{enumerate}
        \item[(8)]
        $$
        \lim_{n \to \infty} \sqrt[n]{\frac{x^{n^2}}{2^n}} = \frac {x^n} 2
        $$

        则当 $x < 1$ 时级数收敛,$x > 1$ 时级数发散.

        $x = 1$ 时,$\frac {x^n} 2 = \frac 1 2 < 1$,则级数收敛.

        故 $R = 1$.
    \end{enumerate}
    \item[3.]
    \begin{enumerate}
        \item[(5)]
        设 $\displaystyle f(x) = \sum_{n=1}^{\infty} \frac{x^{2n-1}}{(2n-1)!!}$,则 $\displaystyle f'(x) = 1 + \sum_{n=1}^{\infty} \frac{x^{2n}}{(2n-1)!!} = 1 + xf(x)$.

        解得 $\displaystyle f(x) = e^{\frac {x^2} 2} \left( \int e^{-\frac {x^2} 2} + C \right)$.
    \end{enumerate}
    \item[4.]
    \begin{enumerate}
        \item[(1)]
        $$
        \begin{aligned}
            \sum_{n=2}^{\infty} \frac 1 {(n^2-1) 2^n} &= \frac 1 4 \sum_{n=2}^{\infty} \frac 1 {(n-1) 2^{n-1}} - \sum_{n=2}^{\infty} \frac 1 {(n+1) 2^{n+1}} \\
            &= \frac 5 8 - \frac 3 4 \sum_{n=1}^{\infty} \frac 1 {n 2^n} \\
            &= \frac 5 8 + \frac 3 4 \ln 2
        \end{aligned}
        $$
        \item[(2)]
        $$
        \begin{aligned}
            \sum_{n=1}^{\infty} \frac{(-1)^n (n^2-n+1)}{2^n} &= \frac 1 2 + \sum_{n=3}^{\infty} n(n-1) \left( - \frac 1 2 \right)^{n-2} + \sum_{n=1}^{\infty} \left( -\frac 1 2 \right)^n \\
            &= \frac 1 2 + \left. \left( \frac{x^3}{1 - x} \right)'' \right|_{x = - \frac 1 2} + \left. \frac x {1 - x} \right|_{x = - \frac 1 2} \\
            &= - \frac{67}{54}
        \end{aligned}
        $$
        \item[(3)]
        $$
        \begin{aligned}
            \sum_{n=0}^{\infty} \frac{(-1)^n}{3n+1} &= - \sum_{n=0}^{\infty} \frac{(-1)^{3n+1}}{3n+1} \\
            &= - \left. \left( \int \frac{\mathrm dx}{1 - x^3} \right) \right|_{x=-1} \\
            &= \frac{\sqrt 3 \pi}{18} + \frac 1 3 \ln 2
        \end{aligned}
        $$
        \item[(4)]
        $$
        \begin{aligned}
            \sum_{n=0}^{\infty} \frac{(n+1)^2}{n!} &= \sum_{n=0}^{\infty} \frac{n^2}{n!} + 2 \sum_{n=0}^{\infty} \frac n {n!} + \sum_{n=0}^{\infty} \frac 1 {n!} \\
            &= \sum_{n=0}^{\infty} \frac{n+1}{n!} + 2 \sum_{n=0}^{\infty} \frac 1 {n!} + e \\
            &= \sum_{n=0}^{\infty} \frac n {n!} + \sum_{n=0}^{\infty} \frac 1 {n!} + 3e \\
            &= 5e
        \end{aligned}
        $$
    \end{enumerate}
    \item[7.]
    $$
    y = \frac 1 {1 + \lambda} x + \frac{\lambda}{6 (1 + \lambda)^4} x^3
    $$
\end{enumerate}

\section{级数的应用}

\begin{enumerate}
    \item[6.]
    由 Stolz 定理知
    
    $$
    \begin{aligned}
        \lim_{n \to \infty} \frac{\ln n!}{\ln n^n} &= \lim_{n \to \infty} \frac{\ln n! - \ln (n-1)!}{\ln n^n - \ln (n-1)^{n-1}} \\
        &= \lim_{n \to \infty} \frac{\ln n}{\ln n + (n-1) \ln\left( 1 + \frac 1 {n-1} \right)} \\
        &= \lim_{n \to \infty} \frac{\ln n}{\ln n + 1} \\
        &= 1
    \end{aligned}
    $$
\end{enumerate}

\section*{第 7 章综合习题}

\begin{enumerate}
    \item[3.]
    先证充分性:不妨设 $|a_n| \leqslant M$,则
    
    $$
    \displaystyle S_n = \sum_{k=1}^n \frac{a_{k+1} - a_k}{a_k} < \sum_{k=1}^n \frac{a_{k+1} - a_k}{a_1} = \frac{a_{n+1}}{a_1} - 1 \leqslant \frac M {a_1} - 1
    $$

    即 $\displaystyle \sum_{n=1}^{\infty} \left( \frac{a_{n+1}}{a_n} - 1 \right)$ 收敛.

    再证必要性:

    由级数收敛知,对任意 $0 < \varepsilon < \frac 1 2$,存在 $N \in \mathbf N_+$,使得当 $m > n \geqslant N$ 时有

    $$
    1 - \frac{a_n}{a_{m}} < \left| \sum_{k=n}^{m-1} a_{k+1} \left( \frac 1 {a_k} - \frac 1 {a_{k+1}} \right) \right| < \varepsilon
    $$

    即对任意 $m > N$ 有 $a_m < \frac{a_N}{1 - \varepsilon} < 2a_N$.
    
    取 $M = \max\left\{ a_1,\ a_2,\ \ldots,\ 2a_N \right\}$,则恒有 $a_n \leqslant M$,即 $\{ a_n \}$ 有界.
    \item[4.]
    \begin{enumerate}
        \item[(a)]
        当 $\alpha \geqslant 1$ 时:

        $$
        S_n = \sum_{k=1}^n \left( 1 - \frac{a_k}{a_{k+1}} \right) \frac 1 {a_k^{\alpha}} \leqslant \sum_{k=1}^n \left( 1 - \frac{a_k^{\alpha}}{a_{k+1}^{\alpha}} \right) \frac 1 {a_k^{\alpha}} = \frac 1 {a_1^{\alpha}} - \frac 1 {a_{n+1}^{\alpha}} < \frac 1 {a_1^{\alpha}}
        $$

        则 $\displaystyle \sum_{n=1}^{\infty} \frac{a_{n+1} - a_n}{a_{n+1} a_n^{\alpha}}$ 收敛.
        \item[(b)]
        当 $0 < \alpha < 1$ 时:

        由 Lagrange 中值定理知,存在 $\xi \in (a_n, a_{n+1})$ 使得

        $$
        \frac{a_{n+1}^{\alpha} - a_n^{\alpha}}{a_{n+1} - a_n} = \alpha \xi^{\alpha - 1} > \alpha a_{n+1}^{\alpha - 1}
        $$

        即 $\displaystyle \frac{a_{n+1} - a_n}{a_{n+1} a_n^{\alpha}} < \frac 1 {\alpha} \left( \frac 1 {a_n^{\alpha}} - \frac 1 {a_{n+1}^{\alpha}} \right)$.

        故

        $$
        S_n < \frac 1 {\alpha a_1^{\alpha}}  - \frac 1 {\alpha a_{n+1}^{\alpha}} < \frac 1 {\alpha a_1^{\alpha}}
        $$

        即 $\displaystyle \sum_{n=1}^{\infty} \frac{a_{n+1} - a_n}{a_{n+1} a_n^{\alpha}}$ 收敛.
    \end{enumerate}
    \item[5.]
    易知

    $$
    \prod_{n=1}^{\infty} (1 + c_n) = \exp\left( \sum_{n=1}^{\infty} \ln(1 + c_n) \right) \leqslant \exp\left( \sum_{n=1}^{\infty} c_n \right) = e^M = M_1
    $$

    则原不等式可化为

    $$
    \frac{a_{n+1}}{\prod_{k=1}^n (1+c_k)} - \frac{a_n}{\prod_{k=1}^{n-1} (1+c_k)} \leqslant - \frac{b_n \Phi(a_n)}{\prod_{k=1}^n (1+c_n)} < 0
    $$

    不妨设 $\displaystyle d_n = \frac{a_n}{\prod_{k=1}^{n-1} (1+c_k)}$,则 $d_n$ 单调递减有下界,即 $\{ d_n \}$ 收敛.

    而若 $\displaystyle \lim_{n \to \infty} d_n = d > 0$,则

    $$
    d_{n+1} - d_n \leqslant -b_n \frac{\Phi(d)}{M_1} \ \Rightarrow \ d_{n+1} \leqslant d_1 - \frac{\Phi(d)}{M_1} \sum_{k=1}^{n} b_k
    $$

    则 $\displaystyle \lim_{n \to \infty} d_n = -\infty$,矛盾,故 $\displaystyle \lim_{n \to \infty} d_n = 0$,进而有 $\displaystyle \lim_{n \to \infty} a_n = 0$.
    \item[6.]
    由 Cauchy-Schwarz 不等式知

    $$
    (a_1 + a_2 + \cdots + a_n)\left( \frac 1 {a_1} + \frac 4 {a_2} + \cdots + \frac{n^2}{a_n} \right) \geqslant \frac {n^2(n+1)^2} 4
    $$

    即

    $$
    \begin{aligned}
        \frac n {a_1 + a_2 + \cdots + a_n} &\leqslant \frac 4 {n(n+1)^2} \sum_{k=1}^n \frac{k^2}{a_k} \\
        &\leqslant 2 \frac{2n+1}{n^2(n+1)^2} \\
        &= 2 \left( \frac 1 {n^2} - \frac 1 {(n+1)^2} \right) \sum_{k=1}^n \frac{k^2}{a_k}
    \end{aligned}
    $$

    对 $n$ 求和得

    $$
    \begin{aligned}
        \sum_{n=1}^N \frac n {a_1 + a_2 + \cdots + a_n} &\leqslant \sum_{n=1}^N 2 \left( \frac 1 {n^2} - \frac 1 {(n+1)^2} \right) \sum_{k=1}^n \frac{k^2}{a_k} \\
        &= 2 \sum_{k=1}^N \frac{k^2}{a_k} \sum_{n=k}^N \left( \frac 1 {n^2} - \frac 1 {(n+1)^2} \right) \\
        &= 2 \sum_{k=1}^N \frac 1 {a_k} - \frac 2 {(N+1)^2} \sum_{k=1}^N \frac{k^2}{a_k}
    \end{aligned}
    $$

    令 $N \to \infty$,则有

    $$
    \sum_{n=1}^{\infty} \frac n {a_1 + a_2 + \cdots + a_n} \leqslant 2 \sum_{n=1}^{\infty} \frac 1 {a_n}
    $$
    
    或见\href{https://math.stackexchange.com/posts/108598/revisions}{此解}及\href{https://www.komal.hu/feladat?a=feladat&f=A709&l=en}{此解}.
    \item[7.]
    不妨令 $a_1 = 0$,则由上述结论知

    $$
    \sum_{n=1}^{\infty} \frac 1 {a_{n+1} - a_n} \geqslant \frac 1 2 \sum_{n=1}^{\infty} \frac n {a_{n+1} - a_1} \geqslant \frac n {(n+1)^2 \ln(n+1)}
    $$

    故 $\displaystyle \sum_{n=1}^{\infty} \frac 1 {a_{n+1} - a_n}$ 发散.
    \item[9.]
    设 $f_n(x) = x^{a_n} \ (n \in \mathbf N)$,则 $f_{n+1}(x) = x^{\frac {a_n + 1} 2} = x^{a_{n+1}}$.

    易证 $\displaystyle \lim_{n \to \infty} a_n = 1$,故 $\displaystyle \lim_{n \to \infty} f_n(x) = x$.
    \item[11.]
    设 $|f_0(x)| \leqslant M$,则 $|f_n(x)| \leqslant \frac{Ma^n}{n!}$,显然一致收敛于 $0$.
    
    事实上有

    $$
    \begin{aligned}
        \int_0^x \frac{(x-t)^n}{n!} f_0(t) \mathrm dt &= \left. \frac{(x-t)^n}{n!} \left( \int_0^t f_0(u) \mathrm du \right) \right|_0^x + \int_0^x \frac{(x-t)^{n-1}}{(n-1)!} \left( \int_0^t f_0(u) \mathrm du \right) \mathrm dt \\
        &= \int_0^x \frac{(x-t)^{n-1}}{(n-1)!} f_1(t) \mathrm dt \\
        &= \cdots \\
        &= \int_0^x f_n(t) \mathrm dt \\
        &= f_{n+1}(x)
    \end{aligned}
    $$

    若我们先假设 $\{ f_n(x) \}$ 收敛至 $f(x)$,则有 $f(x) = f'(x)$,解得 $f(x) = Ce^x$. 由题设知 $f(0) = 0$,故 $C = 0$,也即 $f(x) \equiv 0$.
\end{enumerate}

\end{document}
