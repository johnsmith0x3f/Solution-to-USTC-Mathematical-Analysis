\chapter{单变量函数的连续性}

\section{连续函数的基本概念}

\begin{problem}{MAblue}{2.1.15}
    设 $f(x)$ 在 $\R$ 上连续,且对于任意 $x, y$ 有 $f(x+y) = f(x) + f(y)$. 求证 $f(x) = cx$,其中 $c$ 是常数.
\end{problem}

\begin{proof}
    易知 $f(nx) = nf(x) \ (n \in \Z)$,进而对任意有理数 $\frac p q$ 有
    \[
        f\left( \frac p q x \right) = \frac p q f(x).
    \]
    由有理数的稠密性知,对任意实数 $\alpha$,总存在有理数列 $\{ a_n \}$ 满足 $\lim_{n \to \infty} a_n = \alpha$. 考虑
    \[
       f(a_n x) = a_n f(x),
    \]
    由于 $f(x)$ 连续,令 $n \to \infty$ 即得 $f(\alpha x) = \alpha f(x)$. 故 $f(x) = x f(1)$,也即 $c = f(1)$.
\end{proof}

\section{闭区间上连续函数的性质}

\begin{problem}{MAblue}{2.2.6}
    设函数 $f(x)$ 在 $[0, 2a]$ 上连续,且 $f(0) = f(2a)$. 证明:在区间 $[0, a]$ 上存在某个 $x_0$,使得 $f(x_0) = f(x_0 + a)$.
\end{problem}

\begin{proof}
    设 $g(x) = f(x) - f(x+a) \ (0 \leqslant x \leqslant a)$,则 $g(0) = -g(a)$,由零点定理可得欲证.
\end{proof}

\begin{problem}{MAblue}{2.2.7}
    试证:若函数 $f(x)$ 在 $[a, b]$ 上连续,$x_1, x_2, \ldots, x_n$ 为此区间中的任意值,则在 $[a, b]$ 中有一点 $\xi$,使得
    \[
        f(\xi) = \frac 1 n \left( f(x_1) + f(x_2) + \cdots + f(x_n) \right).
    \]
    更一般地,若 $q_1, q_2, \ldots, q_n \in \R_+$,且 $q_1 + q_2 + \cdots + q_n = 1$,则在 $[a, b]$ 中有一点 $\xi$,使得
    \[
        f(\xi) = q_1f(x_1) + q_2f(x_2) + \cdots + q_nf(x_n).
    \]
\end{problem}

\begin{proof}
    显然有 $\min\left( f(x_1), f(x_2), \ldots, f(x_n) \right) \leqslant f(\xi) \leqslant \max\left( f(x_1), f(x_2), \ldots, f(x_n) \right)$,则由介值定理可得欲证.
\end{proof}

\begin{problem}{MAblue}{2.2.16}
    给出一个在 $(-\infty, +\infty)$ 上连续且有界但不一致连续的函数.
\end{problem}

\begin{solution}
    $f(x) = \sin x^2$. 由连续且有界想到基本初等函数中的 $\sin x$(或 $\cos x$),而不一致连续的条件启发我们寻找一个任意陡的函数,故想到 $\sin x^2$.
\end{solution}

\section*{第 2 章综合习题}
\addcontentsline{toc}{section}{第 2 章综合习题}

\begin{problem}{MAblue}{2.0.5}
    设 $f(x)$ 在区间 $[0, 1]$ 上连续,且 $f(0) = f(1)$. 证明:对任意自然数 $n$,在区间 $\left[ 0, 1 - \frac 1 n \right]$ 中有一点 $\xi$,使得 $f(\xi) = f\left( \xi + \frac 1 n \right)$.
\end{problem}

\begin{proof}
    设 $g(x) = f(x) - f(x+\frac 1 n) \ (0 \leqslant x \leqslant 1 - \frac 1 n)$,若存在 $g(\xi) = 0$,命题得证;否则有 $g(0) + g(\frac 1 n) + \cdots + g(1-\frac 1 n) = 0$,其中必有两项异号,由零点定理可得欲证.
\end{proof}

\begin{problem}{MAblue}{2.0.8}
    设函数 $f(x)$ 定义在区间 $[a, b]$ 上,满足条件:$a \leqslant f(x) \leqslant b$,且对 $[a, b]$ 中任意的 $x, y$ 有
    \[
        |f(x) - f(y)| \leqslant k|x - y|,
    \]
    其中常数 $k \in (0, 1)$,证明
    \begin{enumerate}
        \item[(1)]
        存在唯一的 $x_0 \in [a, b]$,使得 $f(x_0) = x_0$.
        \item[(2)]
        任取 $x_1 \in [a, b]$,并定义数列 $\{ x_n \}$:$x_{n+1} = f(x_n) \ (n = 1, 2, \cdots)$,则 $\lim_{n \to \infty} x_n = x_0$.
        \item[(3)]
        给出一个在实轴上的连续函数,使得对任意 $x \neq y$ 有 $|f(x) - f(y)| < |x-y|$,但方程 $f(x) - x = 0$ 无解.
    \end{enumerate}
\end{problem}

\begin{enumerate}
    \item[(1)]
    \begin{proof}
        设 $g(x) = f(x) - x$,则 $g(a) \geqslant 0,\ g(b) \leqslant 0$. 故由零点定理知存在 $x_0 \in [a, b]$ 使得 $g(x_0) = 0$,也即 $f(x_0) = x_0$. 若存在另一 $x_0' \in [a, b]$ 满足 $f(x_0') = x_0'$,则
        \[
            |f(x_0) - f(x_0')| = |x_0 - x_0'| > k|x_0 - x_0'|,
        \]
        与题设矛盾. 故 $x_0$ 是唯一的.
    \end{proof}
    \item[(2)]
    \begin{proof}
        由题设有 $|x_{n+1} - x_n| \leqslant k|x_n - x_{n-1}| \ (n \geqslant 2)$. 设 $M = \left| \dfrac{x_2 - x_1}{k^2} \right|$,则对任意 $m > n$,有
        \begin{align*}
            |x_n - x_m| &\leqslant |x_n - x_{n+1}| + \cdots + |x_{m-1} - x_m| \\
            &\leqslant M k^n + M k^{n+1} + \cdots + M k^m.
        \end{align*}
        故由 1.2.17.(3) 及夹逼原理知,数列 $\{ x_n \}$ 是基本列. 进而由 Cauchy 收敛准则知,数列 $\{ x_n \}$ 收敛. 设 $\lim_{n \to \infty} x_n = x$,则
        \[
            x = f(x) \ \Rightarrow \ x = x_0.
        \]
        也即 $\lim_{n \to \infty} x_n = x_0$.
    \end{proof}
\end{enumerate}

事实上,此即

\begin{center}
    \begin{minipage}{0.85\textwidth}
        \begin{theorem}{压缩映射原理}{}
            设 $f: [a, b] \to [a, b]$,且存在 $k \in (0, 1)$ 使得
            \[
                |f(x_1) - f(x_2)| \leqslant k|x_1 - x_2|
            \]
            对任意 $x_1, x_2 \in [a, b]$ 均成立,则
            \begin{enumerate}[label=\arabic*$^\circ$]
                \item
                存在唯一的 $\xi \in [a, b]$,使得 $f(\xi) = \xi$;
                \item
                若数列 $\{ x_n \}$ 满足 $x_0 \in [a, b],\ x_{n+1} = f(x_n)$,则 $\lim_{n \to \infty} x_n = \xi$.
            \end{enumerate}
        \end{theorem}
    \end{minipage}
\end{center}

\begin{problem}{MAblue}{2.0.10}
    设 $f(x)$ 在 $[a, b]$ 上连续,且对任意 $x \in [a, b)$ 存在 $y \in (x, b)$ 使得 $f(y) > f(x)$. 求证:$f(b) > f(a)$.
\end{problem}

\begin{proof}
    由题设知,存在数列 $\{ x_n \}$,满足 $x_0 = a,\ x_n < x_{n+1} < b$,且 $f(x_n) < f(x_{n+1})$. 由单调有界原理知数列 $\{ x_n \}$ 收敛,不妨设其收敛到 $c$,则必有 $c = b$,否则对任意 $x \in (c, b)$ 均有 $f(c) \geqslant f(x)$,与题设矛盾. 而当 $n > 1$ 时,我们有
    \[
        a = f(x_0) < f(x_1) < f(x_n).
    \]
    由于 $f(x)$ 连续,令 $n \to \infty$ 即得 $f(a) < f(x_1) \leqslant f(b)$.
\end{proof}