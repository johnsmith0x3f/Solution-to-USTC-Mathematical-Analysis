\chapter{单变量函数的微分学}

\section{导数}

\begin{problem}{MAblue}{3.1.17}
    求下列各式的和
    \begin{enumerate}
        \item[(1)]
        $P_n = 1 + 2x + \cdots + nx^{n-1}$;
        \item[(2)]
        $Q_n = 1^2 + 2^2x + \cdots + n^2x^{n-1}$;
        \item[(3)]
        $R_n = \cos 1 + 2 \cos 2 + \cdots + n \cos n$.
    \end{enumerate}
\end{problem}

\begin{enumerate}
    \item[(1)]
    \begin{solution}
        设
        \[
            f(x) = \frac{x(1-x^n)}{1-x} = x + x^2 + \cdots + x^n \ (x \neq 1),
        \]
        则
        \[
            P_n = f'(x) = \frac{nx^{n+1}-(n+1)x^n+1}{(x-1)^2}.
        \]
    \end{solution}
    \item[(2)]
    \begin{solution}
        \[
            Q_n = f'(x)+xf''(x) = \frac{n^2x^{n+2}-(2n^2+2n-1)x^{n+1}+(n+1)^2x^n-x-1}{(x-1)^3}.
        \]
    \end{solution}
    \item[(3)]
    \begin{solution}
        设
        \[
            g(x) = \sin x + \sin 2x + \cdots + \sin nx = \frac{\cos\frac x 2 - \cos(\frac {2n+1} 2 x)}{2\sin\frac x 2},
        \]
        则
        \[
            R_n = g'(1) = \frac{(n+1)\cos n - n\cos(n+1)-1}{4\sin^2\frac 1 2}.
        \]
    \end{solution}
\end{enumerate}

\section{微分}

\begin{problem}{MAblue}{3.2.3}
    对下列函数,求 $\frac{\mathrm dy}{\mathrm dx}$ 及 $\frac{\mathrm d^2y}{\mathrm dx^2}$.
    \begin{multicols}{2}
        \begin{enumerate}
            \item[(1)]
            $\begin{cases}
                x = \ln(1+t^2), \\
                y = t - \arctan t;
            \end{cases}$
            \item[(2)]
            $\begin{cases}
                x = t - \sin t, \\
                y = 1 - \cos t;
            \end{cases}$
            \item[(3)]
            $\begin{cases}
                x = \varphi \cos \varphi, \\
                y = \varphi \sin \varphi;
            \end{cases}$
            \item[(4)]
            $\begin{cases}
                x = \cos^3 \varphi, \\
                y = \sin^3 \varphi.
            \end{cases}$
        \end{enumerate}
    \end{multicols}
\end{problem}

\begin{enumerate}
    \item[(1)]
    \begin{solution}
    \[
        \begin{cases}
            \mathrm dx = \frac{2t}{1+t^2} \mathrm dt, \\
            \mathrm dy = \frac 1 {1+t^2} \mathrm dt,
        \end{cases}
        \quad \Rightarrow \quad
        \begin{cases}
            \frac{\mathrm dy}{\mathrm dx} = \frac 1 {2t}, \\
            \frac{\mathrm d^2y}{\mathrm dx^2} = \frac{\mathrm d \left( \frac{\mathrm dy}{\mathrm dx} \right)}{\mathrm dx} = -\frac{1+t^2}{4t^3}.
        \end{cases}
    \]
    \end{solution}
\end{enumerate}

\section{微分中值定理}

\begin{problem}{MAblue}{3.3.4}
    证明下列不等式.
    \begin{enumerate}
        \item[(1)]
        当 $a > b > 0,\ n > 1$ 时,有 $nb^{n-1}(a-b) < a^n - b^n < na^{n-1}(a-b)$;
        \item[(2)]
        当 $x > 0$ 时,有 $\frac x {1+x} < \ln(1+x) < x$;
        \item[(3)]
        当 $0 < a < b$ 时,有 $(a+b) \ln \frac {a+b} 2 < a \ln a + b \ln b$;
        \item[(4)]
        当 $0 < \alpha < \beta < \frac \pi 2$ 时,有 $\frac{\beta-\alpha}{\cos^2\alpha} < \tan \beta - \tan \alpha < \frac{\beta-\alpha}{\cos^2\beta}$.
    \end{enumerate}
\end{problem}

\begin{enumerate}
    \item[(3)]
    \begin{proof}
        设 $f(x) = x \ln x$,则在 $\left( a,\ \frac {a+b} 2 \right)$ 和 $\left( \frac {a+b} 2,\ b \right)$ 上,由 Lagrange 中值定理有
        \[
            \frac{f\left( \frac {a+b} 2 \right) - f(a)}{\frac {b-a} 2} = 1 + \ln \xi_1 < 1 + \ln \xi_2 = \frac{f(b) - f\left( \frac {a+b} 2 \right)}{\frac {b-a} 2}
        \]
        整理后即得欲证.
    \end{proof}
    事实上,此为 \textbf{Jensen 不等式}的一种形式:当 $f(x)$ 是凸函数,即 $f''(x) > 0$ 时,有 $\frac {f(a) + f(b)} 2 > f(\frac {a+b} 2)$.
\end{enumerate}

\begin{problem}{MAblue}{3.3.7}
    设函数 $f(x)$ 在 $[0, 1]$ 上连续,在 $(0, 1)$ 内可微,且 $f'(x) < 1$,又 $f(0) = f(1)$,证明:对于 $[0, 1]$ 上的任意两点 $x_1, x_2$,有 $|f(x_1) - f(x_2)| < \frac 1 2$.
\end{problem}

\begin{proof}
    不妨设 $x_1 < x_2$,则由 Lagrange 中值定理知,存在 $\xi \in (x_1, x_2)$,使得
    \[
        \left| \frac{f(x_1) - f(x_2)}{x_1 - x_2} \right| = |f'(\xi)| < 1,
    \]
    也即 $|f(x_1) - f(x_2)| < |x_1 - x_2| = x_2 - x_1$. 则若 $x_2 - x_1 < \frac 1 2$,结论成立;否则有 $|f(x_1) - f(x_2)| \leqslant |f(x_1) - f(0)| + |f(1) - f(x_2)| < 1 - (x_2 - x_1) \leqslant \frac 1 2$.
\end{proof}

\begin{problem}{MAblue}{3.3.12}
    设对所有的实数 $x, y$,不等式 $|f(x) - f(y)| \leqslant M|x-y|^2$ 都成立. 证明:$f(x)$ 恒为常数.
\end{problem}

\begin{proof}
    两边同除 $|x-y|$ 得
    \[
        \left| \frac{f(x)-f(y)}{x-y} \right| \leqslant M|x-y|.
    \]
    令 $y \to x$,得 $f'(x) = 0$,故 $f(x)$ 恒为常数.
\end{proof}

\begin{problem}{MAblue}{3.3.18}
    若 $f(x)$ 在 $[0, +\infty)$ 可微,$f(0) = 0$,$f'(x)$ 严格递增,证明 $\frac {f(x)} x$ 严格递增.
\end{problem}

\begin{proof}
    由题设知对任意 $x > 0$ 存在 $\xi \in (0,\ x)$ 使得 $\frac{f(x)-f(0)}{x-0} = f'(\xi) < f'(x)$,则
    \[
        \frac{\mathrm d \left( \frac{f(x)}{x} \right)}{\mathrm dx} = \frac{xf'(x)-f(x)}{x^2} > 0,
    \]
    即 $\frac{f(x)}{x}$ 严格递增.
\end{proof}

\begin{problem}{MAblue}{3.3.20}
    设 $f(x)$ 在 $[0, 1]$ 上有二阶导函数,且 $f(0) = f'(0),\ f(1) = f'(1)$,求证:存在 $\xi \in (0, 1)$ 满足 $f(\xi) = f''(\xi)$.
\end{problem}

\begin{proof}
    设 $g(x)=e^x\left( f(x)-f'(x) \right)$,则 $g(0) = g(1) = 0$. 进而由 Rolle 定理知,存在 $\xi \in (0,\ 1)$,满足 $g'(\xi) = e^{\xi}\left( f(\xi) - f''(\xi) \right) = 0$,也即 $f(\xi) = f''(\xi)$.
\end{proof}

\begin{problem}{MAblue}{3.3.23}
    证明下列不等式.
    \begin{enumerate}
        \item[(1)]
        $\frac 1 {2^{p-1}} \leqslant x^p + (1-x)^p \leqslant 1,\ x \in (0, 1],\ p > 1$;
        \item[(2)]
        $\tan x > x - \frac {x^3} 3,\ x \in \left( 0, \frac \pi 2 \right)$;
        \item[(3)]
        $\frac{\tan x_2}{\tan x_1} > \frac{x_2}{x_1},\ 0 < x_1 < x_2 < \frac \pi 2$;
        \item[(4)]
        $\ln(1+x) > \frac{\arctan x}{1+x},\ x > 0$;
        \item[(5)]
        $1 + x \ln(x+\sqrt{1+x^2}) \geqslant \sqrt{1+x^2},\ x \in \R$;
        \item[(6)]
        $\frac x {\sin x} > \frac 4 3 - \frac 1 3 \cos x,\ x \in \left( 0, \frac \pi 2 \right)$,且右端的 $\frac 4 3$ 为最优系数;
        \item[(7)]
        $\left( 1 - \frac 1 x \right)^{x-1} \left( 1 + \frac 1 x \right)^{x+1} < 4,\ x \in (1, +\infty)$;
        \item[(8)]
        $x^{a-1} + x^{a+1} \geqslant \left( \frac{1-a}{1+a} \right)^{\frac {a-1} 2} + \left( \frac{1-a}{1+a} \right)^{\frac {a+1} 2},\ x \in (0, 1),\ a \in (0, 1)$.
    \end{enumerate}
\end{problem}

\begin{enumerate}
    \item[(1)]
    \begin{proof}
        先证右边. 显然有 $x^p + (1-x)^p \leqslant x + 1-x = 1$.
    
        对于左边,不妨设 $x < \frac 1 2$,则
        \[
            \frac 1 {2^{p-1}} \leqslant x^p + (1-x)^p \quad \Leftrightarrow \quad \frac 1 {2^p} - x^p \leqslant (1-x)^p - \frac 1 {2^p}.
        \]
        由 Lagrange 中值定理知,存在 $x < \xi_1 < \frac 1 2 < \xi_2 < 1-x$,满足
        \[
            \frac{\frac 1 {2^p} - x^p}{\frac 1 2 - x} = p\xi_1^{p-1} < p\xi_2^{p-1} = \frac{(1-x)^p - \frac 1 {2^p}}{1-x-\frac 1 2},
        \]
        则原式得证.
    \end{proof}
    \item[(3)]
    \begin{proof}
        由 Lagrange 中值定理知,存在 $0 < \xi_1 < x_1 < \xi_2 < x_2 < \frac \pi 2$,满足
        \[
            \frac{\tan x_1-0}{x_1-0} = \frac 1 {\cos^2 \xi_1} < \frac 1 {\cos^2 \xi_2} = \frac{\tan x_2 - \tan x_1}{x_2 - x_1},
        \]
        整理后即得原式.
    \end{proof}
    \item[(5)]
    \begin{proof}
        注意到不等式两侧均为关于 $x$ 的偶函数,而易知 $x = 0$ 时等号成立,故不妨设 $x > 0$. 由 Lagrange 中值定理知,存在 $\xi \in (0,\ x)$,满足
        \[
            \frac{\ln\left( x+\sqrt{1+x^2} \right)-0}{x-0} = \frac 1 {\sqrt{1+\xi^2}} > \frac 1 {\sqrt{1+x^2}},
        \]
        代入原式即得.
    \end{proof}
\end{enumerate}

\section{未定式的极限}

\begin{problem}{MAblue}{3.4.4}
    设 $f(x)$ 在 $[a, b] \ (ab>0)$ 上连续,在 $(a, b)$ 上可微. 求证:存在 $\xi \in (a, b)$,使
    \[
        \frac{af(b) - bf(a)}{a-b} = f(\xi) - \xi f'(\xi).
    \]
\end{problem}

\begin{proof}
    由 Cauchy 中值定理知,存在 $\xi \in (a,\ b)$ 满足
    \[
        \frac{\frac {f(b)} b - \frac {f(a)} a}{\frac 1 b - \frac 1 a} = \frac{\frac{\xi f'(\xi) - f(\xi)}{\xi^2}}{-\frac 1 {\xi^2}} = f(\xi) - \xi f'(\xi),
    \]
    整理后即得.
\end{proof}

\section{函数的单调性和凸性}

\begin{problem}{MAblue}{3.5.1}
    证明 Jensen 不等式.
\end{problem}

\begin{center}
    \begin{minipage}{0.9\textwidth}
        \begin{theorem}{Jensen 不等式}{}
            若 $f(x)$ 是区间 $I$ 上的凸函数,$x_1, x_2, \ldots, x_n$ 是 $I$ 中 $n$ 个点,则对任意满足 $\lambda_1 + \lambda_2 + \cdots + \lambda_n = 1$ 的正数 $\lambda_1, \lambda_2, \ldots, \lambda_n$ 有
            \[
                f(\lambda_1 x_1 + \cdots + \lambda_n x_n) \leqslant \lambda_1 f(x_1) + \cdots + \lambda_n f(x_n).
            \]
        \end{theorem}
    \end{minipage}
\end{center}

\begin{proof}
    归纳证明. 显然 $n = 1$ 时成立. 假设 $n=k \ (\geqslant 1)$ 时结论成立,则当 $n = k + 1$ 时,设 $\lambda_1 + \lambda_2 + \cdots + \lambda_{k+1} = 1$,则
    \[
        f(\frac{\lambda_1x_1 + \lambda_2x_2 + \cdots + \lambda_kx_k}{1-\lambda_{k+1}}) \leqslant \frac{\lambda_1f(x_1) + \lambda_2f(x_2) + \cdots + \lambda_kf(x_k)}{1-\lambda_{k+1}}.
    \]
    由凸函数定义知
    \[
        f\left( (1-\lambda_{k+1}) \frac{\sum_{i=1}^k \lambda_ix_i}{1-\lambda_{k+1}} + \lambda_{k+1}x_{k+1} \right) \leqslant (1-\lambda_{k+1}) \frac{\sum_{i=1}^k \lambda_if(x_i)}{1-\lambda_{k+1}} + \lambda_{k+1}f(x_{k+1}).
    \]
    {\flushleft 整理后即得 $n = k+1$ 时成立. 故原命题成立.}
\end{proof}

\begin{problem}{MAblue}{3.5.2}
    证明加权均值不等式.
\end{problem}

\begin{center}
    \begin{minipage}{0.85\textwidth}
        \begin{theorem}{加权均值不等式}{}
            设 $x_1, x_2, \ldots, x_n$ 和 $\lambda_1, \lambda_2, \ldots, \lambda_n$ 都是正数,且 $\lambda_1 + \lambda_2 + \cdots + \lambda_n = 1$. 则有不等式
            \[
                x_1^{\lambda_1} x_2^{\lambda_2} \cdots x_n^{\lambda_n} \leqslant \lambda_1 x_1 + \lambda_2 x_2 + \cdots + \lambda_n x_n.
            \]
        \end{theorem}
    \end{minipage}
\end{center}

\begin{proof}
    设 $A_n = \lambda_1 x_1 + \lambda_2 x_2 + \cdots + \lambda_n x_n,\ G_n = x_1^{\lambda_1} x_2^{\lambda_2} \cdots x_n^{\lambda_n}$,则
    \[
        \ln\left( \frac{G_n}{A_n} \right) = \sum_{k=1}^n \lambda_i \ln\left( \frac{x_i}{A_n} \right) \leqslant \sum_{k=1}^n \lambda_i \left( \frac{x_i}{A_n} - 1 \right) = 0.
    \]
    整理后即得.
\end{proof}
{\flushleft 亦可借助 Jensen 不等式证明.}

\section{Taylor 展开}

\begin{problem}{MAblue}{3.6.8}
    设函数 $f(x)$ 在 $[0, 2]$ 上二阶可导,且对任意 $x \in [0, 2]$,有 $|f(x)| \leqslant 1$ 及 $|f''(x)| \leqslant 1$. 证明:$|f'(x)| \leqslant 2,\ x \in [0, 2]$.
\end{problem}

\begin{proof}
    对任意 $x \in [0, 2]$,由 Taylor 公式得
    \[
        \begin{cases}
            f(0) = f(x) - f'(x)x + \frac {f''(\xi_1)} 2 (0-x)^2, & 0 \leqslant \xi_1 \leqslant x, \\
            f(2) = f(x) + f'(x)(2-x) + \frac {f''(\xi_2)} 2 (2-x)^2, & x \leqslant \xi_2 \leqslant 2.
        \end{cases}
    \]
    两式相减得
    \[
        |f'(x)| = \left| \frac {f(2)-f(0)} 2 + \frac {x^2 f''(\xi_1) - (x-2)^2 f''(\xi_2)} 4 \right| \leqslant 1 + \frac{x^2 + (x-2)^2} 4 \leqslant 2. \qedhere
    \]
\end{proof}

\section*{第 3 章综合习题}
\addcontentsline{toc}{section}{第 3 章综合习题}

\begin{problem}{MAblue}{3.0.5}
    设 $f(x)$ 在区间 $I$ 上连续,如果任给 $I$ 中两点 $x_1, x_2$,有
    \[
        f \left( \frac {x_1 + x_2} 2 \right) \leqslant \frac {f(x_1) + f(x_2)} 2,
    \]
    则 $f(x)$ 是区间 $I$ 上的凸函数.
\end{problem}

\begin{proof}
    先考虑以下引理:
    \begin{center}
        \begin{minipage}{0.9\textwidth}
            \begin{lemma}{}{}
                对任意 $x_1, x_2 \in I,\ i, n \in \N_+,\ 0 \leqslant i \leqslant 2^n$,有
                \[
                    f\left( \frac{ix_1+(2^n-i)x_2}{2^n} \right) \leqslant \frac i {2^n} f(x_1) + \left( 1 - \frac i {2^n} \right) f(x_2).
                \]
                \tcblower
                \begin{proof}
                    考虑归纳证明. 当 $n = 1$ 时,结论显然成立.

                    若当 $n = k (\geqslant 1)$ 时,结论成立,则当 $n = k + 1$ 时,若 $i = 0$ 或 $i = 2^n$,结论是显然的,故我们考虑证明 $0 < i < 2^{k+1}$ 的情况. 若 $i$ 为偶数,结论显然成立;若 $i$ 为奇数,不妨设 $i < 2^k$,则
                    \begin{align*}
                        f\left( \frac{(i-1)x_1+(2^{k+1}-i+1)x_2}{2^{k+1}} \right) \leqslant \frac {i-1} {2^{k+1}} f(x_1) + \left( 1 - \frac {i-1} {2^{k+1}} \right) f(x_2), \\
                        f\left( \frac{(i+1)x_1+(2^{k+1}-i-1)x_2}{2^{k+1}} \right) \leqslant \frac {i+1} {2^{k+1}} f(x_1) + \left( 1 - \frac {i+1} {2^{k+1}} \right) f(x_2).
                    \end{align*}
                    则将
                    \[
                        x_1' = \frac{(i-1)x_1+(2^{k+1}-i+1)x_2}{2^{k+1}},\ x_2' = \frac{(i+1)x_1+(2^{k+1}-i-1)x_2}{2^{k+1}}
                    \]
                    代入 $f\left( \frac {x_1+x_2} 2 \right) \leqslant \frac {f(x_1)+f(x_2)} 2$ 中即可得结论成立.
                \end{proof}
            \end{lemma}
        \end{minipage}
    \end{center}
    对任意实数 $\lambda \in (0,\ 1)$,可利用上述结论逼近之,再由函数连续性得
    \[
        f(\lambda x_1 + (1 - \lambda) x_2) \leqslant \lambda f(x_1) + (1 - \lambda) f(x_2),
    \]
    即 $f(x)$ 是凸函数.
\end{proof}

\begin{problem}{MAblue}{3.0.6}
    设 $f(x)$ 是 $[0, 1]$ 上的两阶可微函数,$f(0) = f(1) = 0$. 证明:存在 $\xi \in (0, 1)$,使得 $f''(\xi) = \frac{2f'(\xi)}{1-\xi}$.
\end{problem}

\begin{proof}
    设 $g(x) = f'(x)(x-1)^2$,由 Rolle 定理知存在 $\eta \in (0,\ 1)$,满足 $f'(\eta) = 0$. 则 $g(\eta) = g(1) = 0$,故再由 Rolle 定理知存在 $\xi \in (\eta,\ 1)$,满足 $g'(\xi) = 0$,即 $f''(\xi) = \frac{2f'(\xi)}{1-\xi}$.
\end{proof}

\begin{mnote}
    下简述选取辅助函数 $g(x)$ 的思路. 我们试图利用 Rolle 定理证明此题,这要求等式 $f''(\xi) = \frac{2f'(\xi)}{1-\xi}$ 可化成 $g'(\xi) = 0$ 的形式,也即有 $g(x) = C$. 解此常微分方程,得到 $f'(x)(x-1)^2 = C$,故可令 $g(x) = f'(x)(x-1)^2$.
\end{mnote}

\begin{problem}{MAblue}{3.0.9}
    设 $f(x)$ 在 $[0, 1]$ 上可导,$f(0) = 1,\ f(1) = \frac 1 2$. 求证:存在 $\xi \in (0, 1)$ 使得
    \[
        f^2(\xi) + f'(\xi) = 0.
    \]
\end{problem}

\begin{proof}
    设 $g(x) = \frac 1 {f(x)} - x$,则 $g(0) = g(1) = 1$. 由 Rolle 定理知,存在 $\xi \in (0,\ 1)$,满足 $g'(\xi) = - \frac{f'(\xi)}{f^2(\xi)} - 1 = 0$,即 $f^2(\xi) + f'(\xi) = 0$.
\end{proof}

\begin{problem}{MAblue}{3.0.19}
    设函数 $f(x)$ 在闭区间 $[-1, 1]$ 上具有三阶连续导数,且 $f(-1) = 0,\ f(1) = 1,\ f'(0) = 0$. 证明:存在 $\xi \in (-1, 1)$,使得 $f'''(\xi) = 3$.
\end{problem}

\begin{proof}
    设 $g(x) = f(x) - \frac 1 2 x^3 + \left( f(0) - \frac 1 2 \right) x^2$,则 $g(-1) = g(0) = g(1)$. 则由 Rolle 定理知存在 $\xi_1 \in (-1,\ 0)$ 及 $\xi_2 \in (0,\ 1)$,满足 $g'(\xi_1) = g'(\xi_2) = g(0) = 0$. 故知存在 $\eta_1 \in (\xi_1,\ 0)$ 及 $\eta_2 \in (0,\ \xi_2)$,满足 $g''(\eta_1) = g''(\eta_2) = 0$. 进而又知存在 $\zeta \in (\eta_1,\ \eta_2)$,满足 $g'''(\zeta) = 0$,即 $f'''(\zeta) = 3$.
\end{proof}

\begin{mnote}
    此题中 $g(x)$ 的选取思路与本节第 6 题不同. 仍然是考虑 Rolle 定理,则我们需要某个 $\phi(x)$ 满足 $\phi'(\xi) = f'''(\xi) - 3 = 0$,故 $\phi(x) = f''(x) - 3x + a$,依此类推,还应有 $\varphi(x) = f'(x) - \frac 3 2 x^2 + ax + b$ 及 $g(x) = f(x) - \frac 1 2 x^3 + \frac a 2 x^2 + bx + c$,为能导出欲证,应有 $g(-1) = g(0) = g(1)$,由此可解得参数 $a, b, c$.
\end{mnote}

\begin{problem}{MAblue}{3.0.20}
    设 $a > 1$,函数 $f : (0, +\infty) \to (0, +\infty)$ 可微. 求证存在趋于无穷的正数列 $\{ x_n \}$ 使得
    \[
        f'(x_n) < f(ax_n),\ n = 1, 2, \cdots.
    \]
\end{problem}

\begin{proof}
    用反证法. 假设命题不成立,即存在 $X > 0$,使得 $x > X$ 时均有 $f'(x) \geqslant f(ax) > 0$. 故 $f(x)$ 在 $(X,\ +\infty)$ 上单调递增. 由 Lagrange 中值定理知,当 $x > X$ 时存在 $\xi \in (x,\ ax)$,满足
    \[
        \frac{f(ax)-f(x)}{(a-1)x} = f'(\xi) \geqslant f(a\xi) > f(ax).
    \]
    故有 $(1-ax+x)f(ax) \geqslant f(x)$. 则当 $x > \frac 1 {a-1}$ 时有 $f(x) < 0$,与题设矛盾,故命题成立.
\end{proof}

\begin{problem}{MAblue}{3.0.21}
    证明 H{\"o}lder 不等式.
\end{problem}

\begin{center}
    \begin{minipage}{0.85\textwidth}
        \begin{theorem}{H{\"o}lder 不等式}{}
            设 $\{ a_i \}, \{ b_i \} \ (i = 1, 2, \ldots, n)$ 是正数. 有 $p, q \in (1, +\infty)$,且 $\frac 1 p + \frac 1 q = 1$,则有
            \[
                \sum_{i=1}^n a_ib_i \leqslant \left( \sum_{i=1}^n a_i^p \right)^{\frac 1 p} \left( \sum_{i=1}^n a_i^q \right)^{\frac 1 q}.
            \]
        \end{theorem}
    \end{minipage}
\end{center}

\begin{proof}
    考虑以下引理:
    \begin{center}
        \begin{minipage}{0.85\textwidth}
            \begin{lemma}{Young 不等式}{}
                设 $x, y \in (0, +\infty),\ p, q \in (1, +\infty)$,且 $\frac 1 p + \frac 1 q = 1$,则
                \[
                    xy \leqslant \frac 1 p x^p + \frac 1 q y^q.
                \]
                \tcblower
                \begin{proof}
                    由 $\ln x$ 凹性知
                    \[
                        \ln xy = \frac 1 p \ln x^p + \frac 1 q \ln y^q \leqslant \ln\left( \frac 1 p x^p + \frac 1 q y^q \right).
                    \]
                    整理后即得.
                \end{proof}
            \end{lemma}
        \end{minipage}
    \end{center}
    令 $x = \frac{a_i}{\left( \sum_{i=1}^n a_i^p \right)^{\frac 1 p}},\ y = \frac{b_i}{\left( \sum_{i=1}^n b_i^q \right)^{\frac 1 q}}$,则由引理有
    \[
        \frac{a_ib_i}{\left( \sum_{i=1}^n a_i^p \right)^{\frac 1 p} \left( \sum_{i=1}^n b_i^q \right)^{\frac 1 q}} \leqslant \frac{a_i^p}{p \left( \sum_{i=1}^n a_i^p \right)} + \frac{b_i^q}{q \left( \sum_{i=1}^n b_i^q \right)},
    \]
    累加得
    \[
        \frac{\sum_{i=1}^n a_ib_i}{\left( \sum_{i=1}^n a_i^p \right)^{\frac 1 p} \left( \sum_{i=1}^n b_i^q \right)^{\frac 1 q}} \leqslant \frac 1 p + \frac 1 q = 1.
    \]
    整理后即得欲证.
\end{proof}