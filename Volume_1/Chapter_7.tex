\chapter{无穷级数}

\section{数项级数}

\begin{problem}{MAblue}{7.1.2}
    \begin{multicols}{2}
        \begin{enumerate}[label={(\arabic*)}]
            \item $\displaystyle \sum_{n=1}^\infty \sqrt[n]{0.001}$;
            \item $\displaystyle \sum_{n=1}^\infty \frac 1 {n\sqrt{n-1}}$;
            \item $\displaystyle \sum_{n=1}^\infty \frac 1 {\sqrt{(2n-1)(2n+1)}}$;
            \item $\displaystyle \sum_{n=1}^\infty \sin n$;
            \item $\displaystyle \sum_{n=1}^\infty 2^n \frac \pi {3^n}$;
            \item $\displaystyle \sum_{n=1}^\infty \frac 1 {n\sqrt n}$;
            \item $\displaystyle \sum_{n=1}^\infty \frac 1 {\left( 2 + \frac 1 n \right)^n}$;
            \item $\displaystyle \sum_{n=1}^\infty \frac n {\left( n + \frac 1 n \right)^n}$;
            \item $\displaystyle \sum_{n=1}^\infty \arctan \frac \pi {4n}$;
            \item $\displaystyle \sum_{n=1}^\infty \frac{1000^n}{n!}$;
            \item $\displaystyle \sum_{n=1}^\infty \frac{(n!)^2}{(2n)!}$;
            \item $\displaystyle \sum_{n=1}^\infty \frac{3+(-1)^n}{2^n}$;
            \item $\displaystyle \sum_{n=1}^\infty \frac{\ln n}{\sqrt[4]{n^5}}$;
            \item $\displaystyle \sum_{n=2}^\infty \frac 1 {n(\ln n)^k}$;
            \item $\displaystyle \sum_{n=1}^\infty \left( \cos \frac 1 n \right)^{n^3}$;
            \item $\displaystyle \sum_{n=2}^\infty \left( \frac{an}{n+1} \right)^n,\ a > 0$.
        \end{enumerate}
    \end{multicols}
\end{problem}

\begin{enumerate}
    \item[(13)]
    \begin{solution}
        对任意 $0 < k < \frac 1 4$,存在 $N \in \N_+$ 使得当 $n \geqslant N$ 时有 $\ln n < n^k$. 故
        \[
            \sum_{n=N}^{\infty} \frac{\ln n}{\sqrt[4]{n^5}} < \sum_{n=N}^{\infty} \frac 1 {n^{\frac 5 4 - k}}.
        \]
        由比较审敛法知 $\sum_{n=1}^{\infty} \frac{\ln n}{\sqrt[4]{n^5}}$ 收敛.
    \end{solution}
    \item[(14)]
    \begin{solution}
        易知
        \[
            \int_3^{+\infty} \frac 1 {x \ln x (\ln \ln x)^k} = \begin{cases}
                \left. \ln \ln \ln x \right|_3^{+\infty} = +\infty, & k = 1; \\
                \left. \frac{(\ln \ln x)^{1-k}}{1-k} \right|_3^{+\infty} = +\infty, & k \neq 1.
            \end{cases}
        \]
        故由 Cauchy 积分判别法知级数 $\sum_{n=2}^{\infty} \frac 1 {n \ln n (\ln \ln n)^k}$ 发散.
    \end{solution}
    \item[(15)]
    \begin{solution}
        当 $n \to \infty$ 时,有
        \[
            \left( \cos \frac 1 n \right)^{n^3} \sim \left( 1 - \frac 1 {2n^2} \right)^{n^3} = \frac 1 {e^{\frac n 2}}.
        \]
        故原级数收敛.
    \end{solution}
    \item[(16)]
    \begin{solution}
        当 $n \to \infty$ 时,有
        \[
            \left( \frac{an}{n+1} \right)^n = \frac{a^n}{\left( 1 + \frac 1 n \right)^n} \sim \frac {a^n} e.
        \]
        故 $a \geqslant 1$ 时原级数发散,$a < 1$ 时原级数收敛.
    \end{solution}
\end{enumerate}

\begin{problem}{MAblue}{7.1.4}
    证明或回答下列论断:
    \begin{enumerate}[label={(\arabic*)}]
        \item 若 $\lim_{n \to \infty} na_n = a \neq 0$,则级数 $\sum_{n=1}^\infty a_n$ 发散;
        \item 若级数 $\sum_{n=1}^\infty a_n$ 收敛,是否有 $\lim_{n \to \infty} na_n = 0$;
        \item 若 $\lim_{n \to \infty} na_n = a$,且级数 $\sum_{n=1}^\infty n(a_n-a_{n+1})$ 收敛,证明 $\sum_{n=1}^\infty a_n$ 收敛.
    \end{enumerate}
\end{problem}

\begin{enumerate}
    \item[(3)]
    \begin{proof}
        设 $A_n = \sum_{k=1}^n k (a_k - a_{k+1})$,则由题设知 $\lim_{n \to \infty} A_n$ 存在,不妨设之为 $A$. 又由 $\lim_{n \to \infty} n a_n = a$ 知 $\lim_{n \to \infty} a_n = 0$. 故
        \[
            \lim_{n \to \infty} \sum_{k=1}^n a_k = \lim_{n \to \infty} \left( A_n + (n+1)a_{n+1} - a_{n+1} \right) = A + a - 0.
        \]
        故原级数收敛.
    \end{proof}
\end{enumerate}

\begin{problem}{MAblue}{7.1.6}
    设 $\{ a_n \}, \{ b_n \}$ 是两个非负数列,满足 $a_{n+1} < a_n + b_n$,而且 $\sum_{n=1}^\infty b_n$ 收敛. 求证 $\{ a_n \}$ 收敛.
\end{problem}

\begin{proof}
    设 $c_n = a_n - a_1 - \sum_{k=1}^{n-1} b_k$,则由单调有界原理知 $\{ c_n \}$ 收敛,进而得 $\{ a_n \}$ 收敛.
\end{proof}

\begin{problem}{MAblue}{7.1.7}
    证明:若级数 $\sum_{n=1}^\infty a_n^2$ 和 $\sum_{n=1}^\infty b_n^2$ 收敛,则级数 $\sum_{n=1}^\infty |a_nb_n|,\ \sum_{n=1}^\infty (a_n+b_n)^2$,以及 $\sum_{n=1}^\infty \frac {|a_n|} n$ 也收敛.
\end{problem}

\begin{proof}
    注意到
    \[
        \sum_{n=1}^\infty |a_nb_n| \leqslant \sum_{n=1}^\infty \frac {a_n^2+b_n^2} 2 = \frac 1 2 \sum_{n=1}^\infty a_n^2 + \frac 1 2 \sum_{n=1}^\infty b_n^2,
    \]
    且
    \[
        \sum_{n=1}^\infty (a_n+b_n)^2 \leqslant \sum_{n=1}^\infty 2(a_n^2+b_n^2) = 2\sum_{n=1}^\infty a_n^2 + 2\sum_{n=1}^\infty b_n^2.
    \]
    故二者均收敛. 令 $b_n = \frac 1 n$,即可得 $\sum_{n=1}^\infty \frac {|a_n|} n$ 收敛.
\end{proof}

\begin{problem}{MAblue}{7.1.12}
    研究下列级数的条件收敛性与绝对收敛性:
    \begin{multicols}{2}
        \begin{enumerate}[label={(\arabic*)}]
            \item $\displaystyle \sum_{n=1}^\infty (-1)^n \left( \frac{2n+100}{3n+1} \right)^n$;
            \item $\displaystyle \sum_{n=1}^\infty \frac{(-1)^{\frac {n(n-1)} 2}}{2^n}$;
            \item $\displaystyle \sum_{n=1}^\infty (-1)^n \frac{\sqrt n}{n+100}$;
            \item $\displaystyle \sum_{n=1}^\infty (-1)^{n-1} \sin \frac 1 n$;
            \item $\displaystyle \sum_{n=1}^\infty (-1)^{n-1} \frac {\ln n} n$;
            \item $\displaystyle \sum_{n=1}^\infty \frac{(-1)^{n-1}}{n^p}$;
            \item $\displaystyle \sum_{n=1}^\infty (-1)^n \left( e^{\frac 1 n} - 1 \right)$;
            \item $\displaystyle \sum_{n=1}^\infty (-1)^n \left( \frac 1 n - \ln \left( 1 + \frac 1 n \right) \right)$;
            \item $\displaystyle \sum_{n=1}^\infty (-1)^n \left( 1 - \cos \frac p n \right)$;
            \item $\displaystyle \sum_{n=1}^\infty (-1)^n \left( 1 - \cos \frac 1 n \right)^p$.
        \end{enumerate}
    \end{multicols}
\end{problem}

\begin{enumerate}
    \item[(8)]
    \begin{solution}
        由 $\ln\left( 1 + \frac 1 n \right) = \frac 1 n + O \left( \frac 1 {n^2} \right)$ 知 $\lim_{n \to \infty} \left( \frac 1 n - \ln \left( 1 + \frac 1 n \right) \right) = 0$. 故原级数条件收敛,亦绝对收敛.
    \end{solution}
\end{enumerate}

\begin{problem}{MAblue}{7.1.15}
    研究下列级数的敛散性:
    \begin{multicols}{2}
        \begin{enumerate}[label={(\arabic*)}]
            \item $\displaystyle \sum_{n=1}^\infty \frac {\sin nx} n$;
            \item $\displaystyle \sum_{n=2}^\infty \frac{\cos \frac {n\pi} 4}{\ln n}$;
            \item $\displaystyle \sum_{n=1}^\infty \frac{\sin n}{\sqrt n} \left( 1 + \frac 1 n \right)^n$;
            \item $\displaystyle \sum_{n=1}^\infty (-1)^n \frac{n-1}{n+1} \cdot \frac 1 {\sqrt[100]{n}}$.
        \end{enumerate}
    \end{multicols}
\end{problem}

\begin{enumerate}
    \item[(1)]
    \begin{solution}
        熟知
        \[
            \sum_{k=1}^n \sin kx = \frac{\cos \frac x 2 - \cos \left( nx + \frac x 2 \right)}{2 \sin \frac x 2}
        \]
        有界,而数列 $\left\{ \frac 1 n \right\}$ 单调递减趋于零,故由 Dirichlet 判别法知原级数收敛.
    \end{solution}
    \item[(2)]
    \begin{solution}
        取 $a_n = \cos \frac {n \pi} 4,\ b_n = \frac 1 {\ln n}$,由 Dirichlet 判别法易知原级数收敛.
    \end{solution}
    \item[(3)]
    \begin{solution}
        由 Dirichlet 判别法知 $\sum_{n=1}^{\infty} \frac{|\sin n|}{\sqrt n}$ 收敛. 进而由 Abel 判别法知原级数收敛.
    \end{solution}
    \item[(4)]
    \begin{solution}
        易知 $\lim_{n \to \infty} \frac{n-1}{n+1} \cdot \frac 1 {\sqrt[100] n} = 0$,则由 Leibniz 判别法知原级数收敛.
    \end{solution}
\end{enumerate}

\section{函数项级数}

\begin{problem}{MAblue}{7.2.2}
    确定下列函数项级数的收敛域.
    \begin{multicols}{2}
        \begin{enumerate}[label={(\arabic*)}]
            \item $\displaystyle \sum_{n=1}^\infty ne^{-nx}$;
            \item $\displaystyle \sum_{n=2}^\infty \frac {x^{n^2}} n$;
            \item $\displaystyle \sum_{n=1}^\infty \frac{(-1)^n}{2n-1} \left( \frac{1-x}{1+x} \right)^n$;
            \item $\displaystyle \sum_{n=1}^\infty \frac 1 {x^n} \sin \frac \pi {2^n}$;
            \item $\displaystyle \sum_{n=1}^\infty \frac{(x-3)^n}{n-3^n}$;
            \item $\displaystyle \sum_{n=1}^\infty n! \left( \frac x n \right)^n$;
            \item $\displaystyle \sum_{n=1}^\infty \frac{\cos nx}{e^{nx}}$;
            \item $\displaystyle \sum_{n=1}^\infty \frac{x^n}{1-x^n}$.
        \end{enumerate}
    \end{multicols}
\end{problem}

\begin{enumerate}
    \item[(6)]
    \begin{solution}
        由 Stirling 公式知 $n \to \infty$ 时有
        \[
            n! \left( \frac x n \right)^n \sim \sqrt{2 \pi n} \left( \frac n e \right)^n \left( \frac x n \right)^n = \sqrt{2 \pi n} \left( \frac x e \right)^n.
        \]
        故级数收敛域为 $(-e, e)$.
    \end{solution}
\end{enumerate}

\begin{problem}{MAblue}{7.2.10}
    递归定义 $[0, 1)$ 上的连续可微序列 $\{ f_n \}$ 如下:$f_1 = 1$,在 $(0, 1)$ 上有
    \[
        f_{n+1}'(x) = f_n(x)f_{n+1}(x),\ f_{n+1}(0) = 1.
    \]
    求证:对任意 $x \in [0, 1)$ 有 $\lim_{n \to \infty} f_n(x)$ 存在,并求出其极限函数.
\end{problem}

\begin{proof}
    考虑归纳证明. 由 $f_{n+1}'(x) = f_n(x) f_{n+1}(x)$ 及 $f_n(0) = 1$ 解得
    \[
        f_{n+1}(x) = \exp\left( \int_0^x f_n(t) \mathrm dt \right).
    \]
    \begin{enumerate}
        \item[(a)]
        当 $n = 1$ 时,$f_2(x) = \exp\left( \int_0^x f_1(t) \mathrm dt \right) = e^x$,则 $f_1(x) \leqslant f_2(x) \leqslant \frac 1 {1 - x}$.
        \item[(b)]
        若当 $n = k \ (\geqslant 2)$ 时,$f_{k-1}(x) \leqslant f_k(x) \leqslant \frac 1 {1 - x}$ 成立,则
        \[
            \begin{cases}
                f_{k+1}(x) = \exp\left( \int_0^x f_k(t) \mathrm dt \right) \leqslant \exp\left( \int_0^x \frac{\mathrm dt}{1 - t} \right) = \frac 1 {1 - x} \\
                \frac{f_{k+1}(x)}{f_k(x)} = \exp\left( \int_0^x \left( f_k(t) - f_{k-1}(t) \right) \mathrm dt \right) \geqslant e^0 = 1
            \end{cases}
        \]
        即 $f_k(x) \leqslant f_{k+1}(x) \leqslant \frac 1 {1 - x}$,故 $f_n(x)$ 单调有界,收敛至 $\frac 1 {1-x}$.
    \end{enumerate}
\end{proof}

\begin{problem}{MAblue}{7.2.11}
    证明 Dini 定理.
\end{problem}

\begin{center}
    \begin{minipage}{0.85\textwidth}
        \begin{theorem}{Dini 定理}{}
            若函数项级数 $\{ S_n(x) \}$ 在闭区间 $I$ 上逐点收敛到 $S(x)$,且通向 $u_n(x)$ 在区间 $I$ 上是连续且非负(或非正)的,那么 $S(x)$ 在 $I$ 上连续的充要条件是 此级数在 $I$ 上一致收敛.
        \end{theorem}
    \end{minipage}
\end{center}

\begin{proof}
    充分性的证明见定理 7.34. 对于必要性,令 $f_n(x) = S(x) - S_n(x)$,则 $f(x)$ 连续,故有极值,且该极值单调递减(或递增)趋于零,则 $\{ f_n(x) \}$ 在 $I$ 上一致收敛到零,也即函数项级数 $\{ S_n(x) \}$ 在 $I$ 上一致收敛.
\end{proof}

\section{幂级数与 Taylor 展式}

\begin{problem}{MAblue}{7.3.1}
    求下列幂级数的收敛半径.
    \begin{multicols}{2}
        \begin{enumerate}[label={(\arabic*)}]
            \item $\displaystyle \sum_{n=1}^\infty (-1)^{n+1} \frac{x^n}{n^2}$;
            \item $\displaystyle \sum_{n=1}^\infty \frac{(n!)^2}{(2n)!} x^n$;
            \item $\displaystyle \sum_{n=1}^\infty 2^nx^{2n}$;
            \item $\displaystyle \sum_{n=1}^\infty \frac{x^n}{a^n+b^n} \ (a > 0,\ b > 0)$;
            \item $\displaystyle \sum_{n=1}^\infty \frac{(x-2)^{2n-1}}{(2n-1)!}$;
            \item $\displaystyle \sum_{n=1}^\infty \frac {3^n+(-2)^n} n (x+1)^n$;
            \item $\displaystyle \sum_{n=1}^\infty \left( 1 + \frac 1 2 + \cdots + \frac 1 n \right) x^n$;
            \item $\displaystyle \sum_{n=1}^\infty \frac{x^{n^2}}{2^n}$.
        \end{enumerate}
    \end{multicols}
\end{problem}

\begin{enumerate}
    \item[(8)]
    \begin{solution}
        易知
        \[
            \lim_{n \to \infty} \sqrt[n]{\frac{x^{n^2}}{2^n}} = \frac {x^n} 2,
        \]
        则由 Cauchy 判别法知 $x < 1$ 时级数收敛,$x > 1$ 时级数发散,即 $R = 1$.
    \end{solution}
\end{enumerate}

\begin{problem}{MAblue}{7.3.3}
    求下列幂级数的收敛区域及其和函数.
    \begin{multicols}{2}
        \begin{enumerate}[label={(\arabic*)}]
            \item $\displaystyle \sum_{n=0}^\infty (-1)^n \frac{x^{2n+1}}{2n+1}$;
            \item $\displaystyle \sum_{n=0}^\infty (n+1)x^n$;
            \item $\displaystyle \sum_{n=1}^\infty n(n+1)x^{n-1}$;
            \item $\displaystyle \sum_{n=1}^\infty \frac{x^n}{n(n+1)}$;
            \item $\displaystyle \sum_{n=1}^\infty \frac{x^{2n-1}}{(2n-1)!!}$;
        \end{enumerate}
    \end{multicols}
\end{problem}

\begin{enumerate}
    \item[(5)]
    \begin{solution}
        由
        \[
            \lim_{n \to \infty} \frac{\frac{x^{2n+1}}{(2n+1)!!}}{\frac{x^{2n-1}}{(2n-1)!!}} = \lim_{n \to \infty} \frac{x^2}{2n+1} = 0
        \]
        知收敛域为 $\R$.
        
        设 $f(x) = \sum_{n=1}^{\infty} \frac{x^{2n-1}}{(2n-1)!!}$,则 $f'(x) = 1 + \sum_{n=1}^{\infty} \frac{x^{2n}}{(2n-1)!!} = 1 + xf(x)$. 解得
        \[
            f(x) = e^{\frac {x^2} 2} \left( \int e^{-\frac {x^2} 2} + C \right). \qedhere
        \]
    \end{solution}
\end{enumerate}

\begin{problem}{MAblue}{7.3.4}
    求下列级数的和.
    \begin{multicols}{2}
        \begin{enumerate}[label={(\arabic*)}]
            \item $\displaystyle \sum_{n=2}^\infty \frac 1 {(n^2-1)2^n}$;
            \item $\displaystyle \sum_{n=0}^\infty \frac{(-1)^n(n^2-n+1)}{2^n}$;
            \item $\displaystyle \sum_{n=0}^\infty \frac{(-1)^n}{3n+1}$;
            \item $\displaystyle \sum_{n=0}^\infty \frac{(n+1)^2}{n!}$;
        \end{enumerate}
    \end{multicols}
\end{problem}

\begin{enumerate}
    \item[(1)]
    \begin{solution}
        熟知 $\sum_{n=0}^{\infty} x^n = \frac 1 {1-x}$,则
        \[
            \sum_{n=1}^{\infty} \frac {x^n} n = \int_0^x \frac{\mathrm dt}{1-t} = -\ln(1-x).
        \]
        故
        \[
            \sum_{n=2}^{\infty} \frac 1 {(n^2-1) 2^n} = \frac 1 4 \sum_{n=2}^{\infty} \frac 1 {(n-1) 2^{n-1}} - \sum_{n=2}^{\infty} \frac 1 {(n+1) 2^{n+1}} = \frac 5 8 + \frac 3 4 \ln 2. \qedhere
        \]
    \end{solution}
    \item[(2)]
    \begin{solution}
        熟知 $\sum_{n=3}^{\infty} x^n = \frac{x^3}{1-x}$,则
        \[
            \sum_{n=3}^{\infty} n(n-1) x^{n-2} = \left( \frac{x^3}{1-x} \right)'' = \frac{2x^3}{(1-x)^3} + \frac{6x^2}{(1-x)^2} + \frac{6x}{1-x}.
        \]
        故
        \[
            \sum_{n=1}^{\infty} \frac{(-1)^n (n^2-n+1)}{2^n} = \frac 1 2 + \frac 1 4 \sum_{n=3}^{\infty} n(n-1) \left( - \frac 1 2 \right)^{n-2} + \sum_{n=1}^{\infty} \left( -\frac 1 2 \right)^n = -\frac{41}{27}. \qedhere
        \]
    \end{solution}
    \item[(3)]
    \begin{solution}
        熟知 $\frac 1 {1-x^3} = \sum_{n=0}^{\infty} x^{3n}$,则
        \[
            \sum_{n=0}^{\infty} \frac{x^{3n+1}}{3n+1} = \int_0^x \frac{\mathrm dt}{1-t^3} = -\frac 1 3 \ln(1-x) + \frac 1 6 \ln(1 + x + x^2) + \frac {\sqrt 3} 3 \arctan\left( \frac{2x + 1}{\sqrt 3} \right).
        \]
        故
        \[
            \sum_{n=0}^{\infty} \frac{(-1)^n}{3n+1} = - \sum_{n=0}^{\infty} \frac{(-1)^{3n+1}}{3n+1} = \frac{\sqrt 3}{18} \pi + \frac 1 3 \ln 2. \qedhere
        \]
    \end{solution}
    \item[(4)]
    \begin{solution}
        考虑以下引理.
        \begin{center}
            \begin{minipage}{0.85\textwidth}
                \begin{lemma}{}{}
                    对任意 $k \in \N_+$,有
                    \[
                        \sum_{n=0}^\infty \frac{n^k}{n!} = B_k e,
                    \]
                    其中 $B_k$ 为 Bell 数.
                    \tcblower
                    \begin{proof}
                        当 $k \geqslant 1$ 时有
                        \[
                            \sum_{n=0}^\infty \frac{n^k}{n!} = \sum_{n=1}^\infty \frac{n^{k-1}}{(n-1)!} = \sum_{n=0}^\infty \frac{(n+1)^{k-1}}{n!}.
                        \]
                        将二项式展开即得.
                    \end{proof}
                \end{lemma}
            \end{minipage}
        \end{center}
        \[
            \sum_{n=0}^{\infty} \frac{(n+1)^2}{n!} = \sum_{n=0}^{\infty} \frac{n^2}{n!} + 2 \sum_{n=0}^{\infty} \frac n {n!} + \sum_{n=0}^{\infty} \frac 1 {n!} = 2e + 2e + e = 5e. \qedhere
        \]
    \end{solution}
\end{enumerate}

\begin{problem}{MAblue}{7.3.7}
    方程 $y + \lambda \sin y = x \ (\lambda \neq -1)$ 在 $x = 0$ 附近确定了一个隐函数 $y(x)$,试求它的幂级数展开式的前四项.
\end{problem}

\begin{solution}
    不妨设 $y = \sum_{n=0}^{\infty} a_n x^n$. 令 $x = 0$,则有 $a_0 + \lambda \sin a_0 = 0$,而由于 $y(x)$ 是确定的,故只能有 $a_0 = 0$. 原式两边求导,得 $y' + \lambda y' \cos y = 1$,再令 $x = 0$,得 $a_1 + \lambda a_1 = 1$,也即 $a_1 = \frac 1 {1+\lambda}$. 依此类推,可求得
    \[
        y = \frac 1 {1 + \lambda} x + \frac{\lambda}{6 (1 + \lambda)^4} x^3. \qedhere
    \]
\end{solution}

\section{级数的应用}

\begin{problem}{MAblue}{7.4.6}
    证明:当 $n \to \infty$ 时,$\ln(n!) \sim \ln n^n$.
\end{problem}

\begin{proof}
    由 Stolz 定理知
    \begin{align*}
        \lim_{n \to \infty} \frac{\ln n!}{\ln n^n} &= \lim_{n \to \infty} \frac{\ln n! - \ln (n-1)!}{\ln n^n - \ln \left( (n-1)^{n-1} \right)} \\
        &= \lim_{n \to \infty} \frac{\ln n}{\ln n + (n-1) \ln\left( 1 + \frac 1 {n-1} \right)} \\
        &= \lim_{n \to \infty} \frac{\ln n}{\ln n + 1} = 1. \qedhere
    \end{align*}
\end{proof}
{\flushleft 或由 Stirling 公式立得.}

\section*{第 7 章综合习题}

\begin{problem}{MAblue}{7.0.3}
    设 $\{ a_n \}$ 是正的递增数列. 求证:级数 $\sum_{n=1}^\infty \left( \frac{a_{n+1}}{a_n} - 1 \right)$ 收敛的充分必要条件是 $\{ a_n \}$ 有界.
\end{problem}

\begin{proof}
    先证充分性. 不妨设 $a_n \leqslant M$,则
    \[
        S_n = \sum_{k=1}^n \frac{a_{k+1} - a_k}{a_k} < \sum_{k=1}^n \frac{a_{k+1} - a_k}{a_1} = \frac{a_{n+1}}{a_1} - 1 \leqslant \frac M {a_1} - 1.
    \]
    即 $\sum_{n=1}^{\infty} \left( \frac{a_{n+1}}{a_n} - 1 \right)$ 收敛.

    {\flushleft 再证必要性. 由级数收敛知,对任意 $0 < \varepsilon < \frac 1 2$,存在 $N \in \N_+$,使得当 $m > n \geqslant N$ 时有}
    \[
        1 - \frac{a_n}{a_{m}} = a_n \left( \frac 1 {a_n} - \frac 1 {a_m} \right) < \left| \sum_{k=n}^{m-1} a_{k+1} \left( \frac 1 {a_k} - \frac 1 {a_{k+1}} \right) \right| < \varepsilon.
    \]
    故对任意 $m > N$ 有 $a_m < \frac{a_N}{1 - \varepsilon} < 2a_N$. 取 $M = \max\left\{ a_1,\ a_2,\ \ldots,\ 2a_N \right\}$,则恒有 $a_n \leqslant M$,即 $\{ a_n \}$ 有界.
\end{proof}

\begin{problem}{MAblue}{7.0.4}
    设 $\alpha > 0$,$\{ a_n \}$ 是递增正数列. 求证级数 $\sum_{n=1}^\infty \frac{a_{n+1}-a_n}{a_{n+1}a_n^\alpha}$ 收敛.
\end{problem}

\begin{proof}
    \leavevmode
    \begin{enumerate}
        \item[(a)] 当 $\alpha \geqslant 1$ 时:
        \[
            S_n = \sum_{k=1}^n \left( 1 - \frac{a_k}{a_{k+1}} \right) \frac 1 {a_k^{\alpha}} \leqslant \sum_{k=1}^n \left( 1 - \frac{a_k^{\alpha}}{a_{k+1}^{\alpha}} \right) \frac 1 {a_k^{\alpha}} = \frac 1 {a_1^{\alpha}} - \frac 1 {a_{n+1}^{\alpha}} < \frac 1 {a_1^{\alpha}}.
        \]
        故原级数收敛.
        \item[(b)] 当 $0 < \alpha < 1$ 时,由 Lagrange 中值定理知,存在 $\xi \in (a_n, a_{n+1})$ 使得
        \[
            \frac{a_{n+1}^{\alpha} - a_n^{\alpha}}{a_{n+1} - a_n} = \alpha \xi^{\alpha - 1} > \alpha a_{n+1}^{\alpha - 1},
        \]
        即
        \[
            \frac{a_{n+1} - a_n}{a_{n+1} a_n^{\alpha}} < \frac 1 {\alpha} \left( \frac 1 {a_n^{\alpha}} - \frac 1 {a_{n+1}^{\alpha}} \right).
        \]
        故
        \[
            S_n < \frac 1 {\alpha a_1^{\alpha}}  - \frac 1 {\alpha a_{n+1}^{\alpha}} < \frac 1 {\alpha a_1^{\alpha}},
        \]
        即原级数收敛. \qedhere
    \end{enumerate}
\end{proof}

\begin{problem}{MAblue}{7.0.5}
    设 $\Phi(x)$ 是 $(0, +\infty)$ 上正的严格增函数,$\{ a_n \}, \{ b_n \}, \{ c_n \}$ 是三个非负数列,满足
    \[
        a_{n+1} \leqslant a_n - b_n \Phi(a_n) + c_na_n,\ \sum_{n=1}^\infty b_n = \infty,\ \sum_{n=1}^\infty c_n < \infty.
    \]
    求证 $\lim_{n \to \infty} a_n = 0$.
\end{problem}

\begin{proof}
    设 $\sum_{n=1}^\infty c_n = c$,则
    \[
        1 \leqslant \prod_{n=1}^{\infty} (1 + c_n) = \exp\left( \sum_{n=1}^{\infty} \ln(1 + c_n) \right) \leqslant \exp\left( \sum_{n=1}^{\infty} c_n \right) = e^c = C,
    \]
    则原不等式可化为
    \[
        \frac{a_{n+1}}{\prod_{k=1}^n (1+c_k)} - \frac{a_n}{\prod_{k=1}^{n-1} (1+c_k)} \leqslant - \frac{b_n \Phi(a_n)}{\prod_{k=1}^n (1+c_n)} < 0.
    \]
    不妨设 $d_n = \frac{a_n}{\prod_{k=1}^{n-1} (1+c_k)} \geqslant 0$,则 $d_n$ 单调递减有下界,即 $\{ d_n \}$ 收敛. 若 $\lim_{n \to \infty} d_n = d > 0$,则
    \[
        d_{n+1} - d_n \leqslant -b_n \frac{\Phi(d)}{M_1} \quad \Rightarrow \quad d_{n+1} \leqslant d_1 - \frac{\Phi(d)}{M_1} \sum_{k=1}^{n} b_k,
    \]
    则 $\lim_{n \to \infty} d_n = -\infty$,矛盾,故 $\lim_{n \to \infty} d_n = 0$,进而有 $\lim_{n \to \infty} a_n = 0$.
\end{proof}

\begin{problem}{MAblue}{7.0.6}
    设 $\{ a_n \}$ 是正数列使得 $\sum_{n=1}^\infty a_n$ 收敛. 求证:
    \[
        \sum_{n=1}^\infty \frac n {a_1 + a_2 + \cdots + a_n} \leqslant 2 \sum_{n=1}^\infty \frac 1 {a_n},
    \]
    而且上式右端的系数 $2$ 是最佳的.
\end{problem}

\begin{proof}
    由 Cauchy-Schwarz 不等式知
    \[
        (a_1 + a_2 + \cdots + a_n)\left( \frac 1 {a_1} + \frac 4 {a_2} + \cdots + \frac{n^2}{a_n} \right) \geqslant \frac {n^2(n+1)^2} 4,
    \]
    即
    \[
        \frac n {a_1 + a_2 + \cdots + a_n} \leqslant \frac 4 {n(n+1)^2} \sum_{k=1}^n \frac{k^2}{a_k} \leqslant  2 \left( \frac 1 {n^2} - \frac 1 {(n+1)^2} \right) \sum_{k=1}^n \frac{k^2}{a_k}.
    \]
    对 $n$ 求和得
    \begin{align*}
        \sum_{n=1}^N \frac n {a_1 + a_2 + \cdots + a_n} &\leqslant \sum_{n=1}^N 2 \left( \frac 1 {n^2} - \frac 1 {(n+1)^2} \right) \sum_{k=1}^n \frac{k^2}{a_k} \\
        &= 2 \sum_{k=1}^N \frac{k^2}{a_k} \sum_{n=k}^N \left( \frac 1 {n^2} - \frac 1 {(n+1)^2} \right) \\
        &= 2 \sum_{k=1}^N \left( \frac 1 {a_k} - \frac{k^2}{(N+1)^2 a_k} \right).
    \end{align*}
    令 $N \to \infty$,则有
    \[
        \sum_{n=1}^{\infty} \frac n {a_1 + a_2 + \cdots + a_n} \leqslant 2 \sum_{n=1}^{\infty} \frac 1 {a_n}. \qedhere
    \]
\end{proof}
{\flushleft 此题亦可用数学归纳法证明,见\href{https://math.stackexchange.com/posts/108598/revisions}{此解}. 事实上,我们有更强的结论,见\href{https://www.komal.hu/feladat?a=feladat&f=A709&l=en}{此解},其亦确定了最佳系数.}

\begin{problem}{MAblue}{7.0.7}
    设 $\{ a_n \}$ 是一个严格单调递增实数列,且对任意正整数 $n$ 有 $a_n \leqslant n^2 \ln n$,求证:级数 $\sum_{n=1}^\infty \frac 1 {a_{n+1}-a_n}$ 发散.
\end{problem}

\begin{proof}
    不妨令 $a_1 = 0$,则由上题结论知
    \[
        \sum_{n=1}^{\infty} \frac 1 {a_{n+1} - a_n} \geqslant \frac 1 2 \sum_{n=1}^{\infty} \frac n {a_{n+1} - a_1} \geqslant \frac n {(n+1)^2 \ln(n+1)}
    \]
    故 $\sum_{n=1}^{\infty} \frac 1 {a_{n+1} - a_n}$ 发散.
\end{proof}

\begin{problem}{MAblue}{7.0.9}
    设函数列 $\{ f_n(x) \},\ n = 1, 2, \cdots$ 在区间 $[0, 1]$ 上由
    \[
        f_0(x) = 1,\ f_n(x) = \sqrt{xf_{n-1}(x)}
    \]
    定义,证明当 $n \to \infty$ 时,函数列在 $[0, 1]$ 上一致收敛到一个连续函数.
\end{problem}

\begin{proof}
    设 $f_n(x) = x^{a_n} \ (n \in \N)$,则 $f_{n+1}(x) = x^{a_{n+1}} = x^{\frac {a_n + 1} 2}$,也即 $a_0 = 0,\ a_{n+1} = \frac {a_n + 1} 2$,则易证 $\lim_{n \to \infty} a_n = 1$,即 $\lim_{n \to \infty} f_n(x) = x$.
\end{proof}

\begin{problem}{MAblue}{7.0.11}
    设 $f_0(x)$ 是区间 $[0, a]$ 上连续函数,证明按照下列公式
    \[
        f_n(x) = \int_0^x f_{n-1}(u) \mathrm du
    \]
    定义的函数列 $\{ f_n(x) \}$ 在区间 $[0, a]$ 上一致收敛于 $0$.
\end{problem}

\begin{proof}
    设 $|f_0(x)| \leqslant M$,则可归纳证得 $|f_n(x)| \leqslant \frac{Ma^n}{n!}$,显然一致收敛于 $0$.
\end{proof}

{\flushleft 事实上有}
\begin{align*}
    \int_0^x \frac{(x-t)^n}{n!} f_0(t) \mathrm dt &= \left. \frac{(x-t)^n}{n!} f_1(t) \right|_0^x + \int_0^x \frac{(x-t)^{n-1}}{(n-1)!} f_1(t) \mathrm dt \\
    &= \int_0^x \frac{(x-t)^{n-1}}{(n-1)!} f_1(t) \mathrm dt \\
    &= \cdots \\
    &= \int_0^x f_n(t) \mathrm dt \\
    &= f_{n+1}(x).
\end{align*}.