\chapter{常微分方程初步}

\section{一阶微分方程}

\begin{problem}{MAblue}{6.1.4}
    求下列线性方程和贝努利方程的解.
    \begin{multicols}{2}
        \begin{enumerate}
            \item[(1)]
            $(1+x^2)y' - 2xy = (1+x^2)^2$;
            \item[(2)]
            $y' + \frac {1-2x} x = 1$;
            \item[(3)]
            $y' = \frac y {x+y^3}$;
            \item[(4)]
            $y' + \frac y x = y^2 \ln x$;
            \item[(5)]
            $y' = y\tan x + y^2\cos x$;
            \item[(6)]
            $y - y'\cos x = y^2(1-\sin x)\cos x$.
        \end{enumerate}
    \end{multicols}
\end{problem}

\begin{enumerate}
    \item[(3)]
    \begin{solution}
        整理原式得
        \[
            \frac{\mathrm dx}{\mathrm dy} = \frac x y + y^2.
        \]
        若 $y \neq 0$,作代换 $u = \frac x y$,等式化为
        \[
            u + y \frac{\mathrm du}{\mathrm dy} = u + y^2 \ \left( u = \frac x y,\ y \neq 0 \right),
        \]
        则可进一步解得 $x = \frac 1 2 y^3 + Cy$. 验证可知 $y = 0$ 亦为方程的解.
    \end{solution}
\end{enumerate}

\begin{problem}{MAblue}{6.1.6}
    求解下列微分方程.
    \begin{multicols}{2}
        \begin{enumerate}
            \item[(1)]
            $y' + x = \sqrt{x^2 + y}$;
            \item[(2)]
            $y' = \cos(x-y)$;
            \item[(3)]
            $y' - e^{x-y} + e^x = 0$;
            \item[(4)]
            $y'\sin y + x\cos y + x = 0$.
        \end{enumerate}
    \end{multicols}
\end{problem}

\begin{enumerate}
    \item[(1)]
    \begin{solution}
        不妨设 $u = \frac y {x^2}$,则原式化为
        \[
            2xu + x^2 \frac{\mathrm du}{\mathrm dx} = x \left( \sqrt{1+u} - 1 \right),
        \]
        进一步整理得
        \[
            - \frac{\mathrm du}{2u - \sqrt{1+u} + 1} = \frac {\mathrm dx} x.
        \]
        再设 $v = \sqrt{1+u}$,化为
        \[
            - \frac{2v \mathrm dv}{(v-1)(2v+1)} = \frac {\mathrm dx} x,
        \]
        两边积分,得
        \[
            -\frac 1 3 \left( 2 \ln(v-1) + \ln(2v+1) \right) = \ln|x| + C,
        \]
        即
        \[
            (v-1)^2(2v+1) = \frac C {x^3}.
        \]
        则 $y$ 可解.
    \end{solution}
    或设 $u^2 = x^2 + y$,则此题亦可解.
\end{enumerate}

\begin{problem}{MAblue}{6.1.7}
    试用常数变易法导出贝努利方程的通解.
\end{problem}

\begin{solution}
    由定义有
    \begin{align*}
        \frac{\mathrm dy}{\mathrm dx} + P(x) y &= Q(x) y^n \ (n \not\in \{ 0, 1 \} ) \\
        \frac{\mathrm du}{\mathrm dx} + (1-n) P(x) u &= (1-n) Q(x) \ (u = y^{1-n}).
    \end{align*}
    设 $u = C(x) e^{(n-1) \int P(x) \mathrm dx}$,则
    \begin{align*}
        \frac{\mathrm d C(x)}{\mathrm dx} &= (1-n)Q(x)e^{(1-n) \int P(x) \mathrm dx} \\
        C(x) &= \int (1-n)Q(x)e^{(1-n) \int P(x) \mathrm dx} \mathrm dx + C.
    \end{align*}
    故 $u = e^{(n-1) \int P(x) \mathrm dx} \left( \int (1-n)Q(x)e^{(1-n) \int P(x) \mathrm dx} \mathrm dx + C \right)$,进而可得 $y$.
\end{solution}

\begin{problem}{MAblue}{6.1.13}
    求下列二阶方程满足初始条件的特解.
    \begin{enumerate}
        \item[(1)]
        $y'' = \frac {y'} x + \frac{x^2}{y'},\ y(1) = 1,\ y'(1) = 0$;
        \item[(2)]
        $y^3y'' = -1,\ y(1) = 1,\ y'(1) = 0$.
    \end{enumerate}
\end{problem}

\begin{enumerate}
    \item[(2)]
    \begin{solution}
        设 $p = y'$,则原式化为
        \[
            y^3 p \frac{\mathrm dp}{\mathrm dy} = -1,
        \]
        也即 $p \mathrm dp = - \frac{\mathrm dy}{y^3}$,易解得 $y = \sqrt{Cx^2 - \frac 1 C}$.
    \end{solution}
\end{enumerate}

\section{二阶线性微分方程}

\begin{problem}{MAblue}{6.2.1}
    在下列方程中,已知方程的一个特解 $y_1$,试求它们的通解.
    \begin{enumerate}[label={(\arabic*)}]
        \item $y'' + \frac 2 x y' + y = 0,\ y_1 = \frac {\sin x} x$;
        \item $y'' \sin^2x = 2y,\ y_1 = \cot x$;
        \item $(1-x^2)y'' - 2xy' + 2y = 0,\ y_1 = x$.
    \end{enumerate}
\end{problem}

\begin{enumerate}
    \item[(3)]
    \begin{solution}
        由题设知 $y_1(x) = x,\ p(x) = \frac{2x}{x^2 - 1}$,则
        \[
            y_2(x) = x \int \frac 1 {x^2} e^{\int_{x_0}^x \frac{2t}{t^2 - 1} \mathrm dt} \mathrm dx = \frac{x^2 + 1}{x_0^2 - 1}.
        \]
        故通解 $y_0(x) = c_1x + c_2 (x^2 + 1)$.
    \end{solution}
\end{enumerate}

\begin{problem}{MAblue}{6.2.2}
    先用观察法求下列齐次方程的一个非零特解,然后求方程的通解.
    \begin{enumerate}[label={(\arabic*)}]
        \item $x^2y'' - 2xy' + 2y = 0,\ x \neq 0$;
        \item $xy'' - (1+x)y' + y = 0,\ x \neq 0$.
    \end{enumerate}
\end{problem}

\begin{enumerate}
    \item[(2)]
    \begin{solution}
        注意到 $y = x + 1$ 为一特解. 原式变形得
        \begin{align*}
            y - y' &= x(y' - y'') \\
            \frac {\mathrm dx} x &= \frac{\mathrm d(y-y')}{y-y'},
        \end{align*}
        也即 $y - y' = Cx$,解得 $y = C(x+1)$,此即通解.
    \end{solution}
\end{enumerate}

\begin{problem}{MAblue}{6.2.3}
    已知方程 $(1+x^2)y'' + 2xy' - 6x^2 - 2 = 0$ 的一个特解 $y_1 = x^2$,试求该方程满足初始条件 $y(-1) = 0,\ y'(-1) = 0$ 的特解.
\end{problem}

\begin{solution}
    原方程对应的齐次方程为
    \[
        y'' + \frac{2x}{1 + x^2} y' = 0.
    \]
    解得其通解为 $y = c_1 \arctan x + c_2$,则原方程通解为 $y = x^2 + c_1 \arctan x + c_2$. 令 $y(-1) = 0,\ y'(-1) = 0$ 得特解 $y_2 = x^2 + 4\arctan x + \pi-1$.
\end{solution}

\begin{problem}{MAblue}{6.2.5}
    求下列常系数非齐次方程的一个特解.
    \begin{enumerate}[label={(\arabic*)}]
        \item $y'' + y = 2\sin\frac x 2$;
        \item $y'' - 6y' + 9y = (x+1)e^{2x}$.
    \end{enumerate}
\end{problem}

\begin{enumerate}
    \item[(2)]
    \begin{solution}
        易得对应齐次方程的通解为 $y = c_1 e^{3x} + c_2 x e^{3x}$. 则
        \[
            y_0(x) = \int_{x_0}^x \frac{e^{3t} \cdot xe^{3x} - te^{3t} \cdot e^{3x}}{W(t)} f(t) \mathrm dt
        \]
        进而可得原方程通解.
    \end{solution}
\end{enumerate}

\begin{problem}{MAblue}{6.2.9}
    求下列方程的通解.
    \begin{enumerate}[label={(\arabic*)}]
        \item $x''' + 3x'' + 3x' + x = 0$;
        \item $x''' - 2x'' + x' - 2x = 0$;
        \item $x^{(4)} - 8x'' + 18x = 0$;
        \item $x^{(4)} + 2x'' + x = 0$.
    \end{enumerate}
\end{problem}

\begin{enumerate}
    \item[(4)]
    \begin{solution}
        设 $x = e^{\lambda t}$,则得到特征方程 $\lambda^4 + 2\lambda^2 + 1 = 0$. 进而解得 $\lambda = \pm i$,故通解为 $x = c_1 \cos x + c_2 \sin x$.
    \end{solution}
    
\end{enumerate}