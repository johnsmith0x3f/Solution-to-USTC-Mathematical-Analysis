\chapter{单变量函数的积分学}

\section{积分}

\begin{problem}{MAblue}{5.1.2}
    证明:Dirchlet 函数在任意区间 $[a, b]$ 上不可积.(因此有界的函数未必可积.)
\end{problem}

\begin{proof}
    考虑其 Riemann 和. 对任意 $x_{i-1} < x_i \ (i = 1, 2, \ldots, n)$,总能找到一组 $\xi_i \in (x_{i-1}, x_i)$ 使得 $f(\xi_i) \equiv 0$ 或 $f(\xi_i) \equiv 1$,故 $\lim_{\Vert T \Vert \to 0} S_n(T)$ 不存在.
\end{proof}

\begin{problem}{MAblue}{5.1.3}
    举例说明,一个函数的绝对值在 $[a, b]$ 上可积,不能保证该函数在 $[a, b]$ 上可积.(提示:适当地修改 Dirchlet 函数可得出这样的例子. 比较习题 2.1 中第 5 题.)
\end{problem}

\begin{proof}
    取 $f(x) = 2D(x) - 1$,其中 $D(x)$ 为 Dirchlet 函数.
\end{proof}

\begin{problem}{MAblue}{5.1.9}
    \begin{enumerate}
        \item[(1)]
        设函数 $f(x)$ 在 $[a, b]$ 上连续,而 $g(x)$ 在 $[a, b]$ 上可积且是非负(或非正)的,证明:存在 $\xi \in[a, b]$,使得
        \[
            \int_a^b f(x)g(x) \mathrm dx = f(\xi) \int_a^b g(x) \mathrm dx.
        \]
        \item[(2)]
        举例说明,(1) 中对于函数 $g(x)$ 的假设是必不可少的.(提示:在 $[-1, 1]$ 上,取 $f(x) = g(x) = x$.)
    \end{enumerate}
\end{problem}

\begin{enumerate}
    \item[(1)]
    \begin{proof}
        由题设知存在常数 $m, M$ 满足 $m \leqslant f(x) \leqslant M$. 由 $g(x) \geqslant 0$ 知
        \[
            m \int_a^b g(x) \mathrm dx \leqslant \int_a^b f(x)g(x) \mathrm dx \leqslant M \int_a^b g(x) \mathrm dx
        \]
        即存在 $\lambda \in [m, M]$ 使得 $\int_a^b f(x)g(x) \mathrm dx = \lambda \int_a^b g(x) \mathrm dx$. 又 $f(x)$ 连续,则存在 $\xi \in [a, b]$ 使得 $f(\xi) = \lambda$.
    \end{proof}
    \item[(2)]
    \begin{proof}
        令 $f(x) = g(x) = x$,则 $\int_{-1}^1 f(x)g(x) \mathrm dx = 1$,而 $\int_{-1}^1 g(x) \mathrm dx = 0$. 故不存在满足条件的 $\xi$.
    \end{proof}
\end{enumerate}

\begin{problem}{MAblue}{5.1.13}
    设函数 $f(x)$ 处处连续. 记 $F(x) = \int_0^x xf(t) \mathrm dt$,求 $F'(x)$.
\end{problem}

\begin{solution}
    设 $\varphi(x) = \int_0^x f(t) \mathrm dt$,则
    \[
        F'(x) = \left(\int_0^x xf(t) \mathrm dt \right)' = \left( x \varphi(x) \right)' = xf(x) + \varphi(x).
    \]
\end{solution}

\begin{problem}{MAblue}{5.1.18}
    求下列极限.
    \begin{enumerate}
        \item[(1)]
        $\displaystyle \lim_{x \to 0} \left( \frac 1 {x^4} \int_0^x \sin t^3 \mathrm dt \right)$;
        \item[(2)]
        $\displaystyle \lim_{x \to 0} \frac 1 {\sin^3 x} \int_0^{\tan x} \arcsin t^2 \mathrm dt$;
        \item[(3)]
        $\displaystyle \lim_{n \to \infty} \left( \frac 1 {\sqrt{n^2}} + \frac 1 {\sqrt{n^2-1^2}} + \cdots + \frac 1 {\sqrt{n^2-(n-1)^2}} \right)$;
        \item[(4)]
        $\displaystyle \lim_{n \to \infty} \frac{1^p + 2^p + \cdots + n^p}{n^{p+1}}$,其中 $p$ 是正常数.
    \end{enumerate}
\end{problem}

\begin{enumerate}
    \item[(1)]
    \begin{solution}
        由积分中值定理知,存在 $\xi \in (0, x)$,使得 $\int_0^x \sin t^3 \mathrm dt = x \sin \xi^3$. 故
        \[
            \lim_{x \to 0} \left( \frac 1 {x^4} \int_0^x \sin t^3 \mathrm dt \right) = \lim_{x \to 0} \frac{\sin \xi^3}{x^3} = 1.
        \]
    \end{solution}
    \item[(2)]
    \begin{solution}
        由积分中值定理知,存在 $\xi \in (0, \tan x)$,使得 $\int_0^{\tan x} \arcsin t^2 \mathrm dt = \tan x \arcsin \xi^2$. 故
        \[
            \lim_{x \to 0} \frac 1 {\sin^3 x} \int_0^{\tan x} \arcsin t^2 \mathrm dt = \lim_{x \to 0} \frac{\arcsin \xi^2}{\sin^2 x \cos x} = 1.
        \]
    \end{solution}
    \item[(3)]
    \begin{solution}
        \[
            \lim_{n \to \infty} \left( \sum_{k=0}^{n-1} \frac 1 {\sqrt{n^2 - k^2}} \right) = \int_0^1 \frac 1 {\sqrt{1 - x^2}} = \frac \pi 2.
        \]
    \end{solution}
    \item[(4)]
    \begin{solution}
        \[
            \lim_{n \to \infty} \frac{1^p + 2^p + \cdots + n^p}{n^{p+1}} = \int_0^1 x^p = \frac 1 {1+p}.
        \]
    \end{solution}
\end{enumerate}

\begin{problem}{MAblue}{5.1.19}
    求下列极限.
    \begin{enumerate}
        \item[(1)]
        $\displaystyle \lim_{n \to \infty} \int_a^b e^{-nx^2} \mathrm dx$,其中 $a, b$ 为常数,且 $0 < a < b$;
        \item[(2)]
        $\displaystyle \lim_{n \to \infty} \int_0^1 \frac{x^n}{1+x} \mathrm dx$;
        \item[(3)]
        $\displaystyle \lim_{n \to \infty} \int_n^{n+a} \frac {\sin x} x \mathrm dx$,其中 $a$ 为正常数;
    \end{enumerate}
\end{problem}

\begin{enumerate}
    \item[(2)]
    \begin{solution}
        易知 $0 < \frac{x^n}{1+x} < x^n$. 又
        \[
            \lim_{n \to \infty} \int_0^1 x^n \mathrm dx = \lim_{n \to \infty} \left. \frac{x^{n+1}}{n+1} \right|_0^1 = \lim_{n \to \infty} \frac 1 {n+1} = 0.
        \]
        故由夹逼原理知
        \[
            \lim_{n \to \infty} \int_0^1 \frac{x^n}{1+x} \mathrm dx = 0.
        \]
    \end{solution}
\end{enumerate}

\begin{problem}{MAblue}{5.1.22}
    计算下面的积分.
    \begin{multicols}{2}
        \begin{enumerate}
            \item[(1)]
            $\displaystyle \int_0^{2\pi} |\cos x| \mathrm dx$;
            \item[(2)]
            $\displaystyle \int_{-3}^4 [x] \mathrm dx$;
            \item[(3)]
            $\displaystyle \int_{-1}^1 \cos x \ln \frac{1+x}{1-x} \mathrm dx$;
            \item[(4)]
            $\displaystyle \int_{-\frac \pi 2}^{\frac \pi 2} \frac{\cos^3 x}{1+e^x} \mathrm dx$;
            \item[(5)]
            $\displaystyle \int_0^{\ln 2} \sqrt{1-e^{-2x}} \mathrm dx$;
            \item[(6)]
            $\displaystyle \int_0^1 x\arcsin x \mathrm dx$;
            \item[(7)]
            $\displaystyle \int_0^1 x^3e^x \mathrm dx$;
            \item[(8)]
            $\displaystyle \int_0^a \frac{\mathrm dx}{x+\sqrt{a^2-x^2}}$;
            \item[(9)]
            $\displaystyle \int_0^{\frac \pi 4} \sqrt{\tan x} \mathrm dx$;
            \item[(10)]
            $\displaystyle \int_0^{\frac \pi 2} \frac{\mathrm dx}{a^2\sin^2 x + b^2\cos^2 x}$;
            \item[(11)]
            $\displaystyle \int_{-1}^1 x^4 \sqrt{1-x^2} \mathrm dx$;
            \item[(12)]
            $\displaystyle \int_0^{2\pi} \sin^6 x \mathrm dx$;
            \item[(13)]
            $\displaystyle \int_{-1}^1 e^{|x|} \arctan e^x \mathrm dx$;
            \item[(14)]
            $\displaystyle \int_0^\pi \frac{\sec^2x}{2+\tan^2x} \mathrm dx$;
        \end{enumerate}
    \end{multicols}
\end{problem}

\begin{enumerate}
    \item[(14)]
    \begin{solution}
        由
        \[
            \int \frac{\sec^2x}{2+\tan^2x} \mathrm dx = \int \frac{\mathrm d(\tan x)}{2+\tan^2x} = \frac 1 {\sqrt 2} \arctan \frac{\tan x}{\sqrt 2}
        \]
        知
        {\small
            \[
                \int_0^\pi \frac{\sec^2x}{2+\tan^2x} \mathrm dx = \lim_{a \to \frac \pi 2 - 0} \left. \frac 1 {\sqrt 2} \arctan \frac{\tan x}{\sqrt 2} \right|_0^a + \lim_{a \to \frac \pi 2 + 0} \left. \frac 1 {\sqrt 2} \arctan \frac{\tan x}{\sqrt 2} \right|_a^\pi = \frac \pi {\sqrt 2}.
            \]
        }
    \end{solution}
\end{enumerate}

\begin{problem}{MAblue}{5.1.23}
    设 $f(x)$ 在 $[0, 1]$ 上连续,证明:
    \[
        \int_0^\pi x f(\sin x) \mathrm dx = \pi \int_0^{\frac\pi 2} f(\sin x) \mathrm dx,
    \]
    并用这一结果计算 $\int_0^\pi \frac{x\sin x}{1+\cos^2x} \mathrm dx$.
\end{problem}

\begin{proof}
    易知
    \begin{align*}
        \int_0^{\pi} x f(\sin x) \mathrm dx &= \int_0^{\frac \pi 2} x f(\sin x) \mathrm dx + \int_{\frac \pi 2}^{\pi} x f(\sin x) \mathrm dx \\
        &= \int_0^{\frac \pi 2} x f(\sin x) \mathrm dx + \int_0^{\frac \pi 2} (\pi - x) f( \sin(\pi - x) ) \mathrm dx \\
        &= \pi \int_0^{\frac \pi 2} f(\sin x) \mathrm dx. \qedhere
    \end{align*}
\end{proof}
{\flushleft 故}
\[
    \int_0^{\pi} \frac{x \sin x}{1 + \cos^2 x} \mathrm dx = \pi \int_0^{\frac \pi 2} \frac{\sin x}{1 + \cos^2 x} = -\arctan(\cos x) \bigg|_0^{\frac \pi 2} = \frac \pi 4.
\]

\begin{problem}{MAblue}{5.1.24}
    证明:
    \[
        \frac 1 6 < \int_0^1 \sin x^2 \mathrm dx < \frac 1 3.
    \]
\end{problem}

\begin{proof}
    易知当 $0 \leqslant x \leqslant \frac \pi 2$ 时有 $\frac {x^2} 2 < \frac 2 \pi x^2 \leqslant \sin x^2 \leqslant x^2$,在 $[0, 1]$ 上积分即得.
\end{proof}

\begin{problem}{MAblue}{5.1.27}
    \begin{enumerate}
        \item[(1)]
        设 $f(x)$ 是 $[0, 1]$ 上单调递减的连续函数,证明:对任意 $\alpha \in (0, 1)$,有
        \[
            \int_0^\alpha f(x) \mathrm dx \geqslant \alpha \int_0^1 f(x) \mathrm dx.
        \]
        \item[(2)]
        若仅假设 $f(x)$ 在 $[0, 1]$ 上单调递减,证明同样的结论.
    \end{enumerate}
\end{problem}

\begin{proof}
    由 $f(x)$ 单调递减有
    \[
        (1 - \alpha) \int_0^\alpha f(x) \geqslant \alpha (1 - \alpha) f(\alpha) \geqslant \alpha \int_\alpha^1 f(x)
    \]
    整理后即得.
\end{proof}

\phantomsection\label{item:28}
\begin{problem}{MAblue}{5.1.28}
    设函数 $f(x)$ 在 $[a, b]$ 上可微,且对任意 $x \in [a, b]$ 有 $|f'(x)| \leqslant M$.
    \begin{enumerate}
        \item[(1)]
        若 $f(a) = 0$,证明:
        \[
            \int_a^b |f(x)| \mathrm dx \leqslant \frac M 2 (b-a)^2.
        \]
        \item[(2)]
        若 $f(a) = f(b) = 0$,证明:
        \[
            \int_a^b |f(x)| \mathrm dx \leqslant \frac M 4 (b-a)^2.
        \]
    \end{enumerate}
    (提示:通过对积分的上限求导能得出 (1) 的一个证明,即考虑函数 $G(t) = \int_a^t |f(x)| \mathrm dx - \frac M 2 (t-a)^2,\ a \leqslant t \leqslant b$.)
\end{problem}
\begin{enumerate}
    \item[(1)]
    \begin{proof}
        依提示设 $G(t)$,则 $G(a) = 0,\ G(b) = \int_a^b |f(x)| \mathrm dx - \frac M 2 (b-a)^2$,而
        \[
            G'(t) = |f(t)| - M(t - a).
        \]
        又由 Lagrange 中值定理知存在 $\xi \in (a, t)$ 使得
        \[
            \frac{|f(t)|}{t - a} = \frac{|f(t)| - |f(a)|}{t - a} = |f'(\xi)| \leqslant M.
        \]
        故 $G'(t) \leqslant 0$,也即 $G(b) \leqslant G(a) = 0$. 原命题得证.
    \end{proof}
    \item[(2)]
    \begin{proof}
        由 (1) 有
        \[
            \int_a^b |f(x)| \mathrm dx = \int_a^{\frac {a+b} 2} |f(x)| \mathrm dx + \int_{\frac {a+b} 2}^b |f(x)| \mathrm dx \leqslant 2 \cdot \frac M 2 \cdot (\frac {b-a} 2)^2 = \frac M 4 (b-a)^2. \qedhere
        \]
    \end{proof}
    
\end{enumerate}

\section{函数的可积性}

\begin{problem}{MAblue}{5.2.3}
    证明:如果函数 $f(x)$ 在区间 $[a, b]$ 上可积,则 $|f(x)|$ 在区间 $[a, b]$ 上也可积,而且有
    \[
        \left| \int_a^b f(x) \mathrm dx \right| \leqslant \int_a^b |f(x)| \mathrm dx.
    \]
\end{problem}

\begin{proof}
    由题设知对任意划分,有
    \[
        \lim_{\Vert T \Vert \to 0} \sum_{i=1}^n \omega_i \Delta x_i = 0,
    \]
    设 $\omega'_i$ 为 $|f(x)|$ 在 $[x_{i-1}, x_i]$ 上的振幅,则
    \[
        0 < \omega'_i = |f(\xi_i)|_{\max} - |f(\xi_i)|_{\min} < |f(\xi_i)_{\max} - f(\xi_i)_{\min}| = \omega_i.
    \]
    故由夹逼原理知 $\lim_{\Vert T \Vert \to 0} \sum_{i=1}^n \omega'_i \Delta x_i = 0$,即 $|f(x)|$ 可积. 故
    \[
        -\int_a^b |f(x)| \mathrm dx \leqslant \int_a^b f(x) \mathrm dx \leqslant \int_a^b |f(x)| \mathrm dx.
    \]
    也即
    \[
        \left| \int_a^b f(x) \mathrm dx \right| \leqslant \int_a^b |f(x)| \mathrm dx. \qedhere
    \]
\end{proof}

\section{积分的应用}

\begin{problem}{MAblue}{5.3.4}
    求证:以 $R$ 为半径,高为 $h$ 的球缺的体积为 $\pi h^2 \left( R - \frac h 3 \right)$.
\end{problem}

\begin{proof}
    易知
    \[
        V = \int_0^h \pi \left( R^2 - (R-h+x)^2 \right) \mathrm dx = \pi h^2 \left( R - \frac h 3 \right). \qedhere
    \]
\end{proof}

\section{广义积分}

\begin{problem}{MAblue}{5.4.1}
    判断下列广义积分是否收敛,并求出收敛的广义积分的值. 以下 $n$ 均为自然数.
    \begin{multicols}{2}
        \begin{enumerate}
            \item[(1)]
            $\displaystyle \int_0^{+\infty} xe^{-x^2} \mathrm dx$;
            \item[(2)]
            $\displaystyle \int_0^{+\infty} x \sin x \mathrm dx$;
            \item[(3)]
            $\displaystyle \int_2^{+\infty} \frac {\ln x} x \mathrm dx$;
            \item[(4)]
            $\displaystyle \int_1^{+\infty} \frac {\arcsin x} x \mathrm dx$;
            \item[(5)]
            $\displaystyle \int_0^{+\infty} e^{-x} \sin x \mathrm dx$;
            \item[(6)]
            $\displaystyle \int_{-\infty}^{+\infty} \frac{\mathrm dx}{x^2+2x+2}$;
            \item[(7)]
            $\displaystyle \int_0^1 \ln x \mathrm dx$;
            \item[(8)]
            $\displaystyle \int_{-1}^1 \frac{\mathrm dx}{\sqrt{1-x^2}}$;
            \item[(9)]
            $\displaystyle \int_0^1 \frac{x\ln x}{(1-x^2)^{3/2}} \mathrm dx$;
            \item[(10)]
            $\displaystyle \int_0^1 \ln \frac 1 {1-x^2} \mathrm dx$;
            \item[(11)]
            $\displaystyle \int_0^{+\infty} x^ne^{-x} \mathrm dx$;
            \item[(12)]
            $\displaystyle \int_0^1 (\ln x)^n \mathrm dx$.
        \end{enumerate}
    \end{multicols}
\end{problem}

\begin{enumerate}
    \item[(12)]
    \begin{solution}
        \begin{align*}
            \int_0^1 (\ln x)^n \mathrm dx &= (-1)^{n+1} \int_0^{+\infty} t^n e^{-t} \mathrm dt \ (x = e^{-t}) \\
            &= (-1)^{n+1}\Gamma(n+1). \qedhere
        \end{align*}
    \end{solution}
\end{enumerate}

\begin{problem}{MAblue}{5.4.2}
    广义积分 $\int_{-\infty}^{+\infty} f(x) \mathrm dx$ 的柯西主值定义为
    \[
        P.V. \int_{-\infty}^{+\infty} f(x) \mathrm dx = \lim_{b \to +\infty} \int_{-b}^b f(x) \mathrm dx.
    \]
    显然,若广义积分 $\int_{-\infty}^{+\infty} f(x) \mathrm dx$ 收敛,则其主值也收敛,但反过来不一定成立. 研究下列广义积分主值的收敛性.
    \begin{multicols}{2}
        \begin{enumerate}
            \item[(1)]
            $\displaystyle P.V. \int_{-\infty}^{+\infty} \frac x {1+x^2} \mathrm dx$;
            \item[(2)]
            $\displaystyle P.V. \int_{-\infty}^{+\infty} \frac{|x|}{1+x^2} \mathrm dx$.
        \end{enumerate}
    \end{multicols}
\end{problem}

\begin{enumerate}
    \item[(1)]
    \begin{solution}
        由 $\frac x {1+x^2}$ 的奇性知 $\int_{-b}^b \frac x {1+x^2} \mathrm dx \equiv 0$. 然而
        \[
            \int_{-\infty}^{+\infty} \frac x {1+x^2} \mathrm dx = \int_{-\infty}^0 \frac x {1+x^2} \mathrm dx + \int_0^{+\infty} \frac x {1+x^2} \mathrm dx
        \]
        等号右侧两积分均发散,故原积分发散.
    \end{solution}
\end{enumerate}

\begin{problem}{MAblue}{5.4.3}
    若函数 $f(x)$ 在 $(a, +\infty)$ 上连续,并且以 $a$ 为瑕点,则广义积分 $\int_a^{+\infty} f(x) \mathrm dx$ 定义为
    \[
        \int_a^{+\infty} f(x) \mathrm dx = \int_a^b f(x) \mathrm dx + \int_b^{+\infty} f(x) \mathrm dx,
    \]
    这是本节讲的两类广义积分的组合,其中 $b > a$ 是任一个实数. 当上面两个广义积分都收敛时,我们称 $\int_a^{+\infty} f(x) \mathrm dx$ 收敛;否则称为发散.
    \begin{enumerate}
        \item[(1)]
        证明 $\displaystyle \int_1^{+\infty} \frac{\mathrm dx}{x\sqrt{x-1}}$ 收敛,并求其值;
        \item[(2)]
        证明,对任意实数 $\alpha$,$\displaystyle \int_0^{+\infty} \frac{\mathrm dx}{x^\alpha}$ 发散.
    \end{enumerate}
\end{problem}

\begin{enumerate}
    \item[(2)]
    \begin{proof}
        \leavevmode
        \begin{enumerate}
            \item[(a)]
            当 $\alpha = 1$ 时
            \[
                \int_0^{+\infty} \frac {\mathrm dx} x = \ln x \bigg|_0^{+\infty}
            \]
            发散.
            \item[(b)]
            当 $\alpha \neq 1$ 时
            \[
                \int_0^{+\infty} \frac 1 {x^\alpha} = \left. \frac{x^{1-\alpha}}{1-\alpha} \right|_0^{+\infty}
            \]
            亦发散.
        \end{enumerate}
        综上,无论 $\alpha$ 取何值,$\int_0^{+\infty} \frac{\mathrm dx}{x^{\alpha}} $ 必发散.
    \end{proof}
\end{enumerate}

\section*{第 5 章综合习题}
\addcontentsline{toc}{section}{第 5 章综合习题}

\begin{problem}{MAblue}{5.0.1}
    设 $m, n$ 为正整数,证明
    \begin{enumerate}
        \item[(1)]
        $\displaystyle \int_0^{2\pi} \sin mx \cdot \cos nx \mathrm dx = 0$;
        \item[(2)]
        $\displaystyle \int_0^{2\pi} \sin mx \cdot \sin nx \mathrm dx = \int_0^{2\pi} \cos mx \cdot \cos nx \mathrm dx = \begin{cases}
            \pi, & m = n; \\
            0, & m \neq n.
        \end{cases}$
    \end{enumerate}
\end{problem}

\begin{enumerate}
    \item[(1)]
    \begin{proof}
        设 $t = 2 \pi - x$,则
        \[
            \int_0^{2 \pi} \sin mx \cdot \cos nx \mathrm dx = \int_0^\pi \sin mx \cdot \cos nx \mathrm dx - \int_0^\pi \sin mt \cdot \cos nt \mathrm dt = 0. \qedhere
        \]
    \end{proof}
    \item[(2)]
    \begin{proof}
        易知
        \begin{align*}
            \int_0^{2\pi} \sin mx \cdot \sin nx &= \frac 1 2 \int_0^{2\pi} \left( \cos( mx - nx ) - \cos( mx + nx ) \right) \mathrm dx \\
            &= \frac 1 2 \int_0^{2\pi} \cos \left( (m - n)x \right) \mathrm dx - \frac 1 2 \int_0^{2\pi} \cos \left( (m + n)x \right) \mathrm dx,
        \end{align*}
        则结论显然.
    \end{proof}
\end{enumerate}

\begin{problem}{MAblue}{5.0.2}
    设 $m, n$ 为正整数,记
    \[
        B(m, n) = \int_0^1 x^m(1-x)^n \mathrm dx.
    \]
    证明:(1) $B(m, n) = B(n, m)$;\quad(2) $B(m, n) = \frac{m!n!}{(m+n+1)!}$.
\end{problem}

\begin{proof}
    (1) 是显然的,下证 (2). 易知有
    \[
        B(m+1, n-1) = B(m, n-1) - B(m, n),
    \]
    则
    {\small
        \[
            B(m, n) = \int_0^1 x^m (1-x)^n \mathrm dx = 0 + \frac n {m+1} \int_0^1 x^{m+1} (1-x)^{n-1} \mathrm dx = \frac{n (B(m, n-1) - B(m, n))}{m+1},
        \]
    }
    {\flushleft 即 $B(m, n) = \frac n {m+n+1} B(m, n-1)$. 又知 $B(0, 0) = 1$,则}
    \[
        B(m, n) = \frac{n!}{(m+n+1)(m+n) \cdots (m+2)} \cdot \frac{m!}{(m+1)!} = \frac{n!m!}{(m+n+1)!},
    \]
    在 $m, n \geqslant 0$ 时成立.
\end{proof}

\begin{problem}{MAblue}{5.0.3}
    计算下列积分:
    \begin{enumerate}
        \item[(a)]
        $\displaystyle \int_{\frac 1 2}^2 \left( 1 + x - \frac 1 x \right)e^{x+\frac 1 x} \mathrm dx$;
        \item[(b)]
        $\displaystyle \int_0^{n\pi} x|\sin x| \mathrm dx$,其中 $n$ 为自然数;
        \item[(c)]
        设 $\displaystyle f(x) = \int_x^{x+2\pi} \left( 1 + e^{\sin t} - e^{-\sin t} \right) \mathrm dt + \frac 1 {1+x} \int_0^1 f(t) \mathrm dt$,求 $\displaystyle \int_0^1 f(x) \mathrm dx$;
    \end{enumerate}
\end{problem}

\begin{enumerate}
    \item[(c)]
    \begin{solution}
        设 $\varphi(x) = e^{\sin x} - e^{-\sin x}$,则有 $\varphi(x) + \varphi(x+\pi) = 0$,则
        \[
            \int_x^{x+2\pi} (1 + \varphi(t)) \mathrm dt = \int_x^{x+\pi} (1 + \varphi(t) + 1 + \varphi(t+\pi)) \mathrm dt = 2\pi.
        \]
        故 $\int_0^1 f(x) \mathrm dx = (1 + x)(f(x) - 2\pi)$.
    \end{solution}
\end{enumerate}

\begin{problem}{MAblue}{5.0.4}
    证明:
    \[
        \frac 1 {2n+2} < \int_0^{\frac \pi 4} \tan^n x \mathrm dx < \frac 1 {2n} \ (n = 1, 2, \cdots).
    \]
\end{problem}

\begin{proof}
    设 $x = \arctan t$,则
    \[
        \int_0^{\frac \pi 4} \tan^n x \mathrm dx = \int_0^1 \frac{t^n}{1+t^2} \mathrm dt,
    \]
    而
    \[
        \frac 1 {2n+2} = \int_0^1 \frac {t^n} 2 \mathrm dt < \int_0^1 \frac{t^n}{1+t^2} \mathrm dt < \int_0^1 \frac{t^n}{2t} \mathrm dt = \frac 1 {2n},
    \]
    则命题得证.
\end{proof}

\begin{problem}{MAblue}{5.0.10}
    设 $f(x)$ 处处连续,$f(0) = 0$,且 $f'(0)$ 存在. 记 $F(x) = \int_0^x f(xy) \mathrm dy$,证明 $F(x)$ 处处可导,并求出 $F'(x)$.
\end{problem}

\begin{proof}
    显然 $F(0) = 0$. 当 $x \neq 0$ 时,设 $\varphi(x) = \int f(x) \mathrm dx$,则
    \[
        F(x) = \left. \frac {\varphi(xy)} x \right|_{y=0}^1 = \frac {\varphi(x) - \varphi(0)} x.
    \]
    此时显然有 $F'(x) = \frac{xf(x) - \varphi(x) + \varphi(0)}{x^2}$. 又
    \[
        F'(0) = \lim_{x \to 0} \frac{F(x) - F(0)}{x - 0} = \lim_{x \to 0} \frac {f(x)} x = f'(0),
    \]
    则 $F(x)$ 处处可导.
\end{proof}

\begin{problem}{MAblue}{5.0.11}
    \begin{enumerate}
        \item[(1)]
        设
        \[
            f(x) = \begin{cases}
                e^{-x^2}, & |x| \leqslant 1; \\
                1, & |x| > 1.
            \end{cases}
        \]
        记 $F(x) = \int_0^x f(t) \mathrm dt$. 试研究 $F(x)$ 在哪些点可导.
        \item[(2)]
        设 $f(x) = \int_0^x \cos \frac 1 t \mathrm dt$,求证 $f'_+(0) = 0$.
    \end{enumerate}
\end{problem}

\begin{enumerate}
    \item[(2)]
    \begin{proof}
        \begin{align*}
            f'_+(0) &= \lim_{x \to 0_+} \frac{\int_{1/x}^{+\infty} \frac{\cos u}{u^2} \mathrm du - 0}{x - 0} \ \left( u = \frac 1 t \right) \\
            &= \lim_{x \to 0_+} \frac 1 x \left( \left. \frac{\sin u}{u^2} \right|_{\frac 1 x}^{+\infty} \right) + \frac 2 x \int_{\frac 1 x}^{+\infty} \frac{\sin u}{u^3} \mathrm du \\
            &= 2 \lim_{x \to 0_+} \frac{\int_0^x t \cos \frac 1 t}{x} = 2 \lim_{x \to 0_+} x \cos \frac 1 x = 0. \qedhere
        \end{align*}
    \end{proof}
\end{enumerate}

\begin{problem}{MAblue}{5.0.12}
    设函数 $f$ 处处连续. 证明
    \[
        \lim_{h \to 0} \frac 1 h \int_a^b \left( f(x+h)-f(x) \right) \mathrm dx = f(b) - f(a).
    \]
\end{problem}

\begin{proof}
    设 $\varphi(x) = \int f(x) \mathrm dx$,则
    \begin{align*}
        LHS &= \lim_{h \to 0} \frac 1 h \left( \varphi(b+h) - \varphi(b) - \varphi(a+h) + \varphi(a) \right) \\
        &= \lim_{h \to 0} \frac {\varphi(b+h) - \varphi(b)} h - \lim_{h \to 0} \frac {\varphi(a+h) - \varphi(a)} h \\
        &= f(b) - f(a). \qedhere
    \end{align*}
\end{proof}

\begin{problem}{MAblue}{5.0.13}
    设函数 $f(x)$ 在 $[a, b]$ 上连续可微. 证明:
    \[
        \lim_{\lambda \to \infty} \int_a^b f(x) \sin \lambda x \mathrm dx = 0.
    \]
    (提示:分部积分.)
\end{problem}

\begin{proof}
    \[
        \lim_{\lambda \to \infty} \int_a^b f(x) \sin(\lambda x) \mathrm dx = \lim_{\lambda \to \infty} \left( \left. \frac {f(x) \cos(\lambda x)} \lambda \right|_a^b - \frac 1 \lambda \int_a^b f'(x) \cos(\lambda x) \mathrm dx \right) = 0.
    \]
\end{proof}

\begin{problem}{MAblue}{5.0.14}
    证明:
    \[
        \lim_{x \to +\infty} \frac 1 x \int_0^x |\sin t| \mathrm dt = \frac 2 \pi.
    \]
\end{problem}

\begin{proof}
    \begin{align*}
        \lim_{x \to +\infty} \frac 1 x \int_0^{x} |\sin t| \mathrm dt &= \lim_{x \to +\infty} \frac 1 x \int_0^{k\pi} |\sin t| \mathrm dt + \lim_{x \to +\infty} \frac 1 x \int_{k\pi}^x |\sin t| \mathrm dt \ \left( k = \left\lfloor \frac x \pi \right\rfloor \right) \\
        &= \lim_{x \to \infty} \frac {2k} x + 0 = \frac 2 \pi. \qedhere
    \end{align*}
\end{proof}

\begin{problem}{MAblue}{5.0.16}
    设 $f(x)$ 是 $[a, b]$ 上的连续函数,且对任意 $x \in [a, b]$ 有 $f(x) \geqslant 0$. 记 $f(x)$ 在该区间上的最大值为 $M$,证明:
    \[
        \lim_{n \to \infty} \left( \int_a^b f^n(x) \mathrm dx \right)^{\frac 1 n} = M.
    \]
\end{problem}

\begin{proof}
    设 $f(x_0) = M$,对任意 $\varepsilon > 0$,存在 $\delta$ 使得 $f(x) > M - \varepsilon \ \left(x \in U(x_0, \delta) \right)$. 不妨设
    \[
        g_\delta(x) =
        \begin{cases}
            M - \varepsilon, & x \in U(x_0, \delta); \\
            0, & x \not\in U(x_0, \delta),
        \end{cases}
    \]
    则 $0 < g(x) < f(x) \leqslant M$. 故有
    \[
        M - \varepsilon = \lim_{n \to \infty} \left( \int_a^b g_\delta^n(x) \mathrm dx \right)^{\frac 1 n} \leqslant \lim_{n \to \infty} \left( \int_a^b f^n(x) \mathrm dx \right)^{\frac 1 n} \leqslant \lim_{n \to \infty} \left( \int_a^b M \mathrm dx \right)^{\frac 1 n} = M,
    \]
    也即 $\lim_{n \to \infty} \left( \int_a^b f^n(x) \mathrm dx \right)^{\frac 1 n} = M$.
\end{proof}

\begin{problem}{MAblue}{5.0.18}
    证明柯西积分不等式.
\end{problem}

\begin{center}
    \begin{minipage}{0.85\textwidth}
        \begin{theorem}{Cauchy–Schwarz Inequality}{}
            设 $f(x), g(x) \in C[a, b]$,则有
            \[
                \left( \int_a^b f^2(x) \mathrm dx \right) \left( \int_a^b g^2(x) \mathrm dx \right) \geqslant \left( \int_a^b f(x)g(x) \mathrm dx \right)^2.
            \]
            当且仅当 $g(x) \equiv 0$ 或存在实数 $\lambda$ 使得 $f(x) = \lambda g(x)$ 时,等号成立.
        \end{theorem}
    \end{minipage}
\end{center}

\begin{proof}
    若 $\int_a^b g^2(x) \mathrm dx = 0$,则 $g(x) \equiv 0$,显然成立;否则对任意实数 $\lambda$,有
    \[
        0 \leqslant \int_a^b \left( f(x) - \lambda g(x) \right)^2 \mathrm dx = \int_a^b f^2(x) \mathrm dx - 2\lambda \int_a^b f(x)g(x) \mathrm dx + \lambda^2 \int_a^b g^2(x) \mathrm dx.
    \]
    此为关于 $\lambda$ 的一元二次不等式,故有
    \[
        \Delta = 4 \left( \int_a^b f(x)g(x) \mathrm dx \right)^2 - 4 \left( \int_a^b f^2(x) \mathrm dx \right) \left( \int_a^b g^2(x) \mathrm dx \right) \leqslant 0.
    \]
    整理后即得.
\end{proof}

\begin{problem}{MAblue}{5.0.19}
    设 $f(x)$ 在 $[0, 1]$ 上有连续的导数,证明:对任意 $a \in [0, 1]$,有
    \[
        |f(a)| \leqslant \int_0^1 |f(x)| \mathrm dx + \int_0^1 |f'(x)| \mathrm dx.
    \]
\end{problem}

\begin{proof}
    设 $|f(x_0)| = |f(x)|_{\min}$,则
    \[
        |f(a)| = \left| f(x_0) + \int_{x_0}^a f'(x) \mathrm dx \right| \leqslant |f(x_0)| + \left| \int_{x_0}^a f'(x) \mathrm dx \right| \leqslant \int_0^1 |f(x)| \mathrm dx + \int_0^1 |f'(x)| \mathrm dx. \qedhere
    \]
\end{proof}

\begin{problem}{MAblue}{5.0.21}
    设 $f(x)$ 在区间 $[0, 1]$ 上连续可微,且 $|f'(x)| \leqslant M$,证明
    \[
        \left| \int_0^1 f(x) \mathrm dx = \frac 1 n \sum_{k=1}^n f \left( \frac k n \right) \right| \leqslant \frac M {2n}.
    \]
\end{problem}

\begin{proof}
    易知
    {\small
        \[
            \left| \int_0^1 f(x) \mathrm dx - \frac 1 n \sum_{k=1}^n f\left( \frac k n \right) \right| \leqslant \sum_{k=1}^n \left| \int_{\frac {k-1} n}^{\frac k n} f(x) \mathrm dx - \frac 1 n f\left( \frac k n \right) \right| = \sum_{k=1}^n \left| \int_{\frac {k-1} n}^{\frac k n} \left( f(x) - f\left( \frac k n \right) \right) \mathrm dx \right|,
        \]
    }
    {\flushleft 又由 \hyperref[item:28]{\textbf{5.1.28}} 知}
    \[
        \sum_{k=1}^n \left| \int_{\frac {k-1} n}^{\frac k n} \left( f(x) - f\left( \frac k n \right) \right) \mathrm dx \right| \leqslant \sum_{k=1}^n \frac M 2 (\frac k n - \frac {k-1} n)^2 = \frac M {2n},
    \]
    则命题得证.
\end{proof}

\begin{problem}{MAblue}{5.0.22}
    设 $f : \R \to (0, +\infty)$ 是一个可微函数,且对任意实数 $x, y$ 满足
    \[
        |f'(x)-f'(y)| \leqslant |x-y|.
    \]
    求证:对任意实数 $x$,有
    \[
        \left( f'(x) \right)^2 < 2f(x).
    \]
\end{problem}

\begin{proof}
    若 $f'(x) = 0$,则结论显然. 若 $f'(x) > 0$,设 $x_0 = x - f'(x)$,则有
    \[
        f(x) = \int_{x_0}^x f'(t) \mathrm dt + f(x_0) > \int_{x_0}^x f'(t) \mathrm dt \geqslant \int_{x_0}^x (f'(x) - (x-t)) \mathrm dt = \frac 1 2 (f'(x))^2.
    \]
    同理可证 $f'(x) < 0$ 的情况.
\end{proof}
{\flushleft 类似地,我们可以证明若 $|f'(x) - f'(y)| \leqslant L|x-y|$,则有 $(f'(x))^2 \leqslant 2L f(x)$.}