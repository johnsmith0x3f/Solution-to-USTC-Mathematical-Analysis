% !TEX TS-program = xelatex
% !TEX encoding = UTF-8 Unicode

% This is a simple template for a LaTeX document using the "article" class.
% See "book", "report", "letter" for other types of document.

\documentclass[10pt, oneside, fontset=adobe]{ctexbook} % use larger type; default would be 10pt

\usepackage[utf8]{inputenc} % set input encoding (not needed with XeLaTeX)

%%% Examples of Article customizations
% These packages are optional, depending whether you want the features they provide.
% See the LaTeX Companion or other references for full information.

%%% PAGE DIMENSIONS
\usepackage{geometry} % to change the page dimensions
\geometry{a4paper} % or letterpaper (US) or a5paper or....
% \geometry{margin=2in} % for example, change the margins to 2 inches all round
% \geometry{landscape} % set up the page for landscape
%   read geometry.pdf for detailed page layout information

\usepackage{graphicx} % support the \includegraphics command and options

% \usepackage[parfill]{parskip} % Activate to begin paragraphs with an empty line rather than an indent

%%% PACKAGES
\usepackage{booktabs} % for much better looking tables
\usepackage{array} % for better arrays (eg matrices) in maths
\usepackage{paralist} % very flexible & customisable lists (eg. enumerate/itemize, etc.)
\usepackage{verbatim} % adds environment for commenting out blocks of text & for better verbatim
\usepackage{subfig} % make it possible to include more than one captioned figure/table in a single float
% These packages are all incorporated in the memoir class to one degree or another...

%%% HEADERS & FOOTERS
\usepackage{fancyhdr} % This should be set AFTER setting up the page geometry
\pagestyle{fancy} % options: empty , plain , fancy
\setlength{\headheight}{14pt}
\renewcommand{\headrulewidth}{0pt} % customise the layout...
\lhead{}\chead{}\rhead{}
\lfoot{}\cfoot{\thepage}\rfoot{}

%%% SECTION TITLE APPEARANCE
% \usepackage{sectsty}
% \allsectionsfont{\rmfamily\mdseries\upshape} % (See the fntguide.pdf for font help)
% (This matches ConTeXt defaults)

%%% ToC (table of contents) APPEARANCE
\usepackage[nottoc, notlof, notlot]{tocbibind} % Put the bibliography in the ToC
\usepackage[titles, subfigure]{tocloft} % Alter the style of the Table of Contents
\renewcommand{\cftsecfont}{\rmfamily\mdseries\upshape}
\renewcommand{\cftsecpagefont}{\rmfamily\mdseries\upshape} % No bold!

\usepackage[dvipsnames]{xcolor}
\usepackage[most]{tcolorbox}
\definecolor{lightblue}{RGB}{70,130,196}
\definecolor{MAblue}{RGB}{34,53,130}
\definecolor{MAgreen}{RGB}{18,135,108}
\definecolor{MApurple}{RGB}{151,67,155}

\newtcbtheorem{theorem}{Theorem}{
    coltitle=white, fonttitle=\bfseries,
    colback=red!5, colframe=red!75!black,
    fontupper=\kaishu,
}{}

\newtcbtheorem{lemma}{Lemma}{
    coltitle=white, fonttitle=\bfseries,
    colback=SeaGreen!10!CornflowerBlue!10, colframe=RoyalPurple!55!Aquamarine!100!,
    fontupper=\kaishu,
}{}

\newtcolorbox{problem}[3][]{
    title={#3}, coltitle=white, fonttitle=\bfseries,
    colback=#2!5!white, colframe=#2!75!black,
    #1,
}

\usepackage[colorlinks=true, allcolors=lightblue]{hyperref}

\usepackage{enumitem}
\usepackage{multicol}

\usepackage{amsmath}
\usepackage{amssymb}
\usepackage{amsthm}

\newtheorem*{mnote}{注}
\newtheorem*{mmark}{注意}
\newtheorem*{think}{思考}

\newenvironment{solution}{\begin{proof}[解]}{\end{proof}}

\newcommand{\N}{\mathbb N}
\newcommand{\Z}{\mathbb Z}
\newcommand{\Q}{\mathbb Q}
\newcommand{\R}{\mathbb R}
\newcommand{\bs}{\boldsymbol}
\newcommand{\ovec}{\overrightarrow}

%%% END Article customizations

%%% The "real" document content comes below...

\title{《数学分析讲义》习题答案}
\author{johnsmith0x3f}
\date{\today} % Activate to display a given date or no date (if empty),
              % otherwise the current date is printed

\begin{document}

\maketitle

\frontmatter

\tableofcontents

\mainmatter

\part*{数学分析讲义 \ 第一册}
\addcontentsline{toc}{part}{数学分析讲义 \ 第一册}

\chapter{极限}

\section{实数}

\begin{problem}{MAblue}{1.1.1}
    设 $a$ 是有理数,$b$ 是无理数. 求证:$a + b$ 和 $a - b$ 都是无理数;当 $a \neq 0$ 时,$ab$ 和 $\frac b a$ 也是无理数.
\end{problem}

\begin{proof}
    以加法为例. 考虑反证法. 若 $a + b = c \in \Q$,则 $b = c - a \in \Q$,矛盾,故 $a + b$ 是无理数.
\end{proof}

\begin{problem}{MAblue}{1.1.2}
    求证:两个不同的有理数之间有无理数.
\end{problem}

\begin{proof}
    对有理数 $a, b \ (a < b)$,取 $c = a + \frac {\sqrt 2} 2 (b-a) \in (a, b)$ 即可证得.
\end{proof}

\begin{think}
    如何证明任意两实数间均有无理数?
\end{think}

\begin{problem}{MAblue}{1.1.3}
    求证:$\sqrt 2, \sqrt 3$ 以及 $\sqrt 2 + \sqrt 3$ 都是无理数.
\end{problem}

\begin{proof}
    考虑以下引理:
    \begin{center}
        \begin{minipage}{0.85\textwidth}
            \begin{lemma}{}{}
                对任意正整数 $n$,$\sqrt n$ 是有理数当且仅当 $n$ 是完全平方数.
                \tcblower
                \begin{proof}
                    充分性是显然的,下证必要性. 设
                    \[
                        \sqrt n = \frac a b,
                    \]
                    其中 $a, b$ 是互质的正整数. 则
                    \[
                        nb^2 = a^2.
                    \]
                    考虑 $n$ 的某个质因子 $p$,有 $p \mid a^2$,进而 $p \mid a$. 又 $a, b$ 互质,故 $p \nmid b$. 这说明 $n$ 的唯一分解中所有质因子的次数都是偶数,从而 $n$ 是完全平方数.
                \end{proof}
            \end{lemma}
        \end{minipage}
    \end{center}
    由引理知,$\sqrt 2, \sqrt 3, \sqrt 6$ 均为无理数. 则若 $\sqrt 2 + \sqrt 3$ 是有理数,可得 $(\sqrt 2 + \sqrt 3)^2 = 5 + 2 \sqrt 6$ 亦为有理数,这与 $\sqrt 6$ 是无理数矛盾,故 $\sqrt 2 + \sqrt 3$ 是无理数.
\end{proof}

\begin{problem}{MAblue}{1.1.6}
    设实数 $a_1, a_2, \ldots, a_n$ 有相同的符号,且都大于 $-1$,证明:
    \[
        (1+a_1)(1+a_2) \cdots (1+a_n) \geqslant 1 + a_1 + a_2 + \cdots + a_n.
    \]
\end{problem}

\begin{proof}
    设 $x_n = \prod_{i=1}^n (1 + a_i) - (1 + \sum_{i=1}^n a_i)$,则
    \[
        x_{n+1} - x_n = a_{n+1} \left( \prod_{i=1}^n (1 + a_i) - 1 \right).
    \]
    若 $a_i \geqslant 0$,显然有 $x_{n+1} \geqslant x_n$;若 $a_i < 0$,此时 $(1+a_1)(1+a_2)\cdots(1+a_n) < 1$,亦有 $x_{n+1} > x_n$,故 $\{ x_n \}$ 为递增数列. 可得 $x_n \geqslant x_1 = 0$.
\end{proof}

事实上,此即 \textbf{Bernoulli 不等式}的一般形式.
\begin{center}
    \begin{minipage}{0.85\textwidth}
        \begin{theorem}{Bernoulli 不等式}{}
            若 $a_1, a_2, \cdots, a_n$ 同号,且都大于 $-1$,则有
            \[
                (1 + a_1)(1 + a_2) \cdots (1 + a_n) \geqslant 1 + a_1 + a_2 + \cdots + a_n.
            \]
            特别地,当 $a_1 = a_2 = \cdots = a_n = x$ 时,有
            \[
                (1 + x)^n \geqslant 1 + nx,
            \]
            当且仅当 $x = 0$ 时,等号成立.
        \end{theorem}
    \end{minipage}
\end{center}

\begin{mmark}
    Bernoulli 不等式的一般形式仅需其中 $n-1$ 项为零即可取等.
\end{mmark}

\begin{problem}{MAblue}{1.1.7}
    设 $a, b$ 是实数,且 $|a| < 1,\ |b| < 1$,证明:
    \[
        \left| \frac{a+b}{1+ab} \right| < 1.
    \]
\end{problem}

\begin{proof}
    注意到 $(a-1)(b-1) > 0$ 及 $(a+1)(b+1) > 0$,即 $1+ab > \pm(a+b)$. 故
    \[
        |1+ab| = 1+ab > |a+b| \quad \Rightarrow \quad \left| \frac{a+b}{1+ab} \right| < 1. \qedhere
    \]
\end{proof}

\section{数列极限}

\begin{problem}{MAblue}{1.2.15}
    求下列极限:
    \begin{enumerate}
        \item[(1)]
        $\displaystyle \lim_{n \to \infty} \left( \frac 1 {(n+1)^2} + \frac 1 {(n+2)^2} + \cdots + \frac 1 {(2n)^2} \right)$;
        \item[(2)]
        $\displaystyle \lim_{n \to \infty} \left( (n+1)^k - n^k \right)$,其中 $0 < k < 1$;
        \item[(3)]
        $\displaystyle \lim_{n \to \infty} \left( \sqrt 2 \cdot \sqrt[4] 2 \cdots \sqrt[2^n] 2 \right)$;
        \item[(4)]
        $\displaystyle \lim_{n \to \infty} \sqrt[n]{n^2 - n + 2}$;
        \item[(5)]
        $\displaystyle \lim_{n \to \infty} \sqrt[n]{\cos^2 1 + \cos^2 2 + \cdots + \cos^2 n}$.
    \end{enumerate}
\end{problem}

\begin{enumerate}
    \item[(1)]
    \begin{solution}
        注意到
        \[
            0 < \sum_{i=n+1}^{2n} \frac 1 {i^2} < \sum_{i=n+1}^{2n} \frac 1 {(i-1)i} = \frac 1 n - \frac 1 {2n} = \frac 1 {2n}.
        \]
        故由夹逼原理知,$\lim_{n \to \infty} \left( \frac 1 {(n+1)^2} + \frac 1 {(n+2)^2} + \cdots + \frac 1 {(2n)^2} \right) = 0$.
    \end{solution}
    \item[(2)]
    \begin{solution}
        注意到
        \[
            0 < (n+1)^k - n^k = n^k \left( \left( 1 + \frac 1 n \right)^k - 1 \right) < n^k \left( 1 + \frac 1 n - 1 \right) = n^{k-1}.
        \]
        故由夹逼原理知,$\lim_{n \to \infty} \left( (n+1)^k - n^k \right) = 0$.
    \end{solution}
    \item[(3)]
    \begin{solution}
        易知 $\sqrt 2 \cdot \sqrt[4] 2 \cdots \sqrt[2^n] 2 = \frac 2 {\sqrt[2^n] 2}$,故 $\lim_{n \to \infty} \sqrt 2 \cdot \sqrt[4] 2 \cdots \sqrt[2^n] 2 = 2$.
    \end{solution}
    \item[(4)]
    \begin{solution}
        注意到
        \[
            \sqrt[n] n < \sqrt[n]{n^2 - n + 2} < \sqrt[n]{n^2}.
        \]
        故由夹逼原理知,$\lim_{n \to \infty} \sqrt[n]{n^2 - n + 2} = 1$.
    \end{solution}
    \item[(5)]
    \begin{solution}
        注意到 $\cos^2 3 + \cos^2 4 > \cos^2 \frac {3\pi} 4 + \cos^2 \frac {5\pi} 4 = 1$,则当 $n > 3$ 时,有
        \[
            1 < \sqrt[n]{\cos^2 1 + \cos^2 2 + \cdots + \cos^2 n} < \sqrt[n] n.
        \]
        故由夹逼原理知,$\lim_{n \to \infty} \sqrt[n]{\cos^2 1 + \cos^2 2 + \cdots + \cos^2 n} = 1$.
    \end{solution}
\end{enumerate}

\begin{problem}{MAblue}{1.2.16}
    设 $a_1, a_2, \ldots, a_m$ 为 $m$ 个正数,证明:
    \[
        \lim_{n \to \infty} \sqrt[n]{a_1^n + a_2^n + \cdots + a_m^n} = \max(a_1, a_2, \ldots, a_m).
    \]
\end{problem}

\begin{proof}
    不妨设 $A = \max( a_1,\ a_2,\ \ldots,\ a_m)$,则
    \[
        A \leqslant \sqrt[n]{a_1^n+a_2^n+\cdots+a_m^n} \leqslant A \sqrt[n] m
    \]
    故由夹逼原理知,$\lim_{n \to \infty} \sqrt[n]{a_1^n+a_2^n+\cdots+a_m^n} = A$.
\end{proof}

\begin{problem}{MAblue}{1.2.17}
    证明下列数列收敛:
    \begin{enumerate}
        \item[(1)]
        $a_n = \left( 1 - \frac 1 2 \right) \left( 1 - \frac 1 {2^2} \right) \cdots \left( 1 - \frac 1 {2^n} \right)$;
        \item[(2)]
        $a_n = \frac 1 {3+1} + \frac 1 {3^2 + 1} + \cdots + \frac 1 {3^n+1}$;
        \item[(3)]
        $a_n = \alpha_0 + \alpha_1 q + \cdots + \alpha_n q^n$,其中 $|\alpha_k| \leqslant M \ (k = 1, 2, \cdots)$,而 $|q| < 1$;
        \item[(4)]
        $a_n = \frac{\cos 1}{1 \cdot 2} + \frac{\cos 2}{2 \cdot 3} + \cdots + \frac{\cos n}{n \cdot (n+1)}$.
    \end{enumerate}
\end{problem}

\begin{enumerate}
    \item[(1)]
    \begin{proof}
        显然有 $0 < a_{n+1} < a_n$,故由单调有界原理知数列 $\{ a_n \}$ 收敛.
    \end{proof}
    \item[(2)]
    \begin{proof}
        显然有 $a_{n+1} > a_n$,又
        \[
            a_n = \sum_{i=1}^n \frac 1 {3^i+1} < \sum_{i=1}^n \frac 1 {3^i} = \frac 1 2 \left( 1 - \frac 1 {3^n} \right) < \frac 1 2.
        \]
        故由单调有界原理知数列 $\{ a_n \}$ 收敛.
    \end{proof}
    \item[(3)]
    \begin{proof}
        对任意 $\varepsilon > 0$,取 $N = \left[ \log_q \left( \frac {\varepsilon (1-q)} M \right) \right]$,则当 $m > n > N$ 时,有
        \[
            |a_n - a_m| = |q|^n |\alpha_{n+1} q + \alpha_{n+2} q^2 + \cdots + \alpha_m q^{m-n}| < |q|^n \frac M {1-q} < \varepsilon.
        \]
        则由 Cauchy 收敛准则知,数列 $\{ a_n \}$ 收敛.
    \end{proof}
    \item[(4)]
    \begin{proof}
        对任意 $\varepsilon > 0$,取 $N = \left[ \frac 1 \varepsilon \right]$,则当 $m > n > N$ 时,有
        \[
            |a_n - a_m| = \left| \sum_{i=n+1}^m \frac{\cos i}{i(i+1)} \right| < \sum_{i=n+1}^m \left| \frac 1 {i(i+1)} \right| = \frac 1 {n+1} - \frac 1 {m+1} < \varepsilon.
        \]
        则由 Cauchy 收敛准则知,数列 $\{ a_n \}$ 收敛.
    \end{proof}
\end{enumerate}

\begin{problem}{MAblue}{1.2.18}
    证明下列数列收敛,并求出其极限:
    \begin{enumerate}
        \item[(1)]
        $a_n = \frac n {c^n} \ (c > 1)$;
        \item[(2)]
        $a_1 = \frac c 2,\ a_{n+1} = \frac c 2 + \frac {a_n^2} 2 \ (0 \leqslant c \leqslant 1)$;
        \item[(3)]
        $a > 0,\ a_0 > 0,\ a_{n+1} = \frac 1 2 \left( a_n + \frac a {a_n} \right)$;(提示:先证明 $a_n^2 \geqslant a$.)
        \item[(4)]
        $a_0 = 1,\ a_n = 1 + \frac{a_{n-1}}{a_{n-1}+1}$;
        \item[(5)]
        $a_n = \sin \sin \cdots \sin 1$.($n$ 个 $\sin$.)
    \end{enumerate}
\end{problem}

\begin{enumerate}
    \item[(1)]
    \begin{proof}
        不妨设 $c = 1 + a \ (a > 0)$,由此得
        \[
            0 < a_n = \frac n {c^n} = \frac n {1 + na + \frac {n(n-1)} 2 a + \cdots + a^n} < \frac 2 {(n-1)a}.
        \]
        则对任意 $\varepsilon > 0$,取 $N = \left[ \frac 2 {a\varepsilon} \right] + 1$,可得当 $n > N$ 时有 $|a_n| < \varepsilon$,故数列 $\{ a_n \}$ 收敛,且 $\lim_{n \to \infty} a_n = 0$.
    \end{proof}
    \item[(2)]
    \begin{proof}
        注意到 $a_1 = \frac c 2 \leqslant 1 - \sqrt{1-c}$,且
        \[
            a_k \leqslant 1 - \sqrt{1-c} \ \Rightarrow \ a_{k+1} \leqslant \frac c 2 + \frac {(1-\sqrt{1-c})^2} 2 = 1 - \sqrt{1-c}.
        \]
        故 $a_n \leqslant 1 - \sqrt{1-c} \ (n = 1, 2, \cdots)$. 进而
        \[
            a_{n+1} - a_n = \frac 1 2 (a_n - 1)^2 - \frac {1-c} 2 \geqslant \frac 1 2 (1 - \sqrt{1-c} - 1)^2 - \frac {1-c} 2 = 0.
        \]
        即数列 $\{ a_n \}$ 单调递增有上界,故收敛.
    \end{proof}
    设 $\lim_{n \to \infty} a_n = a$,则
    \[
        a = \frac c 2 + \frac {a^2} 2 \leqslant 1 - \sqrt{1-c}.
    \]
    解得 $a = 1 - \sqrt{1-c}$.
    \item[(3)]
    \begin{proof}
        注意到 $a_{n+1} = \frac 1 2 \left( a_n + \frac a {a_n} \right) \geqslant \frac 1 2 \cdot 2 \sqrt{a_n \cdot \frac a {a_n}} = \sqrt a$,则当 $n \geqslant 1$ 时
        \[
            a_{n+1} - a_n = \frac 1 2 \left( \frac a {a_n} - a_n \right) \leqslant 0.
        \]
        即数列 $\{ a_n \}$ 单调递减有下界,故收敛.
    \end{proof}
    设 $\lim_{n \to \infty} a_n = \alpha$,则
    \[
        \alpha = \frac 1 2 \left( \alpha + \frac a \alpha \right) \geqslant \sqrt a.
    \]
    解得 $\alpha = \sqrt a$.
    \item[(4)]
    \begin{proof}
        注意到 $a_0 = 1 \leqslant \frac {\sqrt 5 + 1} 2$,且
        \[
            a_k \leqslant \frac {\sqrt 5 + 1} 2 \ \Rightarrow \ a_{k+1} = 2 - \frac 1 {a_k + 1} \leqslant 2 - \frac 1 {\frac {\sqrt 5 + 1} 2 + 1} = \frac {\sqrt 5 + 1} 2.
        \]
        故 $0 < a_n \leqslant \frac {\sqrt 5 + 1} 2 \ (n = 0, 1, \cdots)$. 进而
        \[
            a_{n+1} - a_n = \frac 1 {a_n + 1} \left( -\left( a_n - \frac 1 2 \right)^2 + \frac 5 4 \right) \geqslant 0.
        \]
        即数列 $\{ a_n \}$ 单调递增有上界,故收敛.
    \end{proof}
    设 $\lim_{n \to \infty} a_n = a$,则
    \[
        0 \leqslant a = 1 + \frac a {a+1} \leqslant \frac {\sqrt 5 + 1} 2.
    \]
    解得 $a = \frac {\sqrt 5 + 1} 2$.
    \item[(5)]
    \begin{proof}
        显然有 $0 < a_{n+1} < a_n$,故由单调有界原理知数列 $\{ a_n \}$ 收敛.
    \end{proof}
    设 $\lim_{n \to \infty} a_n = a$,则有
    \[
        a = \sin a \ \Rightarrow \ a = 0.
    \]
\end{enumerate}

\begin{think}
    如何求一般递推数列的极限?
\end{think}

\begin{problem}{MAblue}{1.2.21}
    设 $\{ a_n \}, \{ b_n \}$ 是正数列,满足 $\frac{a_{n+1}}{a_n} \leqslant \frac{b_{n+1}}{b_n} \ (n = 1, 2, \cdots)$. 求证:若 $\{ b_n \}$ 收敛,则 $\{ a_n \}$ 收敛.
\end{problem}

\begin{proof}
    由题设知 $0 < \frac{a_{n+1}}{b_{n+1}} \leqslant \frac{a_n}{b_n}$,由单调有界原理知数列 $\left\{ \frac{a_n}{b_n} \right\}$ 收敛. 则
    \[
        \lim_{n \to \infty} a_n = \left( \lim_{n \to \infty} \frac{a_n}{b_n} \right) \left(\lim_{n \to \infty} b_n \right). \qedhere
    \]
\end{proof}

\begin{problem}{MAblue}{1.2.25}
    设数列 $\{ a_n \}$ 由 $a_1 = 1,\ a_{n+1} = a_n + \frac 1 {a_n} \ (n \geqslant 1)$ 定义. 证明:$a_n \to +\infty \ (n \to \infty)$.
\end{problem}

\begin{proof}
    考虑反证法. 假设存在 $M > 0$,使 $0 < a_n \leqslant M$ 恒成立,则
    \[
        a_{M^2+1} = 1 + \frac 1 {a_1} + \frac 1 {a_2} + \cdots + \frac 1 {a_{M^2}} \geqslant 1 + M^2 \cdot \frac 1 M > M
    \]
    与假设矛盾. 故 $ a_n \to +\infty \ (n \to \infty)$.
\end{proof}

\begin{problem}{MAblue}{1.2.26}
    给出 $\frac 0 0$ 型 Stolz 定理的证明.
\end{problem}

\begin{proof}
    由题设知 $\forall \varepsilon > 0,\ \exists N \in \N_+$ 使得 $n > N$ 时有
    \[
        A - \varepsilon < \frac{a_{n+1} - a_n}{b_{n+1} - b_n} < A + \varepsilon.
    \]
    由于 $\{ b_n \}$ 严格单调递减,则有
    \[
        (A - \varepsilon)(b_n - b_{n+1}) < a_n - a_{n+1} < (A + \varepsilon)(b_n - b_{n+1}).
    \]
    取正整数 $m > n$,累加得
    \[
        (A - \varepsilon)(b_n - b_m) < a_n - a_m < (A + \varepsilon)(b_n - b_m).
    \]
    即 $\left| \frac{a_n - a_m}{b_n - b_m} - A \right| < \varepsilon$,令 $m \to \infty$ 即得 $\left| \frac{a_n}{b_n} - A \right| < \varepsilon$.
\end{proof}

\section{函数极限}

\begin{problem}{MAblue}{1.3.7}
    求证:
    \[
        \lim_{n \to \infty} \left( \sin \frac \alpha {n^2} + \sin \frac{2\alpha}{n^2} + \cdots + \sin \frac{n\alpha}{n^2} \right) = \frac \alpha 2.
    \]
\end{problem}

\begin{proof}
    注意到 $n \to \infty$ 时,有 $\sin \frac{i\alpha}{n^2} \sim \frac{i\alpha}{n^2} \ (i \in {1, 2, \cdots, n})$. 则
    \[
        \lim_{n \to \infty} \left( \sum_{i=1}^n \sin \frac{i\alpha}{n^2} \right) = \lim_{n \to \infty} \left( \sum_{i=1}^n \frac{i\alpha}{n^2} \right) = \lim_{n \to \infty} \frac{n(n-1)\alpha}{2n^2} = \frac \alpha 2 \qedhere
    \]
\end{proof}

\begin{think}
    如何证明 $\lim_{n \to \infty} \left( \sin \frac \alpha n + \sin \frac {2\alpha} n + \cdots + \sin \frac \alpha n \right) = +\infty$?
\end{think}

\begin{problem}{MAblue}{1.3.9}
    求下列极限:
    \begin{enumerate}
        \item[(1)]
        $\displaystyle \lim_{x \to 0} \frac{\tan 2x}{\sin 5x}$;
        \item[(2)]
        $\displaystyle \lim_{x \to 0} \frac{\cos x - \cos 3x}{x^2}$;
        \item[(3)]
        $\displaystyle \lim_{x \to +\infty} \left( \frac{x+1}{2x-1} \right)^x$;
        \item[(4)]
        $\displaystyle \lim_{x \to \infty} \left( \frac{x^2+1}{x^2-1} \right)^{x^2}$.
    \end{enumerate}
\end{problem}

\begin{enumerate}
    \item[(4)]
    \begin{solution}
        \vspace{0.15em}
        整理得
        \[
            \lim_{x \to \infty} \left(\frac{x^2+1}{x^2-1}\right)^{x^2} = \lim_{x \to \infty} \left(1+\frac 1 {x^2}\right)^{x^2} \cdot \left(1+\frac 1 {x^2-1}\right)^{x^2} = e^2
        \]
    \end{solution}
\end{enumerate}

\section*{第 1 章综合习题}
\addcontentsline{toc}{section}{第 1 章综合习题}

\begin{problem}{MAblue}{1.0.6}
    设 $\{ a_n \}$ 是正严格单调递增数列. 求证:若 $a_{n+1} - a_n$ 有界,则对任意 $\alpha \in (0, 1)$ 有
    \[
        \lim_{n \to \infty} \left( a_{n+1}^\alpha - a_n^\alpha \right) = 0.
    \]
    并说明此结论的逆不对,即,存在正严格递增数列 $\{ a_n \}$ 使得对任意 $\alpha \in (0, 1)$ 有 $\lim_{n \to \infty} \left( a_{n+1}^\alpha - a_n^\alpha \right) = 0$,但是 $a_{n+1} - a_n$ 无界.(提示:考虑 $a_n = n \ln n$.)
\end{problem}

\begin{proof}
    不妨设 $0 < a_{n+1} - a_n \leqslant M \ (M > 0)$,则
    \[
        0 < a_{n+1}^\alpha - a_n^\alpha < a_n^\alpha \left( \left( 1 + \frac M {a_n} \right)^\alpha - 1 \right) < a_n^\alpha \left( 1 + \frac M {a_n} - 1 \right) = M a_n^{\alpha-1}.
    \]
    故由夹逼原理知,$\lim_{n \to \infty} \left(a_{n+1}^{\alpha}-a_n^{\alpha}\right) = 0$.
\end{proof}

{\flushleft 对其逆命题,考虑反例 $a_n = n \ln n$,易得不成立.}

\begin{mnote}
    事实上,若以 $n$ 为标准,逆命题中 $a_{n+1} - a_n$ 无界的条件要求 $a_n$ 是一个 $k (> 1)$ 阶无穷大量,也即 $a_n \sim n^k \ (n \to \infty)$. 然而在我们试取某个具体的 $k$ 后,我们发现使 $\lim_{n \to \infty} \left( a_{n+1}^\alpha - a_n^\alpha \right) = 0$ 成立的 $\alpha$ 的范围缩小到了 $\left( 0, \frac 1 k \right)$. 那么为了证伪逆命题,我们需要 $k$ 无限地接近 $1$. 另一方面,我们知道对任意 $\alpha > 0$ 有 $\lim_{n \to \infty} \frac{\ln n}{n^\alpha} = 0$. 这启发我们将 $\ln n$ 视为一个“$\varepsilon$ 阶无穷大量”,其中的 $\varepsilon$ 蕴含了极限思想,表示“无穷小但不为零”. 如此一来,我们自然地想到取 $a_n = n \ln n$,也即 $k = 1 + \varepsilon$,便得到了我们想要的结果. 此外,将 $\ln x$ 视为“$\varepsilon$ 阶无穷大量”的思想亦有助于我们理解 $\ln x$ 是 $\frac 1 x$ 的原函数这一事实.
\end{mnote}

\begin{problem}{MAblue}{1.0.7}
    设数列 $\{ a_n \}$ 满足 $\lim_{n \to \infty} (a_{n+1} - a_n) = a$,证明:$\lim_{n \to \infty} \frac {a_n} n = a$.
\end{problem}

{\flushleft 此即 \textbf{Cauchy 命题}.}

\begin{center}
    \begin{minipage}{0.85\textwidth}
        \begin{theorem}{Cauchy 命题}{}
            若数列 $\{ a_n \}$ 满足 $\lim_{n \to \infty} a_n = a$,则
            \[
                \lim_{n \to \infty} \frac {a_1 + a_2 + \cdots + a_n} n = a.
            \]
            \tcblower
            \begin{proof}
                由 $\lim_{n \to \infty} a_n = a$ 知,对任意 $\varepsilon > 0$,存在 $N \in \N_+$ 使得当 $n > N$ 时,有
                \[
                    a - \varepsilon < a_n < a + \varepsilon.
                \]
                设 $S = a_1 + a_2 + \cdots + a_N$,则当 $n > N$ 时,有
                \[
                    \frac {S + (n-N)(a-\varepsilon)} n < \frac {a_1 + a_2 + \cdots + a_n} n < \frac {S + (n-N)(a+\varepsilon)} n.
                \]
                令 $n \to \infty$,则由夹逼原理知
                \[
                    \lim_{n \to \infty} \frac {a_1 + a_2 + \cdots + a_n} n = a. \qedhere
                \]
            \end{proof}
            亦可由 Stolz 定理立得.
        \end{theorem}
    \end{minipage}
\end{center}

\begin{problem}{MAblue}{1.0.8}
    证明:若 $\lim_{n \to \infty} a_n = a$,且 $a_n > 0$,则
    \[
        \lim_{n \to \infty} \sqrt[n]{a_1 a_2 \cdots a_n} = a.
    \]
\end{problem}

\begin{proof}
    当 $a \neq 0$ 时,有 $\lim_{n \to \infty} \frac 1 {a_n} = \frac 1 a$. 又由均值不等式有
    \[
        \frac{n}{\frac 1 {a_1} + \frac 1 {a_2} + \cdots + \frac 1 {a_n}} \leqslant \sqrt[n]{a_1a_2\cdots a_n} \leqslant \frac {a_1+a_2+\cdots+a_n} n.
    \]
    由 Cauchy 命题易知
    \[
        \lim_{n \to \infty} \frac{n}{\frac 1 {a_1} + \frac 1 {a_2} + \cdots + \frac 1 {a_n}} = \frac 1 {\frac 1 a} = a = \lim_{n \to \infty} \frac {a_1+a_2+\cdots+a_n} n.
    \]
    故由夹逼原理知,$\lim_{n \to \infty} \sqrt[n]{a_1a_2\cdots a_n} = a$.
    
    {\flushleft 当 $a = 0$ 时,利用不等式 $0 < \sqrt[n]{a_1a_2\cdots a_n} \leqslant \frac {a_1+a_2+\cdots+a_n} n$ 即可类似地证明.}
\end{proof}

{\flushleft 或利用}
\[
    \lim_{n \to \infty} \sqrt[n]{a_1 a_2 \cdots a_n} = \lim_{n \to \infty} \exp\left( \frac {\ln a_1 + \ln a_2 + \cdots + \ln a_n} n \right) = \exp(\ln a) = a,
\]
亦可证明.

\begin{problem}{MAblue}{1.0.12}
    设 $\{ a_n \}$ 且 $a_n \to a \in \R$,又设 $b_n$ 是正数列,且 $c_n = \frac{a_1b_1 + a_2b_2 + \cdots + a_nb_n}{b_1 + b_2 + \cdots + b_n}$. 求证:
    \begin{enumerate}
        \item[(1)]
        数列 $\{ c_n \}$ 收敛;
        \item[(2)]
        若 $(b_1 + b_2 + \cdots + b_n) \to +\infty$,则 $\lim_{n \to \infty} c_n = a$.
    \end{enumerate}
\end{problem}

\begin{proof}
    设 $B_n = b_1 + b_2 + \cdots + b_n$. 若 $n \to \infty$ 时 $B_n \to B \in \R$,则对任意 $\varepsilon > 0$,存在 $N \in \N_+$,使得当 $m > n > N$ 时有
    \[
        |B_n - B_m| = |b_{n+1} + b_{n+2} + \cdots + b_m| < \varepsilon.
    \]
    又 $\{ a_n \}$ 收敛,不妨设 $|a_n| \leqslant M$,则
    \[
        |a_{n+1}b_{n+1} + a_{n+2}b_{n+2} + \cdots + a_mb_m| < M\varepsilon.
    \]
    故由 Cauchy 收敛准则知,数列 $\left\{ \sum_{k=1}^n a_kb_k \right\}$ 收敛,进而数列 $\{ c_n \}$ 收敛.

    {\flushleft 否则若 $n \to \infty$ 时 $B_n \to +\infty$,则由 Stolz 定理知}
    \[
        \lim_{n \to \infty} c_n = \lim_{n \to \infty} \frac{a_nb_n}{b_n} = a.
    \]
    则 (1),(2) 俱得证.
\end{proof}

\begin{problem}{MAblue}{1.0.16}
    设 $\xi$ 是一个无理数. $a, b$ 是实数,且 $a < b$,求证:存在整数 $m, n$ 使得 $m + n \xi \in (a, b)$,即,集合
    \[
        S = \{ m + n \xi \mid m, n \in \Z \}
    \]
    在 $\R$ 稠密.
\end{problem}

\begin{proof}
    不妨设 $\xi > 0$,考虑以下引理:
    \begin{center}
        \begin{minipage}{0.85\textwidth}
            \begin{lemma}{Dirichlet 逼近定理}{}
                对任意 $x \in \R$ 和 $k \in \N_+$,存在 $p, q \in \Z$ 使得 $0 < |qx - p| < \frac 1 k$.
                \tcblower
                \begin{proof}
                    考虑 $k$ 个区间
                    \[
                        \left[ 0, \frac 1 k \right),\ \left[ \frac 1 k, \frac 2 k \right),\ \ldots,\ \left[ 1 - \frac 1 k, 1 \right)
                    \]
                    及 $k + 1$ 个数
                    \[
                        0, \{ x \}, \{ 2x \}, \ldots, \{ kx \},
                    \]
                    则由鸽巢原理知,必有两个数位于同一区间内,不妨设为 $\{ ix \}, \{ jx \}$,则
                    \[
                        0 < |\{ ix \} - \{ jx \}| = |(i-j)x - ([ix] - [jx])| < \frac 1 k.
                    \]
                    取 $p = [ix] - [jx],\ q = i-j$,则引理得证.
                \end{proof}
            \end{lemma}
        \end{minipage}
    \end{center}
    设 $m_0 + n_0 \xi > a$ 是 $S$ 中最小的满足此性质的数,则若 $m_0 + n_0 \xi \geqslant b$,取 $k = \left[ \frac 1 {b - a} \right] + 1$,由引理有 $m_0 + n_0 \xi - |q\xi - p| > b - \frac 1 k > a$,矛盾,故必有 $m_0 + n_0 \xi \in (a, b)$.
\end{proof}

\chapter{单变量函数的连续性}

\section{连续函数的基本概念}

\begin{problem}{MAblue}{2.1.15}
    设 $f(x)$ 在 $\R$ 上连续,且对于任意 $x, y$ 有 $f(x+y) = f(x) + f(y)$. 求证 $f(x) = cx$,其中 $c$ 是常数.
\end{problem}

\begin{proof}
    易知 $f(nx) = nf(x) \ (n \in \Z)$,进而对任意有理数 $\frac p q$ 有
    \[
        f\left( \frac p q x \right) = \frac p q f(x).
    \]
    由有理数的稠密性知,对任意实数 $\alpha$,总存在有理数列 $\{ a_n \}$ 满足 $\lim_{n \to \infty} a_n = \alpha$. 考虑
    \[
       f(a_n x) = a_n f(x),
    \]
    由于 $f(x)$ 连续,令 $n \to \infty$ 即得 $f(\alpha x) = \alpha f(x)$. 故 $f(x) = x f(1)$,也即 $c = f(1)$.
\end{proof}

\section{闭区间上连续函数的性质}

\begin{problem}{MAblue}{2.2.6}
    设函数 $f(x)$ 在 $[0, 2a]$ 上连续,且 $f(0) = f(2a)$. 证明:在区间 $[0, a]$ 上存在某个 $x_0$,使得 $f(x_0) = f(x_0 + a)$.
\end{problem}

\begin{proof}
    设 $g(x) = f(x) - f(x+a) \ (0 \leqslant x \leqslant a)$,则 $g(0) = -g(a)$,由零点定理可得欲证.
\end{proof}

\begin{problem}{MAblue}{2.2.7}
    试证:若函数 $f(x)$ 在 $[a, b]$ 上连续,$x_1, x_2, \ldots, x_n$ 为此区间中的任意值,则在 $[a, b]$ 中有一点 $\xi$,使得
    \[
        f(\xi) = \frac 1 n \left( f(x_1) + f(x_2) + \cdots + f(x_n) \right).
    \]
    更一般地,若 $q_1, q_2, \ldots, q_n \in \R_+$,且 $q_1 + q_2 + \cdots + q_n = 1$,则在 $[a, b]$ 中有一点 $\xi$,使得
    \[
        f(\xi) = q_1f(x_1) + q_2f(x_2) + \cdots + q_nf(x_n).
    \]
\end{problem}

\begin{proof}
    显然有 $\min\left( f(x_1), f(x_2), \ldots, f(x_n) \right) \leqslant f(\xi) \leqslant \max\left( f(x_1), f(x_2), \ldots, f(x_n) \right)$,则由介值定理可得欲证.
\end{proof}

\begin{problem}{MAblue}{2.2.16}
    给出一个在 $(-\infty, +\infty)$ 上连续且有界但不一致连续的函数.
\end{problem}

\begin{solution}
    $f(x) = \sin x^2$. 由连续且有界想到基本初等函数中的 $\sin x$(或 $\cos x$),而不一致连续的条件启发我们寻找一个任意陡的函数,故想到 $\sin x^2$.
\end{solution}

\section*{第 2 章综合习题}
\addcontentsline{toc}{section}{第 2 章综合习题}

\begin{problem}{MAblue}{2.0.5}
    设 $f(x)$ 在区间 $[0, 1]$ 上连续,且 $f(0) = f(1)$. 证明:对任意自然数 $n$,在区间 $\left[ 0, 1 - \frac 1 n \right]$ 中有一点 $\xi$,使得 $f(\xi) = f\left( \xi + \frac 1 n \right)$.
\end{problem}

\begin{proof}
    设 $g(x) = f(x) - f(x+\frac 1 n) \ (0 \leqslant x \leqslant 1 - \frac 1 n)$,若存在 $g(\xi) = 0$,命题得证;否则有 $g(0) + g(\frac 1 n) + \cdots + g(1-\frac 1 n) = 0$,其中必有两项异号,由零点定理可得欲证.
\end{proof}

\begin{problem}{MAblue}{2.0.8}
    设函数 $f(x)$ 定义在区间 $[a, b]$ 上,满足条件:$a \leqslant f(x) \leqslant b$,且对 $[a, b]$ 中任意的 $x, y$ 有
    \[
        |f(x) - f(y)| \leqslant k|x - y|,
    \]
    其中常数 $k \in (0, 1)$,证明
    \begin{enumerate}
        \item[(1)]
        存在唯一的 $x_0 \in [a, b]$,使得 $f(x_0) = x_0$.
        \item[(2)]
        任取 $x_1 \in [a, b]$,并定义数列 $\{ x_n \}$:$x_{n+1} = f(x_n) \ (n = 1, 2, \cdots)$,则 $\lim_{n \to \infty} x_n = x_0$.
        \item[(3)]
        给出一个在实轴上的连续函数,使得对任意 $x \neq y$ 有 $|f(x) - f(y)| < |x-y|$,但方程 $f(x) - x = 0$ 无解.
    \end{enumerate}
\end{problem}

\begin{enumerate}
    \item[(1)]
    \begin{proof}
        设 $g(x) = f(x) - x$,则 $g(a) \geqslant 0,\ g(b) \leqslant 0$. 故由零点定理知存在 $x_0 \in [a, b]$ 使得 $g(x_0) = 0$,也即 $f(x_0) = x_0$. 若存在另一 $x_0' \in [a, b]$ 满足 $f(x_0') = x_0'$,则
        \[
            |f(x_0) - f(x_0')| = |x_0 - x_0'| > k|x_0 - x_0'|,
        \]
        与题设矛盾. 故 $x_0$ 是唯一的.
    \end{proof}
    \item[(2)]
    \begin{proof}
        由题设有 $|x_{n+1} - x_n| \leqslant k|x_n - x_{n-1}| \ (n \geqslant 2)$. 设 $M = \left| \dfrac{x_2 - x_1}{k^2} \right|$,则对任意 $m > n$,有
        \begin{align*}
            |x_n - x_m| &\leqslant |x_n - x_{n+1}| + \cdots + |x_{m-1} - x_m| \\
            &\leqslant M k^n + M k^{n+1} + \cdots + M k^m.
        \end{align*}
        故由 1.2.17.(3) 及夹逼原理知,数列 $\{ x_n \}$ 是基本列. 进而由 Cauchy 收敛准则知,数列 $\{ x_n \}$ 收敛. 设 $\lim_{n \to \infty} x_n = x$,则
        \[
            x = f(x) \ \Rightarrow \ x = x_0.
        \]
        也即 $\lim_{n \to \infty} x_n = x_0$.
    \end{proof}
\end{enumerate}

事实上,此即

\begin{center}
    \begin{minipage}{0.85\textwidth}
        \begin{theorem}{压缩映射原理}{}
            设 $f: [a, b] \to [a, b]$,且存在 $k \in (0, 1)$ 使得
            \[
                |f(x_1) - f(x_2)| \leqslant k|x_1 - x_2|
            \]
            对任意 $x_1, x_2 \in [a, b]$ 均成立,则
            \begin{enumerate}[label=\arabic*$^\circ$]
                \item
                存在唯一的 $\xi \in [a, b]$,使得 $f(\xi) = \xi$;
                \item
                若数列 $\{ x_n \}$ 满足 $x_0 \in [a, b],\ x_{n+1} = f(x_n)$,则 $\lim_{n \to \infty} x_n = \xi$.
            \end{enumerate}
        \end{theorem}
    \end{minipage}
\end{center}

\begin{problem}{MAblue}{2.0.10}
    设 $f(x)$ 在 $[a, b]$ 上连续,且对任意 $x \in [a, b)$ 存在 $y \in (x, b)$ 使得 $f(y) > f(x)$. 求证:$f(b) > f(a)$.
\end{problem}

\begin{proof}
    由题设知,存在数列 $\{ x_n \}$,满足 $x_0 = a,\ x_n < x_{n+1} < b$,且 $f(x_n) < f(x_{n+1})$. 由单调有界原理知数列 $\{ x_n \}$ 收敛,不妨设其收敛到 $c$,则必有 $c = b$,否则对任意 $x \in (c, b)$ 均有 $f(c) \geqslant f(x)$,与题设矛盾. 而当 $n > 1$ 时,我们有
    \[
        a = f(x_0) < f(x_1) < f(x_n).
    \]
    由于 $f(x)$ 连续,令 $n \to \infty$ 即得 $f(a) < f(x_1) \leqslant f(b)$.
\end{proof}

\chapter{单变量函数的微分学}

\section{导数}

\begin{problem}{MAblue}{3.1.17}
    求下列各式的和
    \begin{enumerate}
        \item[(1)]
        $P_n = 1 + 2x + \cdots + nx^{n-1}$;
        \item[(2)]
        $Q_n = 1^2 + 2^2x + \cdots + n^2x^{n-1}$;
        \item[(3)]
        $R_n = \cos 1 + 2 \cos 2 + \cdots + n \cos n$.
    \end{enumerate}
\end{problem}

\begin{enumerate}
    \item[(1)]
    \begin{solution}
        设
        \[
            f(x) = \frac{x(1-x^n)}{1-x} = x + x^2 + \cdots + x^n \ (x \neq 1),
        \]
        则
        \[
            P_n = f'(x) = \frac{nx^{n+1}-(n+1)x^n+1}{(x-1)^2}.
        \]
    \end{solution}
    \item[(2)]
    \begin{solution}
        \[
            Q_n = f'(x)+xf''(x) = \frac{n^2x^{n+2}-(2n^2+2n-1)x^{n+1}+(n+1)^2x^n-x-1}{(x-1)^3}.
        \]
    \end{solution}
    \item[(3)]
    \begin{solution}
        设
        \[
            g(x) = \sin x + \sin 2x + \cdots + \sin nx = \frac{\cos\frac x 2 - \cos(\frac {2n+1} 2 x)}{2\sin\frac x 2},
        \]
        则
        \[
            R_n = g'(1) = \frac{(n+1)\cos n - n\cos(n+1)-1}{4\sin^2\frac 1 2}.
        \]
    \end{solution}
\end{enumerate}

\section{微分}

\begin{problem}{MAblue}{3.2.3}
    对下列函数,求 $\frac{\mathrm dy}{\mathrm dx}$ 及 $\frac{\mathrm d^2y}{\mathrm dx^2}$.
    \begin{multicols}{2}
        \begin{enumerate}
            \item[(1)]
            $\begin{cases}
                x = \ln(1+t^2), \\
                y = t - \arctan t;
            \end{cases}$
            \item[(2)]
            $\begin{cases}
                x = t - \sin t, \\
                y = 1 - \cos t;
            \end{cases}$
            \item[(3)]
            $\begin{cases}
                x = \varphi \cos \varphi, \\
                y = \varphi \sin \varphi;
            \end{cases}$
            \item[(4)]
            $\begin{cases}
                x = \cos^3 \varphi, \\
                y = \sin^3 \varphi.
            \end{cases}$
        \end{enumerate}
    \end{multicols}
\end{problem}

\begin{enumerate}
    \item[(1)]
    \begin{solution}
    \[
        \begin{cases}
            \mathrm dx = \frac{2t}{1+t^2} \mathrm dt, \\
            \mathrm dy = \frac 1 {1+t^2} \mathrm dt,
        \end{cases}
        \quad \Rightarrow \quad
        \begin{cases}
            \frac{\mathrm dy}{\mathrm dx} = \frac 1 {2t}, \\
            \frac{\mathrm d^2y}{\mathrm dx^2} = \frac{\mathrm d \left( \frac{\mathrm dy}{\mathrm dx} \right)}{\mathrm dx} = -\frac{1+t^2}{4t^3}.
        \end{cases}
    \]
    \end{solution}
\end{enumerate}

\section{微分中值定理}

\begin{problem}{MAblue}{3.3.4}
    证明下列不等式.
    \begin{enumerate}
        \item[(1)]
        当 $a > b > 0,\ n > 1$ 时,有 $nb^{n-1}(a-b) < a^n - b^n < na^{n-1}(a-b)$;
        \item[(2)]
        当 $x > 0$ 时,有 $\frac x {1+x} < \ln(1+x) < x$;
        \item[(3)]
        当 $0 < a < b$ 时,有 $(a+b) \ln \frac {a+b} 2 < a \ln a + b \ln b$;
        \item[(4)]
        当 $0 < \alpha < \beta < \frac \pi 2$ 时,有 $\frac{\beta-\alpha}{\cos^2\alpha} < \tan \beta - \tan \alpha < \frac{\beta-\alpha}{\cos^2\beta}$.
    \end{enumerate}
\end{problem}

\begin{enumerate}
    \item[(3)]
    \begin{proof}
        设 $f(x) = x \ln x$,则在 $\left( a,\ \frac {a+b} 2 \right)$ 和 $\left( \frac {a+b} 2,\ b \right)$ 上,由 Lagrange 中值定理有
        \[
            \frac{f\left( \frac {a+b} 2 \right) - f(a)}{\frac {b-a} 2} = 1 + \ln \xi_1 < 1 + \ln \xi_2 = \frac{f(b) - f\left( \frac {a+b} 2 \right)}{\frac {b-a} 2}
        \]
        整理后即得欲证.
    \end{proof}
    事实上,此为 \textbf{Jensen 不等式}的一种形式:当 $f(x)$ 是凸函数,即 $f''(x) > 0$ 时,有 $\frac {f(a) + f(b)} 2 > f(\frac {a+b} 2)$.
\end{enumerate}

\begin{problem}{MAblue}{3.3.7}
    设函数 $f(x)$ 在 $[0, 1]$ 上连续,在 $(0, 1)$ 内可微,且 $f'(x) < 1$,又 $f(0) = f(1)$,证明:对于 $[0, 1]$ 上的任意两点 $x_1, x_2$,有 $|f(x_1) - f(x_2)| < \frac 1 2$.
\end{problem}

\begin{proof}
    不妨设 $x_1 < x_2$,则由 Lagrange 中值定理知,存在 $\xi \in (x_1, x_2)$,使得
    \[
        \left| \frac{f(x_1) - f(x_2)}{x_1 - x_2} \right| = |f'(\xi)| < 1,
    \]
    也即 $|f(x_1) - f(x_2)| < |x_1 - x_2| = x_2 - x_1$. 则若 $x_2 - x_1 < \frac 1 2$,结论成立;否则有 $|f(x_1) - f(x_2)| \leqslant |f(x_1) - f(0)| + |f(1) - f(x_2)| < 1 - (x_2 - x_1) \leqslant \frac 1 2$.
\end{proof}

\begin{problem}{MAblue}{3.3.12}
    设对所有的实数 $x, y$,不等式 $|f(x) - f(y)| \leqslant M|x-y|^2$ 都成立. 证明:$f(x)$ 恒为常数.
\end{problem}

\begin{proof}
    两边同除 $|x-y|$ 得
    \[
        \left| \frac{f(x)-f(y)}{x-y} \right| \leqslant M|x-y|.
    \]
    令 $y \to x$,得 $f'(x) = 0$,故 $f(x)$ 恒为常数.
\end{proof}

\begin{problem}{MAblue}{3.3.18}
    若 $f(x)$ 在 $[0, +\infty)$ 可微,$f(0) = 0$,$f'(x)$ 严格递增,证明 $\frac {f(x)} x$ 严格递增.
\end{problem}

\begin{proof}
    由题设知对任意 $x > 0$ 存在 $\xi \in (0,\ x)$ 使得 $\frac{f(x)-f(0)}{x-0} = f'(\xi) < f'(x)$,则
    \[
        \frac{\mathrm d \left( \frac{f(x)}{x} \right)}{\mathrm dx} = \frac{xf'(x)-f(x)}{x^2} > 0,
    \]
    即 $\frac{f(x)}{x}$ 严格递增.
\end{proof}

\begin{problem}{MAblue}{3.3.20}
    设 $f(x)$ 在 $[0, 1]$ 上有二阶导函数,且 $f(0) = f'(0),\ f(1) = f'(1)$,求证:存在 $\xi \in (0, 1)$ 满足 $f(\xi) = f''(\xi)$.
\end{problem}

\begin{proof}
    设 $g(x)=e^x\left( f(x)-f'(x) \right)$,则 $g(0) = g(1) = 0$. 进而由 Rolle 定理知,存在 $\xi \in (0,\ 1)$,满足 $g'(\xi) = e^{\xi}\left( f(\xi) - f''(\xi) \right) = 0$,也即 $f(\xi) = f''(\xi)$.
\end{proof}

\begin{problem}{MAblue}{3.3.23}
    证明下列不等式.
    \begin{enumerate}
        \item[(1)]
        $\frac 1 {2^{p-1}} \leqslant x^p + (1-x)^p \leqslant 1,\ x \in (0, 1],\ p > 1$;
        \item[(2)]
        $\tan x > x - \frac {x^3} 3,\ x \in \left( 0, \frac \pi 2 \right)$;
        \item[(3)]
        $\frac{\tan x_2}{\tan x_1} > \frac{x_2}{x_1},\ 0 < x_1 < x_2 < \frac \pi 2$;
        \item[(4)]
        $\ln(1+x) > \frac{\arctan x}{1+x},\ x > 0$;
        \item[(5)]
        $1 + x \ln(x+\sqrt{1+x^2}) \geqslant \sqrt{1+x^2},\ x \in \R$;
        \item[(6)]
        $\frac x {\sin x} > \frac 4 3 - \frac 1 3 \cos x,\ x \in \left( 0, \frac \pi 2 \right)$,且右端的 $\frac 4 3$ 为最优系数;
        \item[(7)]
        $\left( 1 - \frac 1 x \right)^{x-1} \left( 1 + \frac 1 x \right)^{x+1} < 4,\ x \in (1, +\infty)$;
        \item[(8)]
        $x^{a-1} + x^{a+1} \geqslant \left( \frac{1-a}{1+a} \right)^{\frac {a-1} 2} + \left( \frac{1-a}{1+a} \right)^{\frac {a+1} 2},\ x \in (0, 1),\ a \in (0, 1)$.
    \end{enumerate}
\end{problem}

\begin{enumerate}
    \item[(1)]
    \begin{proof}
        先证右边. 显然有 $x^p + (1-x)^p \leqslant x + 1-x = 1$.
    
        对于左边,不妨设 $x < \frac 1 2$,则
        \[
            \frac 1 {2^{p-1}} \leqslant x^p + (1-x)^p \quad \Leftrightarrow \quad \frac 1 {2^p} - x^p \leqslant (1-x)^p - \frac 1 {2^p}.
        \]
        由 Lagrange 中值定理知,存在 $x < \xi_1 < \frac 1 2 < \xi_2 < 1-x$,满足
        \[
            \frac{\frac 1 {2^p} - x^p}{\frac 1 2 - x} = p\xi_1^{p-1} < p\xi_2^{p-1} = \frac{(1-x)^p - \frac 1 {2^p}}{1-x-\frac 1 2},
        \]
        则原式得证.
    \end{proof}
    \item[(3)]
    \begin{proof}
        由 Lagrange 中值定理知,存在 $0 < \xi_1 < x_1 < \xi_2 < x_2 < \frac \pi 2$,满足
        \[
            \frac{\tan x_1-0}{x_1-0} = \frac 1 {\cos^2 \xi_1} < \frac 1 {\cos^2 \xi_2} = \frac{\tan x_2 - \tan x_1}{x_2 - x_1},
        \]
        整理后即得原式.
    \end{proof}
    \item[(5)]
    \begin{proof}
        注意到不等式两侧均为关于 $x$ 的偶函数,而易知 $x = 0$ 时等号成立,故不妨设 $x > 0$. 由 Lagrange 中值定理知,存在 $\xi \in (0,\ x)$,满足
        \[
            \frac{\ln\left( x+\sqrt{1+x^2} \right)-0}{x-0} = \frac 1 {\sqrt{1+\xi^2}} > \frac 1 {\sqrt{1+x^2}},
        \]
        代入原式即得.
    \end{proof}
\end{enumerate}

\section{未定式的极限}

\begin{problem}{MAblue}{3.4.4}
    设 $f(x)$ 在 $[a, b] \ (ab>0)$ 上连续,在 $(a, b)$ 上可微. 求证:存在 $\xi \in (a, b)$,使
    \[
        \frac{af(b) - bf(a)}{a-b} = f(\xi) - \xi f'(\xi).
    \]
\end{problem}

\begin{proof}
    由 Cauchy 中值定理知,存在 $\xi \in (a,\ b)$ 满足
    \[
        \frac{\frac {f(b)} b - \frac {f(a)} a}{\frac 1 b - \frac 1 a} = \frac{\frac{\xi f'(\xi) - f(\xi)}{\xi^2}}{-\frac 1 {\xi^2}} = f(\xi) - \xi f'(\xi),
    \]
    整理后即得.
\end{proof}

\section{函数的单调性和凸性}

\begin{problem}{MAblue}{3.5.1}
    证明 Jensen 不等式.
\end{problem}

\begin{center}
    \begin{minipage}{0.9\textwidth}
        \begin{theorem}{Jensen 不等式}{}
            若 $f(x)$ 是区间 $I$ 上的凸函数,$x_1, x_2, \ldots, x_n$ 是 $I$ 中 $n$ 个点,则对任意满足 $\lambda_1 + \lambda_2 + \cdots + \lambda_n = 1$ 的正数 $\lambda_1, \lambda_2, \ldots, \lambda_n$ 有
            \[
                f(\lambda_1 x_1 + \cdots + \lambda_n x_n) \leqslant \lambda_1 f(x_1) + \cdots + \lambda_n f(x_n).
            \]
        \end{theorem}
    \end{minipage}
\end{center}

\begin{proof}
    归纳证明. 显然 $n = 1$ 时成立. 假设 $n=k \ (\geqslant 1)$ 时结论成立,则当 $n = k + 1$ 时,设 $\lambda_1 + \lambda_2 + \cdots + \lambda_{k+1} = 1$,则
    \[
        f(\frac{\lambda_1x_1 + \lambda_2x_2 + \cdots + \lambda_kx_k}{1-\lambda_{k+1}}) \leqslant \frac{\lambda_1f(x_1) + \lambda_2f(x_2) + \cdots + \lambda_kf(x_k)}{1-\lambda_{k+1}}.
    \]
    由凸函数定义知
    \[
        f\left( (1-\lambda_{k+1}) \frac{\sum_{i=1}^k \lambda_ix_i}{1-\lambda_{k+1}} + \lambda_{k+1}x_{k+1} \right) \leqslant (1-\lambda_{k+1}) \frac{\sum_{i=1}^k \lambda_if(x_i)}{1-\lambda_{k+1}} + \lambda_{k+1}f(x_{k+1}).
    \]
    {\flushleft 整理后即得 $n = k+1$ 时成立. 故原命题成立.}
\end{proof}

\begin{problem}{MAblue}{3.5.2}
    证明加权均值不等式.
\end{problem}

\begin{center}
    \begin{minipage}{0.85\textwidth}
        \begin{theorem}{加权均值不等式}{}
            设 $x_1, x_2, \ldots, x_n$ 和 $\lambda_1, \lambda_2, \ldots, \lambda_n$ 都是正数,且 $\lambda_1 + \lambda_2 + \cdots + \lambda_n = 1$. 则有不等式
            \[
                x_1^{\lambda_1} x_2^{\lambda_2} \cdots x_n^{\lambda_n} \leqslant \lambda_1 x_1 + \lambda_2 x_2 + \cdots + \lambda_n x_n.
            \]
        \end{theorem}
    \end{minipage}
\end{center}

\begin{proof}
    设 $A_n = \lambda_1 x_1 + \lambda_2 x_2 + \cdots + \lambda_n x_n,\ G_n = x_1^{\lambda_1} x_2^{\lambda_2} \cdots x_n^{\lambda_n}$,则
    \[
        \ln\left( \frac{G_n}{A_n} \right) = \sum_{k=1}^n \lambda_i \ln\left( \frac{x_i}{A_n} \right) \leqslant \sum_{k=1}^n \lambda_i \left( \frac{x_i}{A_n} - 1 \right) = 0.
    \]
    整理后即得.
\end{proof}
{\flushleft 亦可借助 Jensen 不等式证明.}

\section{Taylor 展开}

\begin{problem}{MAblue}{3.6.8}
    设函数 $f(x)$ 在 $[0, 2]$ 上二阶可导,且对任意 $x \in [0, 2]$,有 $|f(x)| \leqslant 1$ 及 $|f''(x)| \leqslant 1$. 证明:$|f'(x)| \leqslant 2,\ x \in [0, 2]$.
\end{problem}

\begin{proof}
    对任意 $x \in [0, 2]$,由 Taylor 公式得
    \[
        \begin{cases}
            f(0) = f(x) - f'(x)x + \frac {f''(\xi_1)} 2 (0-x)^2, & 0 \leqslant \xi_1 \leqslant x, \\
            f(2) = f(x) + f'(x)(2-x) + \frac {f''(\xi_2)} 2 (2-x)^2, & x \leqslant \xi_2 \leqslant 2.
        \end{cases}
    \]
    两式相减得
    \[
        |f'(x)| = \left| \frac {f(2)-f(0)} 2 + \frac {x^2 f''(\xi_1) - (x-2)^2 f''(\xi_2)} 4 \right| \leqslant 1 + \frac{x^2 + (x-2)^2} 4 \leqslant 2. \qedhere
    \]
\end{proof}

\section*{第 3 章综合习题}
\addcontentsline{toc}{section}{第 3 章综合习题}

\begin{problem}{MAblue}{3.0.5}
    设 $f(x)$ 在区间 $I$ 上连续,如果任给 $I$ 中两点 $x_1, x_2$,有
    \[
        f \left( \frac {x_1 + x_2} 2 \right) \leqslant \frac {f(x_1) + f(x_2)} 2,
    \]
    则 $f(x)$ 是区间 $I$ 上的凸函数.
\end{problem}

\begin{proof}
    先考虑以下引理:
    \begin{center}
        \begin{minipage}{0.9\textwidth}
            \begin{lemma}{}{}
                对任意 $x_1, x_2 \in I,\ i, n \in \N_+,\ 0 \leqslant i \leqslant 2^n$,有
                \[
                    f\left( \frac{ix_1+(2^n-i)x_2}{2^n} \right) \leqslant \frac i {2^n} f(x_1) + \left( 1 - \frac i {2^n} \right) f(x_2).
                \]
                \tcblower
                \begin{proof}
                    考虑归纳证明. 当 $n = 1$ 时,结论显然成立.

                    若当 $n = k (\geqslant 1)$ 时,结论成立,则当 $n = k + 1$ 时,若 $i = 0$ 或 $i = 2^n$,结论是显然的,故我们考虑证明 $0 < i < 2^{k+1}$ 的情况. 若 $i$ 为偶数,结论显然成立;若 $i$ 为奇数,不妨设 $i < 2^k$,则
                    \begin{align*}
                        f\left( \frac{(i-1)x_1+(2^{k+1}-i+1)x_2}{2^{k+1}} \right) \leqslant \frac {i-1} {2^{k+1}} f(x_1) + \left( 1 - \frac {i-1} {2^{k+1}} \right) f(x_2), \\
                        f\left( \frac{(i+1)x_1+(2^{k+1}-i-1)x_2}{2^{k+1}} \right) \leqslant \frac {i+1} {2^{k+1}} f(x_1) + \left( 1 - \frac {i+1} {2^{k+1}} \right) f(x_2).
                    \end{align*}
                    则将
                    \[
                        x_1' = \frac{(i-1)x_1+(2^{k+1}-i+1)x_2}{2^{k+1}},\ x_2' = \frac{(i+1)x_1+(2^{k+1}-i-1)x_2}{2^{k+1}}
                    \]
                    代入 $f\left( \frac {x_1+x_2} 2 \right) \leqslant \frac {f(x_1)+f(x_2)} 2$ 中即可得结论成立.
                \end{proof}
            \end{lemma}
        \end{minipage}
    \end{center}
    对任意实数 $\lambda \in (0,\ 1)$,可利用上述结论逼近之,再由函数连续性得
    \[
        f(\lambda x_1 + (1 - \lambda) x_2) \leqslant \lambda f(x_1) + (1 - \lambda) f(x_2),
    \]
    即 $f(x)$ 是凸函数.
\end{proof}

\begin{problem}{MAblue}{3.0.6}
    设 $f(x)$ 是 $[0, 1]$ 上的两阶可微函数,$f(0) = f(1) = 0$. 证明:存在 $\xi \in (0, 1)$,使得 $f''(\xi) = \frac{2f'(\xi)}{1-\xi}$.
\end{problem}

\begin{proof}
    设 $g(x) = f'(x)(x-1)^2$,由 Rolle 定理知存在 $\eta \in (0,\ 1)$,满足 $f'(\eta) = 0$. 则 $g(\eta) = g(1) = 0$,故再由 Rolle 定理知存在 $\xi \in (\eta,\ 1)$,满足 $g'(\xi) = 0$,即 $f''(\xi) = \frac{2f'(\xi)}{1-\xi}$.
\end{proof}

\begin{mnote}
    下简述选取辅助函数 $g(x)$ 的思路. 我们试图利用 Rolle 定理证明此题,这要求等式 $f''(\xi) = \frac{2f'(\xi)}{1-\xi}$ 可化成 $g'(\xi) = 0$ 的形式,也即有 $g(x) = C$. 解此常微分方程,得到 $f'(x)(x-1)^2 = C$,故可令 $g(x) = f'(x)(x-1)^2$.
\end{mnote}

\begin{problem}{MAblue}{3.0.9}
    设 $f(x)$ 在 $[0, 1]$ 上可导,$f(0) = 1,\ f(1) = \frac 1 2$. 求证:存在 $\xi \in (0, 1)$ 使得
    \[
        f^2(\xi) + f'(\xi) = 0.
    \]
\end{problem}

\begin{proof}
    设 $g(x) = \frac 1 {f(x)} - x$,则 $g(0) = g(1) = 1$. 由 Rolle 定理知,存在 $\xi \in (0,\ 1)$,满足 $g'(\xi) = - \frac{f'(\xi)}{f^2(\xi)} - 1 = 0$,即 $f^2(\xi) + f'(\xi) = 0$.
\end{proof}

\begin{problem}{MAblue}{3.0.19}
    设函数 $f(x)$ 在闭区间 $[-1, 1]$ 上具有三阶连续导数,且 $f(-1) = 0,\ f(1) = 1,\ f'(0) = 0$. 证明:存在 $\xi \in (-1, 1)$,使得 $f'''(\xi) = 3$.
\end{problem}

\begin{proof}
    设 $g(x) = f(x) - \frac 1 2 x^3 + \left( f(0) - \frac 1 2 \right) x^2$,则 $g(-1) = g(0) = g(1)$. 则由 Rolle 定理知存在 $\xi_1 \in (-1,\ 0)$ 及 $\xi_2 \in (0,\ 1)$,满足 $g'(\xi_1) = g'(\xi_2) = g(0) = 0$. 故知存在 $\eta_1 \in (\xi_1,\ 0)$ 及 $\eta_2 \in (0,\ \xi_2)$,满足 $g''(\eta_1) = g''(\eta_2) = 0$. 进而又知存在 $\zeta \in (\eta_1,\ \eta_2)$,满足 $g'''(\zeta) = 0$,即 $f'''(\zeta) = 3$.
\end{proof}

\begin{mnote}
    此题中 $g(x)$ 的选取思路与本节第 6 题不同. 仍然是考虑 Rolle 定理,则我们需要某个 $\phi(x)$ 满足 $\phi'(\xi) = f'''(\xi) - 3 = 0$,故 $\phi(x) = f''(x) - 3x + a$,依此类推,还应有 $\varphi(x) = f'(x) - \frac 3 2 x^2 + ax + b$ 及 $g(x) = f(x) - \frac 1 2 x^3 + \frac a 2 x^2 + bx + c$,为能导出欲证,应有 $g(-1) = g(0) = g(1)$,由此可解得参数 $a, b, c$.
\end{mnote}

\begin{problem}{MAblue}{3.0.20}
    设 $a > 1$,函数 $f : (0, +\infty) \to (0, +\infty)$ 可微. 求证存在趋于无穷的正数列 $\{ x_n \}$ 使得
    \[
        f'(x_n) < f(ax_n),\ n = 1, 2, \cdots.
    \]
\end{problem}

\begin{proof}
    用反证法. 假设命题不成立,即存在 $X > 0$,使得 $x > X$ 时均有 $f'(x) \geqslant f(ax) > 0$. 故 $f(x)$ 在 $(X,\ +\infty)$ 上单调递增. 由 Lagrange 中值定理知,当 $x > X$ 时存在 $\xi \in (x,\ ax)$,满足
    \[
        \frac{f(ax)-f(x)}{(a-1)x} = f'(\xi) \geqslant f(a\xi) > f(ax).
    \]
    故有 $(1-ax+x)f(ax) \geqslant f(x)$. 则当 $x > \frac 1 {a-1}$ 时有 $f(x) < 0$,与题设矛盾,故命题成立.
\end{proof}

\begin{problem}{MAblue}{3.0.21}
    证明 H{\"o}lder 不等式.
\end{problem}

\begin{center}
    \begin{minipage}{0.85\textwidth}
        \begin{theorem}{H{\"o}lder 不等式}{}
            设 $\{ a_i \}, \{ b_i \} \ (i = 1, 2, \ldots, n)$ 是正数. 有 $p, q \in (1, +\infty)$,且 $\frac 1 p + \frac 1 q = 1$,则有
            \[
                \sum_{i=1}^n a_ib_i \leqslant \left( \sum_{i=1}^n a_i^p \right)^{\frac 1 p} \left( \sum_{i=1}^n a_i^q \right)^{\frac 1 q}.
            \]
        \end{theorem}
    \end{minipage}
\end{center}

\begin{proof}
    考虑以下引理:
    \begin{center}
        \begin{minipage}{0.85\textwidth}
            \begin{lemma}{Young 不等式}{}
                设 $x, y \in (0, +\infty),\ p, q \in (1, +\infty)$,且 $\frac 1 p + \frac 1 q = 1$,则
                \[
                    xy \leqslant \frac 1 p x^p + \frac 1 q y^q.
                \]
                \tcblower
                \begin{proof}
                    由 $\ln x$ 凹性知
                    \[
                        \ln xy = \frac 1 p \ln x^p + \frac 1 q \ln y^q \leqslant \ln\left( \frac 1 p x^p + \frac 1 q y^q \right).
                    \]
                    整理后即得.
                \end{proof}
            \end{lemma}
        \end{minipage}
    \end{center}
    令 $x = \frac{a_i}{\left( \sum_{i=1}^n a_i^p \right)^{\frac 1 p}},\ y = \frac{b_i}{\left( \sum_{i=1}^n b_i^q \right)^{\frac 1 q}}$,则由引理有
    \[
        \frac{a_ib_i}{\left( \sum_{i=1}^n a_i^p \right)^{\frac 1 p} \left( \sum_{i=1}^n b_i^q \right)^{\frac 1 q}} \leqslant \frac{a_i^p}{p \left( \sum_{i=1}^n a_i^p \right)} + \frac{b_i^q}{q \left( \sum_{i=1}^n b_i^q \right)},
    \]
    累加得
    \[
        \frac{\sum_{i=1}^n a_ib_i}{\left( \sum_{i=1}^n a_i^p \right)^{\frac 1 p} \left( \sum_{i=1}^n b_i^q \right)^{\frac 1 q}} \leqslant \frac 1 p + \frac 1 q = 1.
    \]
    整理后即得欲证.
\end{proof}

\chapter{不定积分}

\section{不定积分及其基本计算方法}

\begin{problem}{MAblue}{4.1.3}
    用第二代换法求下列不定积分,其中的 $a$ 均为正常数.
    \begin{multicols}{2}
        \begin{enumerate}
            \item[(1)]
            $\displaystyle \int \sqrt{e^x - 2} \mathrm dx$;
            \item[(2)]
            $\displaystyle \int \sqrt{x^2 + a^2} \mathrm dx$;
            \item[(3)]
            $\displaystyle \int \frac{\mathrm dx}{(x^2-a^2)^{3/2}}$;
            \item[(4)]
            $\displaystyle \int \frac{x^2}{\sqrt{a^2-x^2}} \mathrm dx$;
            \item[(5)]
            $\displaystyle \int \frac{\mathrm dx}{1+\sqrt{x+1}}$;
            \item[(6)]
            $\displaystyle \int \frac{1+\ln x}{(1+x^2)^{3/2}} \mathrm dx$;
            \item[(7)]
            $\displaystyle \int \frac{1-\ln x}{(x - \ln x)^2} \mathrm dx$;
            \item[(8)]
            $\displaystyle \int \frac{\mathrm dx}{x^2\sqrt{x^2+a^2}}$;
            \item[(9)]
            $\displaystyle \int \frac{x+2}{\sqrt[3]{2x+1}} \mathrm dx$;
            \item[(10)]
            $\displaystyle \int \frac{x^{1/7}+x^{1/2}}{x^{8/7}+x^{1/14}} \mathrm dx$;
            \item[(11)]
            $\displaystyle \int \frac{x-1}{x^2\sqrt{x^2-1}} \mathrm dx$;
            \item[(12)]
            $\displaystyle \int \frac{\mathrm dx}{x^8(1+x^2)} \mathrm dx$.
        \end{enumerate}
    \end{multicols}
\end{problem}

\begin{enumerate}
    \item[(10)]
    \begin{solution}
        \begin{align*}
            \int \frac{x^{1/7}+x^{1/2}}{x^{8/7}+x^{1/14}} \mathrm dx &= 14 \int \frac{t^{19} + t^{14}}{t^{15} + 1} \mathrm dt \ (x = t^{14}) \\
            &= \frac {14} 5 \int \frac{u^2}{u^2-u+1} \mathrm du \ (u = t^5) \\
            &= \frac {14} 5 \int \mathrm du + \frac 7 5 \int \frac{2u - 1}{u^2-u+1} \mathrm du - \frac 7 5 \int \frac{\mathrm d \left( u - \frac 1 2 \right)}{\left( u-\frac 1 2 \right)^2 + \frac 3 4} \\
            &= \frac {14} 5 x^{5/14} + \ln\left( x^{5/7} - x^{5/14} + 1 \right) + \frac 2 {\sqrt 3} \arctan\left( \frac{2x^{5/14} - 1}{\sqrt 3} \right) + C.
        \end{align*}
    \end{solution}
    \item[(12)]
    \begin{solution}
        \begin{align*}
            \int \frac 1 {x^8(1+x^2)} \mathrm dx &= \int \left( \frac 1 {x^8} - \frac 1 {x^6} + \frac 1 {x^4} - \frac 1 {x^2} + \frac 1 {1+x^2} \right) \mathrm dx \\
            &= - \frac 1 {7x^7} + \frac 1 {5x^5} - \frac 1 {3x^3} + \frac 1 x + \arctan x + C.
        \end{align*}
    \end{solution}
\end{enumerate}

\begin{problem}{MAblue}{3.4.7}
    求下列不定积分.
    \begin{multicols}{2}
        \begin{enumerate}
            \item[(1)]
            $\displaystyle \int \frac{\mathrm dx}{1+e^x}$;
            \item[(2)]
            $\displaystyle \int \frac{x^2-1}{x^4+x^2+1} \mathrm dx$;
            \item[(3)]
            $\displaystyle \int \frac{\mathrm dx}{x^4+x^6}$;
            \item[(4)]
            $\displaystyle \int x \sqrt{x-2} \mathrm dx$;
            \item[(5)]
            $\displaystyle \int \frac {\sqrt{x-1}\arctan\sqrt{x-1}} x \mathrm dx$;
            \item[(6)]
            $\displaystyle \int \frac{xe^x}{\sqrt{e^x-2}} \mathrm dx$;
            \item[(7)]
            $\displaystyle \int xe^x \sin x \mathrm dx$;
            \item[(8)]
            $\displaystyle \int \frac{\mathrm dx}{(1+\tan x)\sin^2 x}$;
            \item[(9)]
            $\displaystyle \int \frac{\sqrt{1-x}}{1-\sqrt x} \mathrm dx$;
            \item[(10)]
            $\displaystyle \int \sqrt{\frac{x-1}{x+1}} \cdot \frac{\mathrm dx}{x^2}$;
            \item[(11)]
            $\displaystyle \int \frac{x \arctan x}{(1+x^2)^3} \mathrm dx$;
            \item[(12)]
            $\displaystyle \int \frac x {1+\sin x} \mathrm dx$;
            \item[(13)]
            $\displaystyle \int \arcsin \sqrt x \mathrm dx$;
            \item[(14)]
            $\displaystyle \int \frac{x + \sin x}{1 + \cos x} \mathrm dx$;
            \item[(15)]
            $\displaystyle \int x \sin^2 x \mathrm dx$;
            \item[(16)]
            $\displaystyle \int \frac{x^3}{\sqrt{1+x^2}} \mathrm dx$;
            \item[(17)]
            $\displaystyle \int \frac{\arctan x}{x^2 (1+x^2)} \mathrm dx$;
            \item[(18)]
            $\displaystyle \int \frac{\arctan e^x}{e^x} \mathrm dx$;
            \item[(19)]
            $\displaystyle \int e^{2x} (1+\tan x)^2 \mathrm dx$;
            \item[(20)]
            $\displaystyle \int \frac{x^2}{(x \sin x + \cos x)^2} \mathrm dx$;
            \item[(21)]
            $\displaystyle \int \cos x \cos 2x \cos 3x \mathrm dx$;
            \item[(22)]
            $\displaystyle \int \frac{\mathrm dx}{\sqrt{x-1}+\sqrt{x+1}}$;
            \item[(23)]
            $\displaystyle \int \frac{\mathrm dx}{\sqrt{\sqrt x - 1}}$;
            \item[(24)]
            $\displaystyle \int \sqrt{\frac x {1-x\sqrt x}} \mathrm dx$;
            \item[(25)]
            $\displaystyle \int e^{-\frac {x^2} 2} \frac{\cos x - 2x \sin x}{2 \sqrt{\sin x}} \mathrm dx$;
            \item[(26)]
            $\displaystyle \int \frac{xe^x}{(1+x)^2} \mathrm dx$.
        \end{enumerate}
    \end{multicols}
\end{problem}

\begin{enumerate}
    \item[(26)]
    \begin{solution}
        \[
            \int \frac{xe^x}{(1+x)^2} \mathrm dx = -\frac{xe^x}{1+x} + \int e^x \mathrm dx = \frac{e^x}{1+x} + C.
        \]
    \end{solution}
\end{enumerate}

\section{有理函数的不定积分}

\begin{problem}{MAblue}{4.2.1}
    求下列有理函数的不定积分.
    \begin{multicols}{2}
        \begin{enumerate}
            \item[(1)]
            $\displaystyle \int \frac{\mathrm dx}{x^2 + x - 2}$;
            \item[(2)]
            $\displaystyle \int \frac{x^4}{x^2 + 1} \mathrm dx$;
            \item[(3)]
            $\displaystyle \int \frac{x^3 + 1}{x^3 - x} \mathrm dx$;
            \item[(4)]
            $\displaystyle \int \frac{\mathrm dx}{(x^2+1)(x^2+x)}$;
            \item[(5)]
            $\displaystyle \int \frac x {(x+1)^2(x^2+x+1)} \mathrm dx$;
            \item[(6)]
            $\displaystyle \int \frac{x^2+1}{x^4+1} \mathrm dx$;
            \item[(7)]
            $\displaystyle \int \frac{x^5-x}{x^8+1} \mathrm dx$;
            \item[(8)]
            $\displaystyle \int \frac{x^{15}}{(x^8+1)^2} \mathrm dx$.
        \end{enumerate}
    \end{multicols}
\end{problem}

\begin{enumerate}
    \item[(8)]
    \begin{solution}
        \[
            \int \frac{x^{15}}{(x^8+1)^2} \mathrm dx = \frac 1 8 \int \frac{x^8}{(x^8+1)^2} \mathrm d(x^8) = \frac 1 8 \ln(x^8+1) + \frac 1 {8(x^8+1)} + C.
        \]
    \end{solution}
\end{enumerate}

\chapter{单变量函数的积分学}

\section{积分}

\begin{problem}{MAblue}{5.1.2}
    证明:Dirchlet 函数在任意区间 $[a, b]$ 上不可积.(因此有界的函数未必可积.)
\end{problem}

\begin{proof}
    考虑其 Riemann 和. 对任意 $x_{i-1} < x_i \ (i = 1, 2, \ldots, n)$,总能找到一组 $\xi_i \in (x_{i-1}, x_i)$ 使得 $f(\xi_i) \equiv 0$ 或 $f(\xi_i) \equiv 1$,故 $\lim_{\Vert T \Vert \to 0} S_n(T)$ 不存在.
\end{proof}

\begin{problem}{MAblue}{5.1.3}
    举例说明,一个函数的绝对值在 $[a, b]$ 上可积,不能保证该函数在 $[a, b]$ 上可积.(提示:适当地修改 Dirchlet 函数可得出这样的例子. 比较习题 2.1 中第 5 题.)
\end{problem}

\begin{proof}
    取 $f(x) = 2D(x) - 1$,其中 $D(x)$ 为 Dirchlet 函数.
\end{proof}

\begin{problem}{MAblue}{5.1.9}
    \begin{enumerate}
        \item[(1)]
        设函数 $f(x)$ 在 $[a, b]$ 上连续,而 $g(x)$ 在 $[a, b]$ 上可积且是非负(或非正)的,证明:存在 $\xi \in[a, b]$,使得
        \[
            \int_a^b f(x)g(x) \mathrm dx = f(\xi) \int_a^b g(x) \mathrm dx.
        \]
        \item[(2)]
        举例说明,(1) 中对于函数 $g(x)$ 的假设是必不可少的.(提示:在 $[-1, 1]$ 上,取 $f(x) = g(x) = x$.)
    \end{enumerate}
\end{problem}

\begin{enumerate}
    \item[(1)]
    \begin{proof}
        由题设知存在常数 $m, M$ 满足 $m \leqslant f(x) \leqslant M$. 由 $g(x) \geqslant 0$ 知
        \[
            m \int_a^b g(x) \mathrm dx \leqslant \int_a^b f(x)g(x) \mathrm dx \leqslant M \int_a^b g(x) \mathrm dx
        \]
        即存在 $\lambda \in [m, M]$ 使得 $\int_a^b f(x)g(x) \mathrm dx = \lambda \int_a^b g(x) \mathrm dx$. 又 $f(x)$ 连续,则存在 $\xi \in [a, b]$ 使得 $f(\xi) = \lambda$.
    \end{proof}
    \item[(2)]
    \begin{proof}
        令 $f(x) = g(x) = x$,则 $\int_{-1}^1 f(x)g(x) \mathrm dx = 1$,而 $\int_{-1}^1 g(x) \mathrm dx = 0$. 故不存在满足条件的 $\xi$.
    \end{proof}
\end{enumerate}

\begin{problem}{MAblue}{5.1.13}
    设函数 $f(x)$ 处处连续. 记 $F(x) = \int_0^x xf(t) \mathrm dt$,求 $F'(x)$.
\end{problem}

\begin{solution}
    设 $\varphi(x) = \int_0^x f(t) \mathrm dt$,则
    \[
        F'(x) = \left(\int_0^x xf(t) \mathrm dt \right)' = \left( x \varphi(x) \right)' = xf(x) + \varphi(x).
    \]
\end{solution}

\begin{problem}{MAblue}{5.1.18}
    求下列极限.
    \begin{enumerate}
        \item[(1)]
        $\displaystyle \lim_{x \to 0} \left( \frac 1 {x^4} \int_0^x \sin t^3 \mathrm dt \right)$;
        \item[(2)]
        $\displaystyle \lim_{x \to 0} \frac 1 {\sin^3 x} \int_0^{\tan x} \arcsin t^2 \mathrm dt$;
        \item[(3)]
        $\displaystyle \lim_{n \to \infty} \left( \frac 1 {\sqrt{n^2}} + \frac 1 {\sqrt{n^2-1^2}} + \cdots + \frac 1 {\sqrt{n^2-(n-1)^2}} \right)$;
        \item[(4)]
        $\displaystyle \lim_{n \to \infty} \frac{1^p + 2^p + \cdots + n^p}{n^{p+1}}$,其中 $p$ 是正常数.
    \end{enumerate}
\end{problem}

\begin{enumerate}
    \item[(1)]
    \begin{solution}
        由积分中值定理知,存在 $\xi \in (0, x)$,使得 $\int_0^x \sin t^3 \mathrm dt = x \sin \xi^3$. 故
        \[
            \lim_{x \to 0} \left( \frac 1 {x^4} \int_0^x \sin t^3 \mathrm dt \right) = \lim_{x \to 0} \frac{\sin \xi^3}{x^3} = 1.
        \]
    \end{solution}
    \item[(2)]
    \begin{solution}
        由积分中值定理知,存在 $\xi \in (0, \tan x)$,使得 $\int_0^{\tan x} \arcsin t^2 \mathrm dt = \tan x \arcsin \xi^2$. 故
        \[
            \lim_{x \to 0} \frac 1 {\sin^3 x} \int_0^{\tan x} \arcsin t^2 \mathrm dt = \lim_{x \to 0} \frac{\arcsin \xi^2}{\sin^2 x \cos x} = 1.
        \]
    \end{solution}
    \item[(3)]
    \begin{solution}
        \[
            \lim_{n \to \infty} \left( \sum_{k=0}^{n-1} \frac 1 {\sqrt{n^2 - k^2}} \right) = \int_0^1 \frac 1 {\sqrt{1 - x^2}} = \frac \pi 2.
        \]
    \end{solution}
    \item[(4)]
    \begin{solution}
        \[
            \lim_{n \to \infty} \frac{1^p + 2^p + \cdots + n^p}{n^{p+1}} = \int_0^1 x^p = \frac 1 {1+p}.
        \]
    \end{solution}
\end{enumerate}

\begin{problem}{MAblue}{5.1.19}
    求下列极限.
    \begin{enumerate}
        \item[(1)]
        $\displaystyle \lim_{n \to \infty} \int_a^b e^{-nx^2} \mathrm dx$,其中 $a, b$ 为常数,且 $0 < a < b$;
        \item[(2)]
        $\displaystyle \lim_{n \to \infty} \int_0^1 \frac{x^n}{1+x} \mathrm dx$;
        \item[(3)]
        $\displaystyle \lim_{n \to \infty} \int_n^{n+a} \frac {\sin x} x \mathrm dx$,其中 $a$ 为正常数;
    \end{enumerate}
\end{problem}

\begin{enumerate}
    \item[(2)]
    \begin{solution}
        易知 $0 < \frac{x^n}{1+x} < x^n$. 又
        \[
            \lim_{n \to \infty} \int_0^1 x^n \mathrm dx = \lim_{n \to \infty} \left. \frac{x^{n+1}}{n+1} \right|_0^1 = \lim_{n \to \infty} \frac 1 {n+1} = 0.
        \]
        故由夹逼原理知
        \[
            \lim_{n \to \infty} \int_0^1 \frac{x^n}{1+x} \mathrm dx = 0.
        \]
    \end{solution}
\end{enumerate}

\begin{problem}{MAblue}{5.1.22}
    计算下面的积分.
    \begin{multicols}{2}
        \begin{enumerate}
            \item[(1)]
            $\displaystyle \int_0^{2\pi} |\cos x| \mathrm dx$;
            \item[(2)]
            $\displaystyle \int_{-3}^4 [x] \mathrm dx$;
            \item[(3)]
            $\displaystyle \int_{-1}^1 \cos x \ln \frac{1+x}{1-x} \mathrm dx$;
            \item[(4)]
            $\displaystyle \int_{-\frac \pi 2}^{\frac \pi 2} \frac{\cos^3 x}{1+e^x} \mathrm dx$;
            \item[(5)]
            $\displaystyle \int_0^{\ln 2} \sqrt{1-e^{-2x}} \mathrm dx$;
            \item[(6)]
            $\displaystyle \int_0^1 x\arcsin x \mathrm dx$;
            \item[(7)]
            $\displaystyle \int_0^1 x^3e^x \mathrm dx$;
            \item[(8)]
            $\displaystyle \int_0^a \frac{\mathrm dx}{x+\sqrt{a^2-x^2}}$;
            \item[(9)]
            $\displaystyle \int_0^{\frac \pi 4} \sqrt{\tan x} \mathrm dx$;
            \item[(10)]
            $\displaystyle \int_0^{\frac \pi 2} \frac{\mathrm dx}{a^2\sin^2 x + b^2\cos^2 x}$;
            \item[(11)]
            $\displaystyle \int_{-1}^1 x^4 \sqrt{1-x^2} \mathrm dx$;
            \item[(12)]
            $\displaystyle \int_0^{2\pi} \sin^6 x \mathrm dx$;
            \item[(13)]
            $\displaystyle \int_{-1}^1 e^{|x|} \arctan e^x \mathrm dx$;
            \item[(14)]
            $\displaystyle \int_0^\pi \frac{\sec^2x}{2+\tan^2x} \mathrm dx$;
        \end{enumerate}
    \end{multicols}
\end{problem}

\begin{enumerate}
    \item[(14)]
    \begin{solution}
        由
        \[
            \int \frac{\sec^2x}{2+\tan^2x} \mathrm dx = \int \frac{\mathrm d(\tan x)}{2+\tan^2x} = \frac 1 {\sqrt 2} \arctan \frac{\tan x}{\sqrt 2}
        \]
        知
        {\small
            \[
                \int_0^\pi \frac{\sec^2x}{2+\tan^2x} \mathrm dx = \lim_{a \to \frac \pi 2 - 0} \left. \frac 1 {\sqrt 2} \arctan \frac{\tan x}{\sqrt 2} \right|_0^a + \lim_{a \to \frac \pi 2 + 0} \left. \frac 1 {\sqrt 2} \arctan \frac{\tan x}{\sqrt 2} \right|_a^\pi = \frac \pi {\sqrt 2}.
            \]
        }
    \end{solution}
\end{enumerate}

\begin{problem}{MAblue}{5.1.23}
    设 $f(x)$ 在 $[0, 1]$ 上连续,证明:
    \[
        \int_0^\pi x f(\sin x) \mathrm dx = \pi \int_0^{\frac\pi 2} f(\sin x) \mathrm dx,
    \]
    并用这一结果计算 $\int_0^\pi \frac{x\sin x}{1+\cos^2x} \mathrm dx$.
\end{problem}

\begin{proof}
    易知
    \begin{align*}
        \int_0^{\pi} x f(\sin x) \mathrm dx &= \int_0^{\frac \pi 2} x f(\sin x) \mathrm dx + \int_{\frac \pi 2}^{\pi} x f(\sin x) \mathrm dx \\
        &= \int_0^{\frac \pi 2} x f(\sin x) \mathrm dx + \int_0^{\frac \pi 2} (\pi - x) f( \sin(\pi - x) ) \mathrm dx \\
        &= \pi \int_0^{\frac \pi 2} f(\sin x) \mathrm dx. \qedhere
    \end{align*}
\end{proof}
{\flushleft 故}
\[
    \int_0^{\pi} \frac{x \sin x}{1 + \cos^2 x} \mathrm dx = \pi \int_0^{\frac \pi 2} \frac{\sin x}{1 + \cos^2 x} = -\arctan(\cos x) \bigg|_0^{\frac \pi 2} = \frac \pi 4.
\]

\begin{problem}{MAblue}{5.1.24}
    证明:
    \[
        \frac 1 6 < \int_0^1 \sin x^2 \mathrm dx < \frac 1 3.
    \]
\end{problem}

\begin{proof}
    易知当 $0 \leqslant x \leqslant \frac \pi 2$ 时有 $\frac {x^2} 2 < \frac 2 \pi x^2 \leqslant \sin x^2 \leqslant x^2$,在 $[0, 1]$ 上积分即得.
\end{proof}

\begin{problem}{MAblue}{5.1.27}
    \begin{enumerate}
        \item[(1)]
        设 $f(x)$ 是 $[0, 1]$ 上单调递减的连续函数,证明:对任意 $\alpha \in (0, 1)$,有
        \[
            \int_0^\alpha f(x) \mathrm dx \geqslant \alpha \int_0^1 f(x) \mathrm dx.
        \]
        \item[(2)]
        若仅假设 $f(x)$ 在 $[0, 1]$ 上单调递减,证明同样的结论.
    \end{enumerate}
\end{problem}

\begin{proof}
    由 $f(x)$ 单调递减有
    \[
        (1 - \alpha) \int_0^\alpha f(x) \geqslant \alpha (1 - \alpha) f(\alpha) \geqslant \alpha \int_\alpha^1 f(x)
    \]
    整理后即得.
\end{proof}

\phantomsection\label{item:28}
\begin{problem}{MAblue}{5.1.28}
    设函数 $f(x)$ 在 $[a, b]$ 上可微,且对任意 $x \in [a, b]$ 有 $|f'(x)| \leqslant M$.
    \begin{enumerate}
        \item[(1)]
        若 $f(a) = 0$,证明:
        \[
            \int_a^b |f(x)| \mathrm dx \leqslant \frac M 2 (b-a)^2.
        \]
        \item[(2)]
        若 $f(a) = f(b) = 0$,证明:
        \[
            \int_a^b |f(x)| \mathrm dx \leqslant \frac M 4 (b-a)^2.
        \]
    \end{enumerate}
    (提示:通过对积分的上限求导能得出 (1) 的一个证明,即考虑函数 $G(t) = \int_a^t |f(x)| \mathrm dx - \frac M 2 (t-a)^2,\ a \leqslant t \leqslant b$.)
\end{problem}
\begin{enumerate}
    \item[(1)]
    \begin{proof}
        依提示设 $G(t)$,则 $G(a) = 0,\ G(b) = \int_a^b |f(x)| \mathrm dx - \frac M 2 (b-a)^2$,而
        \[
            G'(t) = |f(t)| - M(t - a).
        \]
        又由 Lagrange 中值定理知存在 $\xi \in (a, t)$ 使得
        \[
            \frac{|f(t)|}{t - a} = \frac{|f(t)| - |f(a)|}{t - a} = |f'(\xi)| \leqslant M.
        \]
        故 $G'(t) \leqslant 0$,也即 $G(b) \leqslant G(a) = 0$. 原命题得证.
    \end{proof}
    \item[(2)]
    \begin{proof}
        由 (1) 有
        \[
            \int_a^b |f(x)| \mathrm dx = \int_a^{\frac {a+b} 2} |f(x)| \mathrm dx + \int_{\frac {a+b} 2}^b |f(x)| \mathrm dx \leqslant 2 \cdot \frac M 2 \cdot (\frac {b-a} 2)^2 = \frac M 4 (b-a)^2. \qedhere
        \]
    \end{proof}
    
\end{enumerate}

\section{函数的可积性}

\begin{problem}{MAblue}{5.2.3}
    证明:如果函数 $f(x)$ 在区间 $[a, b]$ 上可积,则 $|f(x)|$ 在区间 $[a, b]$ 上也可积,而且有
    \[
        \left| \int_a^b f(x) \mathrm dx \right| \leqslant \int_a^b |f(x)| \mathrm dx.
    \]
\end{problem}

\begin{proof}
    由题设知对任意划分,有
    \[
        \lim_{\Vert T \Vert \to 0} \sum_{i=1}^n \omega_i \Delta x_i = 0,
    \]
    设 $\omega'_i$ 为 $|f(x)|$ 在 $[x_{i-1}, x_i]$ 上的振幅,则
    \[
        0 < \omega'_i = |f(\xi_i)|_{\max} - |f(\xi_i)|_{\min} < |f(\xi_i)_{\max} - f(\xi_i)_{\min}| = \omega_i.
    \]
    故由夹逼原理知 $\lim_{\Vert T \Vert \to 0} \sum_{i=1}^n \omega'_i \Delta x_i = 0$,即 $|f(x)|$ 可积. 故
    \[
        -\int_a^b |f(x)| \mathrm dx \leqslant \int_a^b f(x) \mathrm dx \leqslant \int_a^b |f(x)| \mathrm dx.
    \]
    也即
    \[
        \left| \int_a^b f(x) \mathrm dx \right| \leqslant \int_a^b |f(x)| \mathrm dx. \qedhere
    \]
\end{proof}

\section{积分的应用}

\begin{problem}{MAblue}{5.3.4}
    求证:以 $R$ 为半径,高为 $h$ 的球缺的体积为 $\pi h^2 \left( R - \frac h 3 \right)$.
\end{problem}

\begin{proof}
    易知
    \[
        V = \int_0^h \pi \left( R^2 - (R-h+x)^2 \right) \mathrm dx = \pi h^2 \left( R - \frac h 3 \right). \qedhere
    \]
\end{proof}

\section{广义积分}

\begin{problem}{MAblue}{5.4.1}
    判断下列广义积分是否收敛,并求出收敛的广义积分的值. 以下 $n$ 均为自然数.
    \begin{multicols}{2}
        \begin{enumerate}
            \item[(1)]
            $\displaystyle \int_0^{+\infty} xe^{-x^2} \mathrm dx$;
            \item[(2)]
            $\displaystyle \int_0^{+\infty} x \sin x \mathrm dx$;
            \item[(3)]
            $\displaystyle \int_2^{+\infty} \frac {\ln x} x \mathrm dx$;
            \item[(4)]
            $\displaystyle \int_1^{+\infty} \frac {\arcsin x} x \mathrm dx$;
            \item[(5)]
            $\displaystyle \int_0^{+\infty} e^{-x} \sin x \mathrm dx$;
            \item[(6)]
            $\displaystyle \int_{-\infty}^{+\infty} \frac{\mathrm dx}{x^2+2x+2}$;
            \item[(7)]
            $\displaystyle \int_0^1 \ln x \mathrm dx$;
            \item[(8)]
            $\displaystyle \int_{-1}^1 \frac{\mathrm dx}{\sqrt{1-x^2}}$;
            \item[(9)]
            $\displaystyle \int_0^1 \frac{x\ln x}{(1-x^2)^{3/2}} \mathrm dx$;
            \item[(10)]
            $\displaystyle \int_0^1 \ln \frac 1 {1-x^2} \mathrm dx$;
            \item[(11)]
            $\displaystyle \int_0^{+\infty} x^ne^{-x} \mathrm dx$;
            \item[(12)]
            $\displaystyle \int_0^1 (\ln x)^n \mathrm dx$.
        \end{enumerate}
    \end{multicols}
\end{problem}

\begin{enumerate}
    \item[(12)]
    \begin{solution}
        \begin{align*}
            \int_0^1 (\ln x)^n \mathrm dx &= (-1)^{n+1} \int_0^{+\infty} t^n e^{-t} \mathrm dt \ (x = e^{-t}) \\
            &= (-1)^{n+1}\Gamma(n+1). \qedhere
        \end{align*}
    \end{solution}
\end{enumerate}

\begin{problem}{MAblue}{5.4.2}
    广义积分 $\int_{-\infty}^{+\infty} f(x) \mathrm dx$ 的柯西主值定义为
    \[
        P.V. \int_{-\infty}^{+\infty} f(x) \mathrm dx = \lim_{b \to +\infty} \int_{-b}^b f(x) \mathrm dx.
    \]
    显然,若广义积分 $\int_{-\infty}^{+\infty} f(x) \mathrm dx$ 收敛,则其主值也收敛,但反过来不一定成立. 研究下列广义积分主值的收敛性.
    \begin{multicols}{2}
        \begin{enumerate}
            \item[(1)]
            $\displaystyle P.V. \int_{-\infty}^{+\infty} \frac x {1+x^2} \mathrm dx$;
            \item[(2)]
            $\displaystyle P.V. \int_{-\infty}^{+\infty} \frac{|x|}{1+x^2} \mathrm dx$.
        \end{enumerate}
    \end{multicols}
\end{problem}

\begin{enumerate}
    \item[(1)]
    \begin{solution}
        由 $\frac x {1+x^2}$ 的奇性知 $\int_{-b}^b \frac x {1+x^2} \mathrm dx \equiv 0$. 然而
        \[
            \int_{-\infty}^{+\infty} \frac x {1+x^2} \mathrm dx = \int_{-\infty}^0 \frac x {1+x^2} \mathrm dx + \int_0^{+\infty} \frac x {1+x^2} \mathrm dx
        \]
        等号右侧两积分均发散,故原积分发散.
    \end{solution}
\end{enumerate}

\begin{problem}{MAblue}{5.4.3}
    若函数 $f(x)$ 在 $(a, +\infty)$ 上连续,并且以 $a$ 为瑕点,则广义积分 $\int_a^{+\infty} f(x) \mathrm dx$ 定义为
    \[
        \int_a^{+\infty} f(x) \mathrm dx = \int_a^b f(x) \mathrm dx + \int_b^{+\infty} f(x) \mathrm dx,
    \]
    这是本节讲的两类广义积分的组合,其中 $b > a$ 是任一个实数. 当上面两个广义积分都收敛时,我们称 $\int_a^{+\infty} f(x) \mathrm dx$ 收敛;否则称为发散.
    \begin{enumerate}
        \item[(1)]
        证明 $\displaystyle \int_1^{+\infty} \frac{\mathrm dx}{x\sqrt{x-1}}$ 收敛,并求其值;
        \item[(2)]
        证明,对任意实数 $\alpha$,$\displaystyle \int_0^{+\infty} \frac{\mathrm dx}{x^\alpha}$ 发散.
    \end{enumerate}
\end{problem}

\begin{enumerate}
    \item[(2)]
    \begin{proof}
        \leavevmode
        \begin{enumerate}
            \item[(a)]
            当 $\alpha = 1$ 时
            \[
                \int_0^{+\infty} \frac {\mathrm dx} x = \ln x \bigg|_0^{+\infty}
            \]
            发散.
            \item[(b)]
            当 $\alpha \neq 1$ 时
            \[
                \int_0^{+\infty} \frac 1 {x^\alpha} = \left. \frac{x^{1-\alpha}}{1-\alpha} \right|_0^{+\infty}
            \]
            亦发散.
        \end{enumerate}
        综上,无论 $\alpha$ 取何值,$\int_0^{+\infty} \frac{\mathrm dx}{x^{\alpha}} $ 必发散.
    \end{proof}
\end{enumerate}

\section*{第 5 章综合习题}
\addcontentsline{toc}{section}{第 5 章综合习题}

\begin{problem}{MAblue}{5.0.1}
    设 $m, n$ 为正整数,证明
    \begin{enumerate}
        \item[(1)]
        $\displaystyle \int_0^{2\pi} \sin mx \cdot \cos nx \mathrm dx = 0$;
        \item[(2)]
        $\displaystyle \int_0^{2\pi} \sin mx \cdot \sin nx \mathrm dx = \int_0^{2\pi} \cos mx \cdot \cos nx \mathrm dx = \begin{cases}
            \pi, & m = n; \\
            0, & m \neq n.
        \end{cases}$
    \end{enumerate}
\end{problem}

\begin{enumerate}
    \item[(1)]
    \begin{proof}
        设 $t = 2 \pi - x$,则
        \[
            \int_0^{2 \pi} \sin mx \cdot \cos nx \mathrm dx = \int_0^\pi \sin mx \cdot \cos nx \mathrm dx - \int_0^\pi \sin mt \cdot \cos nt \mathrm dt = 0. \qedhere
        \]
    \end{proof}
    \item[(2)]
    \begin{proof}
        易知
        \begin{align*}
            \int_0^{2\pi} \sin mx \cdot \sin nx &= \frac 1 2 \int_0^{2\pi} \left( \cos( mx - nx ) - \cos( mx + nx ) \right) \mathrm dx \\
            &= \frac 1 2 \int_0^{2\pi} \cos \left( (m - n)x \right) \mathrm dx - \frac 1 2 \int_0^{2\pi} \cos \left( (m + n)x \right) \mathrm dx,
        \end{align*}
        则结论显然.
    \end{proof}
\end{enumerate}

\begin{problem}{MAblue}{5.0.2}
    设 $m, n$ 为正整数,记
    \[
        B(m, n) = \int_0^1 x^m(1-x)^n \mathrm dx.
    \]
    证明:(1) $B(m, n) = B(n, m)$;\quad(2) $B(m, n) = \frac{m!n!}{(m+n+1)!}$.
\end{problem}

\begin{proof}
    (1) 是显然的,下证 (2). 易知有
    \[
        B(m+1, n-1) = B(m, n-1) - B(m, n),
    \]
    则
    {\small
        \[
            B(m, n) = \int_0^1 x^m (1-x)^n \mathrm dx = 0 + \frac n {m+1} \int_0^1 x^{m+1} (1-x)^{n-1} \mathrm dx = \frac{n (B(m, n-1) - B(m, n))}{m+1},
        \]
    }
    {\flushleft 即 $B(m, n) = \frac n {m+n+1} B(m, n-1)$. 又知 $B(0, 0) = 1$,则}
    \[
        B(m, n) = \frac{n!}{(m+n+1)(m+n) \cdots (m+2)} \cdot \frac{m!}{(m+1)!} = \frac{n!m!}{(m+n+1)!},
    \]
    在 $m, n \geqslant 0$ 时成立.
\end{proof}

\begin{problem}{MAblue}{5.0.3}
    计算下列积分:
    \begin{enumerate}
        \item[(a)]
        $\displaystyle \int_{\frac 1 2}^2 \left( 1 + x - \frac 1 x \right)e^{x+\frac 1 x} \mathrm dx$;
        \item[(b)]
        $\displaystyle \int_0^{n\pi} x|\sin x| \mathrm dx$,其中 $n$ 为自然数;
        \item[(c)]
        设 $\displaystyle f(x) = \int_x^{x+2\pi} \left( 1 + e^{\sin t} - e^{-\sin t} \right) \mathrm dt + \frac 1 {1+x} \int_0^1 f(t) \mathrm dt$,求 $\displaystyle \int_0^1 f(x) \mathrm dx$;
    \end{enumerate}
\end{problem}

\begin{enumerate}
    \item[(c)]
    \begin{solution}
        设 $\varphi(x) = e^{\sin x} - e^{-\sin x}$,则有 $\varphi(x) + \varphi(x+\pi) = 0$,则
        \[
            \int_x^{x+2\pi} (1 + \varphi(t)) \mathrm dt = \int_x^{x+\pi} (1 + \varphi(t) + 1 + \varphi(t+\pi)) \mathrm dt = 2\pi.
        \]
        故 $\int_0^1 f(x) \mathrm dx = (1 + x)(f(x) - 2\pi)$.
    \end{solution}
\end{enumerate}

\begin{problem}{MAblue}{5.0.4}
    证明:
    \[
        \frac 1 {2n+2} < \int_0^{\frac \pi 4} \tan^n x \mathrm dx < \frac 1 {2n} \ (n = 1, 2, \cdots).
    \]
\end{problem}

\begin{proof}
    设 $x = \arctan t$,则
    \[
        \int_0^{\frac \pi 4} \tan^n x \mathrm dx = \int_0^1 \frac{t^n}{1+t^2} \mathrm dt,
    \]
    而
    \[
        \frac 1 {2n+2} = \int_0^1 \frac {t^n} 2 \mathrm dt < \int_0^1 \frac{t^n}{1+t^2} \mathrm dt < \int_0^1 \frac{t^n}{2t} \mathrm dt = \frac 1 {2n},
    \]
    则命题得证.
\end{proof}

\begin{problem}{MAblue}{5.0.10}
    设 $f(x)$ 处处连续,$f(0) = 0$,且 $f'(0)$ 存在. 记 $F(x) = \int_0^x f(xy) \mathrm dy$,证明 $F(x)$ 处处可导,并求出 $F'(x)$.
\end{problem}

\begin{proof}
    显然 $F(0) = 0$. 当 $x \neq 0$ 时,设 $\varphi(x) = \int f(x) \mathrm dx$,则
    \[
        F(x) = \left. \frac {\varphi(xy)} x \right|_{y=0}^1 = \frac {\varphi(x) - \varphi(0)} x.
    \]
    此时显然有 $F'(x) = \frac{xf(x) - \varphi(x) + \varphi(0)}{x^2}$. 又
    \[
        F'(0) = \lim_{x \to 0} \frac{F(x) - F(0)}{x - 0} = \lim_{x \to 0} \frac {f(x)} x = f'(0),
    \]
    则 $F(x)$ 处处可导.
\end{proof}

\begin{problem}{MAblue}{5.0.11}
    \begin{enumerate}
        \item[(1)]
        设
        \[
            f(x) = \begin{cases}
                e^{-x^2}, & |x| \leqslant 1; \\
                1, & |x| > 1.
            \end{cases}
        \]
        记 $F(x) = \int_0^x f(t) \mathrm dt$. 试研究 $F(x)$ 在哪些点可导.
        \item[(2)]
        设 $f(x) = \int_0^x \cos \frac 1 t \mathrm dt$,求证 $f'_+(0) = 0$.
    \end{enumerate}
\end{problem}

\begin{enumerate}
    \item[(2)]
    \begin{proof}
        \begin{align*}
            f'_+(0) &= \lim_{x \to 0_+} \frac{\int_{1/x}^{+\infty} \frac{\cos u}{u^2} \mathrm du - 0}{x - 0} \ \left( u = \frac 1 t \right) \\
            &= \lim_{x \to 0_+} \frac 1 x \left( \left. \frac{\sin u}{u^2} \right|_{\frac 1 x}^{+\infty} \right) + \frac 2 x \int_{\frac 1 x}^{+\infty} \frac{\sin u}{u^3} \mathrm du \\
            &= 2 \lim_{x \to 0_+} \frac{\int_0^x t \cos \frac 1 t}{x} = 2 \lim_{x \to 0_+} x \cos \frac 1 x = 0. \qedhere
        \end{align*}
    \end{proof}
\end{enumerate}

\begin{problem}{MAblue}{5.0.12}
    设函数 $f$ 处处连续. 证明
    \[
        \lim_{h \to 0} \frac 1 h \int_a^b \left( f(x+h)-f(x) \right) \mathrm dx = f(b) - f(a).
    \]
\end{problem}

\begin{proof}
    设 $\varphi(x) = \int f(x) \mathrm dx$,则
    \begin{align*}
        LHS &= \lim_{h \to 0} \frac 1 h \left( \varphi(b+h) - \varphi(b) - \varphi(a+h) + \varphi(a) \right) \\
        &= \lim_{h \to 0} \frac {\varphi(b+h) - \varphi(b)} h - \lim_{h \to 0} \frac {\varphi(a+h) - \varphi(a)} h \\
        &= f(b) - f(a). \qedhere
    \end{align*}
\end{proof}

\begin{problem}{MAblue}{5.0.13}
    设函数 $f(x)$ 在 $[a, b]$ 上连续可微. 证明:
    \[
        \lim_{\lambda \to \infty} \int_a^b f(x) \sin \lambda x \mathrm dx = 0.
    \]
    (提示:分部积分.)
\end{problem}

\begin{proof}
    \[
        \lim_{\lambda \to \infty} \int_a^b f(x) \sin(\lambda x) \mathrm dx = \lim_{\lambda \to \infty} \left( \left. \frac {f(x) \cos(\lambda x)} \lambda \right|_a^b - \frac 1 \lambda \int_a^b f'(x) \cos(\lambda x) \mathrm dx \right) = 0.
    \]
\end{proof}

\begin{problem}{MAblue}{5.0.14}
    证明:
    \[
        \lim_{x \to +\infty} \frac 1 x \int_0^x |\sin t| \mathrm dt = \frac 2 \pi.
    \]
\end{problem}

\begin{proof}
    \begin{align*}
        \lim_{x \to +\infty} \frac 1 x \int_0^{x} |\sin t| \mathrm dt &= \lim_{x \to +\infty} \frac 1 x \int_0^{k\pi} |\sin t| \mathrm dt + \lim_{x \to +\infty} \frac 1 x \int_{k\pi}^x |\sin t| \mathrm dt \ \left( k = \left\lfloor \frac x \pi \right\rfloor \right) \\
        &= \lim_{x \to \infty} \frac {2k} x + 0 = \frac 2 \pi. \qedhere
    \end{align*}
\end{proof}

\begin{problem}{MAblue}{5.0.16}
    设 $f(x)$ 是 $[a, b]$ 上的连续函数,且对任意 $x \in [a, b]$ 有 $f(x) \geqslant 0$. 记 $f(x)$ 在该区间上的最大值为 $M$,证明:
    \[
        \lim_{n \to \infty} \left( \int_a^b f^n(x) \mathrm dx \right)^{\frac 1 n} = M.
    \]
\end{problem}

\begin{proof}
    设 $f(x_0) = M$,对任意 $\varepsilon > 0$,存在 $\delta$ 使得 $f(x) > M - \varepsilon \ \left(x \in U(x_0, \delta) \right)$. 不妨设
    \[
        g_\delta(x) =
        \begin{cases}
            M - \varepsilon, & x \in U(x_0, \delta); \\
            0, & x \not\in U(x_0, \delta),
        \end{cases}
    \]
    则 $0 < g(x) < f(x) \leqslant M$. 故有
    \[
        M - \varepsilon = \lim_{n \to \infty} \left( \int_a^b g_\delta^n(x) \mathrm dx \right)^{\frac 1 n} \leqslant \lim_{n \to \infty} \left( \int_a^b f^n(x) \mathrm dx \right)^{\frac 1 n} \leqslant \lim_{n \to \infty} \left( \int_a^b M \mathrm dx \right)^{\frac 1 n} = M,
    \]
    也即 $\lim_{n \to \infty} \left( \int_a^b f^n(x) \mathrm dx \right)^{\frac 1 n} = M$.
\end{proof}

\begin{problem}{MAblue}{5.0.18}
    证明柯西积分不等式.
\end{problem}

\begin{center}
    \begin{minipage}{0.85\textwidth}
        \begin{theorem}{Cauchy–Schwarz Inequality}{}
            设 $f(x), g(x) \in C[a, b]$,则有
            \[
                \left( \int_a^b f^2(x) \mathrm dx \right) \left( \int_a^b g^2(x) \mathrm dx \right) \geqslant \left( \int_a^b f(x)g(x) \mathrm dx \right)^2.
            \]
            当且仅当 $g(x) \equiv 0$ 或存在实数 $\lambda$ 使得 $f(x) = \lambda g(x)$ 时,等号成立.
        \end{theorem}
    \end{minipage}
\end{center}

\begin{proof}
    若 $\int_a^b g^2(x) \mathrm dx = 0$,则 $g(x) \equiv 0$,显然成立;否则对任意实数 $\lambda$,有
    \[
        0 \leqslant \int_a^b \left( f(x) - \lambda g(x) \right)^2 \mathrm dx = \int_a^b f^2(x) \mathrm dx - 2\lambda \int_a^b f(x)g(x) \mathrm dx + \lambda^2 \int_a^b g^2(x) \mathrm dx.
    \]
    此为关于 $\lambda$ 的一元二次不等式,故有
    \[
        \Delta = 4 \left( \int_a^b f(x)g(x) \mathrm dx \right)^2 - 4 \left( \int_a^b f^2(x) \mathrm dx \right) \left( \int_a^b g^2(x) \mathrm dx \right) \leqslant 0.
    \]
    整理后即得.
\end{proof}

\begin{problem}{MAblue}{5.0.19}
    设 $f(x)$ 在 $[0, 1]$ 上有连续的导数,证明:对任意 $a \in [0, 1]$,有
    \[
        |f(a)| \leqslant \int_0^1 |f(x)| \mathrm dx + \int_0^1 |f'(x)| \mathrm dx.
    \]
\end{problem}

\begin{proof}
    设 $|f(x_0)| = |f(x)|_{\min}$,则
    \[
        |f(a)| = \left| f(x_0) + \int_{x_0}^a f'(x) \mathrm dx \right| \leqslant |f(x_0)| + \left| \int_{x_0}^a f'(x) \mathrm dx \right| \leqslant \int_0^1 |f(x)| \mathrm dx + \int_0^1 |f'(x)| \mathrm dx. \qedhere
    \]
\end{proof}

\begin{problem}{MAblue}{5.0.21}
    设 $f(x)$ 在区间 $[0, 1]$ 上连续可微,且 $|f'(x)| \leqslant M$,证明
    \[
        \left| \int_0^1 f(x) \mathrm dx = \frac 1 n \sum_{k=1}^n f \left( \frac k n \right) \right| \leqslant \frac M {2n}.
    \]
\end{problem}

\begin{proof}
    易知
    {\small
        \[
            \left| \int_0^1 f(x) \mathrm dx - \frac 1 n \sum_{k=1}^n f\left( \frac k n \right) \right| \leqslant \sum_{k=1}^n \left| \int_{\frac {k-1} n}^{\frac k n} f(x) \mathrm dx - \frac 1 n f\left( \frac k n \right) \right| = \sum_{k=1}^n \left| \int_{\frac {k-1} n}^{\frac k n} \left( f(x) - f\left( \frac k n \right) \right) \mathrm dx \right|,
        \]
    }
    {\flushleft 又由 \hyperref[item:28]{\textbf{5.1.28}} 知}
    \[
        \sum_{k=1}^n \left| \int_{\frac {k-1} n}^{\frac k n} \left( f(x) - f\left( \frac k n \right) \right) \mathrm dx \right| \leqslant \sum_{k=1}^n \frac M 2 (\frac k n - \frac {k-1} n)^2 = \frac M {2n},
    \]
    则命题得证.
\end{proof}

\begin{problem}{MAblue}{5.0.22}
    设 $f : \R \to (0, +\infty)$ 是一个可微函数,且对任意实数 $x, y$ 满足
    \[
        |f'(x)-f'(y)| \leqslant |x-y|.
    \]
    求证:对任意实数 $x$,有
    \[
        \left( f'(x) \right)^2 < 2f(x).
    \]
\end{problem}

\begin{proof}
    若 $f'(x) = 0$,则结论显然. 若 $f'(x) > 0$,设 $x_0 = x - f'(x)$,则有
    \[
        f(x) = \int_{x_0}^x f'(t) \mathrm dt + f(x_0) > \int_{x_0}^x f'(t) \mathrm dt \geqslant \int_{x_0}^x (f'(x) - (x-t)) \mathrm dt = \frac 1 2 (f'(x))^2.
    \]
    同理可证 $f'(x) < 0$ 的情况.
\end{proof}
{\flushleft 类似地,我们可以证明若 $|f'(x) - f'(y)| \leqslant L|x-y|$,则有 $(f'(x))^2 \leqslant 2L f(x)$.}

\chapter{常微分方程初步}

\section{一阶微分方程}

\begin{problem}{MAblue}{6.1.4}
    求下列线性方程和贝努利方程的解.
    \begin{multicols}{2}
        \begin{enumerate}
            \item[(1)]
            $(1+x^2)y' - 2xy = (1+x^2)^2$;
            \item[(2)]
            $y' + \frac {1-2x} x = 1$;
            \item[(3)]
            $y' = \frac y {x+y^3}$;
            \item[(4)]
            $y' + \frac y x = y^2 \ln x$;
            \item[(5)]
            $y' = y\tan x + y^2\cos x$;
            \item[(6)]
            $y - y'\cos x = y^2(1-\sin x)\cos x$.
        \end{enumerate}
    \end{multicols}
\end{problem}

\begin{enumerate}
    \item[(3)]
    \begin{solution}
        整理原式得
        \[
            \frac{\mathrm dx}{\mathrm dy} = \frac x y + y^2.
        \]
        若 $y \neq 0$,作代换 $u = \frac x y$,等式化为
        \[
            u + y \frac{\mathrm du}{\mathrm dy} = u + y^2 \ \left( u = \frac x y,\ y \neq 0 \right),
        \]
        则可进一步解得 $x = \frac 1 2 y^3 + Cy$. 验证可知 $y = 0$ 亦为方程的解.
    \end{solution}
\end{enumerate}

\begin{problem}{MAblue}{6.1.6}
    求解下列微分方程.
    \begin{multicols}{2}
        \begin{enumerate}
            \item[(1)]
            $y' + x = \sqrt{x^2 + y}$;
            \item[(2)]
            $y' = \cos(x-y)$;
            \item[(3)]
            $y' - e^{x-y} + e^x = 0$;
            \item[(4)]
            $y'\sin y + x\cos y + x = 0$.
        \end{enumerate}
    \end{multicols}
\end{problem}

\begin{enumerate}
    \item[(1)]
    \begin{solution}
        不妨设 $u = \frac y {x^2}$,则原式化为
        \[
            2xu + x^2 \frac{\mathrm du}{\mathrm dx} = x \left( \sqrt{1+u} - 1 \right),
        \]
        进一步整理得
        \[
            - \frac{\mathrm du}{2u - \sqrt{1+u} + 1} = \frac {\mathrm dx} x.
        \]
        再设 $v = \sqrt{1+u}$,化为
        \[
            - \frac{2v \mathrm dv}{(v-1)(2v+1)} = \frac {\mathrm dx} x,
        \]
        两边积分,得
        \[
            -\frac 1 3 \left( 2 \ln(v-1) + \ln(2v+1) \right) = \ln|x| + C,
        \]
        即
        \[
            (v-1)^2(2v+1) = \frac C {x^3}.
        \]
        则 $y$ 可解.
    \end{solution}
    或设 $u^2 = x^2 + y$,则此题亦可解.
\end{enumerate}

\begin{problem}{MAblue}{6.1.7}
    试用常数变易法导出贝努利方程的通解.
\end{problem}

\begin{solution}
    由定义有
    \begin{align*}
        \frac{\mathrm dy}{\mathrm dx} + P(x) y &= Q(x) y^n \ (n \not\in \{ 0, 1 \} ) \\
        \frac{\mathrm du}{\mathrm dx} + (1-n) P(x) u &= (1-n) Q(x) \ (u = y^{1-n}).
    \end{align*}
    设 $u = C(x) e^{(n-1) \int P(x) \mathrm dx}$,则
    \begin{align*}
        \frac{\mathrm d C(x)}{\mathrm dx} &= (1-n)Q(x)e^{(1-n) \int P(x) \mathrm dx} \\
        C(x) &= \int (1-n)Q(x)e^{(1-n) \int P(x) \mathrm dx} \mathrm dx + C.
    \end{align*}
    故 $u = e^{(n-1) \int P(x) \mathrm dx} \left( \int (1-n)Q(x)e^{(1-n) \int P(x) \mathrm dx} \mathrm dx + C \right)$,进而可得 $y$.
\end{solution}

\begin{problem}{MAblue}{6.1.13}
    求下列二阶方程满足初始条件的特解.
    \begin{enumerate}
        \item[(1)]
        $y'' = \frac {y'} x + \frac{x^2}{y'},\ y(1) = 1,\ y'(1) = 0$;
        \item[(2)]
        $y^3y'' = -1,\ y(1) = 1,\ y'(1) = 0$.
    \end{enumerate}
\end{problem}

\begin{enumerate}
    \item[(2)]
    \begin{solution}
        设 $p = y'$,则原式化为
        \[
            y^3 p \frac{\mathrm dp}{\mathrm dy} = -1,
        \]
        也即 $p \mathrm dp = - \frac{\mathrm dy}{y^3}$,易解得 $y = \sqrt{Cx^2 - \frac 1 C}$.
    \end{solution}
\end{enumerate}

\section{二阶线性微分方程}

\begin{problem}{MAblue}{6.2.1}
    在下列方程中,已知方程的一个特解 $y_1$,试求它们的通解.
    \begin{enumerate}[label={(\arabic*)}]
        \item $y'' + \frac 2 x y' + y = 0,\ y_1 = \frac {\sin x} x$;
        \item $y'' \sin^2x = 2y,\ y_1 = \cot x$;
        \item $(1-x^2)y'' - 2xy' + 2y = 0,\ y_1 = x$.
    \end{enumerate}
\end{problem}

\begin{enumerate}
    \item[(3)]
    \begin{solution}
        由题设知 $y_1(x) = x,\ p(x) = \frac{2x}{x^2 - 1}$,则
        \[
            y_2(x) = x \int \frac 1 {x^2} e^{\int_{x_0}^x \frac{2t}{t^2 - 1} \mathrm dt} \mathrm dx = \frac{x^2 + 1}{x_0^2 - 1}.
        \]
        故通解 $y_0(x) = c_1x + c_2 (x^2 + 1)$.
    \end{solution}
\end{enumerate}

\begin{problem}{MAblue}{6.2.2}
    先用观察法求下列齐次方程的一个非零特解,然后求方程的通解.
    \begin{enumerate}[label={(\arabic*)}]
        \item $x^2y'' - 2xy' + 2y = 0,\ x \neq 0$;
        \item $xy'' - (1+x)y' + y = 0,\ x \neq 0$.
    \end{enumerate}
\end{problem}

\begin{enumerate}
    \item[(2)]
    \begin{solution}
        注意到 $y = x + 1$ 为一特解. 原式变形得
        \begin{align*}
            y - y' &= x(y' - y'') \\
            \frac {\mathrm dx} x &= \frac{\mathrm d(y-y')}{y-y'},
        \end{align*}
        也即 $y - y' = Cx$,解得 $y = C(x+1)$,此即通解.
    \end{solution}
\end{enumerate}

\begin{problem}{MAblue}{6.2.3}
    已知方程 $(1+x^2)y'' + 2xy' - 6x^2 - 2 = 0$ 的一个特解 $y_1 = x^2$,试求该方程满足初始条件 $y(-1) = 0,\ y'(-1) = 0$ 的特解.
\end{problem}

\begin{solution}
    原方程对应的齐次方程为
    \[
        y'' + \frac{2x}{1 + x^2} y' = 0.
    \]
    解得其通解为 $y = c_1 \arctan x + c_2$,则原方程通解为 $y = x^2 + c_1 \arctan x + c_2$. 令 $y(-1) = 0,\ y'(-1) = 0$ 得特解 $y_2 = x^2 + 4\arctan x + \pi-1$.
\end{solution}

\begin{problem}{MAblue}{6.2.5}
    求下列常系数非齐次方程的一个特解.
    \begin{enumerate}[label={(\arabic*)}]
        \item $y'' + y = 2\sin\frac x 2$;
        \item $y'' - 6y' + 9y = (x+1)e^{2x}$.
    \end{enumerate}
\end{problem}

\begin{enumerate}
    \item[(2)]
    \begin{solution}
        易得对应齐次方程的通解为 $y = c_1 e^{3x} + c_2 x e^{3x}$. 则
        \[
            y_0(x) = \int_{x_0}^x \frac{e^{3t} \cdot xe^{3x} - te^{3t} \cdot e^{3x}}{W(t)} f(t) \mathrm dt
        \]
        进而可得原方程通解.
    \end{solution}
\end{enumerate}

\begin{problem}{MAblue}{6.2.9}
    求下列方程的通解.
    \begin{enumerate}[label={(\arabic*)}]
        \item $x''' + 3x'' + 3x' + x = 0$;
        \item $x''' - 2x'' + x' - 2x = 0$;
        \item $x^{(4)} - 8x'' + 18x = 0$;
        \item $x^{(4)} + 2x'' + x = 0$.
    \end{enumerate}
\end{problem}

\begin{enumerate}
    \item[(4)]
    \begin{solution}
        设 $x = e^{\lambda t}$,则得到特征方程 $\lambda^4 + 2\lambda^2 + 1 = 0$. 进而解得 $\lambda = \pm i$,故通解为 $x = c_1 \cos x + c_2 \sin x$.
    \end{solution}
    
\end{enumerate}

\chapter{无穷级数}

\section{数项级数}

\begin{problem}{MAblue}{7.1.2}
    \begin{multicols}{2}
        \begin{enumerate}[label={(\arabic*)}]
            \item $\displaystyle \sum_{n=1}^\infty \sqrt[n]{0.001}$;
            \item $\displaystyle \sum_{n=1}^\infty \frac 1 {n\sqrt{n-1}}$;
            \item $\displaystyle \sum_{n=1}^\infty \frac 1 {\sqrt{(2n-1)(2n+1)}}$;
            \item $\displaystyle \sum_{n=1}^\infty \sin n$;
            \item $\displaystyle \sum_{n=1}^\infty 2^n \frac \pi {3^n}$;
            \item $\displaystyle \sum_{n=1}^\infty \frac 1 {n\sqrt n}$;
            \item $\displaystyle \sum_{n=1}^\infty \frac 1 {\left( 2 + \frac 1 n \right)^n}$;
            \item $\displaystyle \sum_{n=1}^\infty \frac n {\left( n + \frac 1 n \right)^n}$;
            \item $\displaystyle \sum_{n=1}^\infty \arctan \frac \pi {4n}$;
            \item $\displaystyle \sum_{n=1}^\infty \frac{1000^n}{n!}$;
            \item $\displaystyle \sum_{n=1}^\infty \frac{(n!)^2}{(2n)!}$;
            \item $\displaystyle \sum_{n=1}^\infty \frac{3+(-1)^n}{2^n}$;
            \item $\displaystyle \sum_{n=1}^\infty \frac{\ln n}{\sqrt[4]{n^5}}$;
            \item $\displaystyle \sum_{n=2}^\infty \frac 1 {n(\ln n)^k}$;
            \item $\displaystyle \sum_{n=1}^\infty \left( \cos \frac 1 n \right)^{n^3}$;
            \item $\displaystyle \sum_{n=2}^\infty \left( \frac{an}{n+1} \right)^n,\ a > 0$.
        \end{enumerate}
    \end{multicols}
\end{problem}

\begin{enumerate}
    \item[(13)]
    \begin{solution}
        对任意 $0 < k < \frac 1 4$,存在 $N \in \N_+$ 使得当 $n \geqslant N$ 时有 $\ln n < n^k$. 故
        \[
            \sum_{n=N}^{\infty} \frac{\ln n}{\sqrt[4]{n^5}} < \sum_{n=N}^{\infty} \frac 1 {n^{\frac 5 4 - k}}.
        \]
        由比较审敛法知 $\sum_{n=1}^{\infty} \frac{\ln n}{\sqrt[4]{n^5}}$ 收敛.
    \end{solution}
    \item[(14)]
    \begin{solution}
        易知
        \[
            \int_3^{+\infty} \frac 1 {x \ln x (\ln \ln x)^k} = \begin{cases}
                \left. \ln \ln \ln x \right|_3^{+\infty} = +\infty, & k = 1; \\
                \left. \frac{(\ln \ln x)^{1-k}}{1-k} \right|_3^{+\infty} = +\infty, & k \neq 1.
            \end{cases}
        \]
        故由 Cauchy 积分判别法知级数 $\sum_{n=2}^{\infty} \frac 1 {n \ln n (\ln \ln n)^k}$ 发散.
    \end{solution}
    \item[(15)]
    \begin{solution}
        当 $n \to \infty$ 时,有
        \[
            \left( \cos \frac 1 n \right)^{n^3} \sim \left( 1 - \frac 1 {2n^2} \right)^{n^3} = \frac 1 {e^{\frac n 2}}.
        \]
        故原级数收敛.
    \end{solution}
    \item[(16)]
    \begin{solution}
        当 $n \to \infty$ 时,有
        \[
            \left( \frac{an}{n+1} \right)^n = \frac{a^n}{\left( 1 + \frac 1 n \right)^n} \sim \frac {a^n} e.
        \]
        故 $a \geqslant 1$ 时原级数发散,$a < 1$ 时原级数收敛.
    \end{solution}
\end{enumerate}

\begin{problem}{MAblue}{7.1.4}
    证明或回答下列论断:
    \begin{enumerate}[label={(\arabic*)}]
        \item 若 $\lim_{n \to \infty} na_n = a \neq 0$,则级数 $\sum_{n=1}^\infty a_n$ 发散;
        \item 若级数 $\sum_{n=1}^\infty a_n$ 收敛,是否有 $\lim_{n \to \infty} na_n = 0$;
        \item 若 $\lim_{n \to \infty} na_n = a$,且级数 $\sum_{n=1}^\infty n(a_n-a_{n+1})$ 收敛,证明 $\sum_{n=1}^\infty a_n$ 收敛.
    \end{enumerate}
\end{problem}

\begin{enumerate}
    \item[(3)]
    \begin{proof}
        设 $A_n = \sum_{k=1}^n k (a_k - a_{k+1})$,则由题设知 $\lim_{n \to \infty} A_n$ 存在,不妨设之为 $A$. 又由 $\lim_{n \to \infty} n a_n = a$ 知 $\lim_{n \to \infty} a_n = 0$. 故
        \[
            \lim_{n \to \infty} \sum_{k=1}^n a_k = \lim_{n \to \infty} \left( A_n + (n+1)a_{n+1} - a_{n+1} \right) = A + a - 0.
        \]
        故原级数收敛.
    \end{proof}
\end{enumerate}

\begin{problem}{MAblue}{7.1.6}
    设 $\{ a_n \}, \{ b_n \}$ 是两个非负数列,满足 $a_{n+1} < a_n + b_n$,而且 $\sum_{n=1}^\infty b_n$ 收敛. 求证 $\{ a_n \}$ 收敛.
\end{problem}

\begin{proof}
    设 $c_n = a_n - a_1 - \sum_{k=1}^{n-1} b_k$,则由单调有界原理知 $\{ c_n \}$ 收敛,进而得 $\{ a_n \}$ 收敛.
\end{proof}

\begin{problem}{MAblue}{7.1.7}
    证明:若级数 $\sum_{n=1}^\infty a_n^2$ 和 $\sum_{n=1}^\infty b_n^2$ 收敛,则级数 $\sum_{n=1}^\infty |a_nb_n|,\ \sum_{n=1}^\infty (a_n+b_n)^2$,以及 $\sum_{n=1}^\infty \frac {|a_n|} n$ 也收敛.
\end{problem}

\begin{proof}
    注意到
    \[
        \sum_{n=1}^\infty |a_nb_n| \leqslant \sum_{n=1}^\infty \frac {a_n^2+b_n^2} 2 = \frac 1 2 \sum_{n=1}^\infty a_n^2 + \frac 1 2 \sum_{n=1}^\infty b_n^2,
    \]
    且
    \[
        \sum_{n=1}^\infty (a_n+b_n)^2 \leqslant \sum_{n=1}^\infty 2(a_n^2+b_n^2) = 2\sum_{n=1}^\infty a_n^2 + 2\sum_{n=1}^\infty b_n^2.
    \]
    故二者均收敛. 令 $b_n = \frac 1 n$,即可得 $\sum_{n=1}^\infty \frac {|a_n|} n$ 收敛.
\end{proof}

\begin{problem}{MAblue}{7.1.12}
    研究下列级数的条件收敛性与绝对收敛性:
    \begin{multicols}{2}
        \begin{enumerate}[label={(\arabic*)}]
            \item $\displaystyle \sum_{n=1}^\infty (-1)^n \left( \frac{2n+100}{3n+1} \right)^n$;
            \item $\displaystyle \sum_{n=1}^\infty \frac{(-1)^{\frac {n(n-1)} 2}}{2^n}$;
            \item $\displaystyle \sum_{n=1}^\infty (-1)^n \frac{\sqrt n}{n+100}$;
            \item $\displaystyle \sum_{n=1}^\infty (-1)^{n-1} \sin \frac 1 n$;
            \item $\displaystyle \sum_{n=1}^\infty (-1)^{n-1} \frac {\ln n} n$;
            \item $\displaystyle \sum_{n=1}^\infty \frac{(-1)^{n-1}}{n^p}$;
            \item $\displaystyle \sum_{n=1}^\infty (-1)^n \left( e^{\frac 1 n} - 1 \right)$;
            \item $\displaystyle \sum_{n=1}^\infty (-1)^n \left( \frac 1 n - \ln \left( 1 + \frac 1 n \right) \right)$;
            \item $\displaystyle \sum_{n=1}^\infty (-1)^n \left( 1 - \cos \frac p n \right)$;
            \item $\displaystyle \sum_{n=1}^\infty (-1)^n \left( 1 - \cos \frac 1 n \right)^p$.
        \end{enumerate}
    \end{multicols}
\end{problem}

\begin{enumerate}
    \item[(8)]
    \begin{solution}
        由 $\ln\left( 1 + \frac 1 n \right) = \frac 1 n + O \left( \frac 1 {n^2} \right)$ 知 $\lim_{n \to \infty} \left( \frac 1 n - \ln \left( 1 + \frac 1 n \right) \right) = 0$. 故原级数条件收敛,亦绝对收敛.
    \end{solution}
\end{enumerate}

\begin{problem}{MAblue}{7.1.15}
    研究下列级数的敛散性:
    \begin{multicols}{2}
        \begin{enumerate}[label={(\arabic*)}]
            \item $\displaystyle \sum_{n=1}^\infty \frac {\sin nx} n$;
            \item $\displaystyle \sum_{n=2}^\infty \frac{\cos \frac {n\pi} 4}{\ln n}$;
            \item $\displaystyle \sum_{n=1}^\infty \frac{\sin n}{\sqrt n} \left( 1 + \frac 1 n \right)^n$;
            \item $\displaystyle \sum_{n=1}^\infty (-1)^n \frac{n-1}{n+1} \cdot \frac 1 {\sqrt[100]{n}}$.
        \end{enumerate}
    \end{multicols}
\end{problem}

\begin{enumerate}
    \item[(1)]
    \begin{solution}
        熟知
        \[
            \sum_{k=1}^n \sin kx = \frac{\cos \frac x 2 - \cos \left( nx + \frac x 2 \right)}{2 \sin \frac x 2}
        \]
        有界,而数列 $\left\{ \frac 1 n \right\}$ 单调递减趋于零,故由 Dirichlet 判别法知原级数收敛.
    \end{solution}
    \item[(2)]
    \begin{solution}
        取 $a_n = \cos \frac {n \pi} 4,\ b_n = \frac 1 {\ln n}$,由 Dirichlet 判别法易知原级数收敛.
    \end{solution}
    \item[(3)]
    \begin{solution}
        由 Dirichlet 判别法知 $\sum_{n=1}^{\infty} \frac{|\sin n|}{\sqrt n}$ 收敛. 进而由 Abel 判别法知原级数收敛.
    \end{solution}
    \item[(4)]
    \begin{solution}
        易知 $\lim_{n \to \infty} \frac{n-1}{n+1} \cdot \frac 1 {\sqrt[100] n} = 0$,则由 Leibniz 判别法知原级数收敛.
    \end{solution}
\end{enumerate}

\section{函数项级数}

\begin{problem}{MAblue}{7.2.2}
    确定下列函数项级数的收敛域.
    \begin{multicols}{2}
        \begin{enumerate}[label={(\arabic*)}]
            \item $\displaystyle \sum_{n=1}^\infty ne^{-nx}$;
            \item $\displaystyle \sum_{n=2}^\infty \frac {x^{n^2}} n$;
            \item $\displaystyle \sum_{n=1}^\infty \frac{(-1)^n}{2n-1} \left( \frac{1-x}{1+x} \right)^n$;
            \item $\displaystyle \sum_{n=1}^\infty \frac 1 {x^n} \sin \frac \pi {2^n}$;
            \item $\displaystyle \sum_{n=1}^\infty \frac{(x-3)^n}{n-3^n}$;
            \item $\displaystyle \sum_{n=1}^\infty n! \left( \frac x n \right)^n$;
            \item $\displaystyle \sum_{n=1}^\infty \frac{\cos nx}{e^{nx}}$;
            \item $\displaystyle \sum_{n=1}^\infty \frac{x^n}{1-x^n}$.
        \end{enumerate}
    \end{multicols}
\end{problem}

\begin{enumerate}
    \item[(6)]
    \begin{solution}
        由 Stirling 公式知 $n \to \infty$ 时有
        \[
            n! \left( \frac x n \right)^n \sim \sqrt{2 \pi n} \left( \frac n e \right)^n \left( \frac x n \right)^n = \sqrt{2 \pi n} \left( \frac x e \right)^n.
        \]
        故级数收敛域为 $(-e, e)$.
    \end{solution}
\end{enumerate}

\begin{problem}{MAblue}{7.2.10}
    递归定义 $[0, 1)$ 上的连续可微序列 $\{ f_n \}$ 如下:$f_1 = 1$,在 $(0, 1)$ 上有
    \[
        f_{n+1}'(x) = f_n(x)f_{n+1}(x),\ f_{n+1}(0) = 1.
    \]
    求证:对任意 $x \in [0, 1)$ 有 $\lim_{n \to \infty} f_n(x)$ 存在,并求出其极限函数.
\end{problem}

\begin{proof}
    考虑归纳证明. 由 $f_{n+1}'(x) = f_n(x) f_{n+1}(x)$ 及 $f_n(0) = 1$ 解得
    \[
        f_{n+1}(x) = \exp\left( \int_0^x f_n(t) \mathrm dt \right).
    \]
    \begin{enumerate}
        \item[(a)]
        当 $n = 1$ 时,$f_2(x) = \exp\left( \int_0^x f_1(t) \mathrm dt \right) = e^x$,则 $f_1(x) \leqslant f_2(x) \leqslant \frac 1 {1 - x}$.
        \item[(b)]
        若当 $n = k \ (\geqslant 2)$ 时,$f_{k-1}(x) \leqslant f_k(x) \leqslant \frac 1 {1 - x}$ 成立,则
        \[
            \begin{cases}
                f_{k+1}(x) = \exp\left( \int_0^x f_k(t) \mathrm dt \right) \leqslant \exp\left( \int_0^x \frac{\mathrm dt}{1 - t} \right) = \frac 1 {1 - x} \\
                \frac{f_{k+1}(x)}{f_k(x)} = \exp\left( \int_0^x \left( f_k(t) - f_{k-1}(t) \right) \mathrm dt \right) \geqslant e^0 = 1
            \end{cases}
        \]
        即 $f_k(x) \leqslant f_{k+1}(x) \leqslant \frac 1 {1 - x}$,故 $f_n(x)$ 单调有界,收敛至 $\frac 1 {1-x}$.
    \end{enumerate}
\end{proof}

\begin{problem}{MAblue}{7.2.11}
    证明 Dini 定理.
\end{problem}

\begin{center}
    \begin{minipage}{0.85\textwidth}
        \begin{theorem}{Dini 定理}{}
            若函数项级数 $\{ S_n(x) \}$ 在闭区间 $I$ 上逐点收敛到 $S(x)$,且通向 $u_n(x)$ 在区间 $I$ 上是连续且非负(或非正)的,那么 $S(x)$ 在 $I$ 上连续的充要条件是 此级数在 $I$ 上一致收敛.
        \end{theorem}
    \end{minipage}
\end{center}

\begin{proof}
    充分性的证明见定理 7.34. 对于必要性,令 $f_n(x) = S(x) - S_n(x)$,则 $f(x)$ 连续,故有极值,且该极值单调递减(或递增)趋于零,则 $\{ f_n(x) \}$ 在 $I$ 上一致收敛到零,也即函数项级数 $\{ S_n(x) \}$ 在 $I$ 上一致收敛.
\end{proof}

\section{幂级数与 Taylor 展式}

\begin{problem}{MAblue}{7.3.1}
    求下列幂级数的收敛半径.
    \begin{multicols}{2}
        \begin{enumerate}[label={(\arabic*)}]
            \item $\displaystyle \sum_{n=1}^\infty (-1)^{n+1} \frac{x^n}{n^2}$;
            \item $\displaystyle \sum_{n=1}^\infty \frac{(n!)^2}{(2n)!} x^n$;
            \item $\displaystyle \sum_{n=1}^\infty 2^nx^{2n}$;
            \item $\displaystyle \sum_{n=1}^\infty \frac{x^n}{a^n+b^n} \ (a > 0,\ b > 0)$;
            \item $\displaystyle \sum_{n=1}^\infty \frac{(x-2)^{2n-1}}{(2n-1)!}$;
            \item $\displaystyle \sum_{n=1}^\infty \frac {3^n+(-2)^n} n (x+1)^n$;
            \item $\displaystyle \sum_{n=1}^\infty \left( 1 + \frac 1 2 + \cdots + \frac 1 n \right) x^n$;
            \item $\displaystyle \sum_{n=1}^\infty \frac{x^{n^2}}{2^n}$.
        \end{enumerate}
    \end{multicols}
\end{problem}

\begin{enumerate}
    \item[(8)]
    \begin{solution}
        易知
        \[
            \lim_{n \to \infty} \sqrt[n]{\frac{x^{n^2}}{2^n}} = \frac {x^n} 2,
        \]
        则由 Cauchy 判别法知 $x < 1$ 时级数收敛,$x > 1$ 时级数发散,即 $R = 1$.
    \end{solution}
\end{enumerate}

\begin{problem}{MAblue}{7.3.3}
    求下列幂级数的收敛区域及其和函数.
    \begin{multicols}{2}
        \begin{enumerate}[label={(\arabic*)}]
            \item $\displaystyle \sum_{n=0}^\infty (-1)^n \frac{x^{2n+1}}{2n+1}$;
            \item $\displaystyle \sum_{n=0}^\infty (n+1)x^n$;
            \item $\displaystyle \sum_{n=1}^\infty n(n+1)x^{n-1}$;
            \item $\displaystyle \sum_{n=1}^\infty \frac{x^n}{n(n+1)}$;
            \item $\displaystyle \sum_{n=1}^\infty \frac{x^{2n-1}}{(2n-1)!!}$;
        \end{enumerate}
    \end{multicols}
\end{problem}

\begin{enumerate}
    \item[(5)]
    \begin{solution}
        由
        \[
            \lim_{n \to \infty} \frac{\frac{x^{2n+1}}{(2n+1)!!}}{\frac{x^{2n-1}}{(2n-1)!!}} = \lim_{n \to \infty} \frac{x^2}{2n+1} = 0
        \]
        知收敛域为 $\R$.
        
        设 $f(x) = \sum_{n=1}^{\infty} \frac{x^{2n-1}}{(2n-1)!!}$,则 $f'(x) = 1 + \sum_{n=1}^{\infty} \frac{x^{2n}}{(2n-1)!!} = 1 + xf(x)$. 解得
        \[
            f(x) = e^{\frac {x^2} 2} \left( \int e^{-\frac {x^2} 2} + C \right). \qedhere
        \]
    \end{solution}
\end{enumerate}

\begin{problem}{MAblue}{7.3.4}
    求下列级数的和.
    \begin{multicols}{2}
        \begin{enumerate}[label={(\arabic*)}]
            \item $\displaystyle \sum_{n=2}^\infty \frac 1 {(n^2-1)2^n}$;
            \item $\displaystyle \sum_{n=0}^\infty \frac{(-1)^n(n^2-n+1)}{2^n}$;
            \item $\displaystyle \sum_{n=0}^\infty \frac{(-1)^n}{3n+1}$;
            \item $\displaystyle \sum_{n=0}^\infty \frac{(n+1)^2}{n!}$;
        \end{enumerate}
    \end{multicols}
\end{problem}

\begin{enumerate}
    \item[(1)]
    \begin{solution}
        熟知 $\sum_{n=0}^{\infty} x^n = \frac 1 {1-x}$,则
        \[
            \sum_{n=1}^{\infty} \frac {x^n} n = \int_0^x \frac{\mathrm dt}{1-t} = -\ln(1-x).
        \]
        故
        \[
            \sum_{n=2}^{\infty} \frac 1 {(n^2-1) 2^n} = \frac 1 4 \sum_{n=2}^{\infty} \frac 1 {(n-1) 2^{n-1}} - \sum_{n=2}^{\infty} \frac 1 {(n+1) 2^{n+1}} = \frac 5 8 + \frac 3 4 \ln 2. \qedhere
        \]
    \end{solution}
    \item[(2)]
    \begin{solution}
        熟知 $\sum_{n=3}^{\infty} x^n = \frac{x^3}{1-x}$,则
        \[
            \sum_{n=3}^{\infty} n(n-1) x^{n-2} = \left( \frac{x^3}{1-x} \right)'' = \frac{2x^3}{(1-x)^3} + \frac{6x^2}{(1-x)^2} + \frac{6x}{1-x}.
        \]
        故
        \[
            \sum_{n=1}^{\infty} \frac{(-1)^n (n^2-n+1)}{2^n} = \frac 1 2 + \frac 1 4 \sum_{n=3}^{\infty} n(n-1) \left( - \frac 1 2 \right)^{n-2} + \sum_{n=1}^{\infty} \left( -\frac 1 2 \right)^n = -\frac{41}{27}. \qedhere
        \]
    \end{solution}
    \item[(3)]
    \begin{solution}
        熟知 $\frac 1 {1-x^3} = \sum_{n=0}^{\infty} x^{3n}$,则
        \[
            \sum_{n=0}^{\infty} \frac{x^{3n+1}}{3n+1} = \int_0^x \frac{\mathrm dt}{1-t^3} = -\frac 1 3 \ln(1-x) + \frac 1 6 \ln(1 + x + x^2) + \frac {\sqrt 3} 3 \arctan\left( \frac{2x + 1}{\sqrt 3} \right).
        \]
        故
        \[
            \sum_{n=0}^{\infty} \frac{(-1)^n}{3n+1} = - \sum_{n=0}^{\infty} \frac{(-1)^{3n+1}}{3n+1} = \frac{\sqrt 3}{18} \pi + \frac 1 3 \ln 2. \qedhere
        \]
    \end{solution}
    \item[(4)]
    \begin{solution}
        考虑以下引理.
        \begin{center}
            \begin{minipage}{0.85\textwidth}
                \begin{lemma}{}{}
                    对任意 $k \in \N_+$,有
                    \[
                        \sum_{n=0}^\infty \frac{n^k}{n!} = B_k e,
                    \]
                    其中 $B_k$ 为 Bell 数.
                    \tcblower
                    \begin{proof}
                        当 $k \geqslant 1$ 时有
                        \[
                            \sum_{n=0}^\infty \frac{n^k}{n!} = \sum_{n=1}^\infty \frac{n^{k-1}}{(n-1)!} = \sum_{n=0}^\infty \frac{(n+1)^{k-1}}{n!}.
                        \]
                        将二项式展开即得.
                    \end{proof}
                \end{lemma}
            \end{minipage}
        \end{center}
        \[
            \sum_{n=0}^{\infty} \frac{(n+1)^2}{n!} = \sum_{n=0}^{\infty} \frac{n^2}{n!} + 2 \sum_{n=0}^{\infty} \frac n {n!} + \sum_{n=0}^{\infty} \frac 1 {n!} = 2e + 2e + e = 5e. \qedhere
        \]
    \end{solution}
\end{enumerate}

\begin{problem}{MAblue}{7.3.7}
    方程 $y + \lambda \sin y = x \ (\lambda \neq -1)$ 在 $x = 0$ 附近确定了一个隐函数 $y(x)$,试求它的幂级数展开式的前四项.
\end{problem}

\begin{solution}
    不妨设 $y = \sum_{n=0}^{\infty} a_n x^n$. 令 $x = 0$,则有 $a_0 + \lambda \sin a_0 = 0$,而由于 $y(x)$ 是确定的,故只能有 $a_0 = 0$. 原式两边求导,得 $y' + \lambda y' \cos y = 1$,再令 $x = 0$,得 $a_1 + \lambda a_1 = 1$,也即 $a_1 = \frac 1 {1+\lambda}$. 依此类推,可求得
    \[
        y = \frac 1 {1 + \lambda} x + \frac{\lambda}{6 (1 + \lambda)^4} x^3. \qedhere
    \]
\end{solution}

\section{级数的应用}

\begin{problem}{MAblue}{7.4.6}
    证明:当 $n \to \infty$ 时,$\ln(n!) \sim \ln n^n$.
\end{problem}

\begin{proof}
    由 Stolz 定理知
    \begin{align*}
        \lim_{n \to \infty} \frac{\ln n!}{\ln n^n} &= \lim_{n \to \infty} \frac{\ln n! - \ln (n-1)!}{\ln n^n - \ln \left( (n-1)^{n-1} \right)} \\
        &= \lim_{n \to \infty} \frac{\ln n}{\ln n + (n-1) \ln\left( 1 + \frac 1 {n-1} \right)} \\
        &= \lim_{n \to \infty} \frac{\ln n}{\ln n + 1} = 1. \qedhere
    \end{align*}
\end{proof}
{\flushleft 或由 Stirling 公式立得.}

\section*{第 7 章综合习题}

\begin{problem}{MAblue}{7.0.3}
    设 $\{ a_n \}$ 是正的递增数列. 求证:级数 $\sum_{n=1}^\infty \left( \frac{a_{n+1}}{a_n} - 1 \right)$ 收敛的充分必要条件是 $\{ a_n \}$ 有界.
\end{problem}

\begin{proof}
    先证充分性. 不妨设 $a_n \leqslant M$,则
    \[
        S_n = \sum_{k=1}^n \frac{a_{k+1} - a_k}{a_k} < \sum_{k=1}^n \frac{a_{k+1} - a_k}{a_1} = \frac{a_{n+1}}{a_1} - 1 \leqslant \frac M {a_1} - 1.
    \]
    即 $\sum_{n=1}^{\infty} \left( \frac{a_{n+1}}{a_n} - 1 \right)$ 收敛.

    {\flushleft 再证必要性. 由级数收敛知,对任意 $0 < \varepsilon < \frac 1 2$,存在 $N \in \N_+$,使得当 $m > n \geqslant N$ 时有}
    \[
        1 - \frac{a_n}{a_{m}} = a_n \left( \frac 1 {a_n} - \frac 1 {a_m} \right) < \left| \sum_{k=n}^{m-1} a_{k+1} \left( \frac 1 {a_k} - \frac 1 {a_{k+1}} \right) \right| < \varepsilon.
    \]
    故对任意 $m > N$ 有 $a_m < \frac{a_N}{1 - \varepsilon} < 2a_N$. 取 $M = \max\left\{ a_1,\ a_2,\ \ldots,\ 2a_N \right\}$,则恒有 $a_n \leqslant M$,即 $\{ a_n \}$ 有界.
\end{proof}

\begin{problem}{MAblue}{7.0.4}
    设 $\alpha > 0$,$\{ a_n \}$ 是递增正数列. 求证级数 $\sum_{n=1}^\infty \frac{a_{n+1}-a_n}{a_{n+1}a_n^\alpha}$ 收敛.
\end{problem}

\begin{proof}
    \leavevmode
    \begin{enumerate}
        \item[(a)] 当 $\alpha \geqslant 1$ 时:
        \[
            S_n = \sum_{k=1}^n \left( 1 - \frac{a_k}{a_{k+1}} \right) \frac 1 {a_k^{\alpha}} \leqslant \sum_{k=1}^n \left( 1 - \frac{a_k^{\alpha}}{a_{k+1}^{\alpha}} \right) \frac 1 {a_k^{\alpha}} = \frac 1 {a_1^{\alpha}} - \frac 1 {a_{n+1}^{\alpha}} < \frac 1 {a_1^{\alpha}}.
        \]
        故原级数收敛.
        \item[(b)] 当 $0 < \alpha < 1$ 时,由 Lagrange 中值定理知,存在 $\xi \in (a_n, a_{n+1})$ 使得
        \[
            \frac{a_{n+1}^{\alpha} - a_n^{\alpha}}{a_{n+1} - a_n} = \alpha \xi^{\alpha - 1} > \alpha a_{n+1}^{\alpha - 1},
        \]
        即
        \[
            \frac{a_{n+1} - a_n}{a_{n+1} a_n^{\alpha}} < \frac 1 {\alpha} \left( \frac 1 {a_n^{\alpha}} - \frac 1 {a_{n+1}^{\alpha}} \right).
        \]
        故
        \[
            S_n < \frac 1 {\alpha a_1^{\alpha}}  - \frac 1 {\alpha a_{n+1}^{\alpha}} < \frac 1 {\alpha a_1^{\alpha}},
        \]
        即原级数收敛. \qedhere
    \end{enumerate}
\end{proof}

\begin{problem}{MAblue}{7.0.5}
    设 $\Phi(x)$ 是 $(0, +\infty)$ 上正的严格增函数,$\{ a_n \}, \{ b_n \}, \{ c_n \}$ 是三个非负数列,满足
    \[
        a_{n+1} \leqslant a_n - b_n \Phi(a_n) + c_na_n,\ \sum_{n=1}^\infty b_n = \infty,\ \sum_{n=1}^\infty c_n < \infty.
    \]
    求证 $\lim_{n \to \infty} a_n = 0$.
\end{problem}

\begin{proof}
    设 $\sum_{n=1}^\infty c_n = c$,则
    \[
        1 \leqslant \prod_{n=1}^{\infty} (1 + c_n) = \exp\left( \sum_{n=1}^{\infty} \ln(1 + c_n) \right) \leqslant \exp\left( \sum_{n=1}^{\infty} c_n \right) = e^c = C,
    \]
    则原不等式可化为
    \[
        \frac{a_{n+1}}{\prod_{k=1}^n (1+c_k)} - \frac{a_n}{\prod_{k=1}^{n-1} (1+c_k)} \leqslant - \frac{b_n \Phi(a_n)}{\prod_{k=1}^n (1+c_n)} < 0.
    \]
    不妨设 $d_n = \frac{a_n}{\prod_{k=1}^{n-1} (1+c_k)} \geqslant 0$,则 $d_n$ 单调递减有下界,即 $\{ d_n \}$ 收敛. 若 $\lim_{n \to \infty} d_n = d > 0$,则
    \[
        d_{n+1} - d_n \leqslant -b_n \frac{\Phi(d)}{M_1} \quad \Rightarrow \quad d_{n+1} \leqslant d_1 - \frac{\Phi(d)}{M_1} \sum_{k=1}^{n} b_k,
    \]
    则 $\lim_{n \to \infty} d_n = -\infty$,矛盾,故 $\lim_{n \to \infty} d_n = 0$,进而有 $\lim_{n \to \infty} a_n = 0$.
\end{proof}

\begin{problem}{MAblue}{7.0.6}
    设 $\{ a_n \}$ 是正数列使得 $\sum_{n=1}^\infty a_n$ 收敛. 求证:
    \[
        \sum_{n=1}^\infty \frac n {a_1 + a_2 + \cdots + a_n} \leqslant 2 \sum_{n=1}^\infty \frac 1 {a_n},
    \]
    而且上式右端的系数 $2$ 是最佳的.
\end{problem}

\begin{proof}
    由 Cauchy-Schwarz 不等式知
    \[
        (a_1 + a_2 + \cdots + a_n)\left( \frac 1 {a_1} + \frac 4 {a_2} + \cdots + \frac{n^2}{a_n} \right) \geqslant \frac {n^2(n+1)^2} 4,
    \]
    即
    \[
        \frac n {a_1 + a_2 + \cdots + a_n} \leqslant \frac 4 {n(n+1)^2} \sum_{k=1}^n \frac{k^2}{a_k} \leqslant  2 \left( \frac 1 {n^2} - \frac 1 {(n+1)^2} \right) \sum_{k=1}^n \frac{k^2}{a_k}.
    \]
    对 $n$ 求和得
    \begin{align*}
        \sum_{n=1}^N \frac n {a_1 + a_2 + \cdots + a_n} &\leqslant \sum_{n=1}^N 2 \left( \frac 1 {n^2} - \frac 1 {(n+1)^2} \right) \sum_{k=1}^n \frac{k^2}{a_k} \\
        &= 2 \sum_{k=1}^N \frac{k^2}{a_k} \sum_{n=k}^N \left( \frac 1 {n^2} - \frac 1 {(n+1)^2} \right) \\
        &= 2 \sum_{k=1}^N \left( \frac 1 {a_k} - \frac{k^2}{(N+1)^2 a_k} \right).
    \end{align*}
    令 $N \to \infty$,则有
    \[
        \sum_{n=1}^{\infty} \frac n {a_1 + a_2 + \cdots + a_n} \leqslant 2 \sum_{n=1}^{\infty} \frac 1 {a_n}. \qedhere
    \]
\end{proof}
{\flushleft 此题亦可用数学归纳法证明,见\href{https://math.stackexchange.com/posts/108598/revisions}{此解}. 事实上,我们有更强的结论,见\href{https://www.komal.hu/feladat?a=feladat&f=A709&l=en}{此解},其亦确定了最佳系数.}

\begin{problem}{MAblue}{7.0.7}
    设 $\{ a_n \}$ 是一个严格单调递增实数列,且对任意正整数 $n$ 有 $a_n \leqslant n^2 \ln n$,求证:级数 $\sum_{n=1}^\infty \frac 1 {a_{n+1}-a_n}$ 发散.
\end{problem}

\begin{proof}
    不妨令 $a_1 = 0$,则由上题结论知
    \[
        \sum_{n=1}^{\infty} \frac 1 {a_{n+1} - a_n} \geqslant \frac 1 2 \sum_{n=1}^{\infty} \frac n {a_{n+1} - a_1} \geqslant \frac n {(n+1)^2 \ln(n+1)}
    \]
    故 $\sum_{n=1}^{\infty} \frac 1 {a_{n+1} - a_n}$ 发散.
\end{proof}

\begin{problem}{MAblue}{7.0.9}
    设函数列 $\{ f_n(x) \},\ n = 1, 2, \cdots$ 在区间 $[0, 1]$ 上由
    \[
        f_0(x) = 1,\ f_n(x) = \sqrt{xf_{n-1}(x)}
    \]
    定义,证明当 $n \to \infty$ 时,函数列在 $[0, 1]$ 上一致收敛到一个连续函数.
\end{problem}

\begin{proof}
    设 $f_n(x) = x^{a_n} \ (n \in \N)$,则 $f_{n+1}(x) = x^{a_{n+1}} = x^{\frac {a_n + 1} 2}$,也即 $a_0 = 0,\ a_{n+1} = \frac {a_n + 1} 2$,则易证 $\lim_{n \to \infty} a_n = 1$,即 $\lim_{n \to \infty} f_n(x) = x$.
\end{proof}

\begin{problem}{MAblue}{7.0.11}
    设 $f_0(x)$ 是区间 $[0, a]$ 上连续函数,证明按照下列公式
    \[
        f_n(x) = \int_0^x f_{n-1}(u) \mathrm du
    \]
    定义的函数列 $\{ f_n(x) \}$ 在区间 $[0, a]$ 上一致收敛于 $0$.
\end{problem}

\begin{proof}
    设 $|f_0(x)| \leqslant M$,则可归纳证得 $|f_n(x)| \leqslant \frac{Ma^n}{n!}$,显然一致收敛于 $0$.
\end{proof}

{\flushleft 事实上有}
\begin{align*}
    \int_0^x \frac{(x-t)^n}{n!} f_0(t) \mathrm dt &= \left. \frac{(x-t)^n}{n!} f_1(t) \right|_0^x + \int_0^x \frac{(x-t)^{n-1}}{(n-1)!} f_1(t) \mathrm dt \\
    &= \int_0^x \frac{(x-t)^{n-1}}{(n-1)!} f_1(t) \mathrm dt \\
    &= \cdots \\
    &= \int_0^x f_n(t) \mathrm dt \\
    &= f_{n+1}(x).
\end{align*}.

\part*{数学分析讲义 \ 第二册}
\addcontentsline{toc}{part}{数学分析讲义 \ 第二册}

\chapter{空间解析几何}

\section{向量与坐标系}

\begin{problem}{MAgreen}{8.1.5}
    证明 $\bs a \times \bs b \cdot \bs c = \bs b \times \bs c \cdot \bs a = \bs c \times \bs a \cdot \bs b$.
\end{problem}

\begin{proof}
    考虑第一个等号. 注意到 $\bs a \times \bs b \cdot \bs a = 0$,则
    \begin{align*}
        \bs a \times \bs b \cdot \bs c &= \bs a \times \bs b \cdot \bs c - \bs c \times \bs b \cdot \bs c + (\bs a - \bs c) \times \bs b \cdot (\bs a - \bs c) - \bs a \times \bs b \cdot \bs a \\
        &= -\bs c \times \bs b \cdot \bs a = \bs b \times \bs c \cdot \bs a.
    \end{align*}
    其余同理.
\end{proof}

\begin{problem}{MAgreen}{8.1.8}
    设 $\bs a, \bs b, \bs c$ 是满足 $\bs a + \bs b + \bs c = \bs 0$ 的单位向量,试求 $\bs a \cdot \bs b + \bs b \cdot \bs c + \bs c \cdot \bs a$ 的值.
\end{problem}

\begin{solution}
    变形 $\bs a + \bs b + \bs c = \bs 0$ 得 $\bs b = -(\bs a + \bs c)$,则
    \[
        \sum_{\mathrm{cyc}} \bs a \cdot \bs b = - \sum_{\mathrm{cyc}} \bs a \cdot (\bs a + \bs c) = -3 - \sum_{\mathrm{cyc}} \bs a \cdot \bs b.
    \]
    故 $\bs a \cdot \bs b + \bs b \cdot \bs c + \bs c \cdot \bs a = -\frac 3 2$.
\end{solution}

\begin{problem}{MAgreen}{8.1.13}
    已知 $\bs a, \bs b, \bs c$ 不共线,且 $\bs a \times \bs b = \bs b \times \bs c = \bs c \times \bs a$,求证:$\bs a + \bs b + \bs c = \bs 0$.
\end{problem}

\begin{proof}
    由 $\bs a \times \bs a = 0$ 知
    \[
        \bs 0 = \bs c \times \bs a - \bs a \times \bs b + \bs a \times \bs a = (\bs a + \bs b + \bs c) \times \bs a.
    \]
    同理有 $(\bs a + \bs b + \bs c) \times \bs b = \bs 0$. 又 $\bs a, \bs b$ 不共线,故 $\bs a + \bs b + \bs c = \bs 0$.
\end{proof}

\section{平面与直线}

\begin{problem}{MAgreen}{8.2.3}
    设平面过点 $(5, -7, 4)$ 且在 $x, y, z$ 三轴上的截距相等,求平面方程.
\end{problem}

\begin{solution}
    设平面法向量为 $\bs n = (a, b, c)$,其在三轴上的截距为 $d$,则有
    \[
        \begin{cases}
            a(d-5) + b(0+7) + c(0-4) = 0, \\
            a(0-5) + b(d+7) + c(0-4) = 0, \\
            a(0-5) + b(0+7) + c(d-4) = 0. \\
        \end{cases}
    \]
    解得 $a = b = c$,故平面方程为 $(x-5) + (y+7) + (z-4) = 0$.
\end{solution}

\section{二次曲面}

\begin{problem}{MAgreen}{8.3.8}
    一动点 $P(x, y, z)$ 到原点的距离等于它到平面 $z = 4$ 的距离,试求此动点 $P$ 的轨迹,并判定它是什么曲面.
\end{problem}

\begin{solution}
    由题对 $P$ 有
    \[
        \sqrt{x^2 + y^2 + z^2} = |z - 4|.
    \]
    整理得 $x^2 + y^2 = -8z + 16$. 故其为曲线
    \[
        \begin{cases}
            x^2 = -8z + 16, \\
            y = 0
        \end{cases}
    \]
    绕 $z$ 轴旋转所得的曲面.
\end{solution}

\section{其它常用坐标系和坐标变换}

\section*{第 8 章综合习题}
\addcontentsline{toc}{section}{第 8 章综合习题}

\backmatter

\end{document}
