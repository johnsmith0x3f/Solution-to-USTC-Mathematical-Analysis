% !TEX TS-program = xelatex
% !TEX encoding = UTF-8 Unicode

% This is a simple template for a LaTeX document using the "article" class.
% See "book", "report", "letter" for other types of document.

\documentclass[11pt,oneside,fontset=fandol]{ctexbook} % use larger type; default would be 10pt

\usepackage[utf8]{inputenc} % set input encoding (not needed with XeLaTeX)

%%% Examples of Article customizations
% These packages are optional, depending whether you want the features they provide.
% See the LaTeX Companion or other references for full information.

%%% PAGE DIMENSIONS
\usepackage{geometry} % to change the page dimensions
\geometry{a4paper} % or letterpaper (US) or a5paper or....
% \geometry{margin=2in} % for example, change the margins to 2 inches all round
% \geometry{landscape} % set up the page for landscape
%   read geometry.pdf for detailed page layout information

\usepackage{graphicx} % support the \includegraphics command and options

% \usepackage[parfill]{parskip} % Activate to begin paragraphs with an empty line rather than an indent

%%% PACKAGES
\usepackage{booktabs} % for much better looking tables
\usepackage{array} % for better arrays (eg matrices) in maths
\usepackage{paralist} % very flexible & customisable lists (eg. enumerate/itemize, etc.)
\usepackage{verbatim} % adds environment for commenting out blocks of text & for better verbatim
\usepackage{subfig} % make it possible to include more than one captioned figure/table in a single float
% These packages are all incorporated in the memoir class to one degree or another...

%%% HEADERS & FOOTERS
% \usepackage{fancyhdr} % This should be set AFTER setting up the page geometry
% \pagestyle{fancy} % options: empty , plain , fancy
% \renewcommand{\headrulewidth}{0pt} % customise the layout...
% \setlength{\headheight}{14pt}
% \lhead{\thesection}\chead{}\rhead{\thepage}
% \lfoot{}\cfoot{}\rfoot{}

%%% SECTION TITLE APPEARANCE
% \usepackage{sectsty}
% \allsectionsfont{\rmfamily\mdseries\upshape} % (See the fntguide.pdf for font help)
% (This matches ConTeXt defaults)

%%% ToC (table of contents) APPEARANCE
\usepackage[nottoc,notlof,notlot]{tocbibind} % Put the bibliography in the ToC
\usepackage[titles,subfigure]{tocloft} % Alter the style of the Table of Contents
\renewcommand{\cftsecfont}{\rmfamily\mdseries\upshape}
\renewcommand{\cftsecpagefont}{\rmfamily\mdseries\upshape} % No bold!

\usepackage{amsmath}
\usepackage{amssymb}
\usepackage{amsthm}

\usepackage{color}
\definecolor{lightblue}{RGB}{70,130,196}
\usepackage[colorlinks=true, allcolors=lightblue]{hyperref}
\usepackage{enumitem}

\renewcommand\labelenumi{\theenumi.}
\renewcommand\theenumi{\thesection.\arabic{enumi}}

%%% END Article customizations

%%% The "real" document content comes below...

\title{《数学分析讲义》习题答案}
\author{johnsmith0x3f}
\date{\today} % Activate to display a given date or no date (if empty),
              % otherwise the current date is printed

\begin{document}

\maketitle

\newpage

\frontmatter

\tableofcontents

\newpage

\mainmatter

\part*{数学分析讲义 \ 第一册}
\addcontentsline{toc}{part}{数学分析讲义 \ 第一册}

\thispagestyle{empty}

\newpage

\chapter{极限}

\section{实数}

\begin{enumerate}
    \item[1.]
    \textbf{证明} \ 以加法为例. 若 $a + b = c \in \mathbf Q$,则 $a = b - c \in \mathbb Q$,矛盾,故 $a + b$ 为无理数.
    $\hfill\qed$
    \item[3.]
    \textbf{证明} \ 若 $\sqrt 2$ 为有理数,则存在最小的正整数 $q$ 使得 $\sqrt 2 q$ 是正整数,然而令 $r = \sqrt 2 q - q < q$,则 $\sqrt 2 r$ 亦为正整数,故 $\sqrt 2$ 为无理数. 同理可得 $\sqrt 3,\ \sqrt 6$ 均为无理数.
    
    对于 $\sqrt 2 + \sqrt 3$,用反证法. 假设其为有理数,则 $(\sqrt 2 + \sqrt 3)^2 = 5 + 2 \sqrt 6$ 为有理数,与 $\sqrt 6$ 为无理数矛盾,故 $\sqrt 2 + \sqrt 3$ 为无理数.
    $\hfill\qed$
    \item[6.]
    \textbf{证明} \ 设 $x_n = \prod_{k=1}^n (1 + a_k) - (1 + \sum_{k=1}^n a_k)$,则
    \[
        x_{n+1} - x_n = a_{n+1} \left( \prod_{k=1}^n (1 + a_k) - 1 \right)
    \]
    
    若 $a_i \geqslant 0$,显然有 $x_{n+1} \geqslant x_n$;若 $-1 \leqslant a_i < 0$,此时 $(1+a_1)(1+a_2)\cdots(1+a_n) < 1$,亦有 $x_{n+1} > x_n$,故 $\{ x_n \}$ 为递增数列. 可得 $x_n \geqslant x_1 = 0$.
    $\hfill\qed$

    事实上,此即 \textbf{Bernoulli 不等式}的一般式.
    \item[7.]
    \textbf{证明} \ 注意到 $(a-1)(b-1) > 0$ 和 $(a+1)(b+1) > 0$,即 $1+ab > \pm(a+b)$. 故
    \[
        |1+ab| = 1+ab > |a+b| \ \Leftrightarrow \ \left| \frac{a+b}{1+ab} \right| < 1
    \]
    $\hfill\qed$
\end{enumerate}

\section{数列极限}

\begin{enumerate}
    \item[15.]
    \begin{enumerate}
        \item[(2)]
        注意到 $0 < (n+1)^k - n^k = n^k \left[(1 + \frac 1 n)^k - 1 \right] < n^k (1 + \frac 1 n - 1) = n^{k-1}$.
        
        则由夹逼原理知,$\lim_{n \to \infty} \left[(n+1)^k - n^k \right] = 0$.
    \end{enumerate}
    \item[16.]
    \textbf{证明} \ 不妨设 $A = \max\{ a_1,\ a_2,\ \ldots,\ a_m \}$,则 $A \leqslant \sqrt[n]{a_1^n+a_2^n+\cdots+a_m^n} \leqslant A \sqrt[n]{m}$.
    
    故由夹逼原理知,$\lim_{n \to \infty} \sqrt[n]{a_1^n+a_2^n+\cdots+a_m^n} = A$.
    $\hfill\qed$
    \item[17.]
    \begin{enumerate}
        \item[(3)(4)]
        利用 Cauchy 收敛准则易证.
    \end{enumerate}
    \item[18.]
    \begin{enumerate}
        \item[(2)(4)]
        实质上是求数列不动点,利用数学归纳法易求得极限.
    \end{enumerate}
    \item[21.]
    \textbf{证明} \ 由题设知 $0 < \frac{a_{n+1}}{b_{n+1}} \leqslant \frac{a_n}{b_n}$,由单调有界原理知 $\{ \frac{a_n}{b_n} \}$ 收敛. 则
    \[
        \lim_{n \to \infty} a_n = \left( \lim_{n \to \infty} \frac{a_n}{b_n} \right) \left(\lim_{n \to \infty} b_n \right)
    \]
    故 $\{ a_n \}$ 收敛.
    $\hfill\qed$
    \item[25.]
    \textbf{证明} \ 显然 $a_n > 0$. 假设存在 $M > 0$,使 $a_n \leqslant M$ 恒成立,则
    \[
        a_{M^2+1} = 1 + \frac 1 {a_1} + \frac 1 {a_2} + \cdots + \frac 1 {a_{M^2}} \geqslant 1 + M^2 \cdot \frac 1 M > M
    \]
    与假设矛盾. 故 $ a_n \to +\infty \ (n \to \infty)$.
    $\hfill\qed$
    \item[26.]
    \textbf{证明} \ 由题设知 $\forall \varepsilon>0,\ \exists N \in \mathbb N_+$ 使得 $n>N$ 时有 $A-\varepsilon < \frac{a_{n+1}-a_n}{b_{n+1}-b_n} < A+\varepsilon$.
    
    由于 $\{ b_n \}$ 严格单调递减,则有
    \[
        (A-\varepsilon)(b_n-b_{n+1}) < a_n - a_{n+1} < (A+\varepsilon)(b_n-b_{n+1})
    \]
    取正整数 $m > n$,累加得
    \[
        (A-\varepsilon)(b_n-b_m) < a_n - a_m < (A+\varepsilon)(b_n-b_m)
    \]
    即 $\left| \frac{a_n-a_m}{b_n-b_m} - A \right| < \varepsilon$,令 $m \to \infty$ 即得 $\left| \frac{a_n}{b_n} - A \right| < \varepsilon$.
    $\hfill\qed$
\end{enumerate}

\section{函数极限}

\begin{enumerate}
    \item[9.]
    \begin{enumerate}
        \item[(4)]
        \[
            \lim_{x \to \infty} \left(\frac{x^2+1}{x^2-1}\right)^{x^2} = \lim_{x \to \infty} \left(1+\frac 1 {x^2}\right)^{x^2} \cdot \left(1+\frac 1 {x^2-1}\right)^{x^2} = e^2
        \]
    \end{enumerate}
\end{enumerate}

\section*{第 1 章综合习题}
\addcontentsline{toc}{section}{第 1 章综合习题}

\begin{enumerate}
    \item[6.]
    \textbf{证明} \ 不妨设 $a_{n+1}-a_n \leqslant k a_1 \ (k>0)$,则
    \[
        \frac{a_{n+1}}{a_n} \leqslant 1 + \frac{ka_1}{a_n} < 1 + k
    \]
    进而有
    \[
        0 < a_{n+1}^{\alpha}-a_n^{\alpha} < a_n^{\alpha}\left((1+k)^{\alpha}-1 \right) < k a_n^{\alpha}
    \]
    故由夹逼原理知,$\lim_{n \to \infty} \left(a_{n+1}^{\alpha}-a_n^{\alpha}\right) = 0$.
    $\hfill\qed$

    对其逆命题,考虑反例 $a_n = n \ln n$,易得不成立. 事实上,若以 $n$ 为标准,逆命题中 $a_{n+1} - a_n$ 无界的条件要求 $a_n$ 是一个 $k (> 1)$ 阶无穷大量,也即 $a_n \sim n^k \ (n \to \infty)$. 然而在我们试取某个具体的 $k$ 后,我们发现使 $\lim_{n \to \infty} \left(a_{n+1}^{\alpha}-a_n^{\alpha}\right) = 0$ 成立的 $\alpha$ 的范围缩小到了 $(0, \frac 1 k)$. 那么为了证伪逆命题,我们需要 $k$ 无限地接近 $1$. 另一方面,我们知道对任意 $\alpha > 0$ 有 $\lim_{n \to \infty} \frac{\ln n}{n^\alpha} = 0$. 这启发我们将 $\ln n$ 视为一个“$\varepsilon$ 阶无穷大量”,其中的 $\varepsilon$ 蕴含了极限思想,表示“无穷小但不为零”. 如此一来,我们自然地想到取 $a_n = n \ln n$,也即 $k = 1 + \varepsilon$,如此便得到了我们想要的结果. 此外,将 $\ln x$ 视为“$\varepsilon$ 阶无穷大量”的思想亦有助于我们理解 $\ln x$ 是 $\frac 1 x$ 的原函数这一事实.
    \item[7.]
    此即 \textbf{Cauchy 命题}.
    \item[8.]
    \textbf{证明} \ 当 $a \neq 0$ 时,有 $\lim_{n \to \infty} \frac 1 {a_n} = \frac 1 a$. 又由均值不等式有
    \[
        \frac{n}{\frac 1 {a_1} + \frac 1 {a_2} + \cdots + \frac 1 {a_n}} \leqslant \sqrt[n]{a_1a_2\cdots a_n} \leqslant \frac {a_1+a_2+\cdots+a_n} n
    \]
    由 Cauchy 命题易知
    \[
        \lim_{n \to \infty} \frac{n}{\frac 1 {a_1} + \frac 1 {a_2} + \cdots + \frac 1 {a_n}} = \frac 1 {\frac 1 a} = a = \lim_{n \to \infty} \frac {a_1+a_2+\cdots+a_n} n
    \]
    故由夹逼原理知,$\lim_{n \to \infty} \sqrt[n]{a_1a_2\cdots a_n} = a$.
    
    当 $a = 0$ 时,利用不等式 $0 < \sqrt[n]{a_1a_2\cdots a_n} \leqslant \frac {a_1+a_2+\cdots+a_n} n$ 即可证明.
    $\hfill\qed$

    或利用复合函数极限的性质
    \[
        \lim_{n \to \infty} \sqrt[n]{a_1 a_2 \cdots a_n} = \lim_{n \to \infty} \exp\left( \frac {\ln a_1 + \ln a_2 + \cdots + \ln a_n} n \right) = \exp(\ln a) = a
    \]
    亦可证明.
    \item[12.]
    \textbf{证明} \ 先证 (2). 由 Stolz 定理知,$\lim_{n \to \infty} c_n = \lim_{n\to \infty} \frac{a_{n+1}b_{n+1}}{b_{n+1}} = a$.

    而当 $(b_1 + b_2 + \cdots + b_n) \to B$ 时,对任意 $\varepsilon > 0$,存在 $N \in \mathbb N_+$,使得当 $n > N$ 时,对任意 $p \in \mathbb N_+$ 有
    \[
        |b_n + b_{n+1} + \cdots + b_{n+p}| < \varepsilon
    \]
    又由 $\{ a_n \}$ 收敛,不妨设 $|a_n| < M$,则
    \[
        |a_n b_n + a_{n+1} b_{n+1} + \cdots + a_{n+p} + b_{n+p}| < M \varepsilon
    \]
    故 $\sum_{k=1}^n a_k b_k$ 收敛,进而得 $c_n$ 收敛,则 (1) 得证.
    $\hfill\qed$
    \item[16.]
    \textbf{证明} \ 由 Dirichlet 逼近定理知,对任意 $k \in \mathbb N_+$,存在 $i,\ j \in \mathbb N_+$,且 $i \neq j$,满足 $|\{ i\xi \}-\{ j\xi \}|<\frac 1 k$. 不妨令 $k>\frac 1 {b-a}$,取最小正整数 $n$ 满足 $n|\{ i\xi \}-\{ j\xi \}|>a$. 若
    \[
        n|\{ i\xi \}-\{ j\xi \}| \geqslant b > a + \frac 1 k > a + |\{ i\xi \}-\{ j\xi \}|
    \]
    则 $(n-1)|\{ i\xi \}-\{ j\xi \}|>a$,矛盾. 故存在 $n|\{ i\xi \}-\{ j\xi \}| \in (a,\ b)$.
    $\hfill\qed$
\end{enumerate}

\newpage

\chapter{单变量函数的连续性}

\section{连续函数的基本概念}

\begin{enumerate}
    \item[15.]
    \textbf{证明} \ 易知 $f(nx) = nf(x) \ (n \in \mathbb N)$,进而有 $f\left(\frac p q x\right) = \frac p q f(x) \ (p,\ q \in \mathbb Z,\ q \neq 0)$. 对任意无理数 $r$,以有理数逼近之,得到收敛于 $r$ 的无穷数列.
    
    又知 $f(x)$ 连续,故 $f(\lambda x) = \lambda f(x)$ 对任意实数 $\lambda$ 成立,故有 $f(x) = f(1) x$.
    $\hfill\qed$
\end{enumerate}

\section{闭区间上连续函数的性质}

\begin{enumerate}
    \item[6.]
    \textbf{证明} \ 设 $g(x) = f(x) - f(x+a) \ (0 \leqslant x \leqslant a)$,则 $g(0) = -g(a)$,由零点定理可得欲证.
    $\hfill\qed$
    \item[7.]
    \textbf{证明} \ 不妨设 $f(x_1) \leqslant f(X_2) \leqslant \cdots \leqslant f(x_n)$,则 $f(x_1) < f(\xi) < f(x_n)$,由介值定理知,显然存在这样的 $\xi$.
    $\hfill\qed$
    \item[16.]
    $f(x) = \sin x^2$. 由连续且有界想到基本初等函数中的 $\sin x$(或 $\cos x$),而不一致连续的条件启发我们寻找一个任意陡的函数,故想到 $\sin x^2$.
\end{enumerate}

\section*{第 2 章综合习题}
\addcontentsline{toc}{section}{第 2 章综合习题}

\begin{enumerate}
    \item[5.]
    \textbf{证明} \ 设 $g(x) = f(x) - f(x+\frac 1 n) \ (0 \leqslant x \leqslant 1 - \frac 1 n)$,若存在 $g(\xi) = 0$,命题得证. 否则有 $g(0) + g(\frac 1 n) + \cdots + g(1-\frac 1 n) = 0$,由零点定理可得欲证.
    $\hfill\qed$
    \item[10.]
    \textbf{证明} \ 设 $x_0 = a,\ x_{n+1} = y(x_n)$,如存在多个 $y(x_n)$,取最大的那一个作为 $x_{n+1}$. 则 $f(x_n)$ 严格单调递增,且可以证明 $\lim_{n \to \infty} x_n = b$:
    
    用反证法. 假设 $\lim_{n \to \infty} x_n = c < b$,则在 $(c, b)$ 上恒有 $f(x) \leqslant f(c)$,与题设矛盾,故 $\lim_{n \to \infty} x_n = b$. 则由函数连续性知 $f(a) < f(b)$.
    $\hfill\qed$

    事实上,可以证明 $f(b) = \max_{a \leqslant x \leqslant b}\{ f(x) \}$.
\end{enumerate}

\newpage

\chapter{单变量函数的微分学}

\section{导数}

\begin{enumerate}
    \item[17.]
    \begin{enumerate}
        \item[(1)]
        设
        \[
            f(x) = x + x^2 + \cdots + x^n = \frac{x(1-x^n)}{1-x} \ (x \neq 1)
        \]
        则
        \[
            P_n = f'(x) = \frac{nx^{n+1}-(n+1)x^n+1}{(x-1)^2}
        \]
        \item[(2)]
        \[
            Q_n = f'(x)+xf''(x) = \frac{n^2x^{n+2}-(2n^2+2n-1)x^{n+1}+(n+1)^2x^n-x-1}{(x-1)^3}
        \]
        \item[(3)]
        设
        \[
            g(x) = \sin x + \sin 2x + \cdots + \sin nx = \frac{\cos\frac x 2 - \cos(\frac {2n+1} 2 x)}{2\sin\frac x 2}
        \]
        则
        \[
            R_n = g'(1) = \frac{(n+1)\cos n - n\cos(n+1)-1}{4\sin^2\frac 1 2}
        \]
    \end{enumerate}
\end{enumerate}

\section{微分}

\begin{enumerate}
    \item[3.]
    \begin{enumerate}
        \item[(1)]
        \[
            \begin{cases}
                \mathrm dx = \frac{2t}{1+t^2} \mathrm dt \\
                \mathrm dy = \frac 1 {1+t^2} \mathrm dt
            \end{cases}
            \ \Rightarrow \
            \begin{cases}
                \frac{\mathrm dy}{\mathrm dx} = \frac 1 {2t} \\
                \frac{\mathrm d^2y}{\mathrm dx^2} = \frac{\mathrm d \left( \frac{\mathrm dy}{\mathrm dx} \right)}{\mathrm dx} = -\frac{1+t^2}{4t^3}
            \end{cases}
        \]
    \end{enumerate}
\end{enumerate}

\section{微分中值定理}

\begin{enumerate}
    \item[4.]
    \begin{enumerate}
        \item[(3)]
        \textbf{证明} \ 设 $f(x) = x \ln x$,则在 $\left( a,\ \frac {a+b} 2 \right)$ 和 $\left( \frac {a+b} 2,\ b \right)$ 上,由 Lagrange 中值定理有
        \[
            \frac{f\left( \frac {a+b} 2 \right) - f(a)}{\frac {b-a} 2} = 1 + \ln \xi_1 < 1 + \ln \xi_2 = \frac{f(b) - f\left( \frac {a+b} 2 \right)}{\frac {b-a} 2}
        \]
        整理后即得欲证.
        $\hfill\qed$

        事实上,此为 \textbf{Jensen 不等式}的一种形式:当 $f(x)$ 是凸函数,即 $f''(x) > 0$ 时,有 $\frac {f(a) + f(b)} 2 > f(\frac {a+b} 2)$.
    \end{enumerate}
    \item[7.]
    \item[18.]
    \textbf{证明} \ 由题设知对任意 $x>0$ 存在 $\xi \in (0,\ x)$ 使得 $\frac{f(x)-f(0)}{x-0} = f'(\xi) < f'(x)$. 则
    \[
        \frac{\mathrm d \frac{f(x)}{x}}{\mathrm dx} = \frac{xf'(x)-f(x)}{x^2} > 0
    \]
    即 $\frac{f(x)}{x}$ 严格递增.
    $\hfill\qed$
    \item[20.]
    \textbf{证明} \ 设 $g(x)=e^x\left( f(x)-f'(x) \right)$,则 $g(0) = g(1) = 0$. 故由 Rolle 定理知,存在 $\xi \in (0,\ 1)$,满足 $g'(\xi) = e^{\xi}\left( f(\xi) - f''(\xi) \right) = 0$,即 $f(\xi) = f''(\xi)$.
    $\hfill\qed$
    \item[23.]
    \begin{enumerate}
        \item[(1)]
        \textbf{证明} \ 先证右边. 显然有 $x^p + (1-x)^p \leqslant x + 1-x = 1$.
        
        对于左边,不妨设 $x < \frac 1 2$,则 $\frac 1 {2^{p-1}} \leqslant x^p + (1-x)^p \ \Leftrightarrow \ \frac 1 {2^p} - x^p \leqslant (1-x)^p - \frac 1 {2^p}$. 由 Lagrange 中值定理知,存在 $x < \xi_1 < \frac 1 2 < \xi_2 < 1-x$,满足
        \[
            \frac{\frac 1 {2^p} - x^p}{\frac 1 2 - x} = p\xi_1^{p-1} < p\xi_2^{p-1} = \frac{(1-x)^p - \frac 1 {2^p}}{1-x-\frac 1 2}
        \]
        则原式得证.
        $\hfill\qed$
        \item[(3)]
        \textbf{证明} \ 由 Lagrange 中值定理知,存在 $0 < \xi_1 < x_1 < \xi_2 < x_2 < \frac \pi 2$,满足
        \[
            \frac{\tan x_1-0}{x_1-0} = \frac 1 {\cos^2 \xi_1} < \frac 1 {\cos^2 \xi_2} = \frac{\tan x_2 - \tan x_1}{x_2 - x_1}
        \]
        整理后即得原式.
        $\hfill\qed$
        \item[(5)]
        \textbf{证明} \ 注意到不等式两侧均为关于 $x$ 的偶函数,而易知 $x = 0$ 时等号成立,故不妨设 $x > 0$. 由 Lagrange 中值定理知,存在 $\xi \in (0,\ x)$,满足
        \[
            \frac{\ln\left( x+\sqrt{1+x^2} \right)-0}{x-0} = \frac 1 {\sqrt{1+\xi^2}} > \frac 1 {\sqrt{1+x^2}}
        \]
        代入原式易证.
        $\hfill\qed$
    \end{enumerate}
\end{enumerate}

\section{未定式的极限}

\begin{enumerate}
    \item[4.]
    \textbf{证明} \ 由 Cauchy 中值定理知,存在 $\xi \in (a,\ b)$ 满足
    \[
        \frac{\frac {f(b)} b - \frac {f(a)} a}{\frac 1 b - \frac 1 a} = \frac{\frac{\xi f'(\xi) - f(\xi)}{\xi^2}}{-\frac 1 {\xi^2}} = f(\xi) - \xi f'(\xi)
    \]
    $\hfill\qed$
\end{enumerate}

\section{函数的单调性和凸性}

\begin{enumerate}
    \item[1.]
    \textbf{证明} \ 归纳证明. 显然 $n = 1$ 时成立. 假设 $n=k \ (\geqslant 1)$ 时结论成立,则当 $n = k + 1$ 时,设 $\alpha_1 + \alpha_2 + \cdots + \alpha_{k+1} = 1$,则
    \[
        f(\frac{\alpha_1x_1 + \alpha_2x_2 + \cdots + \alpha_kx_k}{1-\alpha_{k+1}}) \leqslant \frac{\alpha_1f(x_1) + \alpha_2f(x_2) + \cdots + \alpha_kf(x_k)}{1-\alpha_{k+1}}
    \]
    则由凸函数定义知
    \[
        f\left( (1-\alpha_{k+1}) \frac{\sum_{i=1}^k \alpha_ix_i}{1-\alpha_{k+1}} + \alpha_{k+1}x_{k+1} \right) \leqslant (1-\alpha_{k+1}) \frac{\sum_{i=1}^k \alpha_if(x_i)}{1-\alpha_{k+1}} + \alpha_{k+1}f(x_{k+1})
    \]
    整理后即得 $n = k+1$ 时成立. 故原命题成立.
    $\hfill\qed$
    \item[2.]
    此即为\textbf{加权平均不等式}.
\end{enumerate}

\section{Taylor 展开}

\begin{enumerate}
    \item[8.]
    \textbf{证明} \ 对任意 $x \in (0,\ 2)$,由 Taylor 公式得
    \[
        \begin{cases}
            f(0) = f(x) - f'(x)x + \frac {f''(\xi_1)} 2 (0-x)^2 ,\ \quad \xi_1 \in (0,\ x) \\
            f(2) = f(x) + f'(x)(2-x) + \frac {f''(\xi_2)} 2 (2-x)^2 ,\ \quad \xi_2 \in (x,\ 2)
        \end{cases}
    \]
    两式相减得
    \[
        f'(x) = \frac {f(2)-f(0)} 2 + \frac {x^2 f''(\xi_1) - (x-2)^2 f''(\xi_2)} 4 \leqslant 1 + \frac{x^2 + (x-2)^2} 4 \leqslant 2
    \]
    同理可得 $f'(x) \geqslant -2$,故 $|f'(x)| \leqslant 2$.
    $\hfill\qed$
\end{enumerate}

\section*{第 3 章综合习题}
\addcontentsline{toc}{section}{第 3 章综合习题}

\begin{enumerate}
    \item[5.]
    \textbf{证明} \ 首先证明如下结论:对任意 $x_1, x_2 \in I,\ i, n \in \mathbb N_+,\ 0 \leqslant i \leqslant 2^n$,有
    \[
        f\left( \frac{ix_1+(2^n-i)x_2}{2^n} \right) \leqslant \frac i {2^n} f(x_1) + \left( 1 - \frac i {2^n} \right) f(x_2)
    \]
    考虑归纳证明. 当 $n = 1$ 时,结论显然成立.
    若当 $n = k (\geqslant 1)$ 时,结论成立,则当 $n = k + 1$ 时,若 $i = 0$ 或 $i = 2^n$,结论是显然的,故我们考虑证明 $0 < i < 2^{k+1}$ 的情况. 若 $i$ 为偶数,结论显然成立;若 $i$ 为奇数,不妨设 $i < 2^k$,则
    \begin{align*}
        f\left( \frac{(i-1)x_1+(2^{k+1}-i+1)x_2}{2^{k+1}} \right) \leqslant \frac {i-1} {2^{k+1}} f(x_1) + \left( 1 - \frac {i-1} {2^{k+1}} \right) f(x_2) \\
        f\left( \frac{(i+1)x_1+(2^{k+1}-i-1)x_2}{2^{k+1}} \right) \leqslant \frac {i+1} {2^{k+1}} f(x_1) + \left( 1 - \frac {i+1} {2^{k+1}} \right) f(x_2)
    \end{align*}
    则将
    \[
        x_1' = \frac{(i-1)x_1+(2^{k+1}-i+1)x_2}{2^{k+1}},\ x_2' = \frac{(i+1)x_1+(2^{k+1}-i-1)x_2}{2^{k+1}}
    \]
    代入 $f\left( \frac {x_1+x_2} 2 \right) \leqslant \frac {f(x_1)+f(x_2)} 2$ 中即可得结论成立.

    对任意实数 $\lambda \in (0,\ 1)$,可利用上述结论逼近之,再由函数连续性导出
    \[
        f(\lambda x_1 + (1 - \lambda) x_2) \leqslant \lambda f(x_1) + (1 - \lambda) f(x_2)
    \]
    即 $f(x)$ 是凸函数.
    $\hfill\qed$
    \item[6.]
    \textbf{证明} \ 设 $g(x) = f'(x)(x-1)^2$,由 Rolle 定理知存在 $\eta \in (0,\ 1)$,满足 $f'(\eta) = 0$. 则 $g(\eta) = g(1) = 0$,故再由 Rolle 定理知存在 $\xi \in (\eta,\ 1)$,满足 $g'(\xi) = 0$,即 $f''(\xi) = \frac{2f'(\xi)}{1-\xi}$.
    $\hfill\qed$

    下简述选取辅助函数 $g(x)$ 的思路. 我们试图利用 Rolle 定理证明此题,这要求等式 $f''(\xi) = \frac{2f'(\xi)}{1-\xi}$ 可化成 $g'(\xi) = 0$ 的形式,也即有 $g(x) = C$. 解此常微分方程,得到 $f'(x)(x-1)^2 = e^C$,故可令 $g(x) = f'(x)(x-1)^2$.
    \item[9.]
    \textbf{证明} \ 设 $g(x) = \frac 1 {f(x)} - x$,则 $g(0) = g(1) = 1$. 由 Rolle 定理知,存在 $\xi \in (0,\ 1)$,满足 $g'(\xi) = - \frac{f'(\xi)}{f^2(\xi)} - 1 = 0$,即 $f^2(\xi) + f'(\xi) = 0$.
    $\hfill\qed$
    \item[19.]
    \textbf{证明} \ 设 $g(x) = f(x) - \frac 1 2 x^3 + \left( f(0) - \frac 1 2 \right) x^2$,则 $g(-1) = g(0) = g(1)$. 则由 Rolle 定理知存在 $\xi_1 \in (-1,\ 0)$ 及 $\xi_2 \in (0,\ 1)$,满足 $g'(\xi_1) = g'(\xi_2) = g(0) = 0$. 故知存在 $\eta_1 \in (\xi_1,\ 0)$ 及 $\eta_2 \in (0,\ \xi_2)$,满足 $g''(\eta_1) = g''(\eta_2) = 0$. 进而又知存在 $\zeta \in (\eta_1,\ \eta_2)$,满足 $g'''(\zeta) = 0$,即 $f'''(\zeta) = 3$.
    $\hfill\qed$

    此题中 $g(x)$ 的选取思路与本节第 6 题不同. 仍然是考虑 Rolle 定理,则我们需要某个 $\phi(x)$ 满足 $\phi'(\xi) = f'''(\xi) - 3 = 0$,故 $\phi(x) = f''(x) - 3x + a$,依此类推,还应有 $\varphi(x) = f'(x) - \frac 3 2 x^2 + ax + b$ 及 $g(x) = f(x) - \frac 1 2 x^3 + \frac a 2 x^2 + bx + c$,为能导出欲证,应有 $g(-1) = g(0) = g(1)$,由此可解得参数 $a, b, c$.
    \item[20.]
    \textbf{证明} \ 用反证法. 假设命题不成立,即存在 $X > 0$,使得 $x > X$ 时均有 $f'(x) \geqslant f(ax) > 0$. 故 $f(x)$ 在 $(X,\ +\infty)$ 上单调递增. 由 Lagrange 中值定理知,存在 $\xi \in (x,\ ax)$,满足
    \[
        \frac{f(ax)-f(x)}{(a-1)x} = f'(\xi) \geqslant f(a\xi) > f(ax)
    \]
    故有 $(1-ax+x)f(ax) \geqslant f(x)$. 则当 $x > \frac 1 {a-1}$ 时有 $f(x) < 0$,与题设矛盾,故命题成立.
    $\hfill\qed$
    \item[21.]
    \textbf{证明} \ 首先有如下\textbf{引理}:设 $x,\ y > 0$,$p,\ q$ 是大于 $1$ 的正数,且 $\frac 1 p + \frac 1 q = 1$,则
    \[
        xy \leqslant \frac 1 p x^p + \frac 1 q y^q
    \]
    由 $\ln x$ 凹性知 $\ln xy = \frac 1 p \ln x^p + \frac 1 q \ln y^q \leqslant \ln\left( \frac 1 p x^p + \frac 1 q y^q \right)$,整理则引理得证.
    
    令 $x = \frac{a_i}{\left( \sum_{i=1}^n a_i^p \right)^{\frac 1 p}},\ y = \frac{b_i}{\left( \sum_{i=1}^n b_i^q \right)^{\frac 1 q}}$,则由引理有
    \[
        \frac{a_ib_i}{\left( \sum_{i=1}^n a_i^p \right)^{\frac 1 p} \left( \sum_{i=1}^n b_i^q \right)^{\frac 1 q}} \leqslant \frac{a_i^p}{p \left( \sum_{i=1}^n a_i^p \right)} + \frac{b_i^q}{q \left( \sum_{i=1}^n b_i^q \right)}
    \]
    则
    \[
        \frac{\sum_{i=1}^n a_ib_i}{\left( \sum_{i=1}^n a_i^p \right)^{\frac 1 p} \left( \sum_{i=1}^n b_i^q \right)^{\frac 1 q}} \leqslant \frac 1 p + \frac 1 q = 1
    \]
    整理后即得欲证.
    $\hfill\qed$
\end{enumerate}

\newpage
\chapter{不定积分}

\section{不定积分及其基本计算方法}

\begin{enumerate}
    \item[3.]
    \begin{enumerate}
        \item[(10)]
        多次换元,$t = x^{\frac 1 {14}},\ u = t^{5},\ v = u - \frac 1 2$.
        \item[(12)]
        \begin{align*}
            \int \frac 1 {x^8(1+x^2)} \mathrm dx &= \int \left( \frac 1 {x^8} - \frac 1 {x^6} + \frac 1 {x^4} - \frac 1 {x^2} + \frac 1 {1+x^2} \right) \mathrm dx \\ &= - \frac 1 {7x^7} + \frac 1 {5x^5} - \frac 1 {3x^3} + \frac 1 x + \arctan x + C
        \end{align*}
    \end{enumerate}
    \item[7.]
    \begin{enumerate}
        \item[(26)]
        \begin{align*}
            \int \frac{xe^x}{(1+x)^2} \mathrm dx &= - \frac{xe^x}{1+x} + \int e^x \mathrm dx \\
            &= \frac{e^x}{1+x} + C
        \end{align*}
    \end{enumerate}
\end{enumerate}

\section{有理函数的不定积分}

\begin{enumerate}
    \item[1.]
    \begin{enumerate}
        \item[(8)]
        \begin{align*}
            \int \frac{x^{15}}{(x^8 + 1)^2} \mathrm dx &= - \frac{x^8}{8(x^8 + 1)} + \int \frac{x^7}{x^8 + 1} \mathrm dx \\
            &= \frac 1 8 \ln(x^8 + 1) - \frac{x^8}{8(x^8 + 1)} + C
        \end{align*}
    \end{enumerate}
\end{enumerate}

\newpage

\chapter{单变量函数的积分学}

\section{积分}
\label{sec:5.1}

\begin{enumerate}[label=\theenumi]
    \item[2.]
    \textbf{证明} \ 考虑其 Riemann 和. 对任意 $x_{i-1} < x_i$,总能找到 $\xi_i \in (x_{i-1}, x_i)$ 使得 $f(\xi_i) \equiv 0$ 或 $f(\xi_i) \equiv 1$,故 $\lim_{\Vert T \Vert \to 0} S_n(T)$ 不存在.
    \item[3.]
    取 $f(x) = 2D(x) - 1$,其中 $D(x)$ 为 Dirchlet 函数.
    \item[9.]
    \begin{enumerate}
        \item[(1)]
        \textbf{证明} \ 设 $m \leqslant f(x) \leqslant M$,由 $g(x) \geqslant 0$ 知
        \[
            \int_a^b m g(x) \mathrm dx \leqslant \int_a^b f(x)g(x) \mathrm dx \leqslant \int_a^b M g(x) \mathrm dx
        \]
        即存在 $\lambda \in [m, M]$ 使得 $\int_a^b f(x)g(x) \mathrm dx = \lambda \int_a^b g(x) \mathrm dx$. 又 $f(x)$ 连续,则存在 $\xi \in [a, b]$ 使得 $f(\xi) = \lambda$.
        $\hfill\qed$
        \item[(2)]
        令 $f(x) = (x+1)^2,\ g(x) = x$,则 $\int_a^b f(x)g(x) \mathrm dx = \frac 4 3$,而 $\int_{-1}^1 g(x) \mathrm dx = 0$. 故不存在满足条件的 $\xi$.
    \end{enumerate}
    \item[13.]
    设 $\varphi(x) = \int_0^x f(t) \mathrm dt$,则
    \begin{align*}
        F(x) &= \int_0^x x f(t) \mathrm dt = x \int_0^x f(t) \mathrm dt \\ 
        \Rightarrow \ F'(x) &= x f(x) + \int_0^x f(t) \mathrm dt = xf(x) + \varphi(x)
    \end{align*}
    \item[18.]
    \begin{enumerate}
        \item[(3)]
        \[
            \lim_{n \to \infty} \left( \sum_{k=0}^{n-1} \frac 1 {\sqrt{n^2 - k^2}} \right) = \int_0^1 \frac 1 {\sqrt{1 - x^2}} = \frac \pi 2
        \]
        \item[(4)]
        \[
            \lim_{n \to \infty} \frac{1^p + 2^p + \cdots + n^p}{n^{p+1}} = \int_0^1 x^p = \frac 1 {1+p}
        \]
    \end{enumerate}
    \item[19.]
    \begin{enumerate}
        \item[(2)]
        易知 $0 < \frac{x^n}{1+x} < x^n$. 又
        \[
            \lim_{n \to \infty} \int_0^1 x^n = \lim_{n \to \infty} \left. \frac{x^{n+1}}{n+1} \right|_0^1 = \lim_{n \to \infty} \frac 1 {n+1} = 0
        \]
        故由夹逼原理知 $\lim_{n \to \infty} \int_0^1 \frac{x^n}{1+x} = 0$.
    \end{enumerate}
    \item[22.]
    \begin{enumerate}
        \item[(14)]
        \begin{align*}
            \int_0^{\pi} \frac{\sec^2 x}{2 + \tan^2 x} \mathrm dx &= \int_0^{\frac \pi 2} \frac{\sec^2 x}{2 + \tan^2 x} \mathrm dx + \int_{\frac \pi 2}^{\pi} \frac{\sec^2 x}{2 + \tan^2 x} \mathrm dx \\
            &= \int_{0}^{+\infty} \frac 1 {t^2 + 2} \mathrm dt + \int_{-\infty}^0 \frac 1 {t^2 + 2} \mathrm dt \quad (t = \tan x) \\
            &= \lim_{a \to +\infty} \left. \frac 1 {\sqrt 2} \arctan \left( \frac t {\sqrt 2} \right) \right|_0^a + \lim_{a \to -\infty} \left. \frac 1 {\sqrt 2} \arctan \left( \frac t {\sqrt 2} \right) \right|_a^0 \\
            &= \frac \pi {\sqrt 2}
        \end{align*}
    \end{enumerate}
    \item[23.]
    \textbf{证明} \ 易知
    \begin{align*}
        \int_0^{\pi} x f(\sin x) \mathrm dx &= \int_0^{\frac \pi 2} x f(\sin x) \mathrm dx + \int_{\frac \pi 2}^{\pi} x f(\sin x) \mathrm dx \\
        &= \int_0^{\frac \pi 2} x f(\sin x) \mathrm dx + \int_0^{\frac \pi 2} (\pi - x) f( \sin(\pi - x) ) \mathrm dx \\
        &= \pi \int_0^{\frac \pi 2} f(\sin x) \mathrm dx
    \end{align*}
    $\hfill\qed$

    则
    \begin{align*}
        \int_0^{\pi} \frac{x \sin x}{1 + \cos^2 x} \mathrm dx &= \pi \int_0^{\frac \pi 2} \frac{\sin x}{1 + \cos^2 x} \\
        &= -\arctan(\cos x) \bigg|_0^{\frac \pi 2} \\
        &= \frac \pi 4
    \end{align*}
    \item[24.]
    \textbf{证明} \ 易知在 $(0, 1)$ 上有 $\frac {x^2} 2 < \sin x^2 < x^2$,取积分则命题得证.
    $\hfill\qed$
    \item[27.]
    \textbf{证明} \ 由 $f(x)$ 单调递减有
    \[
        (1 - \alpha) \int_0^\alpha f(x) \geqslant \alpha (1 - \alpha) f(\alpha) \geqslant \alpha \int_\alpha^1 f(x)
    \]
    整理后即得欲证.
    $\hfill\qed$
    \item[28.]\phantomsection\label{item:28}
    \begin{enumerate}
        \item[(1)]
        \textbf{证明} \ 依提示设 $G(t)$,则 $G(a) = 0,\ G(b) = \int_a^b |f(x)| \mathrm dx - \frac M 2 (b-a)^2$. 而
        \[
            G'(t) = |f(t)| - M(t - a)
        \]
        又由 Lagrange 中值定理知
        \[
            \frac{|f(t)|}{t - a} = \frac{|f(t)| - |f(a)|}{t - a} \leqslant |f'(\xi)| \leqslant M \ (a < \xi < t)
        \]
        故 $G'(t) \leqslant 0$,也即 $G(b) \leqslant G(a) = 0$. 原命题得证.
        $\hfill\qed$
        \item[(2)]
        \textbf{证明} \ 由 (1) 有
        \begin{align*}
            \int_a^b |f(x)| \mathrm dx &= \int_a^{\frac {a+b} 2} |f(x)| \mathrm dx + \int_{\frac {a+b} 2}^b |f(x)| \mathrm dx \\
            &\leqslant 2 \cdot \frac M 2 \cdot (\frac {b-a} 2)^2 = \frac M 4 (b-a)^2
        \end{align*}
        $\hfill\qed$
    \end{enumerate}
\end{enumerate}

\section{函数的可积性}

\begin{enumerate}
    \item[3.]
    \textbf{证明} \ 由题设知存在 $m \leqslant f(x) \leqslant M \ (a \leqslant x \leqslant b)$,则 $0 \leqslant |f(x)| \leqslant \max\{ M, -m \}$,则 $|f(x)|$ 在 $[a, b]$ 上可积.
    
    不妨把 $f(x)$ 按取值正负分为两部分,分别记作 $f_1(x)$ 和 $f_2(x)$. 具体地,令
    \[
        f_1(x) = \max\{ f(x), 0 \},\ f_2(x) = \min\{ f(x), 0 \}
    \]
    则
    \begin{align*}
        \left| \int_a^b f(x) \mathrm dx \right| &= \left| \int_a^b f_1(x) \mathrm dx + \int_a^b f_2(x) \mathrm dx \right| \\ &\leqslant \left| \int_a^b f_1(x) \mathrm dx \right| + \left| \int_a^b f_2(x) \mathrm dx \right| \\
        &= \int_a^b |f(x)| \mathrm dx
    \end{align*}
    $\hfill\qed$
\end{enumerate}

\section{积分的应用}

\begin{enumerate}
    \item[4.]
    \textbf{证明}
    \begin{align*}
        V &= \int_0^h \pi \left( R^2 - (R-h+x)^2 \right) \mathrm dx \\
        &= -\pi \int_0^h \left( x^2 + 2 (R-h) x + h (h-2R) \right) \mathrm dx \\
        &= \left. -\pi \left( \frac {x^3} 3 + (R-h) x^2 +  h(h-2R)x \right) \right|_0^h \\
        &= \pi h^2 (R - \frac h 3)
    \end{align*}
    $\hfill\qed$
\end{enumerate}

\section{广义积分}

\begin{enumerate}
    \item[1.]
    \begin{enumerate}
        \item[(12)]
        \begin{align*}
            \int_0^1 (\ln x)^n \mathrm dx &= (-1)^{n+1} \int_0^{+\infty} t^n e^{-t} \mathrm dt \ (x = e^{-t}) \\
            &= (-1)^{n+1}\Gamma(n+1)
        \end{align*}
    \end{enumerate}
    \item[2.]
    \begin{enumerate}
        \item[(1)]
        由 $\frac x {1+x^2}$ 的奇性知 $\int_{-b}^b \frac x {1+x^2} \mathrm dx \equiv 0$. 然而
        \[
            \int_{-\infty}^{+\infty} \frac x {1+x^2} \mathrm dx = \int_{-\infty}^0 \frac x {1+x^2} \mathrm dx + \int_0^{+\infty} \frac x {1+x^2} \mathrm dx
        \]
        等号右侧两积分均发散,故原积分发散.

        我们由此题得出了一条看似矛盾的结论:若 $P.V. \int_{-\infty}^{+\infty} f(x) \mathrm dx$ 收敛,则其本身未必收敛. 我们不妨借此良机,探究一下无穷积分的收敛性.

        首先,对形如 $\int_a^{+\infty} f(x) \mathrm dx$($\int_{-\infty}^a f(x) \mathrm dx$ 同理)的“单侧无穷”的积分,由前述有定义
        \[
            \int_a^{+\infty} f(x) \mathrm dx = \lim_{A \to \infty} \int_a^A f(x) \mathrm dx = \lim_{A \to +\infty} \varphi(A)
        \]
        在此定义下,上述矛盾是不存在的. 故我们主要探究“两侧无穷”的积分的敛散性. 由定义有
        \[
            \int_{-\infty}^{+\infty} f(x) \mathrm dx = \lim_{c \to -\infty} \int_c^a f(x) \mathrm dx + \lim_{b \to +\infty} \int_a^b f(x) \mathrm dx
        \]
        我们注意到其与柯西主值的区别在于积分上下界趋于无穷的同时与否,这便是我们解决矛盾的关键. 若我们认为柯西主值收敛时原积分亦收敛,则相当于认为在 $b \to +\infty$ 时积分的上下界仍保持其“对称性”,也即 $(+\infty) + (-\infty) = 0$,这显然是不成立的,故柯西主值收敛并不能说明原积分收敛. 事实上,由于“无穷”之间不能作上述运算,我们只能在两侧积分均收敛时认为“两侧无穷”的积分收敛.
    \end{enumerate}
    \item[3.]
    \begin{enumerate}
        \item[(2)]
        \textbf{证明}
        \begin{enumerate}
            \item[(a)]
            当 $\alpha = 1$ 时,$\int_0^{+\infty} \frac {\mathrm dx} x = \ln x \bigg|_0^{+\infty}$ 发散.
            \item[(b)]
            当 $\alpha \neq 1$ 时,易知 $\int_0^{+\infty} \frac 1 {x^\alpha} = \left. \frac{x^{1-\alpha}}{1-\alpha} \right|_0^{+\infty}$ 亦发散.
        \end{enumerate}
        综上,无论 $\alpha$ 取何实数值,$\int_0^{+\infty} \frac{\mathrm dx}{x^{\alpha}} $ 必发散.
        $\hfill\qed$
    \end{enumerate}
\end{enumerate}

\section*{第 5 章综合习题}
\addcontentsline{toc}{section}{第 5 章综合习题}

\begin{enumerate}
    \item[1.]
    \begin{enumerate}
        \item[(1)]
        \textbf{证明} \ 设 $t = 2 \pi - x$,则
        \begin{align*}
            \int_0^{2 \pi} \sin mx \cdot \cos nx \mathrm dx &= \int_0^\pi \sin mx \cdot \cos nx \mathrm dx - \int_0^\pi \sin mt \cdot \cos nt \mathrm dt \\
            &= 0
        \end{align*}
        $\hfill\qed$
        \item[(2)]
        \textbf{证明} \ 易知
        \begin{align*}
            \int_0^{2\pi} \sin mx \cdot \sin nx &= \frac 1 2 \int_0^{2\pi} \bigg( \cos( mx - nx ) - \cos( mx + nx ) \bigg) \\
            &= \frac 1 2 \int_0^{2\pi} \cos\big( (m - n)x \big) - \frac 1 2 \int_0^{2\pi} \cos\big( (m + n)x \big)
        \end{align*}
        则结论显然.
        $\hfill\qed$
    \end{enumerate}
    \item[2.]
    \textbf{证明} \ (1) 是显然的,下证 (2). 易知有
    \[
        B(m+1, n-1) = B(m, n-1) - B(m, n)
    \]
    则
    \begin{align*}
        B(m, n) &= \int_0^1 x^m (1-x)^n \mathrm dx \\
        &= \left. \frac{x^{m+1}(1-x)^n}{m+1} \right|_0^1 + \frac n {m+1} \int_0^1 x^{m+1} (1-x)^{n-1} \mathrm dx \\
        &= \frac{n B(m+1, n-1)}{m+1} = \frac{n (B(m, n-1) - B(m, n))}{m+1}
    \end{align*}
    即 $B(m, n) = \frac n {m+n+1} B(m, n-1)$. 又知 $B(0, 0) = 1$,则
    \begin{align*}
        B(m, n) &= \frac{n!}{\frac{(m+n+1)!}{(m+1)!}} \cdot \frac{m!}{(m+1)!} \cdot B(0, 0) \\
        &= \frac{n!m!}{(m+n+1)!}
    \end{align*}
    在 $m, n \geqslant 0$ 时成立.
    $\hfill\qed$
    \item[3.]
    \begin{enumerate}
        \item[(c)]
        设 $\varphi(x) = e^{\sin x} - e^{-\sin x}$,则有 $\varphi(x) + \varphi(x+\pi) = 0$. 故
        \begin{align*}
            \int_x^{x+2\pi} (1 + \varphi(t)) \mathrm dt &= \int_x^{x+\pi} (1 + \varphi(t) + 1 + \varphi(t+\pi)) \mathrm dt \\
            &= 2\pi
        \end{align*}
        则 $\int_0^1 f(x) \mathrm dx = (1 + x)(f(x) - 2\pi)$.
    \end{enumerate}
    \item[4.]
    \textbf{证明} \ 换元得
    \[
        \int_0^{\frac \pi 4} \tan^n x \mathrm dx = \int_0^1 \frac{t^n}{1+t^2} \mathrm dt \ (x = \arctan t)
    \]
    而
    \[
        \frac 1 {2n+2} = \int_0^1 \frac {t^n} 2 \mathrm dt < \int_0^1 \frac{t^n}{1+t^2} \mathrm dt < \int_0^1 \frac{t^n}{2t} \mathrm dt = \frac 1 {2n}
    \]
    则命题得证.
    $\hfill\qed$
    \item[10.]
    \textbf{证明} \ 显然 $F(0) = 0$. 当 $x \neq 0$ 时,设 $\varphi(x) = \int f(x) \mathrm dx$,则
    \[
        F(x) = \left. \frac {\varphi(xy)} x \right|_{y=0}^1 = \frac {\varphi(x) - \varphi(0)} x
    \]
    此时显然有 $F'(x) = \frac{xf(x) - \varphi(x) + \varphi(0)}{x^2}$. 又
    \[
        F'(0) = \lim_{x \to 0} \frac{F(x) - F(0)}{x - 0} = \lim_{x \to 0} \frac {f(x)} x = f'(0)
    \]
    故 $F(x)$ 处处可导.
    $\hfill\qed$
    \item[11.]
    \begin{enumerate}
        \item[(2)]
        \textbf{证明}
        \begin{align*}
            f'_+(0) &= \lim_{x \to 0_+} \frac{\int_{1/x}^{+\infty} \frac{\cos u}{u^2} \mathrm du - 0}{x - 0} \ (u = \frac 1 t) \\
            &= \lim_{x \to 0_+} \frac 1 x \left( \left. \frac{\sin u}{u^2} \right|_{\frac 1 x}^{+\infty} + 2 \int_{\frac 1 x}^{+\infty} \frac{\sin u}{u^3} \right) \\
            &= 2 \lim_{x \to 0_+} \frac{\int_0^x t \cos \frac 1 t}{x} = 2 \lim_{x \to 0_+} x \cos \frac 1 x = 0 \\
        \end{align*}
        $\hfill\qed$
    \end{enumerate}
    \item[12.]
    \textbf{证明} \ 设 $\varphi(x) = \int f(x) \mathrm dx$,则
    \begin{align*}
        LHS &= \lim_{h \to 0} \frac 1 h \left( \varphi(b+h) - \varphi(b) - \varphi(a+h) + \varphi(a) \right) \\
        &= \lim_{h \to 0} \frac {\varphi(b+h) - \varphi(b)} h - \lim_{h \to 0} \frac {\varphi(a+h) - \varphi(a)} h \\
        &= f(b) - f(a)
    \end{align*}
    $\hfill\qed$
    \item[13.]
    \textbf{证明}
    \[
        \lim_{\lambda \to \infty} \int_a^b f(x) \sin(\lambda x) \mathrm dx = \lim_{\lambda \to \infty} \left( \left. \frac {f(x) \cos(\lambda x)} \lambda \right|_a^b - \frac 1 \lambda \int_a^b f'(x) \cos(\lambda x) \mathrm dx \right) = 0
    \]
    $\hfill\qed$
    \item[14.]
    \textbf{证明}
    \begin{align*}
        \lim_{x \to +\infty} \frac 1 x \int_0^{x} |\sin t| \mathrm dt &= \lim_{x \to +\infty} \frac 1 x \int_0^{k\pi} |\sin t| \mathrm dt + \lim_{x \to +\infty} \frac 1 x \int_{k\pi}^x |\sin t| \mathrm dt \ \left( k = \left\lfloor \frac x \pi \right\rfloor \right) \\
        &= \lim_{x \to \infty} \frac {2k} x + 0 = \frac 2 \pi
    \end{align*}
    $\hfill\qed$
    \item[16.]
    \textbf{证明} \ 设 $f(x_0) = M$,对任意 $\varepsilon > 0$,存在 $\delta$ 使得 $f(x) > M - \varepsilon \ \left(x \in U(x_0, \delta) \right)$. 不妨设
    \[
        g_\delta(x) =
        \begin{cases}
            M - \varepsilon & x \in U(x_0, \delta) \\
            0 & x \not\in U(x, \delta)
        \end{cases}
    \]
    则 $0 < g(x) < f(x) \leqslant M$. 故有
    \[
        \lim_{n \to \infty} \left( \int_a^b g_\delta^n(x) \mathrm dx \right)^{\frac 1 n} = M - \varepsilon \leqslant \lim_{n \to \infty} \left( \int_a^b f^n(x) \mathrm dx \right)^{\frac 1 n} \leqslant M
    \]
    由夹逼原理知 $\lim_{n \to \infty} \int_a^b f(x) \mathrm dx = M$.
    $\hfill\qed$
    \item[18.]
    \textbf{证明} \ 考虑 $h(x) = f(x) - \lambda g(x)$,显然有
    \[
        \int_a^b h^2(x) \mathrm dx = \int_a^b f^2(x) \mathrm dx - 2\lambda \int_a^b f(x)g(x) \mathrm dx + \lambda^2 \int_a^b g^2(x) \mathrm dx \geqslant 0
    \]
    上式可看作关于 $\lambda$ 的二次不等式,则有
    \[
        \Delta = 4 \left( \int_a^b f(x)g(x) \mathrm dx \right)^2 - 4 \int_a^b f^2(x) \mathrm dx \int_a^b g^2(x) \mathrm dx \leqslant 0
    \]
    整理后即得欲证.
    $\hfill\qed$
    \item[19.]
    \textbf{证明} \ 设 $\min_{0 \leqslant x \leqslant 1}\{ |f(x)| \} = |f(x_0)|$,则
    \begin{align*}
        |f(a)| &= \left| f(x_0) + \int_{x_0}^a f'(x) \mathrm dx \right| \\
        &\leqslant |f(x_0)| + \left| \int_{x_0}^a f'(x) \mathrm dx \right| \\
        &\leqslant \int_0^1 |f(x)| \mathrm dx + \int_0^1 |f'(x)| \mathrm dx
    \end{align*}
    $\hfill\qed$
    \item[21.]
    \textbf{证明}
    \begin{align*}
        \left| \int_0^1 f(x) \mathrm dx - \frac 1 n \sum_{k=1}^n f\left( \frac k n \right) \right| &\leqslant \sum_{k=1}^n \left| \int_{\frac {k-1} n}^{\frac k n} f(x) \mathrm dx - \frac 1 n f\left( \frac k n \right) \right| \\
        &= \sum_{k=1}^n \left| \int_{\frac {k-1} n}^{\frac k n} \left( f(x) - f\left( \frac k n \right) \right) \mathrm dx \right| \\
        &\leqslant \sum_{k=1}^n \frac M 2 (\frac k n - \frac {k-1} n)^2 \\
        &= \frac M {2n}
    \end{align*}
    $\hfill\qed$

    \textbf{注} \ \textit{此题中用到了} \hyperref[item:28]{\textbf{5.1.28}} \textit{的结论}.
    \item[22.]
    \textbf{证明} \ 若 $f'(x) = 0$,则结论显然. 若 $f'(x) > 0$,设 $x_0 = x - f'(x)$,则有
    \[
        f(x) = \int_{x_0}^x f'(t) \mathrm dt + f(x_0) > \int_{x_0}^x f'(t) \mathrm dt \geqslant \int_{x_0}^x (f'(x) - (x-t)) \mathrm dt = \frac 1 2 (f'(x))^2
    \]
    同理可证 $f'(x) < 0$ 的情况.
    $\hfill\qed$

    一般地,若 $|f'(x) - f'(y)| \leqslant L|x-y|$,则有 $(f'(x))^2 \leqslant 2L f(x)$.\footnote{证明见\href{https://www.zhihu.com/question/611041671/answer/3109739799}{此解}.}
\end{enumerate}

\newpage
\chapter{常微分方程初步}

\section{一阶微分方程}

\begin{enumerate}
    \item[4.]
    \begin{enumerate}
        \item[(3)]
        整理原式得
        \[
            \frac{\mathrm dx}{\mathrm dy} = \frac x y + y^2
        \]
        若 $y \neq 0$,作代换 $u = \frac x y$,等式化为
        \[
            u + y \frac{\mathrm du}{\mathrm dy} = u + y^2 \ \left( u = \frac x y,\ y \neq 0 \right)
        \]
        则可进一步解得 $x = \frac 1 2 y^3 + Cy$. 验证可知 $y = 0$ 亦为方程的解.
    \end{enumerate}
    \item[6.]
    \begin{enumerate}
        \item[(1)]
        不妨设 $u = \frac y {x^2}$,则原式化为
        \[
            2xu + x^2 \frac{\mathrm du}{\mathrm dx} = x \left( \sqrt{1+u} - 1 \right)
        \]
        进一步整理得
        \[
            - \frac{\mathrm du}{2u - \sqrt{1+u} + 1} = \frac {\mathrm dx} x
        \]
        再设 $v = \sqrt{1+u}$,化为
        \[
            - \frac{2v \mathrm dv}{(v-1)(2v+1)} = \frac {\mathrm dx} x
        \]
        两边积分,得
        \[
            -\frac 1 3 \left( 2 \ln(v-1) + \ln(2v+1) \right) = \ln|x| + C
        \]
        即
        \[
        (v-1)^2(2v+1) = \frac{C'}{x^3}.
        \]
        则 $y$ 可解. 或设 $u = \sqrt{x^2 + y}$ 亦可解.\footnote{见\href{https://math.stackexchange.com/questions/3374263/how-to-solve-the-following-ordinary-differential-equation}{此解}.}
    \end{enumerate}
    \item[7.]
    由定义有
    \begin{align*}
        \frac{\mathrm dy}{\mathrm dx} + P(x) y &= Q(x) y^n \ (n \not\in \{ 0, 1 \} ) \\
        \frac{\mathrm du}{\mathrm dx} + (1-n) P(x) u &= (1-n) Q(x) \ (u = y^{1-n})
    \end{align*}
    设 $u = C(x) e^{(n-1) \int P(x) \mathrm dx}$,则
    \begin{align*}
        \frac{\mathrm d C(x)}{\mathrm dx} &= (1-n)Q(x)e^{(1-n) \int P(x) \mathrm dx} \\
        C(x) &= \int (1-n)Q(x)e^{(1-n) \int P(x) \mathrm dx} \mathrm dx + C
    \end{align*}
    故 $u = e^{(n-1) \int P(x) \mathrm dx} \left( \int (1-n)Q(x)e^{(1-n) \int P(x) \mathrm dx} \mathrm dx + C \right)$,进一步可得 $y$.
    \item[13.]
    \begin{enumerate}
        \item[(2)]
        设 $p = y'$,则原式化为
        \[
            y^3 p \frac{\mathrm dp}{\mathrm dy} = -1
        \]
        也即 $p \mathrm dp = - \frac{\mathrm dy}{y^3}$,易解得 $y = \sqrt{Cx^2 - \frac 1 C}$.
    \end{enumerate}
\end{enumerate}

\section{二阶线性微分方程}

\begin{enumerate}
    \item[1.]
    \begin{enumerate}
        \item[(3)]
        由题设知 $y_1(x) = x,\ p(x) = \frac{2x}{x^2 - 1}$,则
        \[
            y_2(x) = x \int \frac 1 {x^2} e^{\int_{x_0}^x \frac{2t}{t^2 - 1} \mathrm dt} \mathrm dx = \frac{x^2 + 1}{x_0^2 - 1}
        \]
        故通解 $y_0(x) = c_1x + c_2 (x^2 + 1)$.
    \end{enumerate}
    \item[2.]
    \begin{enumerate}
        \item[(2)]
        $y = x + 1$ 为一特解.
    \end{enumerate}
    \item[3.]
    其对应的齐次方程为
    \[
        y'' + \frac{2x}{1 + x^2} y' = 0
    \]
    观察知 $y_1 = \arctan x$ 为一特解,则可进一步求出 $y_2$.
    \item[5.]
    \begin{enumerate}
        \item[(2)]
        易得对应齐次方程的通解为 $y = c_1 e^{3x} + c_2 x e^{3x}$. 则
        \[
            y_0(x) = \int_{x_0}^x \frac{e^{3t} \cdot xe^{3x} - te^{3t} \cdot e^{3x}}{W(t)} f(t) \mathrm dt
        \]
        进而可得原方程通解.
    \end{enumerate}
    \item[9.]
    \begin{enumerate}
        \item[(4)]
        设 $x = e^{\lambda t}$,则得到特征方程 $\lambda^4 + 2\lambda^2 + 1 = 0$. 进而解得 $\lambda = \pm i$,故通解为 $x = c_1 \cos x + c_2 \sin x$.
    \end{enumerate}
    
\end{enumerate}

\newpage

\chapter{无穷级数}

\section{数项级数}

\begin{enumerate}
    \item[2.]
    \begin{enumerate}
        \item[(13)]
        对任意 $0 < k < \frac 1 4$,存在 $N \in \mathbf N_+$ 使得当 $n \geqslant N$ 时有 $\ln n < n^k$. 故
        \[
            \sum_{n=N}^{\infty} \frac{\ln n}{\sqrt[4]{n^5}} < \sum_{n=N}^{\infty} \frac 1 {n^{\frac 5 4 - k}}
        \]
        由比较审敛法知 $\sum_{n=1}^{\infty} \frac{\ln n}{\sqrt[4]{n^5}}$ 收敛.
        \item[(14)]
        易知
        \[
            \int_3^{+\infty} \frac 1 {x \ln x (\ln \ln x)^k} = 
            \begin{cases}
                \ln \ln \ln x \bigg|_3^{+\infty} = +\infty & k = 1 \\
                \left. \frac{(\ln \ln x)^{1-k}}{1-k} \right|_3^{+\infty} = +\infty & k \neq 1
            \end{cases}
        \]
        故由 Cauchy 积分判别法知级数 $\sum_{n=2}^{\infty} \frac 1 {n \ln n (\ln \ln n)^k}$ 发散.
        \item[(15)]
        当 $n \to \infty$ 时,有
        \[
            \left( \cos \frac 1 n \right)^{n^3} \sim \left( 1 - \frac 1 {2n^2} \right)^{n^3} = \frac 1 {e^{\frac n 2}}
        \]
        故原级数收敛.
        \item[(16)]
        当 $n \to \infty$ 时,有
        \[
            \left( \frac{an}{n+1} \right)^n = \frac{a^n}{\left( 1 + \frac 1 n \right)^n} \sim \frac {a^n} e
        \]
        故 $a \geqslant 1$ 时原级数发散,$a < 1$ 时原级数收敛.
    \end{enumerate}
    \item[4.]
    \begin{enumerate}
        \item[(3)]
        \textbf{证明} \ 设 $A_n = \sum_{k=1}^n k (a_k - a_{k+1})$,由题设知 $\lim_{n \to \infty} A_n$ 存在,不妨设为 $A$. 则
        \[
            \lim_{n \to \infty} \sum_{k=1}^n a_k = \lim_{n \to \infty} \left( A_n + (n+1)a_{n+1} - a_{n+1} \right)
        \]
        又由 $\lim_{n \to \infty} n a_n = a$ 知 $\lim_{n \to \infty} a_n = 0$. 故
        \[
            \lim_{n \to \infty} \sum_{k=1}^n a_k = A + a - 0
        \]
        故原级数收敛.
        $\hfill\qed$
    \end{enumerate}
    \item[6.]
    \textbf{证明} \ 设 $c_n = a_n - a_1 - \sum_{k=1}^{n-1} b_k$,则由单调有界原理知 $\{ c_n \}$ 收敛,进而得 $\{ a_n \}$ 收敛.
    $\hfill\qed$
    \item[7.]
    前两项易证. 第三项取 $b_n = \frac 1 n$ 即得证.
    \item[12.]
    \begin{enumerate}
        \item[(8)]
        由 $\ln\left( 1 + \frac 1 n \right) = \frac 1 n + o \left( \frac 1 {n^2} \right)$ 知 $\lim_{n \to \infty} \left( \frac 1 n - \ln \left( 1 + \frac 1 n \right) \right) = 0$. 故原级数收敛.
    \end{enumerate}
    \item[15.]
    \begin{enumerate}
        \item[(1)]
        熟知
        \[
            \sum_{k=1}^n \sin kx = \frac{\cos \frac x 2 - \cos \left( nx + \frac x 2 \right)}{2 \sin \frac x 2}
        \]
        有界,而数列 $\{ \frac 1 n \}$ 单调递减趋于零,故由 Dirichlet 判别法知原级数收敛.
        \item[(2)]
        取 $a_n = \cos \frac {n \pi} 4,\ b_n = \frac 1 {\ln n}$,由 Dirichlet 判别法易知原级数收敛.
        \item[(3)]
        由 Dirichlet 判别法知 $\sum_{n=1}^{\infty} \frac{|\sin n|}{\sqrt n}$ 收敛. 进而由 Abel 判别法知原级数收敛.
        \item[(4)]
        易知 $\lim_{n \to \infty} \frac{n-1}{n+1} \cdot \frac 1 {\sqrt[100] n} = 0$,则由 Leibniz 判别法知原级数收敛.
    \end{enumerate}
\end{enumerate}

\section{函数项级数}

\begin{enumerate}
    \item[2.]
    \begin{enumerate}
        \item[(6)]
        由 Stirling 公式知 $n \to \infty$ 时有
        \[
            n! \left( \frac x n \right)^n \sim \sqrt{2 \pi n} \left( \frac n e \right)^n \left( \frac x n \right)^n = \sqrt{2 \pi n} \left( \frac x e \right)^n
        \]
        故级数收敛域为 $(-e, e)$.
    \end{enumerate}
    \item[10.]
    \textbf{证明} \ 考虑归纳证明. 由 $f_{n+1}'(x) = f_n(x) f_{n+1}(x)$ 及 $f_n(0) = 1$ 解得 $f_{n+1}(x) = \exp\left( \int_0^x f_n(t) \mathrm dt \right)$.
    \begin{enumerate}
        \item[(a)]
        当 $n = 1$ 时,$f_2(x) = \exp\left( \int_0^x f_1(t) \mathrm dt \right) = e^x$,则 $f_1(x) \leqslant f_2(x) \leqslant \frac 1 {1 - x}$.
        \item[(b)]
        若当 $n = k \ (\geqslant 2)$ 时,$f_{k-1}(x) \leqslant f_k(x) \leqslant \frac 1 {1 - x}$ 成立,则
        \[
            \begin{cases}
                f_{k+1}(x) = \exp\left( \int_0^x f_k(t) \mathrm dt \right) \leqslant \exp\left( \int_0^x \frac{\mathrm dt}{1 - t} \right) = \frac 1 {1 - x} \\
                \frac{f_{k+1}(x)}{f_k(x)} = \exp\left( \int_0^x \left( f_k(t) - f_{k-1}(t) \right) \mathrm dt \right) \geqslant e^0 = 1
            \end{cases}
        \]
        即 $f_k(x) \leqslant f_{k+1}(x) \leqslant \frac 1 {1 - x}$,故 $f_n(x)$ 单调有界,收敛至 $\frac 1 {1-x}$.
    \end{enumerate}
    \item[11.]
    \begin{enumerate}
        \item[(1)]
        \textbf{证明} \ 由题设知 $u_n(x)$ 在 $[a, b]$ 上有极值,且极值单调递减趋于零,故 $\{ u_n(x) \}$ 一致收敛到零.
        \item[(2)]
        \textbf{证明} \ 充分性的证明见定理 7.34. 对于必要性,令 (1) 中的 $u_n(x) = S(x) - S_n(x)$ 即得.
        $\hfill\qed$
    \end{enumerate}
\end{enumerate}

\section{幂级数与 Taylor 展式}

\begin{enumerate}
    \item[1.]
    \begin{enumerate}
        \item[(8)]
        易知 $\lim_{n \to \infty} \sqrt[n]{\frac{x^{n^2}}{2^n}} = \frac {x^n} 2$. 则由 Cauchy 判别法知 $x < 1$ 时级数收敛,$x > 1$ 时级数发散,即 $R = 1$.
    \end{enumerate}
    \item[3.]
    \begin{enumerate}
        \item[(5)]
        由
        \[
            \lim_{n \to \infty} \frac{\frac{x^{2n+1}}{(2n+1)!!}}{\frac{x^{2n-1}}{(2n-1)!!}} = \lim_{n \to \infty} \frac{x^2}{2n+1} = 0
        \]
        知收敛域为 $\mathbb R$.
        
        设 $f(x) = \sum_{n=1}^{\infty} \frac{x^{2n-1}}{(2n-1)!!}$,则 $f'(x) = 1 + \sum_{n=1}^{\infty} \frac{x^{2n}}{(2n-1)!!} = 1 + xf(x)$. 解得
        \[
            f(x) = e^{\frac {x^2} 2} \left( \int e^{-\frac {x^2} 2} + C \right)
        \]
    \end{enumerate}
    \item[4.]
    \begin{enumerate}
        \item[(1)]
        熟知 $\frac 1 {1-x} = \sum_{n=0}^{\infty} x^n$,则
        \[
            -\ln(1-x) = \int \frac 1 {1-x} = \sum_{n=1}^{\infty} \frac {x^n} n
        \]
        故
        \begin{align*}
            \sum_{n=2}^{\infty} \frac 1 {(n^2-1) 2^n} &= \frac 1 4 \sum_{n=2}^{\infty} \frac 1 {(n-1) 2^{n-1}} - \sum_{n=2}^{\infty} \frac 1 {(n+1) 2^{n+1}} \\
            &= \frac 5 8 - \frac 3 4 \sum_{n=1}^{\infty} \frac 1 {n 2^n} = \frac 5 8 + \frac 3 4 \ln 2
        \end{align*}
        \item[(2)]
        熟知 $\frac 1 {1-x} = \sum_{n=0}^{\infty} x^n,\ \frac{x^3}{1-x} = \sum_{n=3}^{\infty} x^n$,则
        \[
            \left( \frac{x^3}{1-x} \right)'' = \sum_{n=3}^{\infty} n(n-1) x^{n-2}
        \]
        故
        \begin{align*}
            \sum_{n=1}^{\infty} \frac{(-1)^n (n^2-n+1)}{2^n} &= \frac 1 2 + \frac 1 4 \sum_{n=3}^{\infty} n(n-1) \left( - \frac 1 2 \right)^{n-2} + \sum_{n=1}^{\infty} \left( -\frac 1 2 \right)^n \\
            &= - \frac{41}{27}
        \end{align*}
        \item[(3)]
        熟知 $\frac 1 {1-x^3} = \sum_{n=0}^{\infty} x^{3n}$,两边积分得
        \[
            -\frac 1 3 \ln(1-x) + \frac 1 6 \ln(1 + x + x^2) + \frac {2 \sqrt 3} 3 \arctan\left( \frac{2x + 1}{\sqrt 3} \right) = \sum_{n=0}^{\infty} \frac{x^{3n+1}}{3n+1}
        \]
        故
        \[
            \sum_{n=0}^{\infty} \frac{(-1)^n}{3n+1} = - \sum_{n=0}^{\infty} \frac{(-1)^{3n+1}}{3n+1} = \frac {\sqrt 3 \pi} 9 + \frac 1 3 \ln 2
        \]
        \item[(4)]
        有如下结论:$\sum_{n=0}^{\infty} \frac{n^k}{n!} = ke \ (k \in \mathbb N)$,可归纳证明. 则
        \begin{align*}
            \sum_{n=0}^{\infty} \frac{(n+1)^2}{n!} &= \sum_{n=0}^{\infty} \frac{n^2}{n!} + 2 \sum_{n=0}^{\infty} \frac n {n!} + \sum_{n=0}^{\infty} \frac 1 {n!} \\
            &= 2e + 2e + e = 5e
        \end{align*}
    \end{enumerate}
    \item[7.]
    不妨设 $y = \sum_{n=0}^{\infty} a_n x^n$. 令 $x = 0$,则有 $a_0 + \lambda \sin a_0 = 0$,由于 $y(x)$ 是确定的,故只能有 $a_0 = 0$. 原式两边求导,得 $y' + \lambda y' \cos y = 1$,再令 $x = 0$,得 $a_1 + \lambda a_1 = 1$,也即 $a_1 = \frac 1 {1+\lambda}$. 依此类推,可求得
    \[
        y = \frac 1 {1 + \lambda} x + \frac{\lambda}{6 (1 + \lambda)^4} x^3
    \]
\end{enumerate}

\section{级数的应用}

\begin{enumerate}
    \item[6.]
    \textbf{证明} \ 由 Stolz 定理知
    \begin{align*}
        \lim_{n \to \infty} \frac{\ln n!}{\ln n^n} &= \lim_{n \to \infty} \frac{\ln n! - \ln (n-1)!}{\ln n^n - \ln (n-1)^{n-1}} \\
        &= \lim_{n \to \infty} \frac{\ln n}{\ln n + (n-1) \ln\left( 1 + \frac 1 {n-1} \right)} \\
        &= \lim_{n \to \infty} \frac{\ln n}{\ln n + 1} \\
        &= 1
    \end{align*}
    $\hfill\qed$

    或由 Stirling 公式立得.
\end{enumerate}

\section*{第 7 章综合习题}

\begin{enumerate}
    \item[3.]
    \textbf{证明} \ 先证充分性. 不妨设 $|a_n| \leqslant M$,则
    \[
        S_n = \sum_{k=1}^n \frac{a_{k+1} - a_k}{a_k} < \sum_{k=1}^n \frac{a_{k+1} - a_k}{a_1} = \frac{a_{n+1}}{a_1} - 1 \leqslant \frac M {a_1} - 1
    \]
    即 $\sum_{n=1}^{\infty} \left( \frac{a_{n+1}}{a_n} - 1 \right)$ 收敛.

    再证必要性. 由级数收敛知,对任意 $0 < \varepsilon < \frac 1 2$,存在 $N \in \mathbf N_+$,使得当 $m > n \geqslant N$ 时有
    \[
        1 - \frac{a_n}{a_{m}} < \left| \sum_{k=n}^{m-1} a_{k+1} \left( \frac 1 {a_k} - \frac 1 {a_{k+1}} \right) \right| < \varepsilon
    \]
    即对任意 $m > N$ 有 $a_m < \frac{a_N}{1 - \varepsilon} < 2a_N$. 取 $M = \max\left\{ a_1,\ a_2,\ \ldots,\ 2a_N \right\}$,则恒有 $a_n \leqslant M$,即 $\{ a_n \}$ 有界.
    $\hfill\qed$
    \item[4.]
    \textbf{证明}
    \begin{enumerate}
        \item[(a)]
        当 $\alpha \geqslant 1$ 时:
        \[
            S_n = \sum_{k=1}^n \left( 1 - \frac{a_k}{a_{k+1}} \right) \frac 1 {a_k^{\alpha}} \leqslant \sum_{k=1}^n \left( 1 - \frac{a_k^{\alpha}}{a_{k+1}^{\alpha}} \right) \frac 1 {a_k^{\alpha}} = \frac 1 {a_1^{\alpha}} - \frac 1 {a_{n+1}^{\alpha}} < \frac 1 {a_1^{\alpha}}
        \]
        则 $\sum_{n=1}^{\infty} \frac{a_{n+1} - a_n}{a_{n+1} a_n^{\alpha}}$ 收敛.
        \item[(b)]
        当 $0 < \alpha < 1$ 时:

        由 Lagrange 中值定理知,存在 $\xi \in (a_n, a_{n+1})$ 使得
        \[
            \frac{a_{n+1}^{\alpha} - a_n^{\alpha}}{a_{n+1} - a_n} = \alpha \xi^{\alpha - 1} > \alpha a_{n+1}^{\alpha - 1}
        \]
        即 $\frac{a_{n+1} - a_n}{a_{n+1} a_n^{\alpha}} < \frac 1 {\alpha} \left( \frac 1 {a_n^{\alpha}} - \frac 1 {a_{n+1}^{\alpha}} \right)$. 故
        \[
            S_n < \frac 1 {\alpha a_1^{\alpha}}  - \frac 1 {\alpha a_{n+1}^{\alpha}} < \frac 1 {\alpha a_1^{\alpha}}
        \]
        即 $\sum_{n=1}^{\infty} \frac{a_{n+1} - a_n}{a_{n+1} a_n^{\alpha}}$ 收敛.
        $\hfill\qed$
    \end{enumerate}
    \item[5.]
    \textbf{证明} \ 易知
    \[
        \prod_{n=1}^{\infty} (1 + c_n) = \exp\left( \sum_{n=1}^{\infty} \ln(1 + c_n) \right) \leqslant \exp\left( \sum_{n=1}^{\infty} c_n \right) = e^M = M_1
    \]
    则原不等式可化为
    \[
        \frac{a_{n+1}}{\prod_{k=1}^n (1+c_k)} - \frac{a_n}{\prod_{k=1}^{n-1} (1+c_k)} \leqslant - \frac{b_n \Phi(a_n)}{\prod_{k=1}^n (1+c_n)} < 0
    \]
    不妨设 $d_n = \frac{a_n}{\prod_{k=1}^{n-1} (1+c_k)}$,则 $d_n$ 单调递减有下界,即 $\{ d_n \}$ 收敛. 而若 $\lim_{n \to \infty} d_n = d > 0$,则
    \[
        d_{n+1} - d_n \leqslant -b_n \frac{\Phi(d)}{M_1} \ \Rightarrow \ d_{n+1} \leqslant d_1 - \frac{\Phi(d)}{M_1} \sum_{k=1}^{n} b_k
    \]
    则 $\lim_{n \to \infty} d_n = -\infty$,矛盾,故 $\lim_{n \to \infty} d_n = 0$,进而有 $\lim_{n \to \infty} a_n = 0$.
    $\hfill\qed$
    \item[6.]
    \textbf{证明} \ 由 Cauchy-Schwarz 不等式知
    \[
        (a_1 + a_2 + \cdots + a_n)\left( \frac 1 {a_1} + \frac 4 {a_2} + \cdots + \frac{n^2}{a_n} \right) \geqslant \frac {n^2(n+1)^2} 4
    \]
    即
    \begin{align*}
        \frac n {a_1 + a_2 + \cdots + a_n} &\leqslant \frac 4 {n(n+1)^2} \sum_{k=1}^n \frac{k^2}{a_k} \\
        &\leqslant 2 \frac{2n+1}{n^2(n+1)^2} \sum_{k=1}^n \frac{k^2}{a_k} \\
        &= 2 \left( \frac 1 {n^2} - \frac 1 {(n+1)^2} \right) \sum_{k=1}^n \frac{k^2}{a_k}
    \end{align*}
    对 $n$ 求和得
    \begin{align*}
        \sum_{n=1}^N \frac n {a_1 + a_2 + \cdots + a_n} &\leqslant \sum_{n=1}^N 2 \left( \frac 1 {n^2} - \frac 1 {(n+1)^2} \right) \sum_{k=1}^n \frac{k^2}{a_k} \\
        &= 2 \sum_{k=1}^N \frac{k^2}{a_k} \sum_{n=k}^N \left( \frac 1 {n^2} - \frac 1 {(n+1)^2} \right) \\
        &= 2 \sum_{k=1}^N \left( \frac 1 {a_k} - \frac{k^2}{(N+1)^2 a_k} \right)
    \end{align*}
    令 $N \to \infty$,则有
    \[
        \sum_{n=1}^{\infty} \frac n {a_1 + a_2 + \cdots + a_n} \leqslant 2 \sum_{n=1}^{\infty} \frac 1 {a_n}
    \]
    此题亦可用数学归纳法证明\footnote{见\href{https://math.stackexchange.com/posts/108598/revisions}{此解}.},事实上,我们有更强的结论\footnote{见\href{https://www.komal.hu/feladat?a=feladat&f=A709&l=en}{此解},其亦确定了最佳系数.}.
    \item[7.]
    \textbf{证明} \ 不妨令 $a_1 = 0$,则由上述结论知
    \[
        \sum_{n=1}^{\infty} \frac 1 {a_{n+1} - a_n} \geqslant \frac 1 2 \sum_{n=1}^{\infty} \frac n {a_{n+1} - a_1} \geqslant \frac n {(n+1)^2 \ln(n+1)}
    \]
    故 $\sum_{n=1}^{\infty} \frac 1 {a_{n+1} - a_n}$ 发散.
    $\hfill\qed$
    \item[9.]
    \textbf{证明} \ 设 $f_n(x) = x^{a_n} \ (n \in \mathbf N)$,则 $f_{n+1}(x) = x^{\frac {a_n + 1} 2} = x^{a_{n+1}}$. 易证 $\lim_{n \to \infty} a_n = 1$,即 $\lim_{n \to \infty} f_n(x) = x$.
    $\hfill\qed$
    \item[11.]
    \textbf{证明} \ 设 $|f_0(x)| \leqslant M$,则 $|f_n(x)| \leqslant \frac{Ma^n}{n!}$,显然一致收敛于 $0$.
    $\hfill\qed$

    事实上有
    \begin{align*}
        \int_0^x \frac{(x-t)^n}{n!} f_0(t) \mathrm dt &= \left. \frac{(x-t)^n}{n!} f_1(t) \right|_0^x + \int_0^x \frac{(x-t)^{n-1}}{(n-1)!} f_1(t) \mathrm dt \\
        &= \int_0^x \frac{(x-t)^{n-1}}{(n-1)!} f_1(t) \mathrm dt \\
        &= \cdots \\
        &= \int_0^x f_n(t) \mathrm dt = f_{n+1}(x)
    \end{align*}

    若我们先假设 $\{ f_n(x) \}$ 收敛至 $f(x)$,则有 $f(x) = f'(x)$,解得 $f(x) = Ce^x$. 由题设知 $f(0) = 0$,故 $C = 0$,也即 $f(x) \equiv 0$.
\end{enumerate}

\end{document}
